
%!TEX root = main.tex

\section*{Python Cheat Sheet}\label{sec:pythonCheat}
\subsection*{Numerisches,Arrays und Plots}
\begin{table}[H]
    \centering
    \begin{tabular}{|l|l|p{7cm}|}
        \hline
    \textbf{Befehl} & \textbf{Beispiel} & \textbf{Kommentar} \\ \hline
    
    \texttt{\%pylab} & \texttt{\%pylab} & Schneller import von üblichen paketen (\texttt{matplotlib} und \texttt{numpy}). \\ \hline
    \texttt{zeros()} & \texttt{x = zeros(5)} & Generierung eines arrays, mit nullen befüllt. \\ \hline
    \texttt{linspace()} & \texttt{x = linspace(0,1, 5)} & Generierung eines arrays, zahlen linear zwischen Argument 1 und 2 erzeugt, Anzahl der zahlen bestimmt durch argument 3. \\ \hline
    \texttt{plot()} & \texttt{plot(x,y)} & Line-plot erzeugen.  \\ \hline
    \texttt{clf()} & \texttt{clf()} & Clear figure. Aktuellen plot löschen.  \\ \hline
    \texttt{scatter()} & \texttt{scatter(x,y)} & Scatter Plot erzeugen. \\ \hline
    \texttt{stem()} & \texttt{stem(x,y)} & Stem Plot erzeugen. \\ \hline

    \texttt{[]} & \texttt{x[5] = 3} & Element aus array anzeigen holen oder bearbeiten. Im Bsp. wird dem 6. Element des array \texttt{x} der Wert 3 zugewiesen. \\ \hline
    \texttt{random.random()} & \texttt{random.random(50)} & Erzeuge 50 gleichverteilte Zufallszahlen im halboffenen Intervall $[0.0, 1.0)$. \\ \hline
    \texttt{**} & \texttt{x**3} & Exponenzieren. Im Bsp.: $x^3$. \\ \hline
    \texttt{1j} & \texttt{z = 3+5j} & Die Zahl $i$. \\ \hline
    \texttt{abs()} & \texttt{r = abs(z)} & Berechnet den Betrag von \texttt{z}. \\ \hline
    \texttt{angle()} & \texttt{phi = angle(z)} & Berechnet das Argument von \texttt{z}. \\ \hline

    \end{tabular}
\end{table}

\subsection*{Symbolische Mathematik}
\begin{table}[H]
    \centering
    \begin{tabular}{|p{3cm}|p{6cm}|p{6cm}|}
        \hline
    \textbf{Befehl} & \textbf{Beispiel} & \textbf{Kommentar} \\ \hline
    
    \texttt{import sympy as sp} & \texttt{import sympy as sp} & Import des \texttt{sympy} Pakets für symbolische Berechnungen. \\ \hline
    \texttt{from sympy.abc import} & \texttt{from sympy.abc import x,y,z} & Variablen \texttt{x,y,z} als symbolische variablen importieren. \\ \hline
    \texttt{sp.roots()} & \texttt{nullst = sp.roots(p)} & Findet die Nullstellen des polynoms \texttt{p}. \\ \hline
    \texttt{sp.factor()} & \texttt{factors = sp.factor(p)} & Linearfaktor Zerlegung des polynoms \texttt{p}. \\ \hline
    \texttt{sp.apart()} & \texttt{brueche = sp.apart(p, full=True).doit()} & Partialbruch Zerlegung polynoms \texttt{p}. \\ \hline
    \texttt{sp.lambdify()} & \texttt{f  = sp.lambdify(x,p)} & Erzeugt eine ausführbare numerische funktion (\texttt{f}) aus dem symbolischen Ausdruck \texttt{p}. \\ \hline

    \end{tabular}
\end{table}

\subsection*{Sound I/O}
\begin{table}[H]
    \centering
    \begin{tabular}{|p{3cm}|p{6cm}|p{6cm}|}
        \hline
    \textbf{Befehl} & \textbf{Beispiel} & \textbf{Kommentar} \\ \hline
    
    \texttt{import soundfile as sf} & \texttt{import soundfile as sf} & Import des \texttt{soundfile} Pakets für Audio Datei Verarbeitung. \\ \hline
    \texttt{sf.write()} & \texttt{sf.write('test.wav', a, 44100)} & Schreibt den array \texttt{a} in das file 'test.wav' mit samplerate 44100 \\ \hline
    \texttt{sf.read()} & \texttt{x,sr =sf.read('test.wav')} & Lädt das file 'test.wav'. \texttt{x} enthält die sample werte, \texttt{sr} die sampingrate. \\ \hline

    \end{tabular}
\end{table}
