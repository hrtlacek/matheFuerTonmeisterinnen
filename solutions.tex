%!TEX root = main.tex
\chapter{Lösungen}

\section{Kapitel 1}
\begin{enumerate}
    \item Siehe Abb. \ref{fig:dreiMengenLoes}
    \item 
    \begin{enumerate}
\item Siehe Abb \ref{fig:Loes_min}
\item Siehe Abb \ref{fig:Loes_abs}. Die Bedeutung der Parameter ist durchaus greifbar: die eine Zahl schiebt den Graphen auf und ab, die andere nach links und rechts. Siehe Abb. \ref{fig:Loes_abs_var}.
    \end{enumerate}

\item Eine vielleich umständliche aber richtige Lösung wäre mit einer Fallunterscheidung:
$$ f(x) = \begin{cases} 
-0.9 & \text{für } x \leq -0.9 \\ 
0.9 & \text{für } x \geq 0.9 \\
x & \text{für }  -0.9 < x < 0.9
\end{cases}$$

Wesentlich eleganter ist aber: $f(x) = min(0.9, max(-0.9, x))$.

\end{enumerate}

\begin{figure}[h]
    \centering
    \includegraphics[width=0.3\textwidth]{img/dreiMeingenLoesung.png}
    \caption{Drei Mengen, Lösung. }
    \label{fig:dreiMengenLoes}
\end{figure}

\begin{figure}[h]
    \centering
    \includegraphics[width=0.3\textwidth]{img/loesung_min.png}
    \caption{Lösung. }
    \label{fig:Loes_min}
\end{figure}

\begin{figure}[h]
    \centering
    \includegraphics[width=0.3\textwidth]{img/loesung_abs.png}
    \caption{Lösung. }
    \label{fig:Loes_abs}
\end{figure}

\begin{figure}[h]
    \centering
    %% Creator: Matplotlib, PGF backend
%%
%% To include the figure in your LaTeX document, write
%%   \input{<filename>.pgf}
%%
%% Make sure the required packages are loaded in your preamble
%%   \usepackage{pgf}
%%
%% Also ensure that all the required font packages are loaded; for instance,
%% the lmodern package is sometimes necessary when using math font.
%%   \usepackage{lmodern}
%%
%% Figures using additional raster images can only be included by \input if
%% they are in the same directory as the main LaTeX file. For loading figures
%% from other directories you can use the `import` package
%%   \usepackage{import}
%%
%% and then include the figures with
%%   \import{<path to file>}{<filename>.pgf}
%%
%% Matplotlib used the following preamble
%%   \def\mathdefault#1{#1}
%%   \everymath=\expandafter{\the\everymath\displaystyle}
%%   
%%   \usepackage{fontspec}
%%   \setmainfont{VeraSe.ttf}[Path=\detokenize{/usr/share/fonts/TTF/}]
%%   \setsansfont{DejaVuSans.ttf}[Path=\detokenize{/home/pl/miniconda3/lib/python3.12/site-packages/matplotlib/mpl-data/fonts/ttf/}]
%%   \setmonofont{DejaVuSansMono.ttf}[Path=\detokenize{/home/pl/miniconda3/lib/python3.12/site-packages/matplotlib/mpl-data/fonts/ttf/}]
%%   \makeatletter\@ifpackageloaded{underscore}{}{\usepackage[strings]{underscore}}\makeatother
%%
\begingroup%
\makeatletter%
\begin{pgfpicture}%
\pgfpathrectangle{\pgfpointorigin}{\pgfqpoint{5.918612in}{3.167157in}}%
\pgfusepath{use as bounding box, clip}%
\begin{pgfscope}%
\pgfsetbuttcap%
\pgfsetmiterjoin%
\definecolor{currentfill}{rgb}{1.000000,1.000000,1.000000}%
\pgfsetfillcolor{currentfill}%
\pgfsetlinewidth{0.000000pt}%
\definecolor{currentstroke}{rgb}{1.000000,1.000000,1.000000}%
\pgfsetstrokecolor{currentstroke}%
\pgfsetdash{}{0pt}%
\pgfpathmoveto{\pgfqpoint{0.000000in}{0.000000in}}%
\pgfpathlineto{\pgfqpoint{5.918612in}{0.000000in}}%
\pgfpathlineto{\pgfqpoint{5.918612in}{3.167157in}}%
\pgfpathlineto{\pgfqpoint{0.000000in}{3.167157in}}%
\pgfpathlineto{\pgfqpoint{0.000000in}{0.000000in}}%
\pgfpathclose%
\pgfusepath{fill}%
\end{pgfscope}%
\begin{pgfscope}%
\pgfsetbuttcap%
\pgfsetmiterjoin%
\definecolor{currentfill}{rgb}{1.000000,1.000000,1.000000}%
\pgfsetfillcolor{currentfill}%
\pgfsetlinewidth{0.000000pt}%
\definecolor{currentstroke}{rgb}{0.000000,0.000000,0.000000}%
\pgfsetstrokecolor{currentstroke}%
\pgfsetstrokeopacity{0.000000}%
\pgfsetdash{}{0pt}%
\pgfpathmoveto{\pgfqpoint{0.393613in}{0.548486in}}%
\pgfpathlineto{\pgfqpoint{2.859522in}{0.548486in}}%
\pgfpathlineto{\pgfqpoint{2.859522in}{3.014395in}}%
\pgfpathlineto{\pgfqpoint{0.393613in}{3.014395in}}%
\pgfpathlineto{\pgfqpoint{0.393613in}{0.548486in}}%
\pgfpathclose%
\pgfusepath{fill}%
\end{pgfscope}%
\begin{pgfscope}%
\pgfsetbuttcap%
\pgfsetroundjoin%
\definecolor{currentfill}{rgb}{0.000000,0.000000,0.000000}%
\pgfsetfillcolor{currentfill}%
\pgfsetlinewidth{0.803000pt}%
\definecolor{currentstroke}{rgb}{0.000000,0.000000,0.000000}%
\pgfsetstrokecolor{currentstroke}%
\pgfsetdash{}{0pt}%
\pgfsys@defobject{currentmarker}{\pgfqpoint{0.000000in}{-0.048611in}}{\pgfqpoint{0.000000in}{0.000000in}}{%
\pgfpathmoveto{\pgfqpoint{0.000000in}{0.000000in}}%
\pgfpathlineto{\pgfqpoint{0.000000in}{-0.048611in}}%
\pgfusepath{stroke,fill}%
}%
\begin{pgfscope}%
\pgfsys@transformshift{0.804597in}{0.548486in}%
\pgfsys@useobject{currentmarker}{}%
\end{pgfscope}%
\end{pgfscope}%
\begin{pgfscope}%
\definecolor{textcolor}{rgb}{0.000000,0.000000,0.000000}%
\pgfsetstrokecolor{textcolor}%
\pgfsetfillcolor{textcolor}%
\pgftext[x=0.804597in,y=0.451264in,,top]{\color{textcolor}{\rmfamily\fontsize{10.000000}{12.000000}\selectfont\catcode`\^=\active\def^{\ifmmode\sp\else\^{}\fi}\catcode`\%=\active\def%{\%}\ensuremath{-}2}}%
\end{pgfscope}%
\begin{pgfscope}%
\pgfsetbuttcap%
\pgfsetroundjoin%
\definecolor{currentfill}{rgb}{0.000000,0.000000,0.000000}%
\pgfsetfillcolor{currentfill}%
\pgfsetlinewidth{0.803000pt}%
\definecolor{currentstroke}{rgb}{0.000000,0.000000,0.000000}%
\pgfsetstrokecolor{currentstroke}%
\pgfsetdash{}{0pt}%
\pgfsys@defobject{currentmarker}{\pgfqpoint{0.000000in}{-0.048611in}}{\pgfqpoint{0.000000in}{0.000000in}}{%
\pgfpathmoveto{\pgfqpoint{0.000000in}{0.000000in}}%
\pgfpathlineto{\pgfqpoint{0.000000in}{-0.048611in}}%
\pgfusepath{stroke,fill}%
}%
\begin{pgfscope}%
\pgfsys@transformshift{1.626567in}{0.548486in}%
\pgfsys@useobject{currentmarker}{}%
\end{pgfscope}%
\end{pgfscope}%
\begin{pgfscope}%
\definecolor{textcolor}{rgb}{0.000000,0.000000,0.000000}%
\pgfsetstrokecolor{textcolor}%
\pgfsetfillcolor{textcolor}%
\pgftext[x=1.626567in,y=0.451264in,,top]{\color{textcolor}{\rmfamily\fontsize{10.000000}{12.000000}\selectfont\catcode`\^=\active\def^{\ifmmode\sp\else\^{}\fi}\catcode`\%=\active\def%{\%}0}}%
\end{pgfscope}%
\begin{pgfscope}%
\pgfsetbuttcap%
\pgfsetroundjoin%
\definecolor{currentfill}{rgb}{0.000000,0.000000,0.000000}%
\pgfsetfillcolor{currentfill}%
\pgfsetlinewidth{0.803000pt}%
\definecolor{currentstroke}{rgb}{0.000000,0.000000,0.000000}%
\pgfsetstrokecolor{currentstroke}%
\pgfsetdash{}{0pt}%
\pgfsys@defobject{currentmarker}{\pgfqpoint{0.000000in}{-0.048611in}}{\pgfqpoint{0.000000in}{0.000000in}}{%
\pgfpathmoveto{\pgfqpoint{0.000000in}{0.000000in}}%
\pgfpathlineto{\pgfqpoint{0.000000in}{-0.048611in}}%
\pgfusepath{stroke,fill}%
}%
\begin{pgfscope}%
\pgfsys@transformshift{2.448537in}{0.548486in}%
\pgfsys@useobject{currentmarker}{}%
\end{pgfscope}%
\end{pgfscope}%
\begin{pgfscope}%
\definecolor{textcolor}{rgb}{0.000000,0.000000,0.000000}%
\pgfsetstrokecolor{textcolor}%
\pgfsetfillcolor{textcolor}%
\pgftext[x=2.448537in,y=0.451264in,,top]{\color{textcolor}{\rmfamily\fontsize{10.000000}{12.000000}\selectfont\catcode`\^=\active\def^{\ifmmode\sp\else\^{}\fi}\catcode`\%=\active\def%{\%}2}}%
\end{pgfscope}%
\begin{pgfscope}%
\definecolor{textcolor}{rgb}{0.000000,0.000000,0.000000}%
\pgfsetstrokecolor{textcolor}%
\pgfsetfillcolor{textcolor}%
\pgftext[x=1.626567in,y=0.261295in,,top]{\color{textcolor}{\rmfamily\fontsize{12.000000}{14.400000}\selectfont\catcode`\^=\active\def^{\ifmmode\sp\else\^{}\fi}\catcode`\%=\active\def%{\%}$x$}}%
\end{pgfscope}%
\begin{pgfscope}%
\pgfsetbuttcap%
\pgfsetroundjoin%
\definecolor{currentfill}{rgb}{0.000000,0.000000,0.000000}%
\pgfsetfillcolor{currentfill}%
\pgfsetlinewidth{0.803000pt}%
\definecolor{currentstroke}{rgb}{0.000000,0.000000,0.000000}%
\pgfsetstrokecolor{currentstroke}%
\pgfsetdash{}{0pt}%
\pgfsys@defobject{currentmarker}{\pgfqpoint{-0.048611in}{0.000000in}}{\pgfqpoint{-0.000000in}{0.000000in}}{%
\pgfpathmoveto{\pgfqpoint{-0.000000in}{0.000000in}}%
\pgfpathlineto{\pgfqpoint{-0.048611in}{0.000000in}}%
\pgfusepath{stroke,fill}%
}%
\begin{pgfscope}%
\pgfsys@transformshift{0.393613in}{0.548486in}%
\pgfsys@useobject{currentmarker}{}%
\end{pgfscope}%
\end{pgfscope}%
\begin{pgfscope}%
\definecolor{textcolor}{rgb}{0.000000,0.000000,0.000000}%
\pgfsetstrokecolor{textcolor}%
\pgfsetfillcolor{textcolor}%
\pgftext[x=0.100000in, y=0.495724in, left, base]{\color{textcolor}{\rmfamily\fontsize{10.000000}{12.000000}\selectfont\catcode`\^=\active\def^{\ifmmode\sp\else\^{}\fi}\catcode`\%=\active\def%{\%}\ensuremath{-}2}}%
\end{pgfscope}%
\begin{pgfscope}%
\pgfsetbuttcap%
\pgfsetroundjoin%
\definecolor{currentfill}{rgb}{0.000000,0.000000,0.000000}%
\pgfsetfillcolor{currentfill}%
\pgfsetlinewidth{0.803000pt}%
\definecolor{currentstroke}{rgb}{0.000000,0.000000,0.000000}%
\pgfsetstrokecolor{currentstroke}%
\pgfsetdash{}{0pt}%
\pgfsys@defobject{currentmarker}{\pgfqpoint{-0.048611in}{0.000000in}}{\pgfqpoint{-0.000000in}{0.000000in}}{%
\pgfpathmoveto{\pgfqpoint{-0.000000in}{0.000000in}}%
\pgfpathlineto{\pgfqpoint{-0.048611in}{0.000000in}}%
\pgfusepath{stroke,fill}%
}%
\begin{pgfscope}%
\pgfsys@transformshift{0.393613in}{0.959471in}%
\pgfsys@useobject{currentmarker}{}%
\end{pgfscope}%
\end{pgfscope}%
\begin{pgfscope}%
\definecolor{textcolor}{rgb}{0.000000,0.000000,0.000000}%
\pgfsetstrokecolor{textcolor}%
\pgfsetfillcolor{textcolor}%
\pgftext[x=0.100000in, y=0.906709in, left, base]{\color{textcolor}{\rmfamily\fontsize{10.000000}{12.000000}\selectfont\catcode`\^=\active\def^{\ifmmode\sp\else\^{}\fi}\catcode`\%=\active\def%{\%}\ensuremath{-}1}}%
\end{pgfscope}%
\begin{pgfscope}%
\pgfsetbuttcap%
\pgfsetroundjoin%
\definecolor{currentfill}{rgb}{0.000000,0.000000,0.000000}%
\pgfsetfillcolor{currentfill}%
\pgfsetlinewidth{0.803000pt}%
\definecolor{currentstroke}{rgb}{0.000000,0.000000,0.000000}%
\pgfsetstrokecolor{currentstroke}%
\pgfsetdash{}{0pt}%
\pgfsys@defobject{currentmarker}{\pgfqpoint{-0.048611in}{0.000000in}}{\pgfqpoint{-0.000000in}{0.000000in}}{%
\pgfpathmoveto{\pgfqpoint{-0.000000in}{0.000000in}}%
\pgfpathlineto{\pgfqpoint{-0.048611in}{0.000000in}}%
\pgfusepath{stroke,fill}%
}%
\begin{pgfscope}%
\pgfsys@transformshift{0.393613in}{1.370456in}%
\pgfsys@useobject{currentmarker}{}%
\end{pgfscope}%
\end{pgfscope}%
\begin{pgfscope}%
\definecolor{textcolor}{rgb}{0.000000,0.000000,0.000000}%
\pgfsetstrokecolor{textcolor}%
\pgfsetfillcolor{textcolor}%
\pgftext[x=0.208025in, y=1.317694in, left, base]{\color{textcolor}{\rmfamily\fontsize{10.000000}{12.000000}\selectfont\catcode`\^=\active\def^{\ifmmode\sp\else\^{}\fi}\catcode`\%=\active\def%{\%}0}}%
\end{pgfscope}%
\begin{pgfscope}%
\pgfsetbuttcap%
\pgfsetroundjoin%
\definecolor{currentfill}{rgb}{0.000000,0.000000,0.000000}%
\pgfsetfillcolor{currentfill}%
\pgfsetlinewidth{0.803000pt}%
\definecolor{currentstroke}{rgb}{0.000000,0.000000,0.000000}%
\pgfsetstrokecolor{currentstroke}%
\pgfsetdash{}{0pt}%
\pgfsys@defobject{currentmarker}{\pgfqpoint{-0.048611in}{0.000000in}}{\pgfqpoint{-0.000000in}{0.000000in}}{%
\pgfpathmoveto{\pgfqpoint{-0.000000in}{0.000000in}}%
\pgfpathlineto{\pgfqpoint{-0.048611in}{0.000000in}}%
\pgfusepath{stroke,fill}%
}%
\begin{pgfscope}%
\pgfsys@transformshift{0.393613in}{1.781441in}%
\pgfsys@useobject{currentmarker}{}%
\end{pgfscope}%
\end{pgfscope}%
\begin{pgfscope}%
\definecolor{textcolor}{rgb}{0.000000,0.000000,0.000000}%
\pgfsetstrokecolor{textcolor}%
\pgfsetfillcolor{textcolor}%
\pgftext[x=0.208025in, y=1.728679in, left, base]{\color{textcolor}{\rmfamily\fontsize{10.000000}{12.000000}\selectfont\catcode`\^=\active\def^{\ifmmode\sp\else\^{}\fi}\catcode`\%=\active\def%{\%}1}}%
\end{pgfscope}%
\begin{pgfscope}%
\pgfsetbuttcap%
\pgfsetroundjoin%
\definecolor{currentfill}{rgb}{0.000000,0.000000,0.000000}%
\pgfsetfillcolor{currentfill}%
\pgfsetlinewidth{0.803000pt}%
\definecolor{currentstroke}{rgb}{0.000000,0.000000,0.000000}%
\pgfsetstrokecolor{currentstroke}%
\pgfsetdash{}{0pt}%
\pgfsys@defobject{currentmarker}{\pgfqpoint{-0.048611in}{0.000000in}}{\pgfqpoint{-0.000000in}{0.000000in}}{%
\pgfpathmoveto{\pgfqpoint{-0.000000in}{0.000000in}}%
\pgfpathlineto{\pgfqpoint{-0.048611in}{0.000000in}}%
\pgfusepath{stroke,fill}%
}%
\begin{pgfscope}%
\pgfsys@transformshift{0.393613in}{2.192425in}%
\pgfsys@useobject{currentmarker}{}%
\end{pgfscope}%
\end{pgfscope}%
\begin{pgfscope}%
\definecolor{textcolor}{rgb}{0.000000,0.000000,0.000000}%
\pgfsetstrokecolor{textcolor}%
\pgfsetfillcolor{textcolor}%
\pgftext[x=0.208025in, y=2.139664in, left, base]{\color{textcolor}{\rmfamily\fontsize{10.000000}{12.000000}\selectfont\catcode`\^=\active\def^{\ifmmode\sp\else\^{}\fi}\catcode`\%=\active\def%{\%}2}}%
\end{pgfscope}%
\begin{pgfscope}%
\pgfsetbuttcap%
\pgfsetroundjoin%
\definecolor{currentfill}{rgb}{0.000000,0.000000,0.000000}%
\pgfsetfillcolor{currentfill}%
\pgfsetlinewidth{0.803000pt}%
\definecolor{currentstroke}{rgb}{0.000000,0.000000,0.000000}%
\pgfsetstrokecolor{currentstroke}%
\pgfsetdash{}{0pt}%
\pgfsys@defobject{currentmarker}{\pgfqpoint{-0.048611in}{0.000000in}}{\pgfqpoint{-0.000000in}{0.000000in}}{%
\pgfpathmoveto{\pgfqpoint{-0.000000in}{0.000000in}}%
\pgfpathlineto{\pgfqpoint{-0.048611in}{0.000000in}}%
\pgfusepath{stroke,fill}%
}%
\begin{pgfscope}%
\pgfsys@transformshift{0.393613in}{2.603410in}%
\pgfsys@useobject{currentmarker}{}%
\end{pgfscope}%
\end{pgfscope}%
\begin{pgfscope}%
\definecolor{textcolor}{rgb}{0.000000,0.000000,0.000000}%
\pgfsetstrokecolor{textcolor}%
\pgfsetfillcolor{textcolor}%
\pgftext[x=0.208025in, y=2.550649in, left, base]{\color{textcolor}{\rmfamily\fontsize{10.000000}{12.000000}\selectfont\catcode`\^=\active\def^{\ifmmode\sp\else\^{}\fi}\catcode`\%=\active\def%{\%}3}}%
\end{pgfscope}%
\begin{pgfscope}%
\pgfsetbuttcap%
\pgfsetroundjoin%
\definecolor{currentfill}{rgb}{0.000000,0.000000,0.000000}%
\pgfsetfillcolor{currentfill}%
\pgfsetlinewidth{0.803000pt}%
\definecolor{currentstroke}{rgb}{0.000000,0.000000,0.000000}%
\pgfsetstrokecolor{currentstroke}%
\pgfsetdash{}{0pt}%
\pgfsys@defobject{currentmarker}{\pgfqpoint{-0.048611in}{0.000000in}}{\pgfqpoint{-0.000000in}{0.000000in}}{%
\pgfpathmoveto{\pgfqpoint{-0.000000in}{0.000000in}}%
\pgfpathlineto{\pgfqpoint{-0.048611in}{0.000000in}}%
\pgfusepath{stroke,fill}%
}%
\begin{pgfscope}%
\pgfsys@transformshift{0.393613in}{3.014395in}%
\pgfsys@useobject{currentmarker}{}%
\end{pgfscope}%
\end{pgfscope}%
\begin{pgfscope}%
\definecolor{textcolor}{rgb}{0.000000,0.000000,0.000000}%
\pgfsetstrokecolor{textcolor}%
\pgfsetfillcolor{textcolor}%
\pgftext[x=0.208025in, y=2.961634in, left, base]{\color{textcolor}{\rmfamily\fontsize{10.000000}{12.000000}\selectfont\catcode`\^=\active\def^{\ifmmode\sp\else\^{}\fi}\catcode`\%=\active\def%{\%}4}}%
\end{pgfscope}%
\begin{pgfscope}%
\pgfpathrectangle{\pgfqpoint{0.393613in}{0.548486in}}{\pgfqpoint{2.465909in}{2.465909in}}%
\pgfusepath{clip}%
\pgfsetrectcap%
\pgfsetroundjoin%
\pgfsetlinewidth{1.505625pt}%
\definecolor{currentstroke}{rgb}{0.000000,0.000000,0.000000}%
\pgfsetstrokecolor{currentstroke}%
\pgfsetdash{}{0pt}%
\pgfpathmoveto{\pgfqpoint{0.393613in}{3.014395in}}%
\pgfpathlineto{\pgfqpoint{0.418521in}{2.989487in}}%
\pgfpathlineto{\pgfqpoint{0.443429in}{2.964579in}}%
\pgfpathlineto{\pgfqpoint{0.468337in}{2.939671in}}%
\pgfpathlineto{\pgfqpoint{0.493245in}{2.914762in}}%
\pgfpathlineto{\pgfqpoint{0.518153in}{2.889854in}}%
\pgfpathlineto{\pgfqpoint{0.543062in}{2.864946in}}%
\pgfpathlineto{\pgfqpoint{0.567970in}{2.840038in}}%
\pgfpathlineto{\pgfqpoint{0.592878in}{2.815130in}}%
\pgfpathlineto{\pgfqpoint{0.617786in}{2.790222in}}%
\pgfpathlineto{\pgfqpoint{0.642694in}{2.765313in}}%
\pgfpathlineto{\pgfqpoint{0.667602in}{2.740405in}}%
\pgfpathlineto{\pgfqpoint{0.692511in}{2.715497in}}%
\pgfpathlineto{\pgfqpoint{0.717419in}{2.690589in}}%
\pgfpathlineto{\pgfqpoint{0.742327in}{2.665681in}}%
\pgfpathlineto{\pgfqpoint{0.767235in}{2.640772in}}%
\pgfpathlineto{\pgfqpoint{0.792143in}{2.615864in}}%
\pgfpathlineto{\pgfqpoint{0.817051in}{2.590956in}}%
\pgfpathlineto{\pgfqpoint{0.841960in}{2.566048in}}%
\pgfpathlineto{\pgfqpoint{0.866868in}{2.541140in}}%
\pgfpathlineto{\pgfqpoint{0.891776in}{2.516232in}}%
\pgfpathlineto{\pgfqpoint{0.916684in}{2.491323in}}%
\pgfpathlineto{\pgfqpoint{0.941592in}{2.466415in}}%
\pgfpathlineto{\pgfqpoint{0.966500in}{2.441507in}}%
\pgfpathlineto{\pgfqpoint{0.991409in}{2.416599in}}%
\pgfpathlineto{\pgfqpoint{1.016317in}{2.391691in}}%
\pgfpathlineto{\pgfqpoint{1.041225in}{2.366783in}}%
\pgfpathlineto{\pgfqpoint{1.066133in}{2.341874in}}%
\pgfpathlineto{\pgfqpoint{1.091041in}{2.316966in}}%
\pgfpathlineto{\pgfqpoint{1.115950in}{2.292058in}}%
\pgfpathlineto{\pgfqpoint{1.140858in}{2.267150in}}%
\pgfpathlineto{\pgfqpoint{1.165766in}{2.242242in}}%
\pgfpathlineto{\pgfqpoint{1.190674in}{2.217334in}}%
\pgfpathlineto{\pgfqpoint{1.215582in}{2.192425in}}%
\pgfpathlineto{\pgfqpoint{1.240490in}{2.167517in}}%
\pgfpathlineto{\pgfqpoint{1.265399in}{2.142609in}}%
\pgfpathlineto{\pgfqpoint{1.290307in}{2.117701in}}%
\pgfpathlineto{\pgfqpoint{1.315215in}{2.092793in}}%
\pgfpathlineto{\pgfqpoint{1.340123in}{2.067885in}}%
\pgfpathlineto{\pgfqpoint{1.365031in}{2.042976in}}%
\pgfpathlineto{\pgfqpoint{1.389939in}{2.018068in}}%
\pgfpathlineto{\pgfqpoint{1.414848in}{1.993160in}}%
\pgfpathlineto{\pgfqpoint{1.439756in}{1.968252in}}%
\pgfpathlineto{\pgfqpoint{1.464664in}{1.943344in}}%
\pgfpathlineto{\pgfqpoint{1.489572in}{1.918435in}}%
\pgfpathlineto{\pgfqpoint{1.514480in}{1.893527in}}%
\pgfpathlineto{\pgfqpoint{1.539388in}{1.868619in}}%
\pgfpathlineto{\pgfqpoint{1.564297in}{1.843711in}}%
\pgfpathlineto{\pgfqpoint{1.589205in}{1.818803in}}%
\pgfpathlineto{\pgfqpoint{1.614113in}{1.793895in}}%
\pgfpathlineto{\pgfqpoint{1.639021in}{1.768986in}}%
\pgfpathlineto{\pgfqpoint{1.663929in}{1.744078in}}%
\pgfpathlineto{\pgfqpoint{1.688837in}{1.719170in}}%
\pgfpathlineto{\pgfqpoint{1.713746in}{1.694262in}}%
\pgfpathlineto{\pgfqpoint{1.738654in}{1.669354in}}%
\pgfpathlineto{\pgfqpoint{1.763562in}{1.644446in}}%
\pgfpathlineto{\pgfqpoint{1.788470in}{1.619537in}}%
\pgfpathlineto{\pgfqpoint{1.813378in}{1.594629in}}%
\pgfpathlineto{\pgfqpoint{1.838287in}{1.569721in}}%
\pgfpathlineto{\pgfqpoint{1.863195in}{1.544813in}}%
\pgfpathlineto{\pgfqpoint{1.888103in}{1.519905in}}%
\pgfpathlineto{\pgfqpoint{1.913011in}{1.494997in}}%
\pgfpathlineto{\pgfqpoint{1.937919in}{1.470088in}}%
\pgfpathlineto{\pgfqpoint{1.962827in}{1.445180in}}%
\pgfpathlineto{\pgfqpoint{1.987736in}{1.420272in}}%
\pgfpathlineto{\pgfqpoint{2.012644in}{1.395364in}}%
\pgfpathlineto{\pgfqpoint{2.037552in}{1.370456in}}%
\pgfpathlineto{\pgfqpoint{2.062460in}{1.345547in}}%
\pgfpathlineto{\pgfqpoint{2.087368in}{1.320639in}}%
\pgfpathlineto{\pgfqpoint{2.112276in}{1.295731in}}%
\pgfpathlineto{\pgfqpoint{2.137185in}{1.270823in}}%
\pgfpathlineto{\pgfqpoint{2.162093in}{1.245915in}}%
\pgfpathlineto{\pgfqpoint{2.187001in}{1.221007in}}%
\pgfpathlineto{\pgfqpoint{2.211909in}{1.196098in}}%
\pgfpathlineto{\pgfqpoint{2.236817in}{1.171190in}}%
\pgfpathlineto{\pgfqpoint{2.261725in}{1.146282in}}%
\pgfpathlineto{\pgfqpoint{2.286634in}{1.121374in}}%
\pgfpathlineto{\pgfqpoint{2.311542in}{1.096466in}}%
\pgfpathlineto{\pgfqpoint{2.336450in}{1.071558in}}%
\pgfpathlineto{\pgfqpoint{2.361358in}{1.046649in}}%
\pgfpathlineto{\pgfqpoint{2.386266in}{1.021741in}}%
\pgfpathlineto{\pgfqpoint{2.411174in}{0.996833in}}%
\pgfpathlineto{\pgfqpoint{2.436083in}{0.971925in}}%
\pgfpathlineto{\pgfqpoint{2.460991in}{0.971925in}}%
\pgfpathlineto{\pgfqpoint{2.485899in}{0.996833in}}%
\pgfpathlineto{\pgfqpoint{2.510807in}{1.021741in}}%
\pgfpathlineto{\pgfqpoint{2.535715in}{1.046649in}}%
\pgfpathlineto{\pgfqpoint{2.560624in}{1.071558in}}%
\pgfpathlineto{\pgfqpoint{2.585532in}{1.096466in}}%
\pgfpathlineto{\pgfqpoint{2.610440in}{1.121374in}}%
\pgfpathlineto{\pgfqpoint{2.635348in}{1.146282in}}%
\pgfpathlineto{\pgfqpoint{2.660256in}{1.171190in}}%
\pgfpathlineto{\pgfqpoint{2.685164in}{1.196098in}}%
\pgfpathlineto{\pgfqpoint{2.710073in}{1.221007in}}%
\pgfpathlineto{\pgfqpoint{2.734981in}{1.245915in}}%
\pgfpathlineto{\pgfqpoint{2.759889in}{1.270823in}}%
\pgfpathlineto{\pgfqpoint{2.784797in}{1.295731in}}%
\pgfpathlineto{\pgfqpoint{2.809705in}{1.320639in}}%
\pgfpathlineto{\pgfqpoint{2.834613in}{1.345547in}}%
\pgfpathlineto{\pgfqpoint{2.859522in}{1.370456in}}%
\pgfusepath{stroke}%
\end{pgfscope}%
\begin{pgfscope}%
\pgfpathrectangle{\pgfqpoint{0.393613in}{0.548486in}}{\pgfqpoint{2.465909in}{2.465909in}}%
\pgfusepath{clip}%
\pgfsetbuttcap%
\pgfsetroundjoin%
\pgfsetlinewidth{1.505625pt}%
\definecolor{currentstroke}{rgb}{0.000000,0.000000,0.000000}%
\pgfsetstrokecolor{currentstroke}%
\pgfsetdash{{5.550000pt}{2.400000pt}}{0.000000pt}%
\pgfpathmoveto{\pgfqpoint{0.393613in}{2.603410in}}%
\pgfpathlineto{\pgfqpoint{0.418521in}{2.578502in}}%
\pgfpathlineto{\pgfqpoint{0.443429in}{2.553594in}}%
\pgfpathlineto{\pgfqpoint{0.468337in}{2.528686in}}%
\pgfpathlineto{\pgfqpoint{0.493245in}{2.503778in}}%
\pgfpathlineto{\pgfqpoint{0.518153in}{2.478869in}}%
\pgfpathlineto{\pgfqpoint{0.543062in}{2.453961in}}%
\pgfpathlineto{\pgfqpoint{0.567970in}{2.429053in}}%
\pgfpathlineto{\pgfqpoint{0.592878in}{2.404145in}}%
\pgfpathlineto{\pgfqpoint{0.617786in}{2.379237in}}%
\pgfpathlineto{\pgfqpoint{0.642694in}{2.354328in}}%
\pgfpathlineto{\pgfqpoint{0.667602in}{2.329420in}}%
\pgfpathlineto{\pgfqpoint{0.692511in}{2.304512in}}%
\pgfpathlineto{\pgfqpoint{0.717419in}{2.279604in}}%
\pgfpathlineto{\pgfqpoint{0.742327in}{2.254696in}}%
\pgfpathlineto{\pgfqpoint{0.767235in}{2.229788in}}%
\pgfpathlineto{\pgfqpoint{0.792143in}{2.204879in}}%
\pgfpathlineto{\pgfqpoint{0.817051in}{2.179971in}}%
\pgfpathlineto{\pgfqpoint{0.841960in}{2.155063in}}%
\pgfpathlineto{\pgfqpoint{0.866868in}{2.130155in}}%
\pgfpathlineto{\pgfqpoint{0.891776in}{2.105247in}}%
\pgfpathlineto{\pgfqpoint{0.916684in}{2.080339in}}%
\pgfpathlineto{\pgfqpoint{0.941592in}{2.055430in}}%
\pgfpathlineto{\pgfqpoint{0.966500in}{2.030522in}}%
\pgfpathlineto{\pgfqpoint{0.991409in}{2.005614in}}%
\pgfpathlineto{\pgfqpoint{1.016317in}{1.980706in}}%
\pgfpathlineto{\pgfqpoint{1.041225in}{1.955798in}}%
\pgfpathlineto{\pgfqpoint{1.066133in}{1.930890in}}%
\pgfpathlineto{\pgfqpoint{1.091041in}{1.905981in}}%
\pgfpathlineto{\pgfqpoint{1.115950in}{1.881073in}}%
\pgfpathlineto{\pgfqpoint{1.140858in}{1.856165in}}%
\pgfpathlineto{\pgfqpoint{1.165766in}{1.831257in}}%
\pgfpathlineto{\pgfqpoint{1.190674in}{1.806349in}}%
\pgfpathlineto{\pgfqpoint{1.215582in}{1.781441in}}%
\pgfpathlineto{\pgfqpoint{1.240490in}{1.756532in}}%
\pgfpathlineto{\pgfqpoint{1.265399in}{1.731624in}}%
\pgfpathlineto{\pgfqpoint{1.290307in}{1.706716in}}%
\pgfpathlineto{\pgfqpoint{1.315215in}{1.681808in}}%
\pgfpathlineto{\pgfqpoint{1.340123in}{1.656900in}}%
\pgfpathlineto{\pgfqpoint{1.365031in}{1.631991in}}%
\pgfpathlineto{\pgfqpoint{1.389939in}{1.607083in}}%
\pgfpathlineto{\pgfqpoint{1.414848in}{1.582175in}}%
\pgfpathlineto{\pgfqpoint{1.439756in}{1.557267in}}%
\pgfpathlineto{\pgfqpoint{1.464664in}{1.532359in}}%
\pgfpathlineto{\pgfqpoint{1.489572in}{1.507451in}}%
\pgfpathlineto{\pgfqpoint{1.514480in}{1.482542in}}%
\pgfpathlineto{\pgfqpoint{1.539388in}{1.457634in}}%
\pgfpathlineto{\pgfqpoint{1.564297in}{1.432726in}}%
\pgfpathlineto{\pgfqpoint{1.589205in}{1.407818in}}%
\pgfpathlineto{\pgfqpoint{1.614113in}{1.382910in}}%
\pgfpathlineto{\pgfqpoint{1.639021in}{1.358002in}}%
\pgfpathlineto{\pgfqpoint{1.663929in}{1.333093in}}%
\pgfpathlineto{\pgfqpoint{1.688837in}{1.308185in}}%
\pgfpathlineto{\pgfqpoint{1.713746in}{1.283277in}}%
\pgfpathlineto{\pgfqpoint{1.738654in}{1.258369in}}%
\pgfpathlineto{\pgfqpoint{1.763562in}{1.233461in}}%
\pgfpathlineto{\pgfqpoint{1.788470in}{1.208553in}}%
\pgfpathlineto{\pgfqpoint{1.813378in}{1.183644in}}%
\pgfpathlineto{\pgfqpoint{1.838287in}{1.158736in}}%
\pgfpathlineto{\pgfqpoint{1.863195in}{1.133828in}}%
\pgfpathlineto{\pgfqpoint{1.888103in}{1.108920in}}%
\pgfpathlineto{\pgfqpoint{1.913011in}{1.084012in}}%
\pgfpathlineto{\pgfqpoint{1.937919in}{1.059104in}}%
\pgfpathlineto{\pgfqpoint{1.962827in}{1.034195in}}%
\pgfpathlineto{\pgfqpoint{1.987736in}{1.009287in}}%
\pgfpathlineto{\pgfqpoint{2.012644in}{0.984379in}}%
\pgfpathlineto{\pgfqpoint{2.037552in}{0.959471in}}%
\pgfpathlineto{\pgfqpoint{2.062460in}{0.984379in}}%
\pgfpathlineto{\pgfqpoint{2.087368in}{1.009287in}}%
\pgfpathlineto{\pgfqpoint{2.112276in}{1.034195in}}%
\pgfpathlineto{\pgfqpoint{2.137185in}{1.059104in}}%
\pgfpathlineto{\pgfqpoint{2.162093in}{1.084012in}}%
\pgfpathlineto{\pgfqpoint{2.187001in}{1.108920in}}%
\pgfpathlineto{\pgfqpoint{2.211909in}{1.133828in}}%
\pgfpathlineto{\pgfqpoint{2.236817in}{1.158736in}}%
\pgfpathlineto{\pgfqpoint{2.261725in}{1.183644in}}%
\pgfpathlineto{\pgfqpoint{2.286634in}{1.208553in}}%
\pgfpathlineto{\pgfqpoint{2.311542in}{1.233461in}}%
\pgfpathlineto{\pgfqpoint{2.336450in}{1.258369in}}%
\pgfpathlineto{\pgfqpoint{2.361358in}{1.283277in}}%
\pgfpathlineto{\pgfqpoint{2.386266in}{1.308185in}}%
\pgfpathlineto{\pgfqpoint{2.411174in}{1.333093in}}%
\pgfpathlineto{\pgfqpoint{2.436083in}{1.358002in}}%
\pgfpathlineto{\pgfqpoint{2.460991in}{1.382910in}}%
\pgfpathlineto{\pgfqpoint{2.485899in}{1.407818in}}%
\pgfpathlineto{\pgfqpoint{2.510807in}{1.432726in}}%
\pgfpathlineto{\pgfqpoint{2.535715in}{1.457634in}}%
\pgfpathlineto{\pgfqpoint{2.560624in}{1.482542in}}%
\pgfpathlineto{\pgfqpoint{2.585532in}{1.507451in}}%
\pgfpathlineto{\pgfqpoint{2.610440in}{1.532359in}}%
\pgfpathlineto{\pgfqpoint{2.635348in}{1.557267in}}%
\pgfpathlineto{\pgfqpoint{2.660256in}{1.582175in}}%
\pgfpathlineto{\pgfqpoint{2.685164in}{1.607083in}}%
\pgfpathlineto{\pgfqpoint{2.710073in}{1.631991in}}%
\pgfpathlineto{\pgfqpoint{2.734981in}{1.656900in}}%
\pgfpathlineto{\pgfqpoint{2.759889in}{1.681808in}}%
\pgfpathlineto{\pgfqpoint{2.784797in}{1.706716in}}%
\pgfpathlineto{\pgfqpoint{2.809705in}{1.731624in}}%
\pgfpathlineto{\pgfqpoint{2.834613in}{1.756532in}}%
\pgfpathlineto{\pgfqpoint{2.859522in}{1.781441in}}%
\pgfusepath{stroke}%
\end{pgfscope}%
\begin{pgfscope}%
\pgfpathrectangle{\pgfqpoint{0.393613in}{0.548486in}}{\pgfqpoint{2.465909in}{2.465909in}}%
\pgfusepath{clip}%
\pgfsetrectcap%
\pgfsetroundjoin%
\pgfsetlinewidth{1.505625pt}%
\definecolor{currentstroke}{rgb}{1.000000,0.000000,0.000000}%
\pgfsetstrokecolor{currentstroke}%
\pgfsetdash{}{0pt}%
\pgfpathmoveto{\pgfqpoint{0.393613in}{2.192425in}}%
\pgfpathlineto{\pgfqpoint{0.418521in}{2.167517in}}%
\pgfpathlineto{\pgfqpoint{0.443429in}{2.142609in}}%
\pgfpathlineto{\pgfqpoint{0.468337in}{2.117701in}}%
\pgfpathlineto{\pgfqpoint{0.493245in}{2.092793in}}%
\pgfpathlineto{\pgfqpoint{0.518153in}{2.067885in}}%
\pgfpathlineto{\pgfqpoint{0.543062in}{2.042976in}}%
\pgfpathlineto{\pgfqpoint{0.567970in}{2.018068in}}%
\pgfpathlineto{\pgfqpoint{0.592878in}{1.993160in}}%
\pgfpathlineto{\pgfqpoint{0.617786in}{1.968252in}}%
\pgfpathlineto{\pgfqpoint{0.642694in}{1.943344in}}%
\pgfpathlineto{\pgfqpoint{0.667602in}{1.918435in}}%
\pgfpathlineto{\pgfqpoint{0.692511in}{1.893527in}}%
\pgfpathlineto{\pgfqpoint{0.717419in}{1.868619in}}%
\pgfpathlineto{\pgfqpoint{0.742327in}{1.843711in}}%
\pgfpathlineto{\pgfqpoint{0.767235in}{1.818803in}}%
\pgfpathlineto{\pgfqpoint{0.792143in}{1.793895in}}%
\pgfpathlineto{\pgfqpoint{0.817051in}{1.768986in}}%
\pgfpathlineto{\pgfqpoint{0.841960in}{1.744078in}}%
\pgfpathlineto{\pgfqpoint{0.866868in}{1.719170in}}%
\pgfpathlineto{\pgfqpoint{0.891776in}{1.694262in}}%
\pgfpathlineto{\pgfqpoint{0.916684in}{1.669354in}}%
\pgfpathlineto{\pgfqpoint{0.941592in}{1.644446in}}%
\pgfpathlineto{\pgfqpoint{0.966500in}{1.619537in}}%
\pgfpathlineto{\pgfqpoint{0.991409in}{1.594629in}}%
\pgfpathlineto{\pgfqpoint{1.016317in}{1.569721in}}%
\pgfpathlineto{\pgfqpoint{1.041225in}{1.544813in}}%
\pgfpathlineto{\pgfqpoint{1.066133in}{1.519905in}}%
\pgfpathlineto{\pgfqpoint{1.091041in}{1.494997in}}%
\pgfpathlineto{\pgfqpoint{1.115950in}{1.470088in}}%
\pgfpathlineto{\pgfqpoint{1.140858in}{1.445180in}}%
\pgfpathlineto{\pgfqpoint{1.165766in}{1.420272in}}%
\pgfpathlineto{\pgfqpoint{1.190674in}{1.395364in}}%
\pgfpathlineto{\pgfqpoint{1.215582in}{1.370456in}}%
\pgfpathlineto{\pgfqpoint{1.240490in}{1.345547in}}%
\pgfpathlineto{\pgfqpoint{1.265399in}{1.320639in}}%
\pgfpathlineto{\pgfqpoint{1.290307in}{1.295731in}}%
\pgfpathlineto{\pgfqpoint{1.315215in}{1.270823in}}%
\pgfpathlineto{\pgfqpoint{1.340123in}{1.245915in}}%
\pgfpathlineto{\pgfqpoint{1.365031in}{1.221007in}}%
\pgfpathlineto{\pgfqpoint{1.389939in}{1.196098in}}%
\pgfpathlineto{\pgfqpoint{1.414848in}{1.171190in}}%
\pgfpathlineto{\pgfqpoint{1.439756in}{1.146282in}}%
\pgfpathlineto{\pgfqpoint{1.464664in}{1.121374in}}%
\pgfpathlineto{\pgfqpoint{1.489572in}{1.096466in}}%
\pgfpathlineto{\pgfqpoint{1.514480in}{1.071558in}}%
\pgfpathlineto{\pgfqpoint{1.539388in}{1.046649in}}%
\pgfpathlineto{\pgfqpoint{1.564297in}{1.021741in}}%
\pgfpathlineto{\pgfqpoint{1.589205in}{0.996833in}}%
\pgfpathlineto{\pgfqpoint{1.614113in}{0.971925in}}%
\pgfpathlineto{\pgfqpoint{1.639021in}{0.971925in}}%
\pgfpathlineto{\pgfqpoint{1.663929in}{0.996833in}}%
\pgfpathlineto{\pgfqpoint{1.688837in}{1.021741in}}%
\pgfpathlineto{\pgfqpoint{1.713746in}{1.046649in}}%
\pgfpathlineto{\pgfqpoint{1.738654in}{1.071558in}}%
\pgfpathlineto{\pgfqpoint{1.763562in}{1.096466in}}%
\pgfpathlineto{\pgfqpoint{1.788470in}{1.121374in}}%
\pgfpathlineto{\pgfqpoint{1.813378in}{1.146282in}}%
\pgfpathlineto{\pgfqpoint{1.838287in}{1.171190in}}%
\pgfpathlineto{\pgfqpoint{1.863195in}{1.196098in}}%
\pgfpathlineto{\pgfqpoint{1.888103in}{1.221007in}}%
\pgfpathlineto{\pgfqpoint{1.913011in}{1.245915in}}%
\pgfpathlineto{\pgfqpoint{1.937919in}{1.270823in}}%
\pgfpathlineto{\pgfqpoint{1.962827in}{1.295731in}}%
\pgfpathlineto{\pgfqpoint{1.987736in}{1.320639in}}%
\pgfpathlineto{\pgfqpoint{2.012644in}{1.345547in}}%
\pgfpathlineto{\pgfqpoint{2.037552in}{1.370456in}}%
\pgfpathlineto{\pgfqpoint{2.062460in}{1.395364in}}%
\pgfpathlineto{\pgfqpoint{2.087368in}{1.420272in}}%
\pgfpathlineto{\pgfqpoint{2.112276in}{1.445180in}}%
\pgfpathlineto{\pgfqpoint{2.137185in}{1.470088in}}%
\pgfpathlineto{\pgfqpoint{2.162093in}{1.494997in}}%
\pgfpathlineto{\pgfqpoint{2.187001in}{1.519905in}}%
\pgfpathlineto{\pgfqpoint{2.211909in}{1.544813in}}%
\pgfpathlineto{\pgfqpoint{2.236817in}{1.569721in}}%
\pgfpathlineto{\pgfqpoint{2.261725in}{1.594629in}}%
\pgfpathlineto{\pgfqpoint{2.286634in}{1.619537in}}%
\pgfpathlineto{\pgfqpoint{2.311542in}{1.644446in}}%
\pgfpathlineto{\pgfqpoint{2.336450in}{1.669354in}}%
\pgfpathlineto{\pgfqpoint{2.361358in}{1.694262in}}%
\pgfpathlineto{\pgfqpoint{2.386266in}{1.719170in}}%
\pgfpathlineto{\pgfqpoint{2.411174in}{1.744078in}}%
\pgfpathlineto{\pgfqpoint{2.436083in}{1.768986in}}%
\pgfpathlineto{\pgfqpoint{2.460991in}{1.793895in}}%
\pgfpathlineto{\pgfqpoint{2.485899in}{1.818803in}}%
\pgfpathlineto{\pgfqpoint{2.510807in}{1.843711in}}%
\pgfpathlineto{\pgfqpoint{2.535715in}{1.868619in}}%
\pgfpathlineto{\pgfqpoint{2.560624in}{1.893527in}}%
\pgfpathlineto{\pgfqpoint{2.585532in}{1.918435in}}%
\pgfpathlineto{\pgfqpoint{2.610440in}{1.943344in}}%
\pgfpathlineto{\pgfqpoint{2.635348in}{1.968252in}}%
\pgfpathlineto{\pgfqpoint{2.660256in}{1.993160in}}%
\pgfpathlineto{\pgfqpoint{2.685164in}{2.018068in}}%
\pgfpathlineto{\pgfqpoint{2.710073in}{2.042976in}}%
\pgfpathlineto{\pgfqpoint{2.734981in}{2.067885in}}%
\pgfpathlineto{\pgfqpoint{2.759889in}{2.092793in}}%
\pgfpathlineto{\pgfqpoint{2.784797in}{2.117701in}}%
\pgfpathlineto{\pgfqpoint{2.809705in}{2.142609in}}%
\pgfpathlineto{\pgfqpoint{2.834613in}{2.167517in}}%
\pgfpathlineto{\pgfqpoint{2.859522in}{2.192425in}}%
\pgfusepath{stroke}%
\end{pgfscope}%
\begin{pgfscope}%
\pgfpathrectangle{\pgfqpoint{0.393613in}{0.548486in}}{\pgfqpoint{2.465909in}{2.465909in}}%
\pgfusepath{clip}%
\pgfsetbuttcap%
\pgfsetroundjoin%
\pgfsetlinewidth{1.505625pt}%
\definecolor{currentstroke}{rgb}{1.000000,0.000000,0.000000}%
\pgfsetstrokecolor{currentstroke}%
\pgfsetdash{{5.550000pt}{2.400000pt}}{0.000000pt}%
\pgfpathmoveto{\pgfqpoint{0.393613in}{1.781441in}}%
\pgfpathlineto{\pgfqpoint{0.418521in}{1.756532in}}%
\pgfpathlineto{\pgfqpoint{0.443429in}{1.731624in}}%
\pgfpathlineto{\pgfqpoint{0.468337in}{1.706716in}}%
\pgfpathlineto{\pgfqpoint{0.493245in}{1.681808in}}%
\pgfpathlineto{\pgfqpoint{0.518153in}{1.656900in}}%
\pgfpathlineto{\pgfqpoint{0.543062in}{1.631991in}}%
\pgfpathlineto{\pgfqpoint{0.567970in}{1.607083in}}%
\pgfpathlineto{\pgfqpoint{0.592878in}{1.582175in}}%
\pgfpathlineto{\pgfqpoint{0.617786in}{1.557267in}}%
\pgfpathlineto{\pgfqpoint{0.642694in}{1.532359in}}%
\pgfpathlineto{\pgfqpoint{0.667602in}{1.507451in}}%
\pgfpathlineto{\pgfqpoint{0.692511in}{1.482542in}}%
\pgfpathlineto{\pgfqpoint{0.717419in}{1.457634in}}%
\pgfpathlineto{\pgfqpoint{0.742327in}{1.432726in}}%
\pgfpathlineto{\pgfqpoint{0.767235in}{1.407818in}}%
\pgfpathlineto{\pgfqpoint{0.792143in}{1.382910in}}%
\pgfpathlineto{\pgfqpoint{0.817051in}{1.358002in}}%
\pgfpathlineto{\pgfqpoint{0.841960in}{1.333093in}}%
\pgfpathlineto{\pgfqpoint{0.866868in}{1.308185in}}%
\pgfpathlineto{\pgfqpoint{0.891776in}{1.283277in}}%
\pgfpathlineto{\pgfqpoint{0.916684in}{1.258369in}}%
\pgfpathlineto{\pgfqpoint{0.941592in}{1.233461in}}%
\pgfpathlineto{\pgfqpoint{0.966500in}{1.208553in}}%
\pgfpathlineto{\pgfqpoint{0.991409in}{1.183644in}}%
\pgfpathlineto{\pgfqpoint{1.016317in}{1.158736in}}%
\pgfpathlineto{\pgfqpoint{1.041225in}{1.133828in}}%
\pgfpathlineto{\pgfqpoint{1.066133in}{1.108920in}}%
\pgfpathlineto{\pgfqpoint{1.091041in}{1.084012in}}%
\pgfpathlineto{\pgfqpoint{1.115950in}{1.059104in}}%
\pgfpathlineto{\pgfqpoint{1.140858in}{1.034195in}}%
\pgfpathlineto{\pgfqpoint{1.165766in}{1.009287in}}%
\pgfpathlineto{\pgfqpoint{1.190674in}{0.984379in}}%
\pgfpathlineto{\pgfqpoint{1.215582in}{0.959471in}}%
\pgfpathlineto{\pgfqpoint{1.240490in}{0.984379in}}%
\pgfpathlineto{\pgfqpoint{1.265399in}{1.009287in}}%
\pgfpathlineto{\pgfqpoint{1.290307in}{1.034195in}}%
\pgfpathlineto{\pgfqpoint{1.315215in}{1.059104in}}%
\pgfpathlineto{\pgfqpoint{1.340123in}{1.084012in}}%
\pgfpathlineto{\pgfqpoint{1.365031in}{1.108920in}}%
\pgfpathlineto{\pgfqpoint{1.389939in}{1.133828in}}%
\pgfpathlineto{\pgfqpoint{1.414848in}{1.158736in}}%
\pgfpathlineto{\pgfqpoint{1.439756in}{1.183644in}}%
\pgfpathlineto{\pgfqpoint{1.464664in}{1.208553in}}%
\pgfpathlineto{\pgfqpoint{1.489572in}{1.233461in}}%
\pgfpathlineto{\pgfqpoint{1.514480in}{1.258369in}}%
\pgfpathlineto{\pgfqpoint{1.539388in}{1.283277in}}%
\pgfpathlineto{\pgfqpoint{1.564297in}{1.308185in}}%
\pgfpathlineto{\pgfqpoint{1.589205in}{1.333093in}}%
\pgfpathlineto{\pgfqpoint{1.614113in}{1.358002in}}%
\pgfpathlineto{\pgfqpoint{1.639021in}{1.382910in}}%
\pgfpathlineto{\pgfqpoint{1.663929in}{1.407818in}}%
\pgfpathlineto{\pgfqpoint{1.688837in}{1.432726in}}%
\pgfpathlineto{\pgfqpoint{1.713746in}{1.457634in}}%
\pgfpathlineto{\pgfqpoint{1.738654in}{1.482542in}}%
\pgfpathlineto{\pgfqpoint{1.763562in}{1.507451in}}%
\pgfpathlineto{\pgfqpoint{1.788470in}{1.532359in}}%
\pgfpathlineto{\pgfqpoint{1.813378in}{1.557267in}}%
\pgfpathlineto{\pgfqpoint{1.838287in}{1.582175in}}%
\pgfpathlineto{\pgfqpoint{1.863195in}{1.607083in}}%
\pgfpathlineto{\pgfqpoint{1.888103in}{1.631991in}}%
\pgfpathlineto{\pgfqpoint{1.913011in}{1.656900in}}%
\pgfpathlineto{\pgfqpoint{1.937919in}{1.681808in}}%
\pgfpathlineto{\pgfqpoint{1.962827in}{1.706716in}}%
\pgfpathlineto{\pgfqpoint{1.987736in}{1.731624in}}%
\pgfpathlineto{\pgfqpoint{2.012644in}{1.756532in}}%
\pgfpathlineto{\pgfqpoint{2.037552in}{1.781441in}}%
\pgfpathlineto{\pgfqpoint{2.062460in}{1.806349in}}%
\pgfpathlineto{\pgfqpoint{2.087368in}{1.831257in}}%
\pgfpathlineto{\pgfqpoint{2.112276in}{1.856165in}}%
\pgfpathlineto{\pgfqpoint{2.137185in}{1.881073in}}%
\pgfpathlineto{\pgfqpoint{2.162093in}{1.905981in}}%
\pgfpathlineto{\pgfqpoint{2.187001in}{1.930890in}}%
\pgfpathlineto{\pgfqpoint{2.211909in}{1.955798in}}%
\pgfpathlineto{\pgfqpoint{2.236817in}{1.980706in}}%
\pgfpathlineto{\pgfqpoint{2.261725in}{2.005614in}}%
\pgfpathlineto{\pgfqpoint{2.286634in}{2.030522in}}%
\pgfpathlineto{\pgfqpoint{2.311542in}{2.055430in}}%
\pgfpathlineto{\pgfqpoint{2.336450in}{2.080339in}}%
\pgfpathlineto{\pgfqpoint{2.361358in}{2.105247in}}%
\pgfpathlineto{\pgfqpoint{2.386266in}{2.130155in}}%
\pgfpathlineto{\pgfqpoint{2.411174in}{2.155063in}}%
\pgfpathlineto{\pgfqpoint{2.436083in}{2.179971in}}%
\pgfpathlineto{\pgfqpoint{2.460991in}{2.204879in}}%
\pgfpathlineto{\pgfqpoint{2.485899in}{2.229788in}}%
\pgfpathlineto{\pgfqpoint{2.510807in}{2.254696in}}%
\pgfpathlineto{\pgfqpoint{2.535715in}{2.279604in}}%
\pgfpathlineto{\pgfqpoint{2.560624in}{2.304512in}}%
\pgfpathlineto{\pgfqpoint{2.585532in}{2.329420in}}%
\pgfpathlineto{\pgfqpoint{2.610440in}{2.354328in}}%
\pgfpathlineto{\pgfqpoint{2.635348in}{2.379237in}}%
\pgfpathlineto{\pgfqpoint{2.660256in}{2.404145in}}%
\pgfpathlineto{\pgfqpoint{2.685164in}{2.429053in}}%
\pgfpathlineto{\pgfqpoint{2.710073in}{2.453961in}}%
\pgfpathlineto{\pgfqpoint{2.734981in}{2.478869in}}%
\pgfpathlineto{\pgfqpoint{2.759889in}{2.503778in}}%
\pgfpathlineto{\pgfqpoint{2.784797in}{2.528686in}}%
\pgfpathlineto{\pgfqpoint{2.809705in}{2.553594in}}%
\pgfpathlineto{\pgfqpoint{2.834613in}{2.578502in}}%
\pgfpathlineto{\pgfqpoint{2.859522in}{2.603410in}}%
\pgfusepath{stroke}%
\end{pgfscope}%
\begin{pgfscope}%
\pgfpathrectangle{\pgfqpoint{0.393613in}{0.548486in}}{\pgfqpoint{2.465909in}{2.465909in}}%
\pgfusepath{clip}%
\pgfsetrectcap%
\pgfsetroundjoin%
\pgfsetlinewidth{1.505625pt}%
\definecolor{currentstroke}{rgb}{0.000000,0.000000,1.000000}%
\pgfsetstrokecolor{currentstroke}%
\pgfsetdash{}{0pt}%
\pgfpathmoveto{\pgfqpoint{0.393613in}{1.370456in}}%
\pgfpathlineto{\pgfqpoint{0.418521in}{1.345547in}}%
\pgfpathlineto{\pgfqpoint{0.443429in}{1.320639in}}%
\pgfpathlineto{\pgfqpoint{0.468337in}{1.295731in}}%
\pgfpathlineto{\pgfqpoint{0.493245in}{1.270823in}}%
\pgfpathlineto{\pgfqpoint{0.518153in}{1.245915in}}%
\pgfpathlineto{\pgfqpoint{0.543062in}{1.221007in}}%
\pgfpathlineto{\pgfqpoint{0.567970in}{1.196098in}}%
\pgfpathlineto{\pgfqpoint{0.592878in}{1.171190in}}%
\pgfpathlineto{\pgfqpoint{0.617786in}{1.146282in}}%
\pgfpathlineto{\pgfqpoint{0.642694in}{1.121374in}}%
\pgfpathlineto{\pgfqpoint{0.667602in}{1.096466in}}%
\pgfpathlineto{\pgfqpoint{0.692511in}{1.071558in}}%
\pgfpathlineto{\pgfqpoint{0.717419in}{1.046649in}}%
\pgfpathlineto{\pgfqpoint{0.742327in}{1.021741in}}%
\pgfpathlineto{\pgfqpoint{0.767235in}{0.996833in}}%
\pgfpathlineto{\pgfqpoint{0.792143in}{0.971925in}}%
\pgfpathlineto{\pgfqpoint{0.817051in}{0.971925in}}%
\pgfpathlineto{\pgfqpoint{0.841960in}{0.996833in}}%
\pgfpathlineto{\pgfqpoint{0.866868in}{1.021741in}}%
\pgfpathlineto{\pgfqpoint{0.891776in}{1.046649in}}%
\pgfpathlineto{\pgfqpoint{0.916684in}{1.071558in}}%
\pgfpathlineto{\pgfqpoint{0.941592in}{1.096466in}}%
\pgfpathlineto{\pgfqpoint{0.966500in}{1.121374in}}%
\pgfpathlineto{\pgfqpoint{0.991409in}{1.146282in}}%
\pgfpathlineto{\pgfqpoint{1.016317in}{1.171190in}}%
\pgfpathlineto{\pgfqpoint{1.041225in}{1.196098in}}%
\pgfpathlineto{\pgfqpoint{1.066133in}{1.221007in}}%
\pgfpathlineto{\pgfqpoint{1.091041in}{1.245915in}}%
\pgfpathlineto{\pgfqpoint{1.115950in}{1.270823in}}%
\pgfpathlineto{\pgfqpoint{1.140858in}{1.295731in}}%
\pgfpathlineto{\pgfqpoint{1.165766in}{1.320639in}}%
\pgfpathlineto{\pgfqpoint{1.190674in}{1.345547in}}%
\pgfpathlineto{\pgfqpoint{1.215582in}{1.370456in}}%
\pgfpathlineto{\pgfqpoint{1.240490in}{1.395364in}}%
\pgfpathlineto{\pgfqpoint{1.265399in}{1.420272in}}%
\pgfpathlineto{\pgfqpoint{1.290307in}{1.445180in}}%
\pgfpathlineto{\pgfqpoint{1.315215in}{1.470088in}}%
\pgfpathlineto{\pgfqpoint{1.340123in}{1.494997in}}%
\pgfpathlineto{\pgfqpoint{1.365031in}{1.519905in}}%
\pgfpathlineto{\pgfqpoint{1.389939in}{1.544813in}}%
\pgfpathlineto{\pgfqpoint{1.414848in}{1.569721in}}%
\pgfpathlineto{\pgfqpoint{1.439756in}{1.594629in}}%
\pgfpathlineto{\pgfqpoint{1.464664in}{1.619537in}}%
\pgfpathlineto{\pgfqpoint{1.489572in}{1.644446in}}%
\pgfpathlineto{\pgfqpoint{1.514480in}{1.669354in}}%
\pgfpathlineto{\pgfqpoint{1.539388in}{1.694262in}}%
\pgfpathlineto{\pgfqpoint{1.564297in}{1.719170in}}%
\pgfpathlineto{\pgfqpoint{1.589205in}{1.744078in}}%
\pgfpathlineto{\pgfqpoint{1.614113in}{1.768986in}}%
\pgfpathlineto{\pgfqpoint{1.639021in}{1.793895in}}%
\pgfpathlineto{\pgfqpoint{1.663929in}{1.818803in}}%
\pgfpathlineto{\pgfqpoint{1.688837in}{1.843711in}}%
\pgfpathlineto{\pgfqpoint{1.713746in}{1.868619in}}%
\pgfpathlineto{\pgfqpoint{1.738654in}{1.893527in}}%
\pgfpathlineto{\pgfqpoint{1.763562in}{1.918435in}}%
\pgfpathlineto{\pgfqpoint{1.788470in}{1.943344in}}%
\pgfpathlineto{\pgfqpoint{1.813378in}{1.968252in}}%
\pgfpathlineto{\pgfqpoint{1.838287in}{1.993160in}}%
\pgfpathlineto{\pgfqpoint{1.863195in}{2.018068in}}%
\pgfpathlineto{\pgfqpoint{1.888103in}{2.042976in}}%
\pgfpathlineto{\pgfqpoint{1.913011in}{2.067885in}}%
\pgfpathlineto{\pgfqpoint{1.937919in}{2.092793in}}%
\pgfpathlineto{\pgfqpoint{1.962827in}{2.117701in}}%
\pgfpathlineto{\pgfqpoint{1.987736in}{2.142609in}}%
\pgfpathlineto{\pgfqpoint{2.012644in}{2.167517in}}%
\pgfpathlineto{\pgfqpoint{2.037552in}{2.192425in}}%
\pgfpathlineto{\pgfqpoint{2.062460in}{2.217334in}}%
\pgfpathlineto{\pgfqpoint{2.087368in}{2.242242in}}%
\pgfpathlineto{\pgfqpoint{2.112276in}{2.267150in}}%
\pgfpathlineto{\pgfqpoint{2.137185in}{2.292058in}}%
\pgfpathlineto{\pgfqpoint{2.162093in}{2.316966in}}%
\pgfpathlineto{\pgfqpoint{2.187001in}{2.341874in}}%
\pgfpathlineto{\pgfqpoint{2.211909in}{2.366783in}}%
\pgfpathlineto{\pgfqpoint{2.236817in}{2.391691in}}%
\pgfpathlineto{\pgfqpoint{2.261725in}{2.416599in}}%
\pgfpathlineto{\pgfqpoint{2.286634in}{2.441507in}}%
\pgfpathlineto{\pgfqpoint{2.311542in}{2.466415in}}%
\pgfpathlineto{\pgfqpoint{2.336450in}{2.491323in}}%
\pgfpathlineto{\pgfqpoint{2.361358in}{2.516232in}}%
\pgfpathlineto{\pgfqpoint{2.386266in}{2.541140in}}%
\pgfpathlineto{\pgfqpoint{2.411174in}{2.566048in}}%
\pgfpathlineto{\pgfqpoint{2.436083in}{2.590956in}}%
\pgfpathlineto{\pgfqpoint{2.460991in}{2.615864in}}%
\pgfpathlineto{\pgfqpoint{2.485899in}{2.640772in}}%
\pgfpathlineto{\pgfqpoint{2.510807in}{2.665681in}}%
\pgfpathlineto{\pgfqpoint{2.535715in}{2.690589in}}%
\pgfpathlineto{\pgfqpoint{2.560624in}{2.715497in}}%
\pgfpathlineto{\pgfqpoint{2.585532in}{2.740405in}}%
\pgfpathlineto{\pgfqpoint{2.610440in}{2.765313in}}%
\pgfpathlineto{\pgfqpoint{2.635348in}{2.790222in}}%
\pgfpathlineto{\pgfqpoint{2.660256in}{2.815130in}}%
\pgfpathlineto{\pgfqpoint{2.685164in}{2.840038in}}%
\pgfpathlineto{\pgfqpoint{2.710073in}{2.864946in}}%
\pgfpathlineto{\pgfqpoint{2.734981in}{2.889854in}}%
\pgfpathlineto{\pgfqpoint{2.759889in}{2.914762in}}%
\pgfpathlineto{\pgfqpoint{2.784797in}{2.939671in}}%
\pgfpathlineto{\pgfqpoint{2.809705in}{2.964579in}}%
\pgfpathlineto{\pgfqpoint{2.834613in}{2.989487in}}%
\pgfpathlineto{\pgfqpoint{2.859522in}{3.014395in}}%
\pgfusepath{stroke}%
\end{pgfscope}%
\begin{pgfscope}%
\pgfpathrectangle{\pgfqpoint{0.393613in}{0.548486in}}{\pgfqpoint{2.465909in}{2.465909in}}%
\pgfusepath{clip}%
\pgfsetrectcap%
\pgfsetroundjoin%
\pgfsetlinewidth{0.501875pt}%
\definecolor{currentstroke}{rgb}{0.000000,0.000000,0.000000}%
\pgfsetstrokecolor{currentstroke}%
\pgfsetdash{}{0pt}%
\pgfpathmoveto{\pgfqpoint{0.393613in}{1.370456in}}%
\pgfpathlineto{\pgfqpoint{2.859522in}{1.370456in}}%
\pgfusepath{stroke}%
\end{pgfscope}%
\begin{pgfscope}%
\pgfpathrectangle{\pgfqpoint{0.393613in}{0.548486in}}{\pgfqpoint{2.465909in}{2.465909in}}%
\pgfusepath{clip}%
\pgfsetrectcap%
\pgfsetroundjoin%
\pgfsetlinewidth{0.501875pt}%
\definecolor{currentstroke}{rgb}{0.000000,0.000000,0.000000}%
\pgfsetstrokecolor{currentstroke}%
\pgfsetdash{}{0pt}%
\pgfpathmoveto{\pgfqpoint{1.626567in}{0.548486in}}%
\pgfpathlineto{\pgfqpoint{1.626567in}{3.014395in}}%
\pgfusepath{stroke}%
\end{pgfscope}%
\begin{pgfscope}%
\pgfsetbuttcap%
\pgfsetmiterjoin%
\definecolor{currentfill}{rgb}{1.000000,1.000000,1.000000}%
\pgfsetfillcolor{currentfill}%
\pgfsetfillopacity{0.800000}%
\pgfsetlinewidth{1.003750pt}%
\definecolor{currentstroke}{rgb}{0.800000,0.800000,0.800000}%
\pgfsetstrokecolor{currentstroke}%
\pgfsetstrokeopacity{0.800000}%
\pgfsetdash{}{0pt}%
\pgfpathmoveto{\pgfqpoint{0.805098in}{1.854835in}}%
\pgfpathlineto{\pgfqpoint{2.448036in}{1.854835in}}%
\pgfpathquadraticcurveto{\pgfqpoint{2.475814in}{1.854835in}}{\pgfqpoint{2.475814in}{1.882613in}}%
\pgfpathlineto{\pgfqpoint{2.475814in}{2.917173in}}%
\pgfpathquadraticcurveto{\pgfqpoint{2.475814in}{2.944951in}}{\pgfqpoint{2.448036in}{2.944951in}}%
\pgfpathlineto{\pgfqpoint{0.805098in}{2.944951in}}%
\pgfpathquadraticcurveto{\pgfqpoint{0.777321in}{2.944951in}}{\pgfqpoint{0.777321in}{2.917173in}}%
\pgfpathlineto{\pgfqpoint{0.777321in}{1.882613in}}%
\pgfpathquadraticcurveto{\pgfqpoint{0.777321in}{1.854835in}}{\pgfqpoint{0.805098in}{1.854835in}}%
\pgfpathlineto{\pgfqpoint{0.805098in}{1.854835in}}%
\pgfpathclose%
\pgfusepath{stroke,fill}%
\end{pgfscope}%
\begin{pgfscope}%
\pgfsetrectcap%
\pgfsetroundjoin%
\pgfsetlinewidth{1.505625pt}%
\definecolor{currentstroke}{rgb}{0.000000,0.000000,0.000000}%
\pgfsetstrokecolor{currentstroke}%
\pgfsetdash{}{0pt}%
\pgfpathmoveto{\pgfqpoint{0.832876in}{2.832483in}}%
\pgfpathlineto{\pgfqpoint{0.971765in}{2.832483in}}%
\pgfpathlineto{\pgfqpoint{1.110654in}{2.832483in}}%
\pgfusepath{stroke}%
\end{pgfscope}%
\begin{pgfscope}%
\definecolor{textcolor}{rgb}{0.000000,0.000000,0.000000}%
\pgfsetstrokecolor{textcolor}%
\pgfsetfillcolor{textcolor}%
\pgftext[x=1.221765in,y=2.783872in,left,base]{\color{textcolor}{\rmfamily\fontsize{10.000000}{12.000000}\selectfont\catcode`\^=\active\def^{\ifmmode\sp\else\^{}\fi}\catcode`\%=\active\def%{\%}$f(x) = |x + -2| - 1$}}%
\end{pgfscope}%
\begin{pgfscope}%
\pgfsetbuttcap%
\pgfsetroundjoin%
\pgfsetlinewidth{1.505625pt}%
\definecolor{currentstroke}{rgb}{0.000000,0.000000,0.000000}%
\pgfsetstrokecolor{currentstroke}%
\pgfsetdash{{5.550000pt}{2.400000pt}}{0.000000pt}%
\pgfpathmoveto{\pgfqpoint{0.832876in}{2.622793in}}%
\pgfpathlineto{\pgfqpoint{0.971765in}{2.622793in}}%
\pgfpathlineto{\pgfqpoint{1.110654in}{2.622793in}}%
\pgfusepath{stroke}%
\end{pgfscope}%
\begin{pgfscope}%
\definecolor{textcolor}{rgb}{0.000000,0.000000,0.000000}%
\pgfsetstrokecolor{textcolor}%
\pgfsetfillcolor{textcolor}%
\pgftext[x=1.221765in,y=2.574182in,left,base]{\color{textcolor}{\rmfamily\fontsize{10.000000}{12.000000}\selectfont\catcode`\^=\active\def^{\ifmmode\sp\else\^{}\fi}\catcode`\%=\active\def%{\%}$f(x) = |x + -1| - 1$}}%
\end{pgfscope}%
\begin{pgfscope}%
\pgfsetrectcap%
\pgfsetroundjoin%
\pgfsetlinewidth{1.505625pt}%
\definecolor{currentstroke}{rgb}{1.000000,0.000000,0.000000}%
\pgfsetstrokecolor{currentstroke}%
\pgfsetdash{}{0pt}%
\pgfpathmoveto{\pgfqpoint{0.832876in}{2.413104in}}%
\pgfpathlineto{\pgfqpoint{0.971765in}{2.413104in}}%
\pgfpathlineto{\pgfqpoint{1.110654in}{2.413104in}}%
\pgfusepath{stroke}%
\end{pgfscope}%
\begin{pgfscope}%
\definecolor{textcolor}{rgb}{0.000000,0.000000,0.000000}%
\pgfsetstrokecolor{textcolor}%
\pgfsetfillcolor{textcolor}%
\pgftext[x=1.221765in,y=2.364493in,left,base]{\color{textcolor}{\rmfamily\fontsize{10.000000}{12.000000}\selectfont\catcode`\^=\active\def^{\ifmmode\sp\else\^{}\fi}\catcode`\%=\active\def%{\%}$f(x) = |x + 0| - 1$}}%
\end{pgfscope}%
\begin{pgfscope}%
\pgfsetbuttcap%
\pgfsetroundjoin%
\pgfsetlinewidth{1.505625pt}%
\definecolor{currentstroke}{rgb}{1.000000,0.000000,0.000000}%
\pgfsetstrokecolor{currentstroke}%
\pgfsetdash{{5.550000pt}{2.400000pt}}{0.000000pt}%
\pgfpathmoveto{\pgfqpoint{0.832876in}{2.203414in}}%
\pgfpathlineto{\pgfqpoint{0.971765in}{2.203414in}}%
\pgfpathlineto{\pgfqpoint{1.110654in}{2.203414in}}%
\pgfusepath{stroke}%
\end{pgfscope}%
\begin{pgfscope}%
\definecolor{textcolor}{rgb}{0.000000,0.000000,0.000000}%
\pgfsetstrokecolor{textcolor}%
\pgfsetfillcolor{textcolor}%
\pgftext[x=1.221765in,y=2.154803in,left,base]{\color{textcolor}{\rmfamily\fontsize{10.000000}{12.000000}\selectfont\catcode`\^=\active\def^{\ifmmode\sp\else\^{}\fi}\catcode`\%=\active\def%{\%}$f(x) = |x + 1| - 1$}}%
\end{pgfscope}%
\begin{pgfscope}%
\pgfsetrectcap%
\pgfsetroundjoin%
\pgfsetlinewidth{1.505625pt}%
\definecolor{currentstroke}{rgb}{0.000000,0.000000,1.000000}%
\pgfsetstrokecolor{currentstroke}%
\pgfsetdash{}{0pt}%
\pgfpathmoveto{\pgfqpoint{0.832876in}{1.993724in}}%
\pgfpathlineto{\pgfqpoint{0.971765in}{1.993724in}}%
\pgfpathlineto{\pgfqpoint{1.110654in}{1.993724in}}%
\pgfusepath{stroke}%
\end{pgfscope}%
\begin{pgfscope}%
\definecolor{textcolor}{rgb}{0.000000,0.000000,0.000000}%
\pgfsetstrokecolor{textcolor}%
\pgfsetfillcolor{textcolor}%
\pgftext[x=1.221765in,y=1.945113in,left,base]{\color{textcolor}{\rmfamily\fontsize{10.000000}{12.000000}\selectfont\catcode`\^=\active\def^{\ifmmode\sp\else\^{}\fi}\catcode`\%=\active\def%{\%}$f(x) = |x + 2| - 1$}}%
\end{pgfscope}%
\begin{pgfscope}%
\pgfsetbuttcap%
\pgfsetmiterjoin%
\definecolor{currentfill}{rgb}{1.000000,1.000000,1.000000}%
\pgfsetfillcolor{currentfill}%
\pgfsetlinewidth{0.000000pt}%
\definecolor{currentstroke}{rgb}{0.000000,0.000000,0.000000}%
\pgfsetstrokecolor{currentstroke}%
\pgfsetstrokeopacity{0.000000}%
\pgfsetdash{}{0pt}%
\pgfpathmoveto{\pgfqpoint{3.352703in}{0.548486in}}%
\pgfpathlineto{\pgfqpoint{5.818612in}{0.548486in}}%
\pgfpathlineto{\pgfqpoint{5.818612in}{3.014395in}}%
\pgfpathlineto{\pgfqpoint{3.352703in}{3.014395in}}%
\pgfpathlineto{\pgfqpoint{3.352703in}{0.548486in}}%
\pgfpathclose%
\pgfusepath{fill}%
\end{pgfscope}%
\begin{pgfscope}%
\pgfsetbuttcap%
\pgfsetroundjoin%
\definecolor{currentfill}{rgb}{0.000000,0.000000,0.000000}%
\pgfsetfillcolor{currentfill}%
\pgfsetlinewidth{0.803000pt}%
\definecolor{currentstroke}{rgb}{0.000000,0.000000,0.000000}%
\pgfsetstrokecolor{currentstroke}%
\pgfsetdash{}{0pt}%
\pgfsys@defobject{currentmarker}{\pgfqpoint{0.000000in}{-0.048611in}}{\pgfqpoint{0.000000in}{0.000000in}}{%
\pgfpathmoveto{\pgfqpoint{0.000000in}{0.000000in}}%
\pgfpathlineto{\pgfqpoint{0.000000in}{-0.048611in}}%
\pgfusepath{stroke,fill}%
}%
\begin{pgfscope}%
\pgfsys@transformshift{3.763688in}{0.548486in}%
\pgfsys@useobject{currentmarker}{}%
\end{pgfscope}%
\end{pgfscope}%
\begin{pgfscope}%
\definecolor{textcolor}{rgb}{0.000000,0.000000,0.000000}%
\pgfsetstrokecolor{textcolor}%
\pgfsetfillcolor{textcolor}%
\pgftext[x=3.763688in,y=0.451264in,,top]{\color{textcolor}{\rmfamily\fontsize{10.000000}{12.000000}\selectfont\catcode`\^=\active\def^{\ifmmode\sp\else\^{}\fi}\catcode`\%=\active\def%{\%}\ensuremath{-}2}}%
\end{pgfscope}%
\begin{pgfscope}%
\pgfsetbuttcap%
\pgfsetroundjoin%
\definecolor{currentfill}{rgb}{0.000000,0.000000,0.000000}%
\pgfsetfillcolor{currentfill}%
\pgfsetlinewidth{0.803000pt}%
\definecolor{currentstroke}{rgb}{0.000000,0.000000,0.000000}%
\pgfsetstrokecolor{currentstroke}%
\pgfsetdash{}{0pt}%
\pgfsys@defobject{currentmarker}{\pgfqpoint{0.000000in}{-0.048611in}}{\pgfqpoint{0.000000in}{0.000000in}}{%
\pgfpathmoveto{\pgfqpoint{0.000000in}{0.000000in}}%
\pgfpathlineto{\pgfqpoint{0.000000in}{-0.048611in}}%
\pgfusepath{stroke,fill}%
}%
\begin{pgfscope}%
\pgfsys@transformshift{4.585658in}{0.548486in}%
\pgfsys@useobject{currentmarker}{}%
\end{pgfscope}%
\end{pgfscope}%
\begin{pgfscope}%
\definecolor{textcolor}{rgb}{0.000000,0.000000,0.000000}%
\pgfsetstrokecolor{textcolor}%
\pgfsetfillcolor{textcolor}%
\pgftext[x=4.585658in,y=0.451264in,,top]{\color{textcolor}{\rmfamily\fontsize{10.000000}{12.000000}\selectfont\catcode`\^=\active\def^{\ifmmode\sp\else\^{}\fi}\catcode`\%=\active\def%{\%}0}}%
\end{pgfscope}%
\begin{pgfscope}%
\pgfsetbuttcap%
\pgfsetroundjoin%
\definecolor{currentfill}{rgb}{0.000000,0.000000,0.000000}%
\pgfsetfillcolor{currentfill}%
\pgfsetlinewidth{0.803000pt}%
\definecolor{currentstroke}{rgb}{0.000000,0.000000,0.000000}%
\pgfsetstrokecolor{currentstroke}%
\pgfsetdash{}{0pt}%
\pgfsys@defobject{currentmarker}{\pgfqpoint{0.000000in}{-0.048611in}}{\pgfqpoint{0.000000in}{0.000000in}}{%
\pgfpathmoveto{\pgfqpoint{0.000000in}{0.000000in}}%
\pgfpathlineto{\pgfqpoint{0.000000in}{-0.048611in}}%
\pgfusepath{stroke,fill}%
}%
\begin{pgfscope}%
\pgfsys@transformshift{5.407628in}{0.548486in}%
\pgfsys@useobject{currentmarker}{}%
\end{pgfscope}%
\end{pgfscope}%
\begin{pgfscope}%
\definecolor{textcolor}{rgb}{0.000000,0.000000,0.000000}%
\pgfsetstrokecolor{textcolor}%
\pgfsetfillcolor{textcolor}%
\pgftext[x=5.407628in,y=0.451264in,,top]{\color{textcolor}{\rmfamily\fontsize{10.000000}{12.000000}\selectfont\catcode`\^=\active\def^{\ifmmode\sp\else\^{}\fi}\catcode`\%=\active\def%{\%}2}}%
\end{pgfscope}%
\begin{pgfscope}%
\definecolor{textcolor}{rgb}{0.000000,0.000000,0.000000}%
\pgfsetstrokecolor{textcolor}%
\pgfsetfillcolor{textcolor}%
\pgftext[x=4.585658in,y=0.261295in,,top]{\color{textcolor}{\rmfamily\fontsize{12.000000}{14.400000}\selectfont\catcode`\^=\active\def^{\ifmmode\sp\else\^{}\fi}\catcode`\%=\active\def%{\%}$x$}}%
\end{pgfscope}%
\begin{pgfscope}%
\pgfsetbuttcap%
\pgfsetroundjoin%
\definecolor{currentfill}{rgb}{0.000000,0.000000,0.000000}%
\pgfsetfillcolor{currentfill}%
\pgfsetlinewidth{0.803000pt}%
\definecolor{currentstroke}{rgb}{0.000000,0.000000,0.000000}%
\pgfsetstrokecolor{currentstroke}%
\pgfsetdash{}{0pt}%
\pgfsys@defobject{currentmarker}{\pgfqpoint{-0.048611in}{0.000000in}}{\pgfqpoint{-0.000000in}{0.000000in}}{%
\pgfpathmoveto{\pgfqpoint{-0.000000in}{0.000000in}}%
\pgfpathlineto{\pgfqpoint{-0.048611in}{0.000000in}}%
\pgfusepath{stroke,fill}%
}%
\begin{pgfscope}%
\pgfsys@transformshift{3.352703in}{0.548486in}%
\pgfsys@useobject{currentmarker}{}%
\end{pgfscope}%
\end{pgfscope}%
\begin{pgfscope}%
\definecolor{textcolor}{rgb}{0.000000,0.000000,0.000000}%
\pgfsetstrokecolor{textcolor}%
\pgfsetfillcolor{textcolor}%
\pgftext[x=3.059091in, y=0.495724in, left, base]{\color{textcolor}{\rmfamily\fontsize{10.000000}{12.000000}\selectfont\catcode`\^=\active\def^{\ifmmode\sp\else\^{}\fi}\catcode`\%=\active\def%{\%}\ensuremath{-}2}}%
\end{pgfscope}%
\begin{pgfscope}%
\pgfsetbuttcap%
\pgfsetroundjoin%
\definecolor{currentfill}{rgb}{0.000000,0.000000,0.000000}%
\pgfsetfillcolor{currentfill}%
\pgfsetlinewidth{0.803000pt}%
\definecolor{currentstroke}{rgb}{0.000000,0.000000,0.000000}%
\pgfsetstrokecolor{currentstroke}%
\pgfsetdash{}{0pt}%
\pgfsys@defobject{currentmarker}{\pgfqpoint{-0.048611in}{0.000000in}}{\pgfqpoint{-0.000000in}{0.000000in}}{%
\pgfpathmoveto{\pgfqpoint{-0.000000in}{0.000000in}}%
\pgfpathlineto{\pgfqpoint{-0.048611in}{0.000000in}}%
\pgfusepath{stroke,fill}%
}%
\begin{pgfscope}%
\pgfsys@transformshift{3.352703in}{0.959471in}%
\pgfsys@useobject{currentmarker}{}%
\end{pgfscope}%
\end{pgfscope}%
\begin{pgfscope}%
\definecolor{textcolor}{rgb}{0.000000,0.000000,0.000000}%
\pgfsetstrokecolor{textcolor}%
\pgfsetfillcolor{textcolor}%
\pgftext[x=3.059091in, y=0.906709in, left, base]{\color{textcolor}{\rmfamily\fontsize{10.000000}{12.000000}\selectfont\catcode`\^=\active\def^{\ifmmode\sp\else\^{}\fi}\catcode`\%=\active\def%{\%}\ensuremath{-}1}}%
\end{pgfscope}%
\begin{pgfscope}%
\pgfsetbuttcap%
\pgfsetroundjoin%
\definecolor{currentfill}{rgb}{0.000000,0.000000,0.000000}%
\pgfsetfillcolor{currentfill}%
\pgfsetlinewidth{0.803000pt}%
\definecolor{currentstroke}{rgb}{0.000000,0.000000,0.000000}%
\pgfsetstrokecolor{currentstroke}%
\pgfsetdash{}{0pt}%
\pgfsys@defobject{currentmarker}{\pgfqpoint{-0.048611in}{0.000000in}}{\pgfqpoint{-0.000000in}{0.000000in}}{%
\pgfpathmoveto{\pgfqpoint{-0.000000in}{0.000000in}}%
\pgfpathlineto{\pgfqpoint{-0.048611in}{0.000000in}}%
\pgfusepath{stroke,fill}%
}%
\begin{pgfscope}%
\pgfsys@transformshift{3.352703in}{1.370456in}%
\pgfsys@useobject{currentmarker}{}%
\end{pgfscope}%
\end{pgfscope}%
\begin{pgfscope}%
\definecolor{textcolor}{rgb}{0.000000,0.000000,0.000000}%
\pgfsetstrokecolor{textcolor}%
\pgfsetfillcolor{textcolor}%
\pgftext[x=3.167116in, y=1.317694in, left, base]{\color{textcolor}{\rmfamily\fontsize{10.000000}{12.000000}\selectfont\catcode`\^=\active\def^{\ifmmode\sp\else\^{}\fi}\catcode`\%=\active\def%{\%}0}}%
\end{pgfscope}%
\begin{pgfscope}%
\pgfsetbuttcap%
\pgfsetroundjoin%
\definecolor{currentfill}{rgb}{0.000000,0.000000,0.000000}%
\pgfsetfillcolor{currentfill}%
\pgfsetlinewidth{0.803000pt}%
\definecolor{currentstroke}{rgb}{0.000000,0.000000,0.000000}%
\pgfsetstrokecolor{currentstroke}%
\pgfsetdash{}{0pt}%
\pgfsys@defobject{currentmarker}{\pgfqpoint{-0.048611in}{0.000000in}}{\pgfqpoint{-0.000000in}{0.000000in}}{%
\pgfpathmoveto{\pgfqpoint{-0.000000in}{0.000000in}}%
\pgfpathlineto{\pgfqpoint{-0.048611in}{0.000000in}}%
\pgfusepath{stroke,fill}%
}%
\begin{pgfscope}%
\pgfsys@transformshift{3.352703in}{1.781441in}%
\pgfsys@useobject{currentmarker}{}%
\end{pgfscope}%
\end{pgfscope}%
\begin{pgfscope}%
\definecolor{textcolor}{rgb}{0.000000,0.000000,0.000000}%
\pgfsetstrokecolor{textcolor}%
\pgfsetfillcolor{textcolor}%
\pgftext[x=3.167116in, y=1.728679in, left, base]{\color{textcolor}{\rmfamily\fontsize{10.000000}{12.000000}\selectfont\catcode`\^=\active\def^{\ifmmode\sp\else\^{}\fi}\catcode`\%=\active\def%{\%}1}}%
\end{pgfscope}%
\begin{pgfscope}%
\pgfsetbuttcap%
\pgfsetroundjoin%
\definecolor{currentfill}{rgb}{0.000000,0.000000,0.000000}%
\pgfsetfillcolor{currentfill}%
\pgfsetlinewidth{0.803000pt}%
\definecolor{currentstroke}{rgb}{0.000000,0.000000,0.000000}%
\pgfsetstrokecolor{currentstroke}%
\pgfsetdash{}{0pt}%
\pgfsys@defobject{currentmarker}{\pgfqpoint{-0.048611in}{0.000000in}}{\pgfqpoint{-0.000000in}{0.000000in}}{%
\pgfpathmoveto{\pgfqpoint{-0.000000in}{0.000000in}}%
\pgfpathlineto{\pgfqpoint{-0.048611in}{0.000000in}}%
\pgfusepath{stroke,fill}%
}%
\begin{pgfscope}%
\pgfsys@transformshift{3.352703in}{2.192425in}%
\pgfsys@useobject{currentmarker}{}%
\end{pgfscope}%
\end{pgfscope}%
\begin{pgfscope}%
\definecolor{textcolor}{rgb}{0.000000,0.000000,0.000000}%
\pgfsetstrokecolor{textcolor}%
\pgfsetfillcolor{textcolor}%
\pgftext[x=3.167116in, y=2.139664in, left, base]{\color{textcolor}{\rmfamily\fontsize{10.000000}{12.000000}\selectfont\catcode`\^=\active\def^{\ifmmode\sp\else\^{}\fi}\catcode`\%=\active\def%{\%}2}}%
\end{pgfscope}%
\begin{pgfscope}%
\pgfsetbuttcap%
\pgfsetroundjoin%
\definecolor{currentfill}{rgb}{0.000000,0.000000,0.000000}%
\pgfsetfillcolor{currentfill}%
\pgfsetlinewidth{0.803000pt}%
\definecolor{currentstroke}{rgb}{0.000000,0.000000,0.000000}%
\pgfsetstrokecolor{currentstroke}%
\pgfsetdash{}{0pt}%
\pgfsys@defobject{currentmarker}{\pgfqpoint{-0.048611in}{0.000000in}}{\pgfqpoint{-0.000000in}{0.000000in}}{%
\pgfpathmoveto{\pgfqpoint{-0.000000in}{0.000000in}}%
\pgfpathlineto{\pgfqpoint{-0.048611in}{0.000000in}}%
\pgfusepath{stroke,fill}%
}%
\begin{pgfscope}%
\pgfsys@transformshift{3.352703in}{2.603410in}%
\pgfsys@useobject{currentmarker}{}%
\end{pgfscope}%
\end{pgfscope}%
\begin{pgfscope}%
\definecolor{textcolor}{rgb}{0.000000,0.000000,0.000000}%
\pgfsetstrokecolor{textcolor}%
\pgfsetfillcolor{textcolor}%
\pgftext[x=3.167116in, y=2.550649in, left, base]{\color{textcolor}{\rmfamily\fontsize{10.000000}{12.000000}\selectfont\catcode`\^=\active\def^{\ifmmode\sp\else\^{}\fi}\catcode`\%=\active\def%{\%}3}}%
\end{pgfscope}%
\begin{pgfscope}%
\pgfsetbuttcap%
\pgfsetroundjoin%
\definecolor{currentfill}{rgb}{0.000000,0.000000,0.000000}%
\pgfsetfillcolor{currentfill}%
\pgfsetlinewidth{0.803000pt}%
\definecolor{currentstroke}{rgb}{0.000000,0.000000,0.000000}%
\pgfsetstrokecolor{currentstroke}%
\pgfsetdash{}{0pt}%
\pgfsys@defobject{currentmarker}{\pgfqpoint{-0.048611in}{0.000000in}}{\pgfqpoint{-0.000000in}{0.000000in}}{%
\pgfpathmoveto{\pgfqpoint{-0.000000in}{0.000000in}}%
\pgfpathlineto{\pgfqpoint{-0.048611in}{0.000000in}}%
\pgfusepath{stroke,fill}%
}%
\begin{pgfscope}%
\pgfsys@transformshift{3.352703in}{3.014395in}%
\pgfsys@useobject{currentmarker}{}%
\end{pgfscope}%
\end{pgfscope}%
\begin{pgfscope}%
\definecolor{textcolor}{rgb}{0.000000,0.000000,0.000000}%
\pgfsetstrokecolor{textcolor}%
\pgfsetfillcolor{textcolor}%
\pgftext[x=3.167116in, y=2.961634in, left, base]{\color{textcolor}{\rmfamily\fontsize{10.000000}{12.000000}\selectfont\catcode`\^=\active\def^{\ifmmode\sp\else\^{}\fi}\catcode`\%=\active\def%{\%}4}}%
\end{pgfscope}%
\begin{pgfscope}%
\pgfpathrectangle{\pgfqpoint{3.352703in}{0.548486in}}{\pgfqpoint{2.465909in}{2.465909in}}%
\pgfusepath{clip}%
\pgfsetrectcap%
\pgfsetroundjoin%
\pgfsetlinewidth{1.505625pt}%
\definecolor{currentstroke}{rgb}{0.000000,0.000000,0.000000}%
\pgfsetstrokecolor{currentstroke}%
\pgfsetdash{}{0pt}%
\pgfpathmoveto{\pgfqpoint{3.352703in}{3.014395in}}%
\pgfpathlineto{\pgfqpoint{3.377612in}{2.989487in}}%
\pgfpathlineto{\pgfqpoint{3.402520in}{2.964579in}}%
\pgfpathlineto{\pgfqpoint{3.427428in}{2.939671in}}%
\pgfpathlineto{\pgfqpoint{3.452336in}{2.914762in}}%
\pgfpathlineto{\pgfqpoint{3.477244in}{2.889854in}}%
\pgfpathlineto{\pgfqpoint{3.502152in}{2.864946in}}%
\pgfpathlineto{\pgfqpoint{3.527061in}{2.840038in}}%
\pgfpathlineto{\pgfqpoint{3.551969in}{2.815130in}}%
\pgfpathlineto{\pgfqpoint{3.576877in}{2.790222in}}%
\pgfpathlineto{\pgfqpoint{3.601785in}{2.765313in}}%
\pgfpathlineto{\pgfqpoint{3.626693in}{2.740405in}}%
\pgfpathlineto{\pgfqpoint{3.651601in}{2.715497in}}%
\pgfpathlineto{\pgfqpoint{3.676510in}{2.690589in}}%
\pgfpathlineto{\pgfqpoint{3.701418in}{2.665681in}}%
\pgfpathlineto{\pgfqpoint{3.726326in}{2.640772in}}%
\pgfpathlineto{\pgfqpoint{3.751234in}{2.615864in}}%
\pgfpathlineto{\pgfqpoint{3.776142in}{2.590956in}}%
\pgfpathlineto{\pgfqpoint{3.801051in}{2.566048in}}%
\pgfpathlineto{\pgfqpoint{3.825959in}{2.541140in}}%
\pgfpathlineto{\pgfqpoint{3.850867in}{2.516232in}}%
\pgfpathlineto{\pgfqpoint{3.875775in}{2.491323in}}%
\pgfpathlineto{\pgfqpoint{3.900683in}{2.466415in}}%
\pgfpathlineto{\pgfqpoint{3.925591in}{2.441507in}}%
\pgfpathlineto{\pgfqpoint{3.950500in}{2.416599in}}%
\pgfpathlineto{\pgfqpoint{3.975408in}{2.391691in}}%
\pgfpathlineto{\pgfqpoint{4.000316in}{2.366783in}}%
\pgfpathlineto{\pgfqpoint{4.025224in}{2.341874in}}%
\pgfpathlineto{\pgfqpoint{4.050132in}{2.316966in}}%
\pgfpathlineto{\pgfqpoint{4.075040in}{2.292058in}}%
\pgfpathlineto{\pgfqpoint{4.099949in}{2.267150in}}%
\pgfpathlineto{\pgfqpoint{4.124857in}{2.242242in}}%
\pgfpathlineto{\pgfqpoint{4.149765in}{2.217334in}}%
\pgfpathlineto{\pgfqpoint{4.174673in}{2.192425in}}%
\pgfpathlineto{\pgfqpoint{4.199581in}{2.217334in}}%
\pgfpathlineto{\pgfqpoint{4.224489in}{2.242242in}}%
\pgfpathlineto{\pgfqpoint{4.249398in}{2.267150in}}%
\pgfpathlineto{\pgfqpoint{4.274306in}{2.292058in}}%
\pgfpathlineto{\pgfqpoint{4.299214in}{2.316966in}}%
\pgfpathlineto{\pgfqpoint{4.324122in}{2.341874in}}%
\pgfpathlineto{\pgfqpoint{4.349030in}{2.366783in}}%
\pgfpathlineto{\pgfqpoint{4.373938in}{2.391691in}}%
\pgfpathlineto{\pgfqpoint{4.398847in}{2.416599in}}%
\pgfpathlineto{\pgfqpoint{4.423755in}{2.441507in}}%
\pgfpathlineto{\pgfqpoint{4.448663in}{2.466415in}}%
\pgfpathlineto{\pgfqpoint{4.473571in}{2.491323in}}%
\pgfpathlineto{\pgfqpoint{4.498479in}{2.516232in}}%
\pgfpathlineto{\pgfqpoint{4.523388in}{2.541140in}}%
\pgfpathlineto{\pgfqpoint{4.548296in}{2.566048in}}%
\pgfpathlineto{\pgfqpoint{4.573204in}{2.590956in}}%
\pgfpathlineto{\pgfqpoint{4.598112in}{2.615864in}}%
\pgfpathlineto{\pgfqpoint{4.623020in}{2.640772in}}%
\pgfpathlineto{\pgfqpoint{4.647928in}{2.665681in}}%
\pgfpathlineto{\pgfqpoint{4.672837in}{2.690589in}}%
\pgfpathlineto{\pgfqpoint{4.697745in}{2.715497in}}%
\pgfpathlineto{\pgfqpoint{4.722653in}{2.740405in}}%
\pgfpathlineto{\pgfqpoint{4.747561in}{2.765313in}}%
\pgfpathlineto{\pgfqpoint{4.772469in}{2.790222in}}%
\pgfpathlineto{\pgfqpoint{4.797377in}{2.815130in}}%
\pgfpathlineto{\pgfqpoint{4.822286in}{2.840038in}}%
\pgfpathlineto{\pgfqpoint{4.847194in}{2.864946in}}%
\pgfpathlineto{\pgfqpoint{4.872102in}{2.889854in}}%
\pgfpathlineto{\pgfqpoint{4.897010in}{2.914762in}}%
\pgfpathlineto{\pgfqpoint{4.921918in}{2.939671in}}%
\pgfpathlineto{\pgfqpoint{4.946826in}{2.964579in}}%
\pgfpathlineto{\pgfqpoint{4.971735in}{2.989487in}}%
\pgfpathlineto{\pgfqpoint{4.996643in}{3.014395in}}%
\pgfpathlineto{\pgfqpoint{5.006643in}{3.024395in}}%
\pgfusepath{stroke}%
\end{pgfscope}%
\begin{pgfscope}%
\pgfpathrectangle{\pgfqpoint{3.352703in}{0.548486in}}{\pgfqpoint{2.465909in}{2.465909in}}%
\pgfusepath{clip}%
\pgfsetbuttcap%
\pgfsetroundjoin%
\pgfsetlinewidth{1.505625pt}%
\definecolor{currentstroke}{rgb}{0.000000,0.000000,0.000000}%
\pgfsetstrokecolor{currentstroke}%
\pgfsetdash{{5.550000pt}{2.400000pt}}{0.000000pt}%
\pgfpathmoveto{\pgfqpoint{3.352703in}{2.603410in}}%
\pgfpathlineto{\pgfqpoint{3.377612in}{2.578502in}}%
\pgfpathlineto{\pgfqpoint{3.402520in}{2.553594in}}%
\pgfpathlineto{\pgfqpoint{3.427428in}{2.528686in}}%
\pgfpathlineto{\pgfqpoint{3.452336in}{2.503778in}}%
\pgfpathlineto{\pgfqpoint{3.477244in}{2.478869in}}%
\pgfpathlineto{\pgfqpoint{3.502152in}{2.453961in}}%
\pgfpathlineto{\pgfqpoint{3.527061in}{2.429053in}}%
\pgfpathlineto{\pgfqpoint{3.551969in}{2.404145in}}%
\pgfpathlineto{\pgfqpoint{3.576877in}{2.379237in}}%
\pgfpathlineto{\pgfqpoint{3.601785in}{2.354328in}}%
\pgfpathlineto{\pgfqpoint{3.626693in}{2.329420in}}%
\pgfpathlineto{\pgfqpoint{3.651601in}{2.304512in}}%
\pgfpathlineto{\pgfqpoint{3.676510in}{2.279604in}}%
\pgfpathlineto{\pgfqpoint{3.701418in}{2.254696in}}%
\pgfpathlineto{\pgfqpoint{3.726326in}{2.229788in}}%
\pgfpathlineto{\pgfqpoint{3.751234in}{2.204879in}}%
\pgfpathlineto{\pgfqpoint{3.776142in}{2.179971in}}%
\pgfpathlineto{\pgfqpoint{3.801051in}{2.155063in}}%
\pgfpathlineto{\pgfqpoint{3.825959in}{2.130155in}}%
\pgfpathlineto{\pgfqpoint{3.850867in}{2.105247in}}%
\pgfpathlineto{\pgfqpoint{3.875775in}{2.080339in}}%
\pgfpathlineto{\pgfqpoint{3.900683in}{2.055430in}}%
\pgfpathlineto{\pgfqpoint{3.925591in}{2.030522in}}%
\pgfpathlineto{\pgfqpoint{3.950500in}{2.005614in}}%
\pgfpathlineto{\pgfqpoint{3.975408in}{1.980706in}}%
\pgfpathlineto{\pgfqpoint{4.000316in}{1.955798in}}%
\pgfpathlineto{\pgfqpoint{4.025224in}{1.930890in}}%
\pgfpathlineto{\pgfqpoint{4.050132in}{1.905981in}}%
\pgfpathlineto{\pgfqpoint{4.075040in}{1.881073in}}%
\pgfpathlineto{\pgfqpoint{4.099949in}{1.856165in}}%
\pgfpathlineto{\pgfqpoint{4.124857in}{1.831257in}}%
\pgfpathlineto{\pgfqpoint{4.149765in}{1.806349in}}%
\pgfpathlineto{\pgfqpoint{4.174673in}{1.781441in}}%
\pgfpathlineto{\pgfqpoint{4.199581in}{1.806349in}}%
\pgfpathlineto{\pgfqpoint{4.224489in}{1.831257in}}%
\pgfpathlineto{\pgfqpoint{4.249398in}{1.856165in}}%
\pgfpathlineto{\pgfqpoint{4.274306in}{1.881073in}}%
\pgfpathlineto{\pgfqpoint{4.299214in}{1.905981in}}%
\pgfpathlineto{\pgfqpoint{4.324122in}{1.930890in}}%
\pgfpathlineto{\pgfqpoint{4.349030in}{1.955798in}}%
\pgfpathlineto{\pgfqpoint{4.373938in}{1.980706in}}%
\pgfpathlineto{\pgfqpoint{4.398847in}{2.005614in}}%
\pgfpathlineto{\pgfqpoint{4.423755in}{2.030522in}}%
\pgfpathlineto{\pgfqpoint{4.448663in}{2.055430in}}%
\pgfpathlineto{\pgfqpoint{4.473571in}{2.080339in}}%
\pgfpathlineto{\pgfqpoint{4.498479in}{2.105247in}}%
\pgfpathlineto{\pgfqpoint{4.523388in}{2.130155in}}%
\pgfpathlineto{\pgfqpoint{4.548296in}{2.155063in}}%
\pgfpathlineto{\pgfqpoint{4.573204in}{2.179971in}}%
\pgfpathlineto{\pgfqpoint{4.598112in}{2.204879in}}%
\pgfpathlineto{\pgfqpoint{4.623020in}{2.229788in}}%
\pgfpathlineto{\pgfqpoint{4.647928in}{2.254696in}}%
\pgfpathlineto{\pgfqpoint{4.672837in}{2.279604in}}%
\pgfpathlineto{\pgfqpoint{4.697745in}{2.304512in}}%
\pgfpathlineto{\pgfqpoint{4.722653in}{2.329420in}}%
\pgfpathlineto{\pgfqpoint{4.747561in}{2.354328in}}%
\pgfpathlineto{\pgfqpoint{4.772469in}{2.379237in}}%
\pgfpathlineto{\pgfqpoint{4.797377in}{2.404145in}}%
\pgfpathlineto{\pgfqpoint{4.822286in}{2.429053in}}%
\pgfpathlineto{\pgfqpoint{4.847194in}{2.453961in}}%
\pgfpathlineto{\pgfqpoint{4.872102in}{2.478869in}}%
\pgfpathlineto{\pgfqpoint{4.897010in}{2.503778in}}%
\pgfpathlineto{\pgfqpoint{4.921918in}{2.528686in}}%
\pgfpathlineto{\pgfqpoint{4.946826in}{2.553594in}}%
\pgfpathlineto{\pgfqpoint{4.971735in}{2.578502in}}%
\pgfpathlineto{\pgfqpoint{4.996643in}{2.603410in}}%
\pgfpathlineto{\pgfqpoint{5.021551in}{2.628318in}}%
\pgfpathlineto{\pgfqpoint{5.046459in}{2.653227in}}%
\pgfpathlineto{\pgfqpoint{5.071367in}{2.678135in}}%
\pgfpathlineto{\pgfqpoint{5.096275in}{2.703043in}}%
\pgfpathlineto{\pgfqpoint{5.121184in}{2.727951in}}%
\pgfpathlineto{\pgfqpoint{5.146092in}{2.752859in}}%
\pgfpathlineto{\pgfqpoint{5.171000in}{2.777767in}}%
\pgfpathlineto{\pgfqpoint{5.195908in}{2.802676in}}%
\pgfpathlineto{\pgfqpoint{5.220816in}{2.827584in}}%
\pgfpathlineto{\pgfqpoint{5.245725in}{2.852492in}}%
\pgfpathlineto{\pgfqpoint{5.270633in}{2.877400in}}%
\pgfpathlineto{\pgfqpoint{5.295541in}{2.902308in}}%
\pgfpathlineto{\pgfqpoint{5.320449in}{2.927216in}}%
\pgfpathlineto{\pgfqpoint{5.345357in}{2.952125in}}%
\pgfpathlineto{\pgfqpoint{5.370265in}{2.977033in}}%
\pgfpathlineto{\pgfqpoint{5.395174in}{3.001941in}}%
\pgfpathlineto{\pgfqpoint{5.417628in}{3.024395in}}%
\pgfusepath{stroke}%
\end{pgfscope}%
\begin{pgfscope}%
\pgfpathrectangle{\pgfqpoint{3.352703in}{0.548486in}}{\pgfqpoint{2.465909in}{2.465909in}}%
\pgfusepath{clip}%
\pgfsetrectcap%
\pgfsetroundjoin%
\pgfsetlinewidth{1.505625pt}%
\definecolor{currentstroke}{rgb}{1.000000,0.000000,0.000000}%
\pgfsetstrokecolor{currentstroke}%
\pgfsetdash{}{0pt}%
\pgfpathmoveto{\pgfqpoint{3.352703in}{2.192425in}}%
\pgfpathlineto{\pgfqpoint{3.377612in}{2.167517in}}%
\pgfpathlineto{\pgfqpoint{3.402520in}{2.142609in}}%
\pgfpathlineto{\pgfqpoint{3.427428in}{2.117701in}}%
\pgfpathlineto{\pgfqpoint{3.452336in}{2.092793in}}%
\pgfpathlineto{\pgfqpoint{3.477244in}{2.067885in}}%
\pgfpathlineto{\pgfqpoint{3.502152in}{2.042976in}}%
\pgfpathlineto{\pgfqpoint{3.527061in}{2.018068in}}%
\pgfpathlineto{\pgfqpoint{3.551969in}{1.993160in}}%
\pgfpathlineto{\pgfqpoint{3.576877in}{1.968252in}}%
\pgfpathlineto{\pgfqpoint{3.601785in}{1.943344in}}%
\pgfpathlineto{\pgfqpoint{3.626693in}{1.918435in}}%
\pgfpathlineto{\pgfqpoint{3.651601in}{1.893527in}}%
\pgfpathlineto{\pgfqpoint{3.676510in}{1.868619in}}%
\pgfpathlineto{\pgfqpoint{3.701418in}{1.843711in}}%
\pgfpathlineto{\pgfqpoint{3.726326in}{1.818803in}}%
\pgfpathlineto{\pgfqpoint{3.751234in}{1.793895in}}%
\pgfpathlineto{\pgfqpoint{3.776142in}{1.768986in}}%
\pgfpathlineto{\pgfqpoint{3.801051in}{1.744078in}}%
\pgfpathlineto{\pgfqpoint{3.825959in}{1.719170in}}%
\pgfpathlineto{\pgfqpoint{3.850867in}{1.694262in}}%
\pgfpathlineto{\pgfqpoint{3.875775in}{1.669354in}}%
\pgfpathlineto{\pgfqpoint{3.900683in}{1.644446in}}%
\pgfpathlineto{\pgfqpoint{3.925591in}{1.619537in}}%
\pgfpathlineto{\pgfqpoint{3.950500in}{1.594629in}}%
\pgfpathlineto{\pgfqpoint{3.975408in}{1.569721in}}%
\pgfpathlineto{\pgfqpoint{4.000316in}{1.544813in}}%
\pgfpathlineto{\pgfqpoint{4.025224in}{1.519905in}}%
\pgfpathlineto{\pgfqpoint{4.050132in}{1.494997in}}%
\pgfpathlineto{\pgfqpoint{4.075040in}{1.470088in}}%
\pgfpathlineto{\pgfqpoint{4.099949in}{1.445180in}}%
\pgfpathlineto{\pgfqpoint{4.124857in}{1.420272in}}%
\pgfpathlineto{\pgfqpoint{4.149765in}{1.395364in}}%
\pgfpathlineto{\pgfqpoint{4.174673in}{1.370456in}}%
\pgfpathlineto{\pgfqpoint{4.199581in}{1.395364in}}%
\pgfpathlineto{\pgfqpoint{4.224489in}{1.420272in}}%
\pgfpathlineto{\pgfqpoint{4.249398in}{1.445180in}}%
\pgfpathlineto{\pgfqpoint{4.274306in}{1.470088in}}%
\pgfpathlineto{\pgfqpoint{4.299214in}{1.494997in}}%
\pgfpathlineto{\pgfqpoint{4.324122in}{1.519905in}}%
\pgfpathlineto{\pgfqpoint{4.349030in}{1.544813in}}%
\pgfpathlineto{\pgfqpoint{4.373938in}{1.569721in}}%
\pgfpathlineto{\pgfqpoint{4.398847in}{1.594629in}}%
\pgfpathlineto{\pgfqpoint{4.423755in}{1.619537in}}%
\pgfpathlineto{\pgfqpoint{4.448663in}{1.644446in}}%
\pgfpathlineto{\pgfqpoint{4.473571in}{1.669354in}}%
\pgfpathlineto{\pgfqpoint{4.498479in}{1.694262in}}%
\pgfpathlineto{\pgfqpoint{4.523388in}{1.719170in}}%
\pgfpathlineto{\pgfqpoint{4.548296in}{1.744078in}}%
\pgfpathlineto{\pgfqpoint{4.573204in}{1.768986in}}%
\pgfpathlineto{\pgfqpoint{4.598112in}{1.793895in}}%
\pgfpathlineto{\pgfqpoint{4.623020in}{1.818803in}}%
\pgfpathlineto{\pgfqpoint{4.647928in}{1.843711in}}%
\pgfpathlineto{\pgfqpoint{4.672837in}{1.868619in}}%
\pgfpathlineto{\pgfqpoint{4.697745in}{1.893527in}}%
\pgfpathlineto{\pgfqpoint{4.722653in}{1.918435in}}%
\pgfpathlineto{\pgfqpoint{4.747561in}{1.943344in}}%
\pgfpathlineto{\pgfqpoint{4.772469in}{1.968252in}}%
\pgfpathlineto{\pgfqpoint{4.797377in}{1.993160in}}%
\pgfpathlineto{\pgfqpoint{4.822286in}{2.018068in}}%
\pgfpathlineto{\pgfqpoint{4.847194in}{2.042976in}}%
\pgfpathlineto{\pgfqpoint{4.872102in}{2.067885in}}%
\pgfpathlineto{\pgfqpoint{4.897010in}{2.092793in}}%
\pgfpathlineto{\pgfqpoint{4.921918in}{2.117701in}}%
\pgfpathlineto{\pgfqpoint{4.946826in}{2.142609in}}%
\pgfpathlineto{\pgfqpoint{4.971735in}{2.167517in}}%
\pgfpathlineto{\pgfqpoint{4.996643in}{2.192425in}}%
\pgfpathlineto{\pgfqpoint{5.021551in}{2.217334in}}%
\pgfpathlineto{\pgfqpoint{5.046459in}{2.242242in}}%
\pgfpathlineto{\pgfqpoint{5.071367in}{2.267150in}}%
\pgfpathlineto{\pgfqpoint{5.096275in}{2.292058in}}%
\pgfpathlineto{\pgfqpoint{5.121184in}{2.316966in}}%
\pgfpathlineto{\pgfqpoint{5.146092in}{2.341874in}}%
\pgfpathlineto{\pgfqpoint{5.171000in}{2.366783in}}%
\pgfpathlineto{\pgfqpoint{5.195908in}{2.391691in}}%
\pgfpathlineto{\pgfqpoint{5.220816in}{2.416599in}}%
\pgfpathlineto{\pgfqpoint{5.245725in}{2.441507in}}%
\pgfpathlineto{\pgfqpoint{5.270633in}{2.466415in}}%
\pgfpathlineto{\pgfqpoint{5.295541in}{2.491323in}}%
\pgfpathlineto{\pgfqpoint{5.320449in}{2.516232in}}%
\pgfpathlineto{\pgfqpoint{5.345357in}{2.541140in}}%
\pgfpathlineto{\pgfqpoint{5.370265in}{2.566048in}}%
\pgfpathlineto{\pgfqpoint{5.395174in}{2.590956in}}%
\pgfpathlineto{\pgfqpoint{5.420082in}{2.615864in}}%
\pgfpathlineto{\pgfqpoint{5.444990in}{2.640772in}}%
\pgfpathlineto{\pgfqpoint{5.469898in}{2.665681in}}%
\pgfpathlineto{\pgfqpoint{5.494806in}{2.690589in}}%
\pgfpathlineto{\pgfqpoint{5.519714in}{2.715497in}}%
\pgfpathlineto{\pgfqpoint{5.544623in}{2.740405in}}%
\pgfpathlineto{\pgfqpoint{5.569531in}{2.765313in}}%
\pgfpathlineto{\pgfqpoint{5.594439in}{2.790222in}}%
\pgfpathlineto{\pgfqpoint{5.619347in}{2.815130in}}%
\pgfpathlineto{\pgfqpoint{5.644255in}{2.840038in}}%
\pgfpathlineto{\pgfqpoint{5.669163in}{2.864946in}}%
\pgfpathlineto{\pgfqpoint{5.694072in}{2.889854in}}%
\pgfpathlineto{\pgfqpoint{5.718980in}{2.914762in}}%
\pgfpathlineto{\pgfqpoint{5.743888in}{2.939671in}}%
\pgfpathlineto{\pgfqpoint{5.768796in}{2.964579in}}%
\pgfpathlineto{\pgfqpoint{5.793704in}{2.989487in}}%
\pgfpathlineto{\pgfqpoint{5.818612in}{3.014395in}}%
\pgfusepath{stroke}%
\end{pgfscope}%
\begin{pgfscope}%
\pgfpathrectangle{\pgfqpoint{3.352703in}{0.548486in}}{\pgfqpoint{2.465909in}{2.465909in}}%
\pgfusepath{clip}%
\pgfsetbuttcap%
\pgfsetroundjoin%
\pgfsetlinewidth{1.505625pt}%
\definecolor{currentstroke}{rgb}{1.000000,0.000000,0.000000}%
\pgfsetstrokecolor{currentstroke}%
\pgfsetdash{{5.550000pt}{2.400000pt}}{0.000000pt}%
\pgfpathmoveto{\pgfqpoint{3.352703in}{1.781441in}}%
\pgfpathlineto{\pgfqpoint{3.377612in}{1.756532in}}%
\pgfpathlineto{\pgfqpoint{3.402520in}{1.731624in}}%
\pgfpathlineto{\pgfqpoint{3.427428in}{1.706716in}}%
\pgfpathlineto{\pgfqpoint{3.452336in}{1.681808in}}%
\pgfpathlineto{\pgfqpoint{3.477244in}{1.656900in}}%
\pgfpathlineto{\pgfqpoint{3.502152in}{1.631991in}}%
\pgfpathlineto{\pgfqpoint{3.527061in}{1.607083in}}%
\pgfpathlineto{\pgfqpoint{3.551969in}{1.582175in}}%
\pgfpathlineto{\pgfqpoint{3.576877in}{1.557267in}}%
\pgfpathlineto{\pgfqpoint{3.601785in}{1.532359in}}%
\pgfpathlineto{\pgfqpoint{3.626693in}{1.507451in}}%
\pgfpathlineto{\pgfqpoint{3.651601in}{1.482542in}}%
\pgfpathlineto{\pgfqpoint{3.676510in}{1.457634in}}%
\pgfpathlineto{\pgfqpoint{3.701418in}{1.432726in}}%
\pgfpathlineto{\pgfqpoint{3.726326in}{1.407818in}}%
\pgfpathlineto{\pgfqpoint{3.751234in}{1.382910in}}%
\pgfpathlineto{\pgfqpoint{3.776142in}{1.358002in}}%
\pgfpathlineto{\pgfqpoint{3.801051in}{1.333093in}}%
\pgfpathlineto{\pgfqpoint{3.825959in}{1.308185in}}%
\pgfpathlineto{\pgfqpoint{3.850867in}{1.283277in}}%
\pgfpathlineto{\pgfqpoint{3.875775in}{1.258369in}}%
\pgfpathlineto{\pgfqpoint{3.900683in}{1.233461in}}%
\pgfpathlineto{\pgfqpoint{3.925591in}{1.208553in}}%
\pgfpathlineto{\pgfqpoint{3.950500in}{1.183644in}}%
\pgfpathlineto{\pgfqpoint{3.975408in}{1.158736in}}%
\pgfpathlineto{\pgfqpoint{4.000316in}{1.133828in}}%
\pgfpathlineto{\pgfqpoint{4.025224in}{1.108920in}}%
\pgfpathlineto{\pgfqpoint{4.050132in}{1.084012in}}%
\pgfpathlineto{\pgfqpoint{4.075040in}{1.059104in}}%
\pgfpathlineto{\pgfqpoint{4.099949in}{1.034195in}}%
\pgfpathlineto{\pgfqpoint{4.124857in}{1.009287in}}%
\pgfpathlineto{\pgfqpoint{4.149765in}{0.984379in}}%
\pgfpathlineto{\pgfqpoint{4.174673in}{0.959471in}}%
\pgfpathlineto{\pgfqpoint{4.199581in}{0.984379in}}%
\pgfpathlineto{\pgfqpoint{4.224489in}{1.009287in}}%
\pgfpathlineto{\pgfqpoint{4.249398in}{1.034195in}}%
\pgfpathlineto{\pgfqpoint{4.274306in}{1.059104in}}%
\pgfpathlineto{\pgfqpoint{4.299214in}{1.084012in}}%
\pgfpathlineto{\pgfqpoint{4.324122in}{1.108920in}}%
\pgfpathlineto{\pgfqpoint{4.349030in}{1.133828in}}%
\pgfpathlineto{\pgfqpoint{4.373938in}{1.158736in}}%
\pgfpathlineto{\pgfqpoint{4.398847in}{1.183644in}}%
\pgfpathlineto{\pgfqpoint{4.423755in}{1.208553in}}%
\pgfpathlineto{\pgfqpoint{4.448663in}{1.233461in}}%
\pgfpathlineto{\pgfqpoint{4.473571in}{1.258369in}}%
\pgfpathlineto{\pgfqpoint{4.498479in}{1.283277in}}%
\pgfpathlineto{\pgfqpoint{4.523388in}{1.308185in}}%
\pgfpathlineto{\pgfqpoint{4.548296in}{1.333093in}}%
\pgfpathlineto{\pgfqpoint{4.573204in}{1.358002in}}%
\pgfpathlineto{\pgfqpoint{4.598112in}{1.382910in}}%
\pgfpathlineto{\pgfqpoint{4.623020in}{1.407818in}}%
\pgfpathlineto{\pgfqpoint{4.647928in}{1.432726in}}%
\pgfpathlineto{\pgfqpoint{4.672837in}{1.457634in}}%
\pgfpathlineto{\pgfqpoint{4.697745in}{1.482542in}}%
\pgfpathlineto{\pgfqpoint{4.722653in}{1.507451in}}%
\pgfpathlineto{\pgfqpoint{4.747561in}{1.532359in}}%
\pgfpathlineto{\pgfqpoint{4.772469in}{1.557267in}}%
\pgfpathlineto{\pgfqpoint{4.797377in}{1.582175in}}%
\pgfpathlineto{\pgfqpoint{4.822286in}{1.607083in}}%
\pgfpathlineto{\pgfqpoint{4.847194in}{1.631991in}}%
\pgfpathlineto{\pgfqpoint{4.872102in}{1.656900in}}%
\pgfpathlineto{\pgfqpoint{4.897010in}{1.681808in}}%
\pgfpathlineto{\pgfqpoint{4.921918in}{1.706716in}}%
\pgfpathlineto{\pgfqpoint{4.946826in}{1.731624in}}%
\pgfpathlineto{\pgfqpoint{4.971735in}{1.756532in}}%
\pgfpathlineto{\pgfqpoint{4.996643in}{1.781441in}}%
\pgfpathlineto{\pgfqpoint{5.021551in}{1.806349in}}%
\pgfpathlineto{\pgfqpoint{5.046459in}{1.831257in}}%
\pgfpathlineto{\pgfqpoint{5.071367in}{1.856165in}}%
\pgfpathlineto{\pgfqpoint{5.096275in}{1.881073in}}%
\pgfpathlineto{\pgfqpoint{5.121184in}{1.905981in}}%
\pgfpathlineto{\pgfqpoint{5.146092in}{1.930890in}}%
\pgfpathlineto{\pgfqpoint{5.171000in}{1.955798in}}%
\pgfpathlineto{\pgfqpoint{5.195908in}{1.980706in}}%
\pgfpathlineto{\pgfqpoint{5.220816in}{2.005614in}}%
\pgfpathlineto{\pgfqpoint{5.245725in}{2.030522in}}%
\pgfpathlineto{\pgfqpoint{5.270633in}{2.055430in}}%
\pgfpathlineto{\pgfqpoint{5.295541in}{2.080339in}}%
\pgfpathlineto{\pgfqpoint{5.320449in}{2.105247in}}%
\pgfpathlineto{\pgfqpoint{5.345357in}{2.130155in}}%
\pgfpathlineto{\pgfqpoint{5.370265in}{2.155063in}}%
\pgfpathlineto{\pgfqpoint{5.395174in}{2.179971in}}%
\pgfpathlineto{\pgfqpoint{5.420082in}{2.204879in}}%
\pgfpathlineto{\pgfqpoint{5.444990in}{2.229788in}}%
\pgfpathlineto{\pgfqpoint{5.469898in}{2.254696in}}%
\pgfpathlineto{\pgfqpoint{5.494806in}{2.279604in}}%
\pgfpathlineto{\pgfqpoint{5.519714in}{2.304512in}}%
\pgfpathlineto{\pgfqpoint{5.544623in}{2.329420in}}%
\pgfpathlineto{\pgfqpoint{5.569531in}{2.354328in}}%
\pgfpathlineto{\pgfqpoint{5.594439in}{2.379237in}}%
\pgfpathlineto{\pgfqpoint{5.619347in}{2.404145in}}%
\pgfpathlineto{\pgfqpoint{5.644255in}{2.429053in}}%
\pgfpathlineto{\pgfqpoint{5.669163in}{2.453961in}}%
\pgfpathlineto{\pgfqpoint{5.694072in}{2.478869in}}%
\pgfpathlineto{\pgfqpoint{5.718980in}{2.503778in}}%
\pgfpathlineto{\pgfqpoint{5.743888in}{2.528686in}}%
\pgfpathlineto{\pgfqpoint{5.768796in}{2.553594in}}%
\pgfpathlineto{\pgfqpoint{5.793704in}{2.578502in}}%
\pgfpathlineto{\pgfqpoint{5.818612in}{2.603410in}}%
\pgfusepath{stroke}%
\end{pgfscope}%
\begin{pgfscope}%
\pgfpathrectangle{\pgfqpoint{3.352703in}{0.548486in}}{\pgfqpoint{2.465909in}{2.465909in}}%
\pgfusepath{clip}%
\pgfsetrectcap%
\pgfsetroundjoin%
\pgfsetlinewidth{1.505625pt}%
\definecolor{currentstroke}{rgb}{0.000000,0.000000,1.000000}%
\pgfsetstrokecolor{currentstroke}%
\pgfsetdash{}{0pt}%
\pgfpathmoveto{\pgfqpoint{3.352703in}{1.370456in}}%
\pgfpathlineto{\pgfqpoint{3.377612in}{1.345547in}}%
\pgfpathlineto{\pgfqpoint{3.402520in}{1.320639in}}%
\pgfpathlineto{\pgfqpoint{3.427428in}{1.295731in}}%
\pgfpathlineto{\pgfqpoint{3.452336in}{1.270823in}}%
\pgfpathlineto{\pgfqpoint{3.477244in}{1.245915in}}%
\pgfpathlineto{\pgfqpoint{3.502152in}{1.221007in}}%
\pgfpathlineto{\pgfqpoint{3.527061in}{1.196098in}}%
\pgfpathlineto{\pgfqpoint{3.551969in}{1.171190in}}%
\pgfpathlineto{\pgfqpoint{3.576877in}{1.146282in}}%
\pgfpathlineto{\pgfqpoint{3.601785in}{1.121374in}}%
\pgfpathlineto{\pgfqpoint{3.626693in}{1.096466in}}%
\pgfpathlineto{\pgfqpoint{3.651601in}{1.071558in}}%
\pgfpathlineto{\pgfqpoint{3.676510in}{1.046649in}}%
\pgfpathlineto{\pgfqpoint{3.701418in}{1.021741in}}%
\pgfpathlineto{\pgfqpoint{3.726326in}{0.996833in}}%
\pgfpathlineto{\pgfqpoint{3.751234in}{0.971925in}}%
\pgfpathlineto{\pgfqpoint{3.776142in}{0.947017in}}%
\pgfpathlineto{\pgfqpoint{3.801051in}{0.922109in}}%
\pgfpathlineto{\pgfqpoint{3.825959in}{0.897200in}}%
\pgfpathlineto{\pgfqpoint{3.850867in}{0.872292in}}%
\pgfpathlineto{\pgfqpoint{3.875775in}{0.847384in}}%
\pgfpathlineto{\pgfqpoint{3.900683in}{0.822476in}}%
\pgfpathlineto{\pgfqpoint{3.925591in}{0.797568in}}%
\pgfpathlineto{\pgfqpoint{3.950500in}{0.772660in}}%
\pgfpathlineto{\pgfqpoint{3.975408in}{0.747751in}}%
\pgfpathlineto{\pgfqpoint{4.000316in}{0.722843in}}%
\pgfpathlineto{\pgfqpoint{4.025224in}{0.697935in}}%
\pgfpathlineto{\pgfqpoint{4.050132in}{0.673027in}}%
\pgfpathlineto{\pgfqpoint{4.075040in}{0.648119in}}%
\pgfpathlineto{\pgfqpoint{4.099949in}{0.623210in}}%
\pgfpathlineto{\pgfqpoint{4.124857in}{0.598302in}}%
\pgfpathlineto{\pgfqpoint{4.149765in}{0.573394in}}%
\pgfpathlineto{\pgfqpoint{4.174673in}{0.548486in}}%
\pgfpathlineto{\pgfqpoint{4.199581in}{0.573394in}}%
\pgfpathlineto{\pgfqpoint{4.224489in}{0.598302in}}%
\pgfpathlineto{\pgfqpoint{4.249398in}{0.623210in}}%
\pgfpathlineto{\pgfqpoint{4.274306in}{0.648119in}}%
\pgfpathlineto{\pgfqpoint{4.299214in}{0.673027in}}%
\pgfpathlineto{\pgfqpoint{4.324122in}{0.697935in}}%
\pgfpathlineto{\pgfqpoint{4.349030in}{0.722843in}}%
\pgfpathlineto{\pgfqpoint{4.373938in}{0.747751in}}%
\pgfpathlineto{\pgfqpoint{4.398847in}{0.772660in}}%
\pgfpathlineto{\pgfqpoint{4.423755in}{0.797568in}}%
\pgfpathlineto{\pgfqpoint{4.448663in}{0.822476in}}%
\pgfpathlineto{\pgfqpoint{4.473571in}{0.847384in}}%
\pgfpathlineto{\pgfqpoint{4.498479in}{0.872292in}}%
\pgfpathlineto{\pgfqpoint{4.523388in}{0.897200in}}%
\pgfpathlineto{\pgfqpoint{4.548296in}{0.922109in}}%
\pgfpathlineto{\pgfqpoint{4.573204in}{0.947017in}}%
\pgfpathlineto{\pgfqpoint{4.598112in}{0.971925in}}%
\pgfpathlineto{\pgfqpoint{4.623020in}{0.996833in}}%
\pgfpathlineto{\pgfqpoint{4.647928in}{1.021741in}}%
\pgfpathlineto{\pgfqpoint{4.672837in}{1.046649in}}%
\pgfpathlineto{\pgfqpoint{4.697745in}{1.071558in}}%
\pgfpathlineto{\pgfqpoint{4.722653in}{1.096466in}}%
\pgfpathlineto{\pgfqpoint{4.747561in}{1.121374in}}%
\pgfpathlineto{\pgfqpoint{4.772469in}{1.146282in}}%
\pgfpathlineto{\pgfqpoint{4.797377in}{1.171190in}}%
\pgfpathlineto{\pgfqpoint{4.822286in}{1.196098in}}%
\pgfpathlineto{\pgfqpoint{4.847194in}{1.221007in}}%
\pgfpathlineto{\pgfqpoint{4.872102in}{1.245915in}}%
\pgfpathlineto{\pgfqpoint{4.897010in}{1.270823in}}%
\pgfpathlineto{\pgfqpoint{4.921918in}{1.295731in}}%
\pgfpathlineto{\pgfqpoint{4.946826in}{1.320639in}}%
\pgfpathlineto{\pgfqpoint{4.971735in}{1.345547in}}%
\pgfpathlineto{\pgfqpoint{4.996643in}{1.370456in}}%
\pgfpathlineto{\pgfqpoint{5.021551in}{1.395364in}}%
\pgfpathlineto{\pgfqpoint{5.046459in}{1.420272in}}%
\pgfpathlineto{\pgfqpoint{5.071367in}{1.445180in}}%
\pgfpathlineto{\pgfqpoint{5.096275in}{1.470088in}}%
\pgfpathlineto{\pgfqpoint{5.121184in}{1.494997in}}%
\pgfpathlineto{\pgfqpoint{5.146092in}{1.519905in}}%
\pgfpathlineto{\pgfqpoint{5.171000in}{1.544813in}}%
\pgfpathlineto{\pgfqpoint{5.195908in}{1.569721in}}%
\pgfpathlineto{\pgfqpoint{5.220816in}{1.594629in}}%
\pgfpathlineto{\pgfqpoint{5.245725in}{1.619537in}}%
\pgfpathlineto{\pgfqpoint{5.270633in}{1.644446in}}%
\pgfpathlineto{\pgfqpoint{5.295541in}{1.669354in}}%
\pgfpathlineto{\pgfqpoint{5.320449in}{1.694262in}}%
\pgfpathlineto{\pgfqpoint{5.345357in}{1.719170in}}%
\pgfpathlineto{\pgfqpoint{5.370265in}{1.744078in}}%
\pgfpathlineto{\pgfqpoint{5.395174in}{1.768986in}}%
\pgfpathlineto{\pgfqpoint{5.420082in}{1.793895in}}%
\pgfpathlineto{\pgfqpoint{5.444990in}{1.818803in}}%
\pgfpathlineto{\pgfqpoint{5.469898in}{1.843711in}}%
\pgfpathlineto{\pgfqpoint{5.494806in}{1.868619in}}%
\pgfpathlineto{\pgfqpoint{5.519714in}{1.893527in}}%
\pgfpathlineto{\pgfqpoint{5.544623in}{1.918435in}}%
\pgfpathlineto{\pgfqpoint{5.569531in}{1.943344in}}%
\pgfpathlineto{\pgfqpoint{5.594439in}{1.968252in}}%
\pgfpathlineto{\pgfqpoint{5.619347in}{1.993160in}}%
\pgfpathlineto{\pgfqpoint{5.644255in}{2.018068in}}%
\pgfpathlineto{\pgfqpoint{5.669163in}{2.042976in}}%
\pgfpathlineto{\pgfqpoint{5.694072in}{2.067885in}}%
\pgfpathlineto{\pgfqpoint{5.718980in}{2.092793in}}%
\pgfpathlineto{\pgfqpoint{5.743888in}{2.117701in}}%
\pgfpathlineto{\pgfqpoint{5.768796in}{2.142609in}}%
\pgfpathlineto{\pgfqpoint{5.793704in}{2.167517in}}%
\pgfpathlineto{\pgfqpoint{5.818612in}{2.192425in}}%
\pgfusepath{stroke}%
\end{pgfscope}%
\begin{pgfscope}%
\pgfpathrectangle{\pgfqpoint{3.352703in}{0.548486in}}{\pgfqpoint{2.465909in}{2.465909in}}%
\pgfusepath{clip}%
\pgfsetrectcap%
\pgfsetroundjoin%
\pgfsetlinewidth{0.501875pt}%
\definecolor{currentstroke}{rgb}{0.000000,0.000000,0.000000}%
\pgfsetstrokecolor{currentstroke}%
\pgfsetdash{}{0pt}%
\pgfpathmoveto{\pgfqpoint{3.352703in}{1.370456in}}%
\pgfpathlineto{\pgfqpoint{5.818612in}{1.370456in}}%
\pgfusepath{stroke}%
\end{pgfscope}%
\begin{pgfscope}%
\pgfpathrectangle{\pgfqpoint{3.352703in}{0.548486in}}{\pgfqpoint{2.465909in}{2.465909in}}%
\pgfusepath{clip}%
\pgfsetrectcap%
\pgfsetroundjoin%
\pgfsetlinewidth{0.501875pt}%
\definecolor{currentstroke}{rgb}{0.000000,0.000000,0.000000}%
\pgfsetstrokecolor{currentstroke}%
\pgfsetdash{}{0pt}%
\pgfpathmoveto{\pgfqpoint{4.585658in}{0.548486in}}%
\pgfpathlineto{\pgfqpoint{4.585658in}{3.014395in}}%
\pgfusepath{stroke}%
\end{pgfscope}%
\begin{pgfscope}%
\pgfsetbuttcap%
\pgfsetmiterjoin%
\definecolor{currentfill}{rgb}{1.000000,1.000000,1.000000}%
\pgfsetfillcolor{currentfill}%
\pgfsetfillopacity{0.800000}%
\pgfsetlinewidth{1.003750pt}%
\definecolor{currentstroke}{rgb}{0.800000,0.800000,0.800000}%
\pgfsetstrokecolor{currentstroke}%
\pgfsetstrokeopacity{0.800000}%
\pgfsetdash{}{0pt}%
\pgfpathmoveto{\pgfqpoint{4.078453in}{0.617930in}}%
\pgfpathlineto{\pgfqpoint{5.721390in}{0.617930in}}%
\pgfpathquadraticcurveto{\pgfqpoint{5.749168in}{0.617930in}}{\pgfqpoint{5.749168in}{0.645708in}}%
\pgfpathlineto{\pgfqpoint{5.749168in}{1.680268in}}%
\pgfpathquadraticcurveto{\pgfqpoint{5.749168in}{1.708046in}}{\pgfqpoint{5.721390in}{1.708046in}}%
\pgfpathlineto{\pgfqpoint{4.078453in}{1.708046in}}%
\pgfpathquadraticcurveto{\pgfqpoint{4.050675in}{1.708046in}}{\pgfqpoint{4.050675in}{1.680268in}}%
\pgfpathlineto{\pgfqpoint{4.050675in}{0.645708in}}%
\pgfpathquadraticcurveto{\pgfqpoint{4.050675in}{0.617930in}}{\pgfqpoint{4.078453in}{0.617930in}}%
\pgfpathlineto{\pgfqpoint{4.078453in}{0.617930in}}%
\pgfpathclose%
\pgfusepath{stroke,fill}%
\end{pgfscope}%
\begin{pgfscope}%
\pgfsetrectcap%
\pgfsetroundjoin%
\pgfsetlinewidth{1.505625pt}%
\definecolor{currentstroke}{rgb}{0.000000,0.000000,0.000000}%
\pgfsetstrokecolor{currentstroke}%
\pgfsetdash{}{0pt}%
\pgfpathmoveto{\pgfqpoint{4.106231in}{1.595578in}}%
\pgfpathlineto{\pgfqpoint{4.245119in}{1.595578in}}%
\pgfpathlineto{\pgfqpoint{4.384008in}{1.595578in}}%
\pgfusepath{stroke}%
\end{pgfscope}%
\begin{pgfscope}%
\definecolor{textcolor}{rgb}{0.000000,0.000000,0.000000}%
\pgfsetstrokecolor{textcolor}%
\pgfsetfillcolor{textcolor}%
\pgftext[x=4.495119in,y=1.546967in,left,base]{\color{textcolor}{\rmfamily\fontsize{10.000000}{12.000000}\selectfont\catcode`\^=\active\def^{\ifmmode\sp\else\^{}\fi}\catcode`\%=\active\def%{\%}$f(x) = |x + 1| - -2$}}%
\end{pgfscope}%
\begin{pgfscope}%
\pgfsetbuttcap%
\pgfsetroundjoin%
\pgfsetlinewidth{1.505625pt}%
\definecolor{currentstroke}{rgb}{0.000000,0.000000,0.000000}%
\pgfsetstrokecolor{currentstroke}%
\pgfsetdash{{5.550000pt}{2.400000pt}}{0.000000pt}%
\pgfpathmoveto{\pgfqpoint{4.106231in}{1.385888in}}%
\pgfpathlineto{\pgfqpoint{4.245119in}{1.385888in}}%
\pgfpathlineto{\pgfqpoint{4.384008in}{1.385888in}}%
\pgfusepath{stroke}%
\end{pgfscope}%
\begin{pgfscope}%
\definecolor{textcolor}{rgb}{0.000000,0.000000,0.000000}%
\pgfsetstrokecolor{textcolor}%
\pgfsetfillcolor{textcolor}%
\pgftext[x=4.495119in,y=1.337277in,left,base]{\color{textcolor}{\rmfamily\fontsize{10.000000}{12.000000}\selectfont\catcode`\^=\active\def^{\ifmmode\sp\else\^{}\fi}\catcode`\%=\active\def%{\%}$f(x) = |x + 1| - -1$}}%
\end{pgfscope}%
\begin{pgfscope}%
\pgfsetrectcap%
\pgfsetroundjoin%
\pgfsetlinewidth{1.505625pt}%
\definecolor{currentstroke}{rgb}{1.000000,0.000000,0.000000}%
\pgfsetstrokecolor{currentstroke}%
\pgfsetdash{}{0pt}%
\pgfpathmoveto{\pgfqpoint{4.106231in}{1.176199in}}%
\pgfpathlineto{\pgfqpoint{4.245119in}{1.176199in}}%
\pgfpathlineto{\pgfqpoint{4.384008in}{1.176199in}}%
\pgfusepath{stroke}%
\end{pgfscope}%
\begin{pgfscope}%
\definecolor{textcolor}{rgb}{0.000000,0.000000,0.000000}%
\pgfsetstrokecolor{textcolor}%
\pgfsetfillcolor{textcolor}%
\pgftext[x=4.495119in,y=1.127588in,left,base]{\color{textcolor}{\rmfamily\fontsize{10.000000}{12.000000}\selectfont\catcode`\^=\active\def^{\ifmmode\sp\else\^{}\fi}\catcode`\%=\active\def%{\%}$f(x) = |x + 1| - 0$}}%
\end{pgfscope}%
\begin{pgfscope}%
\pgfsetbuttcap%
\pgfsetroundjoin%
\pgfsetlinewidth{1.505625pt}%
\definecolor{currentstroke}{rgb}{1.000000,0.000000,0.000000}%
\pgfsetstrokecolor{currentstroke}%
\pgfsetdash{{5.550000pt}{2.400000pt}}{0.000000pt}%
\pgfpathmoveto{\pgfqpoint{4.106231in}{0.966509in}}%
\pgfpathlineto{\pgfqpoint{4.245119in}{0.966509in}}%
\pgfpathlineto{\pgfqpoint{4.384008in}{0.966509in}}%
\pgfusepath{stroke}%
\end{pgfscope}%
\begin{pgfscope}%
\definecolor{textcolor}{rgb}{0.000000,0.000000,0.000000}%
\pgfsetstrokecolor{textcolor}%
\pgfsetfillcolor{textcolor}%
\pgftext[x=4.495119in,y=0.917898in,left,base]{\color{textcolor}{\rmfamily\fontsize{10.000000}{12.000000}\selectfont\catcode`\^=\active\def^{\ifmmode\sp\else\^{}\fi}\catcode`\%=\active\def%{\%}$f(x) = |x + 1| - 1$}}%
\end{pgfscope}%
\begin{pgfscope}%
\pgfsetrectcap%
\pgfsetroundjoin%
\pgfsetlinewidth{1.505625pt}%
\definecolor{currentstroke}{rgb}{0.000000,0.000000,1.000000}%
\pgfsetstrokecolor{currentstroke}%
\pgfsetdash{}{0pt}%
\pgfpathmoveto{\pgfqpoint{4.106231in}{0.756819in}}%
\pgfpathlineto{\pgfqpoint{4.245119in}{0.756819in}}%
\pgfpathlineto{\pgfqpoint{4.384008in}{0.756819in}}%
\pgfusepath{stroke}%
\end{pgfscope}%
\begin{pgfscope}%
\definecolor{textcolor}{rgb}{0.000000,0.000000,0.000000}%
\pgfsetstrokecolor{textcolor}%
\pgfsetfillcolor{textcolor}%
\pgftext[x=4.495119in,y=0.708208in,left,base]{\color{textcolor}{\rmfamily\fontsize{10.000000}{12.000000}\selectfont\catcode`\^=\active\def^{\ifmmode\sp\else\^{}\fi}\catcode`\%=\active\def%{\%}$f(x) = |x + 1| - 2$}}%
\end{pgfscope}%
\end{pgfpicture}%
\makeatother%
\endgroup%

    \caption{Lösung, Variation der Parameter.}
    \label{fig:Loes_abs_var}
\end{figure}

