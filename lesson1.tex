%!TEX root = main.tex
\chapter{Grundlagen}

\section{Aussagen}

\begin{table}[h!]
    \centering
    \renewcommand{\arraystretch}{1.5} % Vergrößert den Zeilenabstand
    \begin{tabular}{|c|c|}
        \hline
        \textbf{Operation} & \textbf{Mathematische Aussage} \\
        \hline
        Negation & $\neg A$ oder $\bar A$ (nicht A) \\
        \hline
        Konjunktion & $A \land B$ (A und B) \\
        \hline
        Disjunktion & $A \lor B$ (A oder B) \\
        \hline
        Implikation & $A \implies B$ (Wenn A, dann B) \\
        \hline
        Äquivalenz & $A \iff B$ (A genau dann, wenn B) \\
        \hline
        Tautologie & $\top$ (immer wahr) \\
        \hline
        Kontradiktion & $\bot$ (immer falsch) \\
        \hline
    \end{tabular}
    \caption{Tabelle über mathematische Aussagen und Operationen}
\end{table}

\todo{Boolsche logik}

\section{Mengen}

\section{Zahlen}

\section{Rechenoperationen}




