%!TEX root = main.tex
\chapter{Grundlagen}

\section{Motivation}



\section{Aussagen}

\begin{table}[h!]
    \centering
    \renewcommand{\arraystretch}{1.5} % Vergrößert den Zeilenabstand
    \begin{tabular}{|c|c|}
        \hline
        \textbf{Operation} & \textbf{Mathematische Aussage} \\
        \hline
        Negation & $\neg A$ oder $\bar A$ (nicht A) \\
        \hline
        Konjunktion & $A \land B$ (A und B) \\
        \hline
        Disjunktion & $A \lor B$ (A oder B) \\
        \hline
        Implikation & $A \implies B$ (Wenn A, dann B) \\
        \hline
        Äquivalenz & $A \iff B$ (A genau dann, wenn B) \\
        \hline
        Tautologie & $\top$ (immer wahr) \\
        \hline
        Kontradiktion & $\bot$ (immer falsch) \\
        \hline
    \end{tabular}
    \caption{Tabelle über mathematische Aussagen und Operationen}
\end{table}



\subsection{Boolsche Logik}
\label{sec:bool}

\begin{table}[h!]
    \centering
    \begin{tabular}{|c|c|c|c|}
        \hline
        $A$ & $B$ & $A \land B$ & $A \lor B$ \\
        \hline
        0 & 0 & 0 & 0 \\
        0 & 1 & 0 & 1 \\
        1 & 0 & 0 & 1 \\
        1 & 1 & 1 & 1 \\
        \hline
    \end{tabular}
    \caption{Boolesche Algebra: UND und ODER Operationen}
\end{table}

\begin{table}[h!]
    \centering
    \begin{tabular}{|c|c|}
        \hline
        $A$ & $\neg A$ \\
        \hline
        0 & 1 \\
        1 & 0 \\
        \hline
    \end{tabular}
    \caption{Boolesche Algebra: NICHT Operation}
\end{table}

\praxis{Boolsche Logik: Programmieren}

\section{Mengen}

\important{"Eine Menge ist eine Zusammenfassung unterscheidbarer Objekte zu einer Gesamtheit. Die Objekte werden 'Elemente' genannt"}

% \equation{ 
Beispiele für \emph{explizite} Mengen:
$$ A = \{ \text{bla}, \text{franz}, \text{hubert} \} $$ 
$$ B = \{ 1, 3, 5, 8 \} $$ 

Beispiel für Mengen durch \emph{beschreibung}:

$$ C = \{x \, | \, x \, \text{ist eine Geige} \} $$

\subsection{Operationen}
Vergl. \ref{sec:bool}

% \begin{figure}[h!]
%     \centering
%     \includesvg[width=0.5\textwidth]{img/test1} % Dateiname ohne .svg-Endung
%     \caption{Eine SVG-Grafik}
% \end{figure}


\section{Zahlen}

\section{Rechenoperationen}



\section{Aufgaben und Beispiele}
