%!TEX root = main.tex
\chapter{Lineare Algebra, Vektoren und Matrizen}
\todo{Probabilistische system entwicklung, https://www.youtube.com/watch?v=K-8F\_zDMDUI\&list=PLMrJAkhIeNNTYaOnVI3QpH7jgULnAmvPA\&index=3}
\todo[inline]{Digital Audio als vektor, Vector scope/oscilloscope music, stereo sachen, Rotationsmatrix, Hadamardmatrix, Householdermatrix,  state space formulations, modal analyse}

\citep{strang2020linear}

% \begin{multicols}{2}

\section{Motivation}
Vektoren und Matrizen findet man sehr häufig in der Audiotechnischen Literatur. Wie bereits in Kapitel \ref{chap:complex} beschrieben, sind komplexe Zahlen allgegenwärtig und deren Interpretation als Vektor bedeutsam und praktisch. Andererseits kann zB. ein Stereosignal als Vektor aufgefasst werden: Zu jedem Zeitpunkt hat es 2 Zahlen, diese können als Vektor im 2-dimensionalen Raum aufgefasst werden, was wiederum diverse Operationen nahelegt. Ebenso scheint es spätestens an diesem Punkt naheliegend ein $n$-Kanal Signal als Vektor mit $n$ Einträgen zu verstehen und womöglich hieraus Schlüsse ziehen zu dürfen. Wenn es um verräumlichung von Signalen (zB. Ambisonics) geht und daher Positionen im Dreidimensionalen Raum ($\mathbb{R}^3$) benötigt werden ist klar, dass wir diese mittels Vektoren beschreiben können. Aber auch wenn es um Direktionalität von Lautsprechern oder Mikrophonen geht werden letztlich 2 bis 3 Dimensionale Felder beschrieben und auch hier kommen Vektoren zum Einsatz. Es sei aber an dieser Stelle betont dass Vektoren und Matrizen nicht ausschließlich für Audioanwendungen im Dreidimensionalen Raum zum Einsatz kommen. Rotations- Hadamard- und Householdermatrizen \citep{PASPWEB2010}\footnote{https://ccrma.stanford.edu/~jos/pasp/Hadamard\_Matrix.html} werden für Hall-Algorithmen und Resonatoren verwendet , Filtersysteme können über Zustandsraumdarstellungen via Vektoren und Matrizen beschrieben werden \citep{bilbao2009numerical} und vieles mehr.       

\section{Vektoren}
Es wird nun einführendes über Vektoren gesagt aber recht kurz und bündig da 1. angenommen wird, dass insbesondere 2 und 3 dimensionale Vektoren zu den weniger abstrakten und aus der Schule bekannten Themen gehören und 2. Vektoren als Spezialfall von Matrizen gesehen werden können und wir unsere Aufmerksamkeit so schnell wie möglich diesen zuwenden wollen.

\important{
Ein Vektor in einem $n$-Dimensionalen Raum ($\mathbb{R}^n$) ist eine geordnete Liste von Zahlen.

\begin{equation}
\vec{a} = \begin{bmatrix}
a_1 & a_2 & \dots & a_n
\end{bmatrix}
\end{equation}


Der Vektor $\vec{a}$ und seine Elemente $a_1, a_2, \dots a_n$ sind hier als \textbf{Reihenvektor} (auch 'Zeilenvektor', engl.'Row vector') dargestellt. Ebenso können wir \textbf{Spaltenvektoren} (engl. 'column vector') darstellen:
\begin{equation}
\vec{a} = \begin{bmatrix}
a_1 \\
a_2 \\
\vdots \\
a_n
\end{bmatrix}
\end{equation}

\textbf{Konventionen} \\
Dieses Dokument wird die Konvention 'Reihen' statt 'Zeilen' verwenden. Vorteilhaft ist daran, dass 'R' vor 'S' im Alphabet kommt und es hilft uns uns die \emph{Reihenfolge} von Indizes von Matrizen zu merken: \textbf{Reihe, Spalte}.

Es gibt unterschiedliche Schreibweisen in Unterschiedlicher Literatur. Dieses Dokument verwendet die Schreibweise $\vec{a}$. Manchmal findet man auch fett gedruckte Buchstaben um zu notieren, dass es sich um einen Vektor handelt: $\mathbf{a}$. Zudem gibt es Literatur die runde Klammern bevorzugt statt eckigen wie in diesem Dokument: $\mathbf{a} = \begin{pmatrix}
a_1 & a_2 & \dots & a_n
\end{pmatrix}$. All das ist lediglich ein Unterschied in Konventionen.
}


\begin{python}{Vektoren: Arrays in Python}
In [1]: %pylab
In [2]: a = array([1,2,3]) #Vektor Definition
   ...: print(a.shape)
   ...: print(a)
(3,)
[1 2 3]
In [3]: b = array([[5, 6, 7]]) # Matrix, 1 Reihenvektor
   ...: print(b.shape)
   ...: print(b)
(1, 3)
[[5 6 7]]
In [4]: c = array([[8, 9, 10]]).T # Matrix, 1 Spaltenvektor
   ...: print(c.shape)
   ...: print(c)
(3, 1)
[[ 8]
 [ 9]
 [10]]
\end{python}

\subsection*{Hadamard\footnote{Jacques Hadamard (Jacques Salomon Hadamard; * 8. Dezember 1865 in Versailles; † 17. Oktober 1963 in Paris) war ein französischer Mathematiker.} Produkt}
\index{Hadamard Produkt}

Auch Elementweises oder Komponentenweises Produkt. Einfache elementweise Multiplikation.
$$ \vec{a}\circ \vec{b} = \begin{bmatrix}a_1b_1 & a_2b_2 & \dots & a_n b_n\end{bmatrix}$$ 

Leider wird üblicherweise die selbe Notation benutzt um die Verkettung von Funktionen zu beschreiben: $(f \circ g)(x) = f(g(x))$. Welche Operation nun gemeint ist, erschließt sich typischerweise aus dem Kontext. Beide Notationen kommen ohnehin nicht wahnsinnig oft vor.

In Python\footnote{Hier ist ein wichtiger unterschied zu \texttt{MATLAB} Code da dort punktweise Multiplikation mit \texttt{.*} ausgeführt wird.} kann elementweise Multiplikation durch einfache Multiplikation ausgeführt werden: \pythoninline{c = a*b}.

\begin{python}{Elementweise Multiplikation von 2 Arrays in Python.}
In [1]: %pylab
In [2]: a = array([1,2,3])
In [3]: b = array([5, 6, 7])
In [4]: a*b
Out[4]: array([ 5, 12, 21])
\end{python}

\begin{question}
In Python, definiere 2 Arrays, $\vec{a}$ und $\vec{b}$. Diese sollen jeweils eine Sinus Schwingung enthalten, aber jeweils mit anderer Frequenz. Multipliziere sie, plotte das Ergebnis und höre dir das Ergebnis an. Befehle um Arrays als Audio file auf die Festplatte zu schreiben sind im Python Cheat-Sheet zu finden in Abschnitt \ref{sec:pythonCheat}. Was passiert hier aus Audio Sicht?
\end{question}

\begin{answer}
Zunächst, was passiert ist eine 'Ringmodulation' (sozusagen eine Variante der Amplitudenmodulation). Um eine Sinus Schwingung mit Frequenz $f_a$ definieren wir eine Funktion $a(t)$:
\begin{equation}
a(t) = sin(2 \pi t f_a)
\end{equation}
Um hier eine Sekunde Sound zu erzeugen mit Samplingrate $f_s = 44100$ benötigen wir einen Vektor $\vec{t}$ der die Zeit in Sekunden enthält: $\vec{t}:= \begin{bmatrix} 0 & \Delta_t& 2 \Delta_t &  \dots 1-\Delta_t \end{bmatrix}$ wobei $\Delta_t$ das Sampling-Intervall $\Delta_t = \frac{1}{f_s}$. Praktisch ist es am einfachsten man erzeugt man einen Array der einen Index enthält: $ns := \begin{bmatrix} 0 & 1 & 2 &  N-1 \end{bmatrix}$. Dies kann man in python via \texttt{arange(N)} erzielen. Diese kann anschliessend durch die samplingrate dividiert werden um $\vec{t}$ zu erhalten. Da wir eine Sekunde Audio erzeugen wollen ist $N=f_s$ die Samplingrate.


\begin{python}{Ringmodulation.}
In [1]: %pylab
In [2]: import soundfile as sf
In [3]: sr= 44100
In [4]: ns = arange(sr)
In [5]: t = ns/sr
In [6]: a = sin(2*pi*t*100)
In [7]: b = sin(2*pi*t*50)
In [8]: c = a*b
In [9]: plot(t,c)
In [10]: sf.write('ringmodTest.wav', c, sr)
\end{python}
\end{answer}


\subsection*{Skalarprodukt}
\index{Skalarprodukt}

Auch 'inneres Produkt' oder 'Punktprodukt', engl. 'Dot product'. Ergibt im fall von Vektoren einen Skalar aber eigentlich ein Spezialfall der Matrixmultiplikation!
$$\vec{a}\cdot \vec{b} = |\vec{a}|\cdot |\vec{b}|\cdot cos \sphericalangle ({\vec {a}},{\vec {b}}) = a_1 b_1 + a_2 b_2 + \dots +a_n b_n = \sum_{i=1}^N a_ib_i$$
Siehe \citep[p.~45]{Westermmann2008}  für Herleitung.

Der Teil $|\vec{a}|\cdot |\vec{b}|\cdot cos \sphericalangle ({\vec {a}},{\vec {b}})$ mag weniger intuitiv sein als die summe der Produkte, mag auch weniger Praktisch sein in der Anwendung. Aus diesem Teil ergibt sich aber eine entscheidende Eigenschaft des Skalarprodukts: \emph{Es ist 0 wenn die Vektoren Senkrecht aufeinander stehen.}

Python: \pythoninline{dot(a,b)}

\begin{python}{Skalarprodukt in Python}
In [1]: %pylab
In [2]: a = array([2,3,4])
In [3]: b = array([5,6,7])
In [4]: 2*5+3*6+4*7
Out[4]: 56
In [5]: dot(a,b)
Out[5]: 56
\end{python}

% \end{multicols}

\praxis{
    Hat all dies irgendeine Relevanz in der Audio-Verarbeitung?
    Eine einfache Bezugnahme kann hergestellt werden indem wir uns verdeutlichen dass die Energie eines zeitdiskreten Signals \citep{christensen2019introduction} folgendermaßen berechnet werden kann:
    $$ E = \sum_{n=0}^{N-1} |x[n]| ^2$$
    \index{Energie}
    Also wir nehmen den Betrag von allen Elementen\footnote{Man mag sich fragen wozu man den Betrag nimmt um anschließend zu quadrieren. Nach dem quadrieren kann doch ohnehin kein negatives Vorzeichen mehr erscheinen. Diese Definition lässt jedoch auch komplexe Signale zu!}, quadrieren sie alle und summieren sie. Dies ist im Falle eines Realwertigen Signals das selbe wie das Skalare Produkt des Vektors mit sich selbst ($aa = a^2$). \index{RMS}\index{Quadratisches Mittel} Die Formel kann ein wenig modifiziert werden um uns den 'Root Mean Squared' (RMS, das Quadratische Mittel) zu berechnen:
    $$ RMS = \sqrt{\frac{1}{N}\sum^N_{i=1} x_i^2}$$
    \index{Korrelation}
    Eine Weiteres Beispiel aus der Praxis ist in Abbildung \ref{fig:dotProdCorr} veranschaulicht. Die in den Titeln der Plots ersichtlichen Skalaren Produkte legen nahe dass hier vielleicht ein Normalisierungsfaktor fehlt um den Korrelationsgrad zu errechnen.
    Abbildung \ref{fig:dotProdCorr} wurde durch dieses Notebook erstellt: \colab{https://colab.research.google.com/github/hrtlacek/matheFuerTonmeisterinnen/blob/master/python/notebooks/dotProduct.ipynb}.

\begin{figure}[H]
    \centering
    %% Creator: Matplotlib, PGF backend
%%
%% To include the figure in your LaTeX document, write
%%   \input{<filename>.pgf}
%%
%% Make sure the required packages are loaded in your preamble
%%   \usepackage{pgf}
%%
%% Also ensure that all the required font packages are loaded; for instance,
%% the lmodern package is sometimes necessary when using math font.
%%   \usepackage{lmodern}
%%
%% Figures using additional raster images can only be included by \input if
%% they are in the same directory as the main LaTeX file. For loading figures
%% from other directories you can use the `import` package
%%   \usepackage{import}
%%
%% and then include the figures with
%%   \import{<path to file>}{<filename>.pgf}
%%
%% Matplotlib used the following preamble
%%   \def\mathdefault#1{#1}
%%   \everymath=\expandafter{\the\everymath\displaystyle}
%%   
%%   \usepackage{fontspec}
%%   \setmainfont{VeraSe.ttf}[Path=\detokenize{/usr/share/fonts/TTF/}]
%%   \setsansfont{DejaVuSans.ttf}[Path=\detokenize{/home/pl/miniconda3/lib/python3.12/site-packages/matplotlib/mpl-data/fonts/ttf/}]
%%   \setmonofont{DejaVuSansMono.ttf}[Path=\detokenize{/home/pl/miniconda3/lib/python3.12/site-packages/matplotlib/mpl-data/fonts/ttf/}]
%%   \makeatletter\@ifpackageloaded{underscore}{}{\usepackage[strings]{underscore}}\makeatother
%%
\begingroup%
\makeatletter%
\begin{pgfpicture}%
\pgfpathrectangle{\pgfpointorigin}{\pgfqpoint{5.818249in}{3.166819in}}%
\pgfusepath{use as bounding box, clip}%
\begin{pgfscope}%
\pgfsetbuttcap%
\pgfsetmiterjoin%
\pgfsetlinewidth{0.000000pt}%
\definecolor{currentstroke}{rgb}{0.000000,0.000000,0.000000}%
\pgfsetstrokecolor{currentstroke}%
\pgfsetstrokeopacity{0.000000}%
\pgfsetdash{}{0pt}%
\pgfpathmoveto{\pgfqpoint{0.000000in}{0.000000in}}%
\pgfpathlineto{\pgfqpoint{5.818249in}{0.000000in}}%
\pgfpathlineto{\pgfqpoint{5.818249in}{3.166819in}}%
\pgfpathlineto{\pgfqpoint{0.000000in}{3.166819in}}%
\pgfpathlineto{\pgfqpoint{0.000000in}{0.000000in}}%
\pgfpathclose%
\pgfusepath{}%
\end{pgfscope}%
\begin{pgfscope}%
\pgfsetbuttcap%
\pgfsetmiterjoin%
\pgfsetlinewidth{0.000000pt}%
\definecolor{currentstroke}{rgb}{0.000000,0.000000,0.000000}%
\pgfsetstrokecolor{currentstroke}%
\pgfsetstrokeopacity{0.000000}%
\pgfsetdash{}{0pt}%
\pgfpathmoveto{\pgfqpoint{0.526127in}{0.548486in}}%
\pgfpathlineto{\pgfqpoint{2.007744in}{0.548486in}}%
\pgfpathlineto{\pgfqpoint{2.007744in}{2.858486in}}%
\pgfpathlineto{\pgfqpoint{0.526127in}{2.858486in}}%
\pgfpathlineto{\pgfqpoint{0.526127in}{0.548486in}}%
\pgfpathclose%
\pgfusepath{}%
\end{pgfscope}%
\begin{pgfscope}%
\pgfsetbuttcap%
\pgfsetroundjoin%
\definecolor{currentfill}{rgb}{0.000000,0.000000,0.000000}%
\pgfsetfillcolor{currentfill}%
\pgfsetlinewidth{0.803000pt}%
\definecolor{currentstroke}{rgb}{0.000000,0.000000,0.000000}%
\pgfsetstrokecolor{currentstroke}%
\pgfsetdash{}{0pt}%
\pgfsys@defobject{currentmarker}{\pgfqpoint{0.000000in}{-0.048611in}}{\pgfqpoint{0.000000in}{0.000000in}}{%
\pgfpathmoveto{\pgfqpoint{0.000000in}{0.000000in}}%
\pgfpathlineto{\pgfqpoint{0.000000in}{-0.048611in}}%
\pgfusepath{stroke,fill}%
}%
\begin{pgfscope}%
\pgfsys@transformshift{0.526127in}{0.548486in}%
\pgfsys@useobject{currentmarker}{}%
\end{pgfscope}%
\end{pgfscope}%
\begin{pgfscope}%
\definecolor{textcolor}{rgb}{0.000000,0.000000,0.000000}%
\pgfsetstrokecolor{textcolor}%
\pgfsetfillcolor{textcolor}%
\pgftext[x=0.526127in,y=0.451264in,,top]{\color{textcolor}{\rmfamily\fontsize{10.000000}{12.000000}\selectfont\catcode`\^=\active\def^{\ifmmode\sp\else\^{}\fi}\catcode`\%=\active\def%{\%}0.00}}%
\end{pgfscope}%
\begin{pgfscope}%
\pgfsetbuttcap%
\pgfsetroundjoin%
\definecolor{currentfill}{rgb}{0.000000,0.000000,0.000000}%
\pgfsetfillcolor{currentfill}%
\pgfsetlinewidth{0.803000pt}%
\definecolor{currentstroke}{rgb}{0.000000,0.000000,0.000000}%
\pgfsetstrokecolor{currentstroke}%
\pgfsetdash{}{0pt}%
\pgfsys@defobject{currentmarker}{\pgfqpoint{0.000000in}{-0.048611in}}{\pgfqpoint{0.000000in}{0.000000in}}{%
\pgfpathmoveto{\pgfqpoint{0.000000in}{0.000000in}}%
\pgfpathlineto{\pgfqpoint{0.000000in}{-0.048611in}}%
\pgfusepath{stroke,fill}%
}%
\begin{pgfscope}%
\pgfsys@transformshift{1.266935in}{0.548486in}%
\pgfsys@useobject{currentmarker}{}%
\end{pgfscope}%
\end{pgfscope}%
\begin{pgfscope}%
\definecolor{textcolor}{rgb}{0.000000,0.000000,0.000000}%
\pgfsetstrokecolor{textcolor}%
\pgfsetfillcolor{textcolor}%
\pgftext[x=1.266935in,y=0.451264in,,top]{\color{textcolor}{\rmfamily\fontsize{10.000000}{12.000000}\selectfont\catcode`\^=\active\def^{\ifmmode\sp\else\^{}\fi}\catcode`\%=\active\def%{\%}0.05}}%
\end{pgfscope}%
\begin{pgfscope}%
\pgfsetbuttcap%
\pgfsetroundjoin%
\definecolor{currentfill}{rgb}{0.000000,0.000000,0.000000}%
\pgfsetfillcolor{currentfill}%
\pgfsetlinewidth{0.803000pt}%
\definecolor{currentstroke}{rgb}{0.000000,0.000000,0.000000}%
\pgfsetstrokecolor{currentstroke}%
\pgfsetdash{}{0pt}%
\pgfsys@defobject{currentmarker}{\pgfqpoint{0.000000in}{-0.048611in}}{\pgfqpoint{0.000000in}{0.000000in}}{%
\pgfpathmoveto{\pgfqpoint{0.000000in}{0.000000in}}%
\pgfpathlineto{\pgfqpoint{0.000000in}{-0.048611in}}%
\pgfusepath{stroke,fill}%
}%
\begin{pgfscope}%
\pgfsys@transformshift{2.007744in}{0.548486in}%
\pgfsys@useobject{currentmarker}{}%
\end{pgfscope}%
\end{pgfscope}%
\begin{pgfscope}%
\definecolor{textcolor}{rgb}{0.000000,0.000000,0.000000}%
\pgfsetstrokecolor{textcolor}%
\pgfsetfillcolor{textcolor}%
\pgftext[x=2.007744in,y=0.451264in,,top]{\color{textcolor}{\rmfamily\fontsize{10.000000}{12.000000}\selectfont\catcode`\^=\active\def^{\ifmmode\sp\else\^{}\fi}\catcode`\%=\active\def%{\%}0.10}}%
\end{pgfscope}%
\begin{pgfscope}%
\definecolor{textcolor}{rgb}{0.000000,0.000000,0.000000}%
\pgfsetstrokecolor{textcolor}%
\pgfsetfillcolor{textcolor}%
\pgftext[x=1.266935in,y=0.261295in,,top]{\color{textcolor}{\rmfamily\fontsize{12.000000}{14.400000}\selectfont\catcode`\^=\active\def^{\ifmmode\sp\else\^{}\fi}\catcode`\%=\active\def%{\%}$t$}}%
\end{pgfscope}%
\begin{pgfscope}%
\pgfsetbuttcap%
\pgfsetroundjoin%
\definecolor{currentfill}{rgb}{0.000000,0.000000,0.000000}%
\pgfsetfillcolor{currentfill}%
\pgfsetlinewidth{0.803000pt}%
\definecolor{currentstroke}{rgb}{0.000000,0.000000,0.000000}%
\pgfsetstrokecolor{currentstroke}%
\pgfsetdash{}{0pt}%
\pgfsys@defobject{currentmarker}{\pgfqpoint{-0.048611in}{0.000000in}}{\pgfqpoint{-0.000000in}{0.000000in}}{%
\pgfpathmoveto{\pgfqpoint{-0.000000in}{0.000000in}}%
\pgfpathlineto{\pgfqpoint{-0.048611in}{0.000000in}}%
\pgfusepath{stroke,fill}%
}%
\begin{pgfscope}%
\pgfsys@transformshift{0.526127in}{0.653486in}%
\pgfsys@useobject{currentmarker}{}%
\end{pgfscope}%
\end{pgfscope}%
\begin{pgfscope}%
\definecolor{textcolor}{rgb}{0.000000,0.000000,0.000000}%
\pgfsetstrokecolor{textcolor}%
\pgfsetfillcolor{textcolor}%
\pgftext[x=0.100000in, y=0.600724in, left, base]{\color{textcolor}{\rmfamily\fontsize{10.000000}{12.000000}\selectfont\catcode`\^=\active\def^{\ifmmode\sp\else\^{}\fi}\catcode`\%=\active\def%{\%}\ensuremath{-}1.0}}%
\end{pgfscope}%
\begin{pgfscope}%
\pgfsetbuttcap%
\pgfsetroundjoin%
\definecolor{currentfill}{rgb}{0.000000,0.000000,0.000000}%
\pgfsetfillcolor{currentfill}%
\pgfsetlinewidth{0.803000pt}%
\definecolor{currentstroke}{rgb}{0.000000,0.000000,0.000000}%
\pgfsetstrokecolor{currentstroke}%
\pgfsetdash{}{0pt}%
\pgfsys@defobject{currentmarker}{\pgfqpoint{-0.048611in}{0.000000in}}{\pgfqpoint{-0.000000in}{0.000000in}}{%
\pgfpathmoveto{\pgfqpoint{-0.000000in}{0.000000in}}%
\pgfpathlineto{\pgfqpoint{-0.048611in}{0.000000in}}%
\pgfusepath{stroke,fill}%
}%
\begin{pgfscope}%
\pgfsys@transformshift{0.526127in}{1.178486in}%
\pgfsys@useobject{currentmarker}{}%
\end{pgfscope}%
\end{pgfscope}%
\begin{pgfscope}%
\definecolor{textcolor}{rgb}{0.000000,0.000000,0.000000}%
\pgfsetstrokecolor{textcolor}%
\pgfsetfillcolor{textcolor}%
\pgftext[x=0.100000in, y=1.125724in, left, base]{\color{textcolor}{\rmfamily\fontsize{10.000000}{12.000000}\selectfont\catcode`\^=\active\def^{\ifmmode\sp\else\^{}\fi}\catcode`\%=\active\def%{\%}\ensuremath{-}0.5}}%
\end{pgfscope}%
\begin{pgfscope}%
\pgfsetbuttcap%
\pgfsetroundjoin%
\definecolor{currentfill}{rgb}{0.000000,0.000000,0.000000}%
\pgfsetfillcolor{currentfill}%
\pgfsetlinewidth{0.803000pt}%
\definecolor{currentstroke}{rgb}{0.000000,0.000000,0.000000}%
\pgfsetstrokecolor{currentstroke}%
\pgfsetdash{}{0pt}%
\pgfsys@defobject{currentmarker}{\pgfqpoint{-0.048611in}{0.000000in}}{\pgfqpoint{-0.000000in}{0.000000in}}{%
\pgfpathmoveto{\pgfqpoint{-0.000000in}{0.000000in}}%
\pgfpathlineto{\pgfqpoint{-0.048611in}{0.000000in}}%
\pgfusepath{stroke,fill}%
}%
\begin{pgfscope}%
\pgfsys@transformshift{0.526127in}{1.703486in}%
\pgfsys@useobject{currentmarker}{}%
\end{pgfscope}%
\end{pgfscope}%
\begin{pgfscope}%
\definecolor{textcolor}{rgb}{0.000000,0.000000,0.000000}%
\pgfsetstrokecolor{textcolor}%
\pgfsetfillcolor{textcolor}%
\pgftext[x=0.208025in, y=1.650724in, left, base]{\color{textcolor}{\rmfamily\fontsize{10.000000}{12.000000}\selectfont\catcode`\^=\active\def^{\ifmmode\sp\else\^{}\fi}\catcode`\%=\active\def%{\%}0.0}}%
\end{pgfscope}%
\begin{pgfscope}%
\pgfsetbuttcap%
\pgfsetroundjoin%
\definecolor{currentfill}{rgb}{0.000000,0.000000,0.000000}%
\pgfsetfillcolor{currentfill}%
\pgfsetlinewidth{0.803000pt}%
\definecolor{currentstroke}{rgb}{0.000000,0.000000,0.000000}%
\pgfsetstrokecolor{currentstroke}%
\pgfsetdash{}{0pt}%
\pgfsys@defobject{currentmarker}{\pgfqpoint{-0.048611in}{0.000000in}}{\pgfqpoint{-0.000000in}{0.000000in}}{%
\pgfpathmoveto{\pgfqpoint{-0.000000in}{0.000000in}}%
\pgfpathlineto{\pgfqpoint{-0.048611in}{0.000000in}}%
\pgfusepath{stroke,fill}%
}%
\begin{pgfscope}%
\pgfsys@transformshift{0.526127in}{2.228486in}%
\pgfsys@useobject{currentmarker}{}%
\end{pgfscope}%
\end{pgfscope}%
\begin{pgfscope}%
\definecolor{textcolor}{rgb}{0.000000,0.000000,0.000000}%
\pgfsetstrokecolor{textcolor}%
\pgfsetfillcolor{textcolor}%
\pgftext[x=0.208025in, y=2.175724in, left, base]{\color{textcolor}{\rmfamily\fontsize{10.000000}{12.000000}\selectfont\catcode`\^=\active\def^{\ifmmode\sp\else\^{}\fi}\catcode`\%=\active\def%{\%}0.5}}%
\end{pgfscope}%
\begin{pgfscope}%
\pgfsetbuttcap%
\pgfsetroundjoin%
\definecolor{currentfill}{rgb}{0.000000,0.000000,0.000000}%
\pgfsetfillcolor{currentfill}%
\pgfsetlinewidth{0.803000pt}%
\definecolor{currentstroke}{rgb}{0.000000,0.000000,0.000000}%
\pgfsetstrokecolor{currentstroke}%
\pgfsetdash{}{0pt}%
\pgfsys@defobject{currentmarker}{\pgfqpoint{-0.048611in}{0.000000in}}{\pgfqpoint{-0.000000in}{0.000000in}}{%
\pgfpathmoveto{\pgfqpoint{-0.000000in}{0.000000in}}%
\pgfpathlineto{\pgfqpoint{-0.048611in}{0.000000in}}%
\pgfusepath{stroke,fill}%
}%
\begin{pgfscope}%
\pgfsys@transformshift{0.526127in}{2.753486in}%
\pgfsys@useobject{currentmarker}{}%
\end{pgfscope}%
\end{pgfscope}%
\begin{pgfscope}%
\definecolor{textcolor}{rgb}{0.000000,0.000000,0.000000}%
\pgfsetstrokecolor{textcolor}%
\pgfsetfillcolor{textcolor}%
\pgftext[x=0.208025in, y=2.700724in, left, base]{\color{textcolor}{\rmfamily\fontsize{10.000000}{12.000000}\selectfont\catcode`\^=\active\def^{\ifmmode\sp\else\^{}\fi}\catcode`\%=\active\def%{\%}1.0}}%
\end{pgfscope}%
\begin{pgfscope}%
\pgfpathrectangle{\pgfqpoint{0.526127in}{0.548486in}}{\pgfqpoint{1.481618in}{2.310000in}}%
\pgfusepath{clip}%
\pgfsetrectcap%
\pgfsetroundjoin%
\pgfsetlinewidth{1.505625pt}%
\definecolor{currentstroke}{rgb}{0.000000,0.000000,0.000000}%
\pgfsetstrokecolor{currentstroke}%
\pgfsetdash{}{0pt}%
\pgfpathmoveto{\pgfqpoint{0.526127in}{1.703486in}}%
\pgfpathlineto{\pgfqpoint{0.540943in}{2.027954in}}%
\pgfpathlineto{\pgfqpoint{0.555759in}{2.320660in}}%
\pgfpathlineto{\pgfqpoint{0.570575in}{2.552954in}}%
\pgfpathlineto{\pgfqpoint{0.585391in}{2.702095in}}%
\pgfpathlineto{\pgfqpoint{0.600208in}{2.753486in}}%
\pgfpathlineto{\pgfqpoint{0.615024in}{2.702095in}}%
\pgfpathlineto{\pgfqpoint{0.629840in}{2.552954in}}%
\pgfpathlineto{\pgfqpoint{0.644656in}{2.320660in}}%
\pgfpathlineto{\pgfqpoint{0.659472in}{2.027954in}}%
\pgfpathlineto{\pgfqpoint{0.674288in}{1.703486in}}%
\pgfpathlineto{\pgfqpoint{0.689105in}{1.379018in}}%
\pgfpathlineto{\pgfqpoint{0.703921in}{1.086311in}}%
\pgfpathlineto{\pgfqpoint{0.718737in}{0.854018in}}%
\pgfpathlineto{\pgfqpoint{0.733553in}{0.704877in}}%
\pgfpathlineto{\pgfqpoint{0.748369in}{0.653486in}}%
\pgfpathlineto{\pgfqpoint{0.763185in}{0.704877in}}%
\pgfpathlineto{\pgfqpoint{0.778002in}{0.854018in}}%
\pgfpathlineto{\pgfqpoint{0.792818in}{1.086311in}}%
\pgfpathlineto{\pgfqpoint{0.807634in}{1.379018in}}%
\pgfpathlineto{\pgfqpoint{0.822450in}{1.703486in}}%
\pgfpathlineto{\pgfqpoint{0.837266in}{2.027954in}}%
\pgfpathlineto{\pgfqpoint{0.852083in}{2.320660in}}%
\pgfpathlineto{\pgfqpoint{0.866899in}{2.552954in}}%
\pgfpathlineto{\pgfqpoint{0.881715in}{2.702095in}}%
\pgfpathlineto{\pgfqpoint{0.896531in}{2.753486in}}%
\pgfpathlineto{\pgfqpoint{0.911347in}{2.702095in}}%
\pgfpathlineto{\pgfqpoint{0.926163in}{2.552954in}}%
\pgfpathlineto{\pgfqpoint{0.940980in}{2.320660in}}%
\pgfpathlineto{\pgfqpoint{0.955796in}{2.027954in}}%
\pgfpathlineto{\pgfqpoint{0.970612in}{1.703486in}}%
\pgfpathlineto{\pgfqpoint{0.985428in}{1.379018in}}%
\pgfpathlineto{\pgfqpoint{1.000244in}{1.086311in}}%
\pgfpathlineto{\pgfqpoint{1.015060in}{0.854018in}}%
\pgfpathlineto{\pgfqpoint{1.029877in}{0.704877in}}%
\pgfpathlineto{\pgfqpoint{1.044693in}{0.653486in}}%
\pgfpathlineto{\pgfqpoint{1.059509in}{0.704877in}}%
\pgfpathlineto{\pgfqpoint{1.074325in}{0.854018in}}%
\pgfpathlineto{\pgfqpoint{1.089141in}{1.086311in}}%
\pgfpathlineto{\pgfqpoint{1.103958in}{1.379018in}}%
\pgfpathlineto{\pgfqpoint{1.118774in}{1.703486in}}%
\pgfpathlineto{\pgfqpoint{1.133590in}{2.027954in}}%
\pgfpathlineto{\pgfqpoint{1.148406in}{2.320660in}}%
\pgfpathlineto{\pgfqpoint{1.163222in}{2.552954in}}%
\pgfpathlineto{\pgfqpoint{1.178038in}{2.702095in}}%
\pgfpathlineto{\pgfqpoint{1.192855in}{2.753486in}}%
\pgfpathlineto{\pgfqpoint{1.207671in}{2.702095in}}%
\pgfpathlineto{\pgfqpoint{1.222487in}{2.552954in}}%
\pgfpathlineto{\pgfqpoint{1.237303in}{2.320660in}}%
\pgfpathlineto{\pgfqpoint{1.252119in}{2.027954in}}%
\pgfpathlineto{\pgfqpoint{1.266935in}{1.703486in}}%
\pgfpathlineto{\pgfqpoint{1.281752in}{1.379018in}}%
\pgfpathlineto{\pgfqpoint{1.296568in}{1.086311in}}%
\pgfpathlineto{\pgfqpoint{1.311384in}{0.854018in}}%
\pgfpathlineto{\pgfqpoint{1.326200in}{0.704877in}}%
\pgfpathlineto{\pgfqpoint{1.341016in}{0.653486in}}%
\pgfpathlineto{\pgfqpoint{1.355833in}{0.704877in}}%
\pgfpathlineto{\pgfqpoint{1.370649in}{0.854018in}}%
\pgfpathlineto{\pgfqpoint{1.385465in}{1.086311in}}%
\pgfpathlineto{\pgfqpoint{1.400281in}{1.379018in}}%
\pgfpathlineto{\pgfqpoint{1.415097in}{1.703486in}}%
\pgfpathlineto{\pgfqpoint{1.429913in}{2.027954in}}%
\pgfpathlineto{\pgfqpoint{1.444730in}{2.320660in}}%
\pgfpathlineto{\pgfqpoint{1.459546in}{2.552954in}}%
\pgfpathlineto{\pgfqpoint{1.474362in}{2.702095in}}%
\pgfpathlineto{\pgfqpoint{1.489178in}{2.753486in}}%
\pgfpathlineto{\pgfqpoint{1.503994in}{2.702095in}}%
\pgfpathlineto{\pgfqpoint{1.518810in}{2.552954in}}%
\pgfpathlineto{\pgfqpoint{1.533627in}{2.320660in}}%
\pgfpathlineto{\pgfqpoint{1.548443in}{2.027954in}}%
\pgfpathlineto{\pgfqpoint{1.563259in}{1.703486in}}%
\pgfpathlineto{\pgfqpoint{1.578075in}{1.379018in}}%
\pgfpathlineto{\pgfqpoint{1.592891in}{1.086311in}}%
\pgfpathlineto{\pgfqpoint{1.607708in}{0.854018in}}%
\pgfpathlineto{\pgfqpoint{1.622524in}{0.704877in}}%
\pgfpathlineto{\pgfqpoint{1.637340in}{0.653486in}}%
\pgfpathlineto{\pgfqpoint{1.652156in}{0.704877in}}%
\pgfpathlineto{\pgfqpoint{1.666972in}{0.854018in}}%
\pgfpathlineto{\pgfqpoint{1.681788in}{1.086311in}}%
\pgfpathlineto{\pgfqpoint{1.696605in}{1.379018in}}%
\pgfpathlineto{\pgfqpoint{1.711421in}{1.703486in}}%
\pgfpathlineto{\pgfqpoint{1.726237in}{2.027954in}}%
\pgfpathlineto{\pgfqpoint{1.741053in}{2.320660in}}%
\pgfpathlineto{\pgfqpoint{1.755869in}{2.552954in}}%
\pgfpathlineto{\pgfqpoint{1.770685in}{2.702095in}}%
\pgfpathlineto{\pgfqpoint{1.785502in}{2.753486in}}%
\pgfpathlineto{\pgfqpoint{1.800318in}{2.702095in}}%
\pgfpathlineto{\pgfqpoint{1.815134in}{2.552954in}}%
\pgfpathlineto{\pgfqpoint{1.829950in}{2.320660in}}%
\pgfpathlineto{\pgfqpoint{1.844766in}{2.027954in}}%
\pgfpathlineto{\pgfqpoint{1.859583in}{1.703486in}}%
\pgfpathlineto{\pgfqpoint{1.874399in}{1.379018in}}%
\pgfpathlineto{\pgfqpoint{1.889215in}{1.086311in}}%
\pgfpathlineto{\pgfqpoint{1.904031in}{0.854018in}}%
\pgfpathlineto{\pgfqpoint{1.918847in}{0.704877in}}%
\pgfpathlineto{\pgfqpoint{1.933663in}{0.653486in}}%
\pgfpathlineto{\pgfqpoint{1.948480in}{0.704877in}}%
\pgfpathlineto{\pgfqpoint{1.963296in}{0.854018in}}%
\pgfpathlineto{\pgfqpoint{1.978112in}{1.086311in}}%
\pgfpathlineto{\pgfqpoint{1.992928in}{1.379018in}}%
\pgfpathlineto{\pgfqpoint{2.007744in}{1.703486in}}%
\pgfpathlineto{\pgfqpoint{2.017744in}{1.922482in}}%
\pgfusepath{stroke}%
\end{pgfscope}%
\begin{pgfscope}%
\pgfpathrectangle{\pgfqpoint{0.526127in}{0.548486in}}{\pgfqpoint{1.481618in}{2.310000in}}%
\pgfusepath{clip}%
\pgfsetbuttcap%
\pgfsetroundjoin%
\pgfsetlinewidth{1.505625pt}%
\definecolor{currentstroke}{rgb}{1.000000,0.000000,0.000000}%
\pgfsetstrokecolor{currentstroke}%
\pgfsetdash{{5.550000pt}{2.400000pt}}{0.000000pt}%
\pgfpathmoveto{\pgfqpoint{0.526127in}{1.703486in}}%
\pgfpathlineto{\pgfqpoint{0.540943in}{2.027954in}}%
\pgfpathlineto{\pgfqpoint{0.555759in}{2.320660in}}%
\pgfpathlineto{\pgfqpoint{0.570575in}{2.552954in}}%
\pgfpathlineto{\pgfqpoint{0.585391in}{2.702095in}}%
\pgfpathlineto{\pgfqpoint{0.600208in}{2.753486in}}%
\pgfpathlineto{\pgfqpoint{0.615024in}{2.702095in}}%
\pgfpathlineto{\pgfqpoint{0.629840in}{2.552954in}}%
\pgfpathlineto{\pgfqpoint{0.644656in}{2.320660in}}%
\pgfpathlineto{\pgfqpoint{0.659472in}{2.027954in}}%
\pgfpathlineto{\pgfqpoint{0.674288in}{1.703486in}}%
\pgfpathlineto{\pgfqpoint{0.689105in}{1.379018in}}%
\pgfpathlineto{\pgfqpoint{0.703921in}{1.086311in}}%
\pgfpathlineto{\pgfqpoint{0.718737in}{0.854018in}}%
\pgfpathlineto{\pgfqpoint{0.733553in}{0.704877in}}%
\pgfpathlineto{\pgfqpoint{0.748369in}{0.653486in}}%
\pgfpathlineto{\pgfqpoint{0.763185in}{0.704877in}}%
\pgfpathlineto{\pgfqpoint{0.778002in}{0.854018in}}%
\pgfpathlineto{\pgfqpoint{0.792818in}{1.086311in}}%
\pgfpathlineto{\pgfqpoint{0.807634in}{1.379018in}}%
\pgfpathlineto{\pgfqpoint{0.822450in}{1.703486in}}%
\pgfpathlineto{\pgfqpoint{0.837266in}{2.027954in}}%
\pgfpathlineto{\pgfqpoint{0.852083in}{2.320660in}}%
\pgfpathlineto{\pgfqpoint{0.866899in}{2.552954in}}%
\pgfpathlineto{\pgfqpoint{0.881715in}{2.702095in}}%
\pgfpathlineto{\pgfqpoint{0.896531in}{2.753486in}}%
\pgfpathlineto{\pgfqpoint{0.911347in}{2.702095in}}%
\pgfpathlineto{\pgfqpoint{0.926163in}{2.552954in}}%
\pgfpathlineto{\pgfqpoint{0.940980in}{2.320660in}}%
\pgfpathlineto{\pgfqpoint{0.955796in}{2.027954in}}%
\pgfpathlineto{\pgfqpoint{0.970612in}{1.703486in}}%
\pgfpathlineto{\pgfqpoint{0.985428in}{1.379018in}}%
\pgfpathlineto{\pgfqpoint{1.000244in}{1.086311in}}%
\pgfpathlineto{\pgfqpoint{1.015060in}{0.854018in}}%
\pgfpathlineto{\pgfqpoint{1.029877in}{0.704877in}}%
\pgfpathlineto{\pgfqpoint{1.044693in}{0.653486in}}%
\pgfpathlineto{\pgfqpoint{1.059509in}{0.704877in}}%
\pgfpathlineto{\pgfqpoint{1.074325in}{0.854018in}}%
\pgfpathlineto{\pgfqpoint{1.089141in}{1.086311in}}%
\pgfpathlineto{\pgfqpoint{1.103958in}{1.379018in}}%
\pgfpathlineto{\pgfqpoint{1.118774in}{1.703486in}}%
\pgfpathlineto{\pgfqpoint{1.133590in}{2.027954in}}%
\pgfpathlineto{\pgfqpoint{1.148406in}{2.320660in}}%
\pgfpathlineto{\pgfqpoint{1.163222in}{2.552954in}}%
\pgfpathlineto{\pgfqpoint{1.178038in}{2.702095in}}%
\pgfpathlineto{\pgfqpoint{1.192855in}{2.753486in}}%
\pgfpathlineto{\pgfqpoint{1.207671in}{2.702095in}}%
\pgfpathlineto{\pgfqpoint{1.222487in}{2.552954in}}%
\pgfpathlineto{\pgfqpoint{1.237303in}{2.320660in}}%
\pgfpathlineto{\pgfqpoint{1.252119in}{2.027954in}}%
\pgfpathlineto{\pgfqpoint{1.266935in}{1.703486in}}%
\pgfpathlineto{\pgfqpoint{1.281752in}{1.379018in}}%
\pgfpathlineto{\pgfqpoint{1.296568in}{1.086311in}}%
\pgfpathlineto{\pgfqpoint{1.311384in}{0.854018in}}%
\pgfpathlineto{\pgfqpoint{1.326200in}{0.704877in}}%
\pgfpathlineto{\pgfqpoint{1.341016in}{0.653486in}}%
\pgfpathlineto{\pgfqpoint{1.355833in}{0.704877in}}%
\pgfpathlineto{\pgfqpoint{1.370649in}{0.854018in}}%
\pgfpathlineto{\pgfqpoint{1.385465in}{1.086311in}}%
\pgfpathlineto{\pgfqpoint{1.400281in}{1.379018in}}%
\pgfpathlineto{\pgfqpoint{1.415097in}{1.703486in}}%
\pgfpathlineto{\pgfqpoint{1.429913in}{2.027954in}}%
\pgfpathlineto{\pgfqpoint{1.444730in}{2.320660in}}%
\pgfpathlineto{\pgfqpoint{1.459546in}{2.552954in}}%
\pgfpathlineto{\pgfqpoint{1.474362in}{2.702095in}}%
\pgfpathlineto{\pgfqpoint{1.489178in}{2.753486in}}%
\pgfpathlineto{\pgfqpoint{1.503994in}{2.702095in}}%
\pgfpathlineto{\pgfqpoint{1.518810in}{2.552954in}}%
\pgfpathlineto{\pgfqpoint{1.533627in}{2.320660in}}%
\pgfpathlineto{\pgfqpoint{1.548443in}{2.027954in}}%
\pgfpathlineto{\pgfqpoint{1.563259in}{1.703486in}}%
\pgfpathlineto{\pgfqpoint{1.578075in}{1.379018in}}%
\pgfpathlineto{\pgfqpoint{1.592891in}{1.086311in}}%
\pgfpathlineto{\pgfqpoint{1.607708in}{0.854018in}}%
\pgfpathlineto{\pgfqpoint{1.622524in}{0.704877in}}%
\pgfpathlineto{\pgfqpoint{1.637340in}{0.653486in}}%
\pgfpathlineto{\pgfqpoint{1.652156in}{0.704877in}}%
\pgfpathlineto{\pgfqpoint{1.666972in}{0.854018in}}%
\pgfpathlineto{\pgfqpoint{1.681788in}{1.086311in}}%
\pgfpathlineto{\pgfqpoint{1.696605in}{1.379018in}}%
\pgfpathlineto{\pgfqpoint{1.711421in}{1.703486in}}%
\pgfpathlineto{\pgfqpoint{1.726237in}{2.027954in}}%
\pgfpathlineto{\pgfqpoint{1.741053in}{2.320660in}}%
\pgfpathlineto{\pgfqpoint{1.755869in}{2.552954in}}%
\pgfpathlineto{\pgfqpoint{1.770685in}{2.702095in}}%
\pgfpathlineto{\pgfqpoint{1.785502in}{2.753486in}}%
\pgfpathlineto{\pgfqpoint{1.800318in}{2.702095in}}%
\pgfpathlineto{\pgfqpoint{1.815134in}{2.552954in}}%
\pgfpathlineto{\pgfqpoint{1.829950in}{2.320660in}}%
\pgfpathlineto{\pgfqpoint{1.844766in}{2.027954in}}%
\pgfpathlineto{\pgfqpoint{1.859583in}{1.703486in}}%
\pgfpathlineto{\pgfqpoint{1.874399in}{1.379018in}}%
\pgfpathlineto{\pgfqpoint{1.889215in}{1.086311in}}%
\pgfpathlineto{\pgfqpoint{1.904031in}{0.854018in}}%
\pgfpathlineto{\pgfqpoint{1.918847in}{0.704877in}}%
\pgfpathlineto{\pgfqpoint{1.933663in}{0.653486in}}%
\pgfpathlineto{\pgfqpoint{1.948480in}{0.704877in}}%
\pgfpathlineto{\pgfqpoint{1.963296in}{0.854018in}}%
\pgfpathlineto{\pgfqpoint{1.978112in}{1.086311in}}%
\pgfpathlineto{\pgfqpoint{1.992928in}{1.379018in}}%
\pgfpathlineto{\pgfqpoint{2.007744in}{1.703486in}}%
\pgfpathlineto{\pgfqpoint{2.017744in}{1.922482in}}%
\pgfusepath{stroke}%
\end{pgfscope}%
\begin{pgfscope}%
\pgfpathrectangle{\pgfqpoint{0.526127in}{0.548486in}}{\pgfqpoint{1.481618in}{2.310000in}}%
\pgfusepath{clip}%
\pgfsetrectcap%
\pgfsetroundjoin%
\pgfsetlinewidth{0.501875pt}%
\definecolor{currentstroke}{rgb}{0.000000,0.000000,0.000000}%
\pgfsetstrokecolor{currentstroke}%
\pgfsetdash{}{0pt}%
\pgfpathmoveto{\pgfqpoint{0.526127in}{1.703486in}}%
\pgfpathlineto{\pgfqpoint{2.007744in}{1.703486in}}%
\pgfusepath{stroke}%
\end{pgfscope}%
\begin{pgfscope}%
\definecolor{textcolor}{rgb}{0.000000,0.000000,0.000000}%
\pgfsetstrokecolor{textcolor}%
\pgfsetfillcolor{textcolor}%
\pgftext[x=1.266935in,y=2.941819in,,base]{\color{textcolor}{\rmfamily\fontsize{12.000000}{14.400000}\selectfont\catcode`\^=\active\def^{\ifmmode\sp\else\^{}\fi}\catcode`\%=\active\def%{\%}$dot(a,b)=$1500.00}}%
\end{pgfscope}%
\begin{pgfscope}%
\pgfsetbuttcap%
\pgfsetmiterjoin%
\definecolor{currentfill}{rgb}{1.000000,1.000000,1.000000}%
\pgfsetfillcolor{currentfill}%
\pgfsetfillopacity{0.800000}%
\pgfsetlinewidth{1.003750pt}%
\definecolor{currentstroke}{rgb}{0.800000,0.800000,0.800000}%
\pgfsetstrokecolor{currentstroke}%
\pgfsetstrokeopacity{0.800000}%
\pgfsetdash{}{0pt}%
\pgfpathmoveto{\pgfqpoint{0.623349in}{0.617930in}}%
\pgfpathlineto{\pgfqpoint{1.973378in}{0.617930in}}%
\pgfpathquadraticcurveto{\pgfqpoint{2.001156in}{0.617930in}}{\pgfqpoint{2.001156in}{0.645708in}}%
\pgfpathlineto{\pgfqpoint{2.001156in}{1.051199in}}%
\pgfpathquadraticcurveto{\pgfqpoint{2.001156in}{1.078977in}}{\pgfqpoint{1.973378in}{1.078977in}}%
\pgfpathlineto{\pgfqpoint{0.623349in}{1.078977in}}%
\pgfpathquadraticcurveto{\pgfqpoint{0.595571in}{1.078977in}}{\pgfqpoint{0.595571in}{1.051199in}}%
\pgfpathlineto{\pgfqpoint{0.595571in}{0.645708in}}%
\pgfpathquadraticcurveto{\pgfqpoint{0.595571in}{0.617930in}}{\pgfqpoint{0.623349in}{0.617930in}}%
\pgfpathlineto{\pgfqpoint{0.623349in}{0.617930in}}%
\pgfpathclose%
\pgfusepath{stroke,fill}%
\end{pgfscope}%
\begin{pgfscope}%
\pgfsetrectcap%
\pgfsetroundjoin%
\pgfsetlinewidth{1.505625pt}%
\definecolor{currentstroke}{rgb}{0.000000,0.000000,0.000000}%
\pgfsetstrokecolor{currentstroke}%
\pgfsetdash{}{0pt}%
\pgfpathmoveto{\pgfqpoint{0.651127in}{0.966509in}}%
\pgfpathlineto{\pgfqpoint{0.790016in}{0.966509in}}%
\pgfpathlineto{\pgfqpoint{0.928904in}{0.966509in}}%
\pgfusepath{stroke}%
\end{pgfscope}%
\begin{pgfscope}%
\definecolor{textcolor}{rgb}{0.000000,0.000000,0.000000}%
\pgfsetstrokecolor{textcolor}%
\pgfsetfillcolor{textcolor}%
\pgftext[x=1.040016in,y=0.917898in,left,base]{\color{textcolor}{\rmfamily\fontsize{10.000000}{12.000000}\selectfont\catcode`\^=\active\def^{\ifmmode\sp\else\^{}\fi}\catcode`\%=\active\def%{\%}$a = sin(2\pi 50t)$}}%
\end{pgfscope}%
\begin{pgfscope}%
\pgfsetbuttcap%
\pgfsetroundjoin%
\pgfsetlinewidth{1.505625pt}%
\definecolor{currentstroke}{rgb}{1.000000,0.000000,0.000000}%
\pgfsetstrokecolor{currentstroke}%
\pgfsetdash{{5.550000pt}{2.400000pt}}{0.000000pt}%
\pgfpathmoveto{\pgfqpoint{0.651127in}{0.756819in}}%
\pgfpathlineto{\pgfqpoint{0.790016in}{0.756819in}}%
\pgfpathlineto{\pgfqpoint{0.928904in}{0.756819in}}%
\pgfusepath{stroke}%
\end{pgfscope}%
\begin{pgfscope}%
\definecolor{textcolor}{rgb}{0.000000,0.000000,0.000000}%
\pgfsetstrokecolor{textcolor}%
\pgfsetfillcolor{textcolor}%
\pgftext[x=1.040016in,y=0.708208in,left,base]{\color{textcolor}{\rmfamily\fontsize{10.000000}{12.000000}\selectfont\catcode`\^=\active\def^{\ifmmode\sp\else\^{}\fi}\catcode`\%=\active\def%{\%}$b = sin(2\pi 50t)$}}%
\end{pgfscope}%
\begin{pgfscope}%
\pgfsetbuttcap%
\pgfsetmiterjoin%
\pgfsetlinewidth{0.000000pt}%
\definecolor{currentstroke}{rgb}{0.000000,0.000000,0.000000}%
\pgfsetstrokecolor{currentstroke}%
\pgfsetstrokeopacity{0.000000}%
\pgfsetdash{}{0pt}%
\pgfpathmoveto{\pgfqpoint{2.304068in}{0.548486in}}%
\pgfpathlineto{\pgfqpoint{3.785685in}{0.548486in}}%
\pgfpathlineto{\pgfqpoint{3.785685in}{2.858486in}}%
\pgfpathlineto{\pgfqpoint{2.304068in}{2.858486in}}%
\pgfpathlineto{\pgfqpoint{2.304068in}{0.548486in}}%
\pgfpathclose%
\pgfusepath{}%
\end{pgfscope}%
\begin{pgfscope}%
\pgfsetbuttcap%
\pgfsetroundjoin%
\definecolor{currentfill}{rgb}{0.000000,0.000000,0.000000}%
\pgfsetfillcolor{currentfill}%
\pgfsetlinewidth{0.803000pt}%
\definecolor{currentstroke}{rgb}{0.000000,0.000000,0.000000}%
\pgfsetstrokecolor{currentstroke}%
\pgfsetdash{}{0pt}%
\pgfsys@defobject{currentmarker}{\pgfqpoint{0.000000in}{-0.048611in}}{\pgfqpoint{0.000000in}{0.000000in}}{%
\pgfpathmoveto{\pgfqpoint{0.000000in}{0.000000in}}%
\pgfpathlineto{\pgfqpoint{0.000000in}{-0.048611in}}%
\pgfusepath{stroke,fill}%
}%
\begin{pgfscope}%
\pgfsys@transformshift{2.304068in}{0.548486in}%
\pgfsys@useobject{currentmarker}{}%
\end{pgfscope}%
\end{pgfscope}%
\begin{pgfscope}%
\definecolor{textcolor}{rgb}{0.000000,0.000000,0.000000}%
\pgfsetstrokecolor{textcolor}%
\pgfsetfillcolor{textcolor}%
\pgftext[x=2.304068in,y=0.451264in,,top]{\color{textcolor}{\rmfamily\fontsize{10.000000}{12.000000}\selectfont\catcode`\^=\active\def^{\ifmmode\sp\else\^{}\fi}\catcode`\%=\active\def%{\%}0.00}}%
\end{pgfscope}%
\begin{pgfscope}%
\pgfsetbuttcap%
\pgfsetroundjoin%
\definecolor{currentfill}{rgb}{0.000000,0.000000,0.000000}%
\pgfsetfillcolor{currentfill}%
\pgfsetlinewidth{0.803000pt}%
\definecolor{currentstroke}{rgb}{0.000000,0.000000,0.000000}%
\pgfsetstrokecolor{currentstroke}%
\pgfsetdash{}{0pt}%
\pgfsys@defobject{currentmarker}{\pgfqpoint{0.000000in}{-0.048611in}}{\pgfqpoint{0.000000in}{0.000000in}}{%
\pgfpathmoveto{\pgfqpoint{0.000000in}{0.000000in}}%
\pgfpathlineto{\pgfqpoint{0.000000in}{-0.048611in}}%
\pgfusepath{stroke,fill}%
}%
\begin{pgfscope}%
\pgfsys@transformshift{3.044877in}{0.548486in}%
\pgfsys@useobject{currentmarker}{}%
\end{pgfscope}%
\end{pgfscope}%
\begin{pgfscope}%
\definecolor{textcolor}{rgb}{0.000000,0.000000,0.000000}%
\pgfsetstrokecolor{textcolor}%
\pgfsetfillcolor{textcolor}%
\pgftext[x=3.044877in,y=0.451264in,,top]{\color{textcolor}{\rmfamily\fontsize{10.000000}{12.000000}\selectfont\catcode`\^=\active\def^{\ifmmode\sp\else\^{}\fi}\catcode`\%=\active\def%{\%}0.05}}%
\end{pgfscope}%
\begin{pgfscope}%
\pgfsetbuttcap%
\pgfsetroundjoin%
\definecolor{currentfill}{rgb}{0.000000,0.000000,0.000000}%
\pgfsetfillcolor{currentfill}%
\pgfsetlinewidth{0.803000pt}%
\definecolor{currentstroke}{rgb}{0.000000,0.000000,0.000000}%
\pgfsetstrokecolor{currentstroke}%
\pgfsetdash{}{0pt}%
\pgfsys@defobject{currentmarker}{\pgfqpoint{0.000000in}{-0.048611in}}{\pgfqpoint{0.000000in}{0.000000in}}{%
\pgfpathmoveto{\pgfqpoint{0.000000in}{0.000000in}}%
\pgfpathlineto{\pgfqpoint{0.000000in}{-0.048611in}}%
\pgfusepath{stroke,fill}%
}%
\begin{pgfscope}%
\pgfsys@transformshift{3.785685in}{0.548486in}%
\pgfsys@useobject{currentmarker}{}%
\end{pgfscope}%
\end{pgfscope}%
\begin{pgfscope}%
\definecolor{textcolor}{rgb}{0.000000,0.000000,0.000000}%
\pgfsetstrokecolor{textcolor}%
\pgfsetfillcolor{textcolor}%
\pgftext[x=3.785685in,y=0.451264in,,top]{\color{textcolor}{\rmfamily\fontsize{10.000000}{12.000000}\selectfont\catcode`\^=\active\def^{\ifmmode\sp\else\^{}\fi}\catcode`\%=\active\def%{\%}0.10}}%
\end{pgfscope}%
\begin{pgfscope}%
\definecolor{textcolor}{rgb}{0.000000,0.000000,0.000000}%
\pgfsetstrokecolor{textcolor}%
\pgfsetfillcolor{textcolor}%
\pgftext[x=3.044877in,y=0.261295in,,top]{\color{textcolor}{\rmfamily\fontsize{12.000000}{14.400000}\selectfont\catcode`\^=\active\def^{\ifmmode\sp\else\^{}\fi}\catcode`\%=\active\def%{\%}$t$}}%
\end{pgfscope}%
\begin{pgfscope}%
\pgfpathrectangle{\pgfqpoint{2.304068in}{0.548486in}}{\pgfqpoint{1.481618in}{2.310000in}}%
\pgfusepath{clip}%
\pgfsetrectcap%
\pgfsetroundjoin%
\pgfsetlinewidth{1.505625pt}%
\definecolor{currentstroke}{rgb}{0.000000,0.000000,0.000000}%
\pgfsetstrokecolor{currentstroke}%
\pgfsetdash{}{0pt}%
\pgfpathmoveto{\pgfqpoint{2.304068in}{1.703486in}}%
\pgfpathlineto{\pgfqpoint{2.318884in}{2.027954in}}%
\pgfpathlineto{\pgfqpoint{2.333700in}{2.320660in}}%
\pgfpathlineto{\pgfqpoint{2.348516in}{2.552954in}}%
\pgfpathlineto{\pgfqpoint{2.363333in}{2.702095in}}%
\pgfpathlineto{\pgfqpoint{2.378149in}{2.753486in}}%
\pgfpathlineto{\pgfqpoint{2.392965in}{2.702095in}}%
\pgfpathlineto{\pgfqpoint{2.407781in}{2.552954in}}%
\pgfpathlineto{\pgfqpoint{2.422597in}{2.320660in}}%
\pgfpathlineto{\pgfqpoint{2.437413in}{2.027954in}}%
\pgfpathlineto{\pgfqpoint{2.452230in}{1.703486in}}%
\pgfpathlineto{\pgfqpoint{2.467046in}{1.379018in}}%
\pgfpathlineto{\pgfqpoint{2.481862in}{1.086311in}}%
\pgfpathlineto{\pgfqpoint{2.496678in}{0.854018in}}%
\pgfpathlineto{\pgfqpoint{2.511494in}{0.704877in}}%
\pgfpathlineto{\pgfqpoint{2.526310in}{0.653486in}}%
\pgfpathlineto{\pgfqpoint{2.541127in}{0.704877in}}%
\pgfpathlineto{\pgfqpoint{2.555943in}{0.854018in}}%
\pgfpathlineto{\pgfqpoint{2.570759in}{1.086311in}}%
\pgfpathlineto{\pgfqpoint{2.585575in}{1.379018in}}%
\pgfpathlineto{\pgfqpoint{2.600391in}{1.703486in}}%
\pgfpathlineto{\pgfqpoint{2.615208in}{2.027954in}}%
\pgfpathlineto{\pgfqpoint{2.630024in}{2.320660in}}%
\pgfpathlineto{\pgfqpoint{2.644840in}{2.552954in}}%
\pgfpathlineto{\pgfqpoint{2.659656in}{2.702095in}}%
\pgfpathlineto{\pgfqpoint{2.674472in}{2.753486in}}%
\pgfpathlineto{\pgfqpoint{2.689288in}{2.702095in}}%
\pgfpathlineto{\pgfqpoint{2.704105in}{2.552954in}}%
\pgfpathlineto{\pgfqpoint{2.718921in}{2.320660in}}%
\pgfpathlineto{\pgfqpoint{2.733737in}{2.027954in}}%
\pgfpathlineto{\pgfqpoint{2.748553in}{1.703486in}}%
\pgfpathlineto{\pgfqpoint{2.763369in}{1.379018in}}%
\pgfpathlineto{\pgfqpoint{2.778185in}{1.086311in}}%
\pgfpathlineto{\pgfqpoint{2.793002in}{0.854018in}}%
\pgfpathlineto{\pgfqpoint{2.807818in}{0.704877in}}%
\pgfpathlineto{\pgfqpoint{2.822634in}{0.653486in}}%
\pgfpathlineto{\pgfqpoint{2.837450in}{0.704877in}}%
\pgfpathlineto{\pgfqpoint{2.852266in}{0.854018in}}%
\pgfpathlineto{\pgfqpoint{2.867083in}{1.086311in}}%
\pgfpathlineto{\pgfqpoint{2.881899in}{1.379018in}}%
\pgfpathlineto{\pgfqpoint{2.896715in}{1.703486in}}%
\pgfpathlineto{\pgfqpoint{2.911531in}{2.027954in}}%
\pgfpathlineto{\pgfqpoint{2.926347in}{2.320660in}}%
\pgfpathlineto{\pgfqpoint{2.941163in}{2.552954in}}%
\pgfpathlineto{\pgfqpoint{2.955980in}{2.702095in}}%
\pgfpathlineto{\pgfqpoint{2.970796in}{2.753486in}}%
\pgfpathlineto{\pgfqpoint{2.985612in}{2.702095in}}%
\pgfpathlineto{\pgfqpoint{3.000428in}{2.552954in}}%
\pgfpathlineto{\pgfqpoint{3.015244in}{2.320660in}}%
\pgfpathlineto{\pgfqpoint{3.030060in}{2.027954in}}%
\pgfpathlineto{\pgfqpoint{3.044877in}{1.703486in}}%
\pgfpathlineto{\pgfqpoint{3.059693in}{1.379018in}}%
\pgfpathlineto{\pgfqpoint{3.074509in}{1.086311in}}%
\pgfpathlineto{\pgfqpoint{3.089325in}{0.854018in}}%
\pgfpathlineto{\pgfqpoint{3.104141in}{0.704877in}}%
\pgfpathlineto{\pgfqpoint{3.118958in}{0.653486in}}%
\pgfpathlineto{\pgfqpoint{3.133774in}{0.704877in}}%
\pgfpathlineto{\pgfqpoint{3.148590in}{0.854018in}}%
\pgfpathlineto{\pgfqpoint{3.163406in}{1.086311in}}%
\pgfpathlineto{\pgfqpoint{3.178222in}{1.379018in}}%
\pgfpathlineto{\pgfqpoint{3.193038in}{1.703486in}}%
\pgfpathlineto{\pgfqpoint{3.207855in}{2.027954in}}%
\pgfpathlineto{\pgfqpoint{3.222671in}{2.320660in}}%
\pgfpathlineto{\pgfqpoint{3.237487in}{2.552954in}}%
\pgfpathlineto{\pgfqpoint{3.252303in}{2.702095in}}%
\pgfpathlineto{\pgfqpoint{3.267119in}{2.753486in}}%
\pgfpathlineto{\pgfqpoint{3.281935in}{2.702095in}}%
\pgfpathlineto{\pgfqpoint{3.296752in}{2.552954in}}%
\pgfpathlineto{\pgfqpoint{3.311568in}{2.320660in}}%
\pgfpathlineto{\pgfqpoint{3.326384in}{2.027954in}}%
\pgfpathlineto{\pgfqpoint{3.341200in}{1.703486in}}%
\pgfpathlineto{\pgfqpoint{3.356016in}{1.379018in}}%
\pgfpathlineto{\pgfqpoint{3.370833in}{1.086311in}}%
\pgfpathlineto{\pgfqpoint{3.385649in}{0.854018in}}%
\pgfpathlineto{\pgfqpoint{3.400465in}{0.704877in}}%
\pgfpathlineto{\pgfqpoint{3.415281in}{0.653486in}}%
\pgfpathlineto{\pgfqpoint{3.430097in}{0.704877in}}%
\pgfpathlineto{\pgfqpoint{3.444913in}{0.854018in}}%
\pgfpathlineto{\pgfqpoint{3.459730in}{1.086311in}}%
\pgfpathlineto{\pgfqpoint{3.474546in}{1.379018in}}%
\pgfpathlineto{\pgfqpoint{3.489362in}{1.703486in}}%
\pgfpathlineto{\pgfqpoint{3.504178in}{2.027954in}}%
\pgfpathlineto{\pgfqpoint{3.518994in}{2.320660in}}%
\pgfpathlineto{\pgfqpoint{3.533810in}{2.552954in}}%
\pgfpathlineto{\pgfqpoint{3.548627in}{2.702095in}}%
\pgfpathlineto{\pgfqpoint{3.563443in}{2.753486in}}%
\pgfpathlineto{\pgfqpoint{3.578259in}{2.702095in}}%
\pgfpathlineto{\pgfqpoint{3.593075in}{2.552954in}}%
\pgfpathlineto{\pgfqpoint{3.607891in}{2.320660in}}%
\pgfpathlineto{\pgfqpoint{3.622708in}{2.027954in}}%
\pgfpathlineto{\pgfqpoint{3.637524in}{1.703486in}}%
\pgfpathlineto{\pgfqpoint{3.652340in}{1.379018in}}%
\pgfpathlineto{\pgfqpoint{3.667156in}{1.086311in}}%
\pgfpathlineto{\pgfqpoint{3.681972in}{0.854018in}}%
\pgfpathlineto{\pgfqpoint{3.696788in}{0.704877in}}%
\pgfpathlineto{\pgfqpoint{3.711605in}{0.653486in}}%
\pgfpathlineto{\pgfqpoint{3.726421in}{0.704877in}}%
\pgfpathlineto{\pgfqpoint{3.741237in}{0.854018in}}%
\pgfpathlineto{\pgfqpoint{3.756053in}{1.086311in}}%
\pgfpathlineto{\pgfqpoint{3.770869in}{1.379018in}}%
\pgfpathlineto{\pgfqpoint{3.785685in}{1.703486in}}%
\pgfpathlineto{\pgfqpoint{3.795685in}{1.922482in}}%
\pgfusepath{stroke}%
\end{pgfscope}%
\begin{pgfscope}%
\pgfpathrectangle{\pgfqpoint{2.304068in}{0.548486in}}{\pgfqpoint{1.481618in}{2.310000in}}%
\pgfusepath{clip}%
\pgfsetbuttcap%
\pgfsetroundjoin%
\pgfsetlinewidth{1.505625pt}%
\definecolor{currentstroke}{rgb}{1.000000,0.000000,0.000000}%
\pgfsetstrokecolor{currentstroke}%
\pgfsetdash{{5.550000pt}{2.400000pt}}{0.000000pt}%
\pgfpathmoveto{\pgfqpoint{2.304068in}{1.703486in}}%
\pgfpathlineto{\pgfqpoint{2.318884in}{1.379018in}}%
\pgfpathlineto{\pgfqpoint{2.333700in}{1.086311in}}%
\pgfpathlineto{\pgfqpoint{2.348516in}{0.854018in}}%
\pgfpathlineto{\pgfqpoint{2.363333in}{0.704877in}}%
\pgfpathlineto{\pgfqpoint{2.378149in}{0.653486in}}%
\pgfpathlineto{\pgfqpoint{2.392965in}{0.704877in}}%
\pgfpathlineto{\pgfqpoint{2.407781in}{0.854018in}}%
\pgfpathlineto{\pgfqpoint{2.422597in}{1.086311in}}%
\pgfpathlineto{\pgfqpoint{2.437413in}{1.379018in}}%
\pgfpathlineto{\pgfqpoint{2.452230in}{1.703486in}}%
\pgfpathlineto{\pgfqpoint{2.467046in}{2.027954in}}%
\pgfpathlineto{\pgfqpoint{2.481862in}{2.320660in}}%
\pgfpathlineto{\pgfqpoint{2.496678in}{2.552954in}}%
\pgfpathlineto{\pgfqpoint{2.511494in}{2.702095in}}%
\pgfpathlineto{\pgfqpoint{2.526310in}{2.753486in}}%
\pgfpathlineto{\pgfqpoint{2.541127in}{2.702095in}}%
\pgfpathlineto{\pgfqpoint{2.555943in}{2.552954in}}%
\pgfpathlineto{\pgfqpoint{2.570759in}{2.320660in}}%
\pgfpathlineto{\pgfqpoint{2.585575in}{2.027954in}}%
\pgfpathlineto{\pgfqpoint{2.600391in}{1.703486in}}%
\pgfpathlineto{\pgfqpoint{2.615208in}{1.379018in}}%
\pgfpathlineto{\pgfqpoint{2.630024in}{1.086311in}}%
\pgfpathlineto{\pgfqpoint{2.644840in}{0.854018in}}%
\pgfpathlineto{\pgfqpoint{2.659656in}{0.704877in}}%
\pgfpathlineto{\pgfqpoint{2.674472in}{0.653486in}}%
\pgfpathlineto{\pgfqpoint{2.689288in}{0.704877in}}%
\pgfpathlineto{\pgfqpoint{2.704105in}{0.854018in}}%
\pgfpathlineto{\pgfqpoint{2.718921in}{1.086311in}}%
\pgfpathlineto{\pgfqpoint{2.733737in}{1.379018in}}%
\pgfpathlineto{\pgfqpoint{2.748553in}{1.703486in}}%
\pgfpathlineto{\pgfqpoint{2.763369in}{2.027954in}}%
\pgfpathlineto{\pgfqpoint{2.778185in}{2.320660in}}%
\pgfpathlineto{\pgfqpoint{2.793002in}{2.552954in}}%
\pgfpathlineto{\pgfqpoint{2.807818in}{2.702095in}}%
\pgfpathlineto{\pgfqpoint{2.822634in}{2.753486in}}%
\pgfpathlineto{\pgfqpoint{2.837450in}{2.702095in}}%
\pgfpathlineto{\pgfqpoint{2.852266in}{2.552954in}}%
\pgfpathlineto{\pgfqpoint{2.867083in}{2.320660in}}%
\pgfpathlineto{\pgfqpoint{2.881899in}{2.027954in}}%
\pgfpathlineto{\pgfqpoint{2.896715in}{1.703486in}}%
\pgfpathlineto{\pgfqpoint{2.911531in}{1.379018in}}%
\pgfpathlineto{\pgfqpoint{2.926347in}{1.086311in}}%
\pgfpathlineto{\pgfqpoint{2.941163in}{0.854018in}}%
\pgfpathlineto{\pgfqpoint{2.955980in}{0.704877in}}%
\pgfpathlineto{\pgfqpoint{2.970796in}{0.653486in}}%
\pgfpathlineto{\pgfqpoint{2.985612in}{0.704877in}}%
\pgfpathlineto{\pgfqpoint{3.000428in}{0.854018in}}%
\pgfpathlineto{\pgfqpoint{3.015244in}{1.086311in}}%
\pgfpathlineto{\pgfqpoint{3.030060in}{1.379018in}}%
\pgfpathlineto{\pgfqpoint{3.044877in}{1.703486in}}%
\pgfpathlineto{\pgfqpoint{3.059693in}{2.027954in}}%
\pgfpathlineto{\pgfqpoint{3.074509in}{2.320660in}}%
\pgfpathlineto{\pgfqpoint{3.089325in}{2.552954in}}%
\pgfpathlineto{\pgfqpoint{3.104141in}{2.702095in}}%
\pgfpathlineto{\pgfqpoint{3.118958in}{2.753486in}}%
\pgfpathlineto{\pgfqpoint{3.133774in}{2.702095in}}%
\pgfpathlineto{\pgfqpoint{3.148590in}{2.552954in}}%
\pgfpathlineto{\pgfqpoint{3.163406in}{2.320660in}}%
\pgfpathlineto{\pgfqpoint{3.178222in}{2.027954in}}%
\pgfpathlineto{\pgfqpoint{3.193038in}{1.703486in}}%
\pgfpathlineto{\pgfqpoint{3.207855in}{1.379018in}}%
\pgfpathlineto{\pgfqpoint{3.222671in}{1.086311in}}%
\pgfpathlineto{\pgfqpoint{3.237487in}{0.854018in}}%
\pgfpathlineto{\pgfqpoint{3.252303in}{0.704877in}}%
\pgfpathlineto{\pgfqpoint{3.267119in}{0.653486in}}%
\pgfpathlineto{\pgfqpoint{3.281935in}{0.704877in}}%
\pgfpathlineto{\pgfqpoint{3.296752in}{0.854018in}}%
\pgfpathlineto{\pgfqpoint{3.311568in}{1.086311in}}%
\pgfpathlineto{\pgfqpoint{3.326384in}{1.379018in}}%
\pgfpathlineto{\pgfqpoint{3.341200in}{1.703486in}}%
\pgfpathlineto{\pgfqpoint{3.356016in}{2.027954in}}%
\pgfpathlineto{\pgfqpoint{3.370833in}{2.320660in}}%
\pgfpathlineto{\pgfqpoint{3.385649in}{2.552954in}}%
\pgfpathlineto{\pgfqpoint{3.400465in}{2.702095in}}%
\pgfpathlineto{\pgfqpoint{3.415281in}{2.753486in}}%
\pgfpathlineto{\pgfqpoint{3.430097in}{2.702095in}}%
\pgfpathlineto{\pgfqpoint{3.444913in}{2.552954in}}%
\pgfpathlineto{\pgfqpoint{3.459730in}{2.320660in}}%
\pgfpathlineto{\pgfqpoint{3.474546in}{2.027954in}}%
\pgfpathlineto{\pgfqpoint{3.489362in}{1.703486in}}%
\pgfpathlineto{\pgfqpoint{3.504178in}{1.379018in}}%
\pgfpathlineto{\pgfqpoint{3.518994in}{1.086311in}}%
\pgfpathlineto{\pgfqpoint{3.533810in}{0.854018in}}%
\pgfpathlineto{\pgfqpoint{3.548627in}{0.704877in}}%
\pgfpathlineto{\pgfqpoint{3.563443in}{0.653486in}}%
\pgfpathlineto{\pgfqpoint{3.578259in}{0.704877in}}%
\pgfpathlineto{\pgfqpoint{3.593075in}{0.854018in}}%
\pgfpathlineto{\pgfqpoint{3.607891in}{1.086311in}}%
\pgfpathlineto{\pgfqpoint{3.622708in}{1.379018in}}%
\pgfpathlineto{\pgfqpoint{3.637524in}{1.703486in}}%
\pgfpathlineto{\pgfqpoint{3.652340in}{2.027954in}}%
\pgfpathlineto{\pgfqpoint{3.667156in}{2.320660in}}%
\pgfpathlineto{\pgfqpoint{3.681972in}{2.552954in}}%
\pgfpathlineto{\pgfqpoint{3.696788in}{2.702095in}}%
\pgfpathlineto{\pgfqpoint{3.711605in}{2.753486in}}%
\pgfpathlineto{\pgfqpoint{3.726421in}{2.702095in}}%
\pgfpathlineto{\pgfqpoint{3.741237in}{2.552954in}}%
\pgfpathlineto{\pgfqpoint{3.756053in}{2.320660in}}%
\pgfpathlineto{\pgfqpoint{3.770869in}{2.027954in}}%
\pgfpathlineto{\pgfqpoint{3.785685in}{1.703486in}}%
\pgfpathlineto{\pgfqpoint{3.795685in}{1.484490in}}%
\pgfusepath{stroke}%
\end{pgfscope}%
\begin{pgfscope}%
\pgfpathrectangle{\pgfqpoint{2.304068in}{0.548486in}}{\pgfqpoint{1.481618in}{2.310000in}}%
\pgfusepath{clip}%
\pgfsetrectcap%
\pgfsetroundjoin%
\pgfsetlinewidth{0.501875pt}%
\definecolor{currentstroke}{rgb}{0.000000,0.000000,0.000000}%
\pgfsetstrokecolor{currentstroke}%
\pgfsetdash{}{0pt}%
\pgfpathmoveto{\pgfqpoint{2.304068in}{1.703486in}}%
\pgfpathlineto{\pgfqpoint{3.785685in}{1.703486in}}%
\pgfusepath{stroke}%
\end{pgfscope}%
\begin{pgfscope}%
\definecolor{textcolor}{rgb}{0.000000,0.000000,0.000000}%
\pgfsetstrokecolor{textcolor}%
\pgfsetfillcolor{textcolor}%
\pgftext[x=3.044877in,y=2.941819in,,base]{\color{textcolor}{\rmfamily\fontsize{12.000000}{14.400000}\selectfont\catcode`\^=\active\def^{\ifmmode\sp\else\^{}\fi}\catcode`\%=\active\def%{\%}$dot(a,b)=$-1500.00}}%
\end{pgfscope}%
\begin{pgfscope}%
\pgfsetbuttcap%
\pgfsetmiterjoin%
\definecolor{currentfill}{rgb}{1.000000,1.000000,1.000000}%
\pgfsetfillcolor{currentfill}%
\pgfsetfillopacity{0.800000}%
\pgfsetlinewidth{1.003750pt}%
\definecolor{currentstroke}{rgb}{0.800000,0.800000,0.800000}%
\pgfsetstrokecolor{currentstroke}%
\pgfsetstrokeopacity{0.800000}%
\pgfsetdash{}{0pt}%
\pgfpathmoveto{\pgfqpoint{2.401290in}{0.617930in}}%
\pgfpathlineto{\pgfqpoint{3.845535in}{0.617930in}}%
\pgfpathquadraticcurveto{\pgfqpoint{3.873313in}{0.617930in}}{\pgfqpoint{3.873313in}{0.645708in}}%
\pgfpathlineto{\pgfqpoint{3.873313in}{1.051199in}}%
\pgfpathquadraticcurveto{\pgfqpoint{3.873313in}{1.078977in}}{\pgfqpoint{3.845535in}{1.078977in}}%
\pgfpathlineto{\pgfqpoint{2.401290in}{1.078977in}}%
\pgfpathquadraticcurveto{\pgfqpoint{2.373512in}{1.078977in}}{\pgfqpoint{2.373512in}{1.051199in}}%
\pgfpathlineto{\pgfqpoint{2.373512in}{0.645708in}}%
\pgfpathquadraticcurveto{\pgfqpoint{2.373512in}{0.617930in}}{\pgfqpoint{2.401290in}{0.617930in}}%
\pgfpathlineto{\pgfqpoint{2.401290in}{0.617930in}}%
\pgfpathclose%
\pgfusepath{stroke,fill}%
\end{pgfscope}%
\begin{pgfscope}%
\pgfsetrectcap%
\pgfsetroundjoin%
\pgfsetlinewidth{1.505625pt}%
\definecolor{currentstroke}{rgb}{0.000000,0.000000,0.000000}%
\pgfsetstrokecolor{currentstroke}%
\pgfsetdash{}{0pt}%
\pgfpathmoveto{\pgfqpoint{2.429068in}{0.966509in}}%
\pgfpathlineto{\pgfqpoint{2.567957in}{0.966509in}}%
\pgfpathlineto{\pgfqpoint{2.706846in}{0.966509in}}%
\pgfusepath{stroke}%
\end{pgfscope}%
\begin{pgfscope}%
\definecolor{textcolor}{rgb}{0.000000,0.000000,0.000000}%
\pgfsetstrokecolor{textcolor}%
\pgfsetfillcolor{textcolor}%
\pgftext[x=2.817957in,y=0.917898in,left,base]{\color{textcolor}{\rmfamily\fontsize{10.000000}{12.000000}\selectfont\catcode`\^=\active\def^{\ifmmode\sp\else\^{}\fi}\catcode`\%=\active\def%{\%}$a = sin(2\pi 50t)$}}%
\end{pgfscope}%
\begin{pgfscope}%
\pgfsetbuttcap%
\pgfsetroundjoin%
\pgfsetlinewidth{1.505625pt}%
\definecolor{currentstroke}{rgb}{1.000000,0.000000,0.000000}%
\pgfsetstrokecolor{currentstroke}%
\pgfsetdash{{5.550000pt}{2.400000pt}}{0.000000pt}%
\pgfpathmoveto{\pgfqpoint{2.429068in}{0.756819in}}%
\pgfpathlineto{\pgfqpoint{2.567957in}{0.756819in}}%
\pgfpathlineto{\pgfqpoint{2.706846in}{0.756819in}}%
\pgfusepath{stroke}%
\end{pgfscope}%
\begin{pgfscope}%
\definecolor{textcolor}{rgb}{0.000000,0.000000,0.000000}%
\pgfsetstrokecolor{textcolor}%
\pgfsetfillcolor{textcolor}%
\pgftext[x=2.817957in,y=0.708208in,left,base]{\color{textcolor}{\rmfamily\fontsize{10.000000}{12.000000}\selectfont\catcode`\^=\active\def^{\ifmmode\sp\else\^{}\fi}\catcode`\%=\active\def%{\%}$b = -sin(2\pi 50t)$}}%
\end{pgfscope}%
\begin{pgfscope}%
\pgfsetbuttcap%
\pgfsetmiterjoin%
\pgfsetlinewidth{0.000000pt}%
\definecolor{currentstroke}{rgb}{0.000000,0.000000,0.000000}%
\pgfsetstrokecolor{currentstroke}%
\pgfsetstrokeopacity{0.000000}%
\pgfsetdash{}{0pt}%
\pgfpathmoveto{\pgfqpoint{4.082009in}{0.548486in}}%
\pgfpathlineto{\pgfqpoint{5.563627in}{0.548486in}}%
\pgfpathlineto{\pgfqpoint{5.563627in}{2.858486in}}%
\pgfpathlineto{\pgfqpoint{4.082009in}{2.858486in}}%
\pgfpathlineto{\pgfqpoint{4.082009in}{0.548486in}}%
\pgfpathclose%
\pgfusepath{}%
\end{pgfscope}%
\begin{pgfscope}%
\pgfsetbuttcap%
\pgfsetroundjoin%
\definecolor{currentfill}{rgb}{0.000000,0.000000,0.000000}%
\pgfsetfillcolor{currentfill}%
\pgfsetlinewidth{0.803000pt}%
\definecolor{currentstroke}{rgb}{0.000000,0.000000,0.000000}%
\pgfsetstrokecolor{currentstroke}%
\pgfsetdash{}{0pt}%
\pgfsys@defobject{currentmarker}{\pgfqpoint{0.000000in}{-0.048611in}}{\pgfqpoint{0.000000in}{0.000000in}}{%
\pgfpathmoveto{\pgfqpoint{0.000000in}{0.000000in}}%
\pgfpathlineto{\pgfqpoint{0.000000in}{-0.048611in}}%
\pgfusepath{stroke,fill}%
}%
\begin{pgfscope}%
\pgfsys@transformshift{4.082009in}{0.548486in}%
\pgfsys@useobject{currentmarker}{}%
\end{pgfscope}%
\end{pgfscope}%
\begin{pgfscope}%
\definecolor{textcolor}{rgb}{0.000000,0.000000,0.000000}%
\pgfsetstrokecolor{textcolor}%
\pgfsetfillcolor{textcolor}%
\pgftext[x=4.082009in,y=0.451264in,,top]{\color{textcolor}{\rmfamily\fontsize{10.000000}{12.000000}\selectfont\catcode`\^=\active\def^{\ifmmode\sp\else\^{}\fi}\catcode`\%=\active\def%{\%}0.00}}%
\end{pgfscope}%
\begin{pgfscope}%
\pgfsetbuttcap%
\pgfsetroundjoin%
\definecolor{currentfill}{rgb}{0.000000,0.000000,0.000000}%
\pgfsetfillcolor{currentfill}%
\pgfsetlinewidth{0.803000pt}%
\definecolor{currentstroke}{rgb}{0.000000,0.000000,0.000000}%
\pgfsetstrokecolor{currentstroke}%
\pgfsetdash{}{0pt}%
\pgfsys@defobject{currentmarker}{\pgfqpoint{0.000000in}{-0.048611in}}{\pgfqpoint{0.000000in}{0.000000in}}{%
\pgfpathmoveto{\pgfqpoint{0.000000in}{0.000000in}}%
\pgfpathlineto{\pgfqpoint{0.000000in}{-0.048611in}}%
\pgfusepath{stroke,fill}%
}%
\begin{pgfscope}%
\pgfsys@transformshift{4.822818in}{0.548486in}%
\pgfsys@useobject{currentmarker}{}%
\end{pgfscope}%
\end{pgfscope}%
\begin{pgfscope}%
\definecolor{textcolor}{rgb}{0.000000,0.000000,0.000000}%
\pgfsetstrokecolor{textcolor}%
\pgfsetfillcolor{textcolor}%
\pgftext[x=4.822818in,y=0.451264in,,top]{\color{textcolor}{\rmfamily\fontsize{10.000000}{12.000000}\selectfont\catcode`\^=\active\def^{\ifmmode\sp\else\^{}\fi}\catcode`\%=\active\def%{\%}0.05}}%
\end{pgfscope}%
\begin{pgfscope}%
\pgfsetbuttcap%
\pgfsetroundjoin%
\definecolor{currentfill}{rgb}{0.000000,0.000000,0.000000}%
\pgfsetfillcolor{currentfill}%
\pgfsetlinewidth{0.803000pt}%
\definecolor{currentstroke}{rgb}{0.000000,0.000000,0.000000}%
\pgfsetstrokecolor{currentstroke}%
\pgfsetdash{}{0pt}%
\pgfsys@defobject{currentmarker}{\pgfqpoint{0.000000in}{-0.048611in}}{\pgfqpoint{0.000000in}{0.000000in}}{%
\pgfpathmoveto{\pgfqpoint{0.000000in}{0.000000in}}%
\pgfpathlineto{\pgfqpoint{0.000000in}{-0.048611in}}%
\pgfusepath{stroke,fill}%
}%
\begin{pgfscope}%
\pgfsys@transformshift{5.563627in}{0.548486in}%
\pgfsys@useobject{currentmarker}{}%
\end{pgfscope}%
\end{pgfscope}%
\begin{pgfscope}%
\definecolor{textcolor}{rgb}{0.000000,0.000000,0.000000}%
\pgfsetstrokecolor{textcolor}%
\pgfsetfillcolor{textcolor}%
\pgftext[x=5.563627in,y=0.451264in,,top]{\color{textcolor}{\rmfamily\fontsize{10.000000}{12.000000}\selectfont\catcode`\^=\active\def^{\ifmmode\sp\else\^{}\fi}\catcode`\%=\active\def%{\%}0.10}}%
\end{pgfscope}%
\begin{pgfscope}%
\definecolor{textcolor}{rgb}{0.000000,0.000000,0.000000}%
\pgfsetstrokecolor{textcolor}%
\pgfsetfillcolor{textcolor}%
\pgftext[x=4.822818in,y=0.261295in,,top]{\color{textcolor}{\rmfamily\fontsize{12.000000}{14.400000}\selectfont\catcode`\^=\active\def^{\ifmmode\sp\else\^{}\fi}\catcode`\%=\active\def%{\%}$t$}}%
\end{pgfscope}%
\begin{pgfscope}%
\pgfpathrectangle{\pgfqpoint{4.082009in}{0.548486in}}{\pgfqpoint{1.481618in}{2.310000in}}%
\pgfusepath{clip}%
\pgfsetrectcap%
\pgfsetroundjoin%
\pgfsetlinewidth{1.505625pt}%
\definecolor{currentstroke}{rgb}{0.000000,0.000000,0.000000}%
\pgfsetstrokecolor{currentstroke}%
\pgfsetdash{}{0pt}%
\pgfpathmoveto{\pgfqpoint{4.082009in}{1.703486in}}%
\pgfpathlineto{\pgfqpoint{4.096825in}{2.027954in}}%
\pgfpathlineto{\pgfqpoint{4.111641in}{2.320660in}}%
\pgfpathlineto{\pgfqpoint{4.126458in}{2.552954in}}%
\pgfpathlineto{\pgfqpoint{4.141274in}{2.702095in}}%
\pgfpathlineto{\pgfqpoint{4.156090in}{2.753486in}}%
\pgfpathlineto{\pgfqpoint{4.170906in}{2.702095in}}%
\pgfpathlineto{\pgfqpoint{4.185722in}{2.552954in}}%
\pgfpathlineto{\pgfqpoint{4.200538in}{2.320660in}}%
\pgfpathlineto{\pgfqpoint{4.215355in}{2.027954in}}%
\pgfpathlineto{\pgfqpoint{4.230171in}{1.703486in}}%
\pgfpathlineto{\pgfqpoint{4.244987in}{1.379018in}}%
\pgfpathlineto{\pgfqpoint{4.259803in}{1.086311in}}%
\pgfpathlineto{\pgfqpoint{4.274619in}{0.854018in}}%
\pgfpathlineto{\pgfqpoint{4.289435in}{0.704877in}}%
\pgfpathlineto{\pgfqpoint{4.304252in}{0.653486in}}%
\pgfpathlineto{\pgfqpoint{4.319068in}{0.704877in}}%
\pgfpathlineto{\pgfqpoint{4.333884in}{0.854018in}}%
\pgfpathlineto{\pgfqpoint{4.348700in}{1.086311in}}%
\pgfpathlineto{\pgfqpoint{4.363516in}{1.379018in}}%
\pgfpathlineto{\pgfqpoint{4.378333in}{1.703486in}}%
\pgfpathlineto{\pgfqpoint{4.393149in}{2.027954in}}%
\pgfpathlineto{\pgfqpoint{4.407965in}{2.320660in}}%
\pgfpathlineto{\pgfqpoint{4.422781in}{2.552954in}}%
\pgfpathlineto{\pgfqpoint{4.437597in}{2.702095in}}%
\pgfpathlineto{\pgfqpoint{4.452413in}{2.753486in}}%
\pgfpathlineto{\pgfqpoint{4.467230in}{2.702095in}}%
\pgfpathlineto{\pgfqpoint{4.482046in}{2.552954in}}%
\pgfpathlineto{\pgfqpoint{4.496862in}{2.320660in}}%
\pgfpathlineto{\pgfqpoint{4.511678in}{2.027954in}}%
\pgfpathlineto{\pgfqpoint{4.526494in}{1.703486in}}%
\pgfpathlineto{\pgfqpoint{4.541310in}{1.379018in}}%
\pgfpathlineto{\pgfqpoint{4.556127in}{1.086311in}}%
\pgfpathlineto{\pgfqpoint{4.570943in}{0.854018in}}%
\pgfpathlineto{\pgfqpoint{4.585759in}{0.704877in}}%
\pgfpathlineto{\pgfqpoint{4.600575in}{0.653486in}}%
\pgfpathlineto{\pgfqpoint{4.615391in}{0.704877in}}%
\pgfpathlineto{\pgfqpoint{4.630208in}{0.854018in}}%
\pgfpathlineto{\pgfqpoint{4.645024in}{1.086311in}}%
\pgfpathlineto{\pgfqpoint{4.659840in}{1.379018in}}%
\pgfpathlineto{\pgfqpoint{4.674656in}{1.703486in}}%
\pgfpathlineto{\pgfqpoint{4.689472in}{2.027954in}}%
\pgfpathlineto{\pgfqpoint{4.704288in}{2.320660in}}%
\pgfpathlineto{\pgfqpoint{4.719105in}{2.552954in}}%
\pgfpathlineto{\pgfqpoint{4.733921in}{2.702095in}}%
\pgfpathlineto{\pgfqpoint{4.748737in}{2.753486in}}%
\pgfpathlineto{\pgfqpoint{4.763553in}{2.702095in}}%
\pgfpathlineto{\pgfqpoint{4.778369in}{2.552954in}}%
\pgfpathlineto{\pgfqpoint{4.793185in}{2.320660in}}%
\pgfpathlineto{\pgfqpoint{4.808002in}{2.027954in}}%
\pgfpathlineto{\pgfqpoint{4.822818in}{1.703486in}}%
\pgfpathlineto{\pgfqpoint{4.837634in}{1.379018in}}%
\pgfpathlineto{\pgfqpoint{4.852450in}{1.086311in}}%
\pgfpathlineto{\pgfqpoint{4.867266in}{0.854018in}}%
\pgfpathlineto{\pgfqpoint{4.882083in}{0.704877in}}%
\pgfpathlineto{\pgfqpoint{4.896899in}{0.653486in}}%
\pgfpathlineto{\pgfqpoint{4.911715in}{0.704877in}}%
\pgfpathlineto{\pgfqpoint{4.926531in}{0.854018in}}%
\pgfpathlineto{\pgfqpoint{4.941347in}{1.086311in}}%
\pgfpathlineto{\pgfqpoint{4.956163in}{1.379018in}}%
\pgfpathlineto{\pgfqpoint{4.970980in}{1.703486in}}%
\pgfpathlineto{\pgfqpoint{4.985796in}{2.027954in}}%
\pgfpathlineto{\pgfqpoint{5.000612in}{2.320660in}}%
\pgfpathlineto{\pgfqpoint{5.015428in}{2.552954in}}%
\pgfpathlineto{\pgfqpoint{5.030244in}{2.702095in}}%
\pgfpathlineto{\pgfqpoint{5.045060in}{2.753486in}}%
\pgfpathlineto{\pgfqpoint{5.059877in}{2.702095in}}%
\pgfpathlineto{\pgfqpoint{5.074693in}{2.552954in}}%
\pgfpathlineto{\pgfqpoint{5.089509in}{2.320660in}}%
\pgfpathlineto{\pgfqpoint{5.104325in}{2.027954in}}%
\pgfpathlineto{\pgfqpoint{5.119141in}{1.703486in}}%
\pgfpathlineto{\pgfqpoint{5.133958in}{1.379018in}}%
\pgfpathlineto{\pgfqpoint{5.148774in}{1.086311in}}%
\pgfpathlineto{\pgfqpoint{5.163590in}{0.854018in}}%
\pgfpathlineto{\pgfqpoint{5.178406in}{0.704877in}}%
\pgfpathlineto{\pgfqpoint{5.193222in}{0.653486in}}%
\pgfpathlineto{\pgfqpoint{5.208038in}{0.704877in}}%
\pgfpathlineto{\pgfqpoint{5.222855in}{0.854018in}}%
\pgfpathlineto{\pgfqpoint{5.237671in}{1.086311in}}%
\pgfpathlineto{\pgfqpoint{5.252487in}{1.379018in}}%
\pgfpathlineto{\pgfqpoint{5.267303in}{1.703486in}}%
\pgfpathlineto{\pgfqpoint{5.282119in}{2.027954in}}%
\pgfpathlineto{\pgfqpoint{5.296935in}{2.320660in}}%
\pgfpathlineto{\pgfqpoint{5.311752in}{2.552954in}}%
\pgfpathlineto{\pgfqpoint{5.326568in}{2.702095in}}%
\pgfpathlineto{\pgfqpoint{5.341384in}{2.753486in}}%
\pgfpathlineto{\pgfqpoint{5.356200in}{2.702095in}}%
\pgfpathlineto{\pgfqpoint{5.371016in}{2.552954in}}%
\pgfpathlineto{\pgfqpoint{5.385833in}{2.320660in}}%
\pgfpathlineto{\pgfqpoint{5.400649in}{2.027954in}}%
\pgfpathlineto{\pgfqpoint{5.415465in}{1.703486in}}%
\pgfpathlineto{\pgfqpoint{5.430281in}{1.379018in}}%
\pgfpathlineto{\pgfqpoint{5.445097in}{1.086311in}}%
\pgfpathlineto{\pgfqpoint{5.459913in}{0.854018in}}%
\pgfpathlineto{\pgfqpoint{5.474730in}{0.704877in}}%
\pgfpathlineto{\pgfqpoint{5.489546in}{0.653486in}}%
\pgfpathlineto{\pgfqpoint{5.504362in}{0.704877in}}%
\pgfpathlineto{\pgfqpoint{5.519178in}{0.854018in}}%
\pgfpathlineto{\pgfqpoint{5.533994in}{1.086311in}}%
\pgfpathlineto{\pgfqpoint{5.548810in}{1.379018in}}%
\pgfpathlineto{\pgfqpoint{5.563627in}{1.703486in}}%
\pgfpathlineto{\pgfqpoint{5.573627in}{1.922482in}}%
\pgfusepath{stroke}%
\end{pgfscope}%
\begin{pgfscope}%
\pgfpathrectangle{\pgfqpoint{4.082009in}{0.548486in}}{\pgfqpoint{1.481618in}{2.310000in}}%
\pgfusepath{clip}%
\pgfsetbuttcap%
\pgfsetroundjoin%
\pgfsetlinewidth{1.505625pt}%
\definecolor{currentstroke}{rgb}{1.000000,0.000000,0.000000}%
\pgfsetstrokecolor{currentstroke}%
\pgfsetdash{{5.550000pt}{2.400000pt}}{0.000000pt}%
\pgfpathmoveto{\pgfqpoint{4.082009in}{2.175804in}}%
\pgfpathlineto{\pgfqpoint{4.096825in}{1.943043in}}%
\pgfpathlineto{\pgfqpoint{4.111641in}{1.656362in}}%
\pgfpathlineto{\pgfqpoint{4.126458in}{2.551737in}}%
\pgfpathlineto{\pgfqpoint{4.141274in}{2.294115in}}%
\pgfpathlineto{\pgfqpoint{4.156090in}{2.004471in}}%
\pgfpathlineto{\pgfqpoint{4.170906in}{1.996060in}}%
\pgfpathlineto{\pgfqpoint{4.185722in}{0.881261in}}%
\pgfpathlineto{\pgfqpoint{4.200538in}{2.428550in}}%
\pgfpathlineto{\pgfqpoint{4.215355in}{2.194813in}}%
\pgfpathlineto{\pgfqpoint{4.230171in}{1.568995in}}%
\pgfpathlineto{\pgfqpoint{4.244987in}{1.207698in}}%
\pgfpathlineto{\pgfqpoint{4.259803in}{1.875577in}}%
\pgfpathlineto{\pgfqpoint{4.274619in}{1.228183in}}%
\pgfpathlineto{\pgfqpoint{4.289435in}{2.319805in}}%
\pgfpathlineto{\pgfqpoint{4.304252in}{0.949353in}}%
\pgfpathlineto{\pgfqpoint{4.319068in}{2.455759in}}%
\pgfpathlineto{\pgfqpoint{4.333884in}{1.749518in}}%
\pgfpathlineto{\pgfqpoint{4.348700in}{0.862720in}}%
\pgfpathlineto{\pgfqpoint{4.363516in}{2.339680in}}%
\pgfpathlineto{\pgfqpoint{4.378333in}{2.725955in}}%
\pgfpathlineto{\pgfqpoint{4.393149in}{2.296532in}}%
\pgfpathlineto{\pgfqpoint{4.407965in}{2.464601in}}%
\pgfpathlineto{\pgfqpoint{4.422781in}{0.932512in}}%
\pgfpathlineto{\pgfqpoint{4.437597in}{2.375815in}}%
\pgfpathlineto{\pgfqpoint{4.452413in}{1.564146in}}%
\pgfpathlineto{\pgfqpoint{4.467230in}{1.547566in}}%
\pgfpathlineto{\pgfqpoint{4.482046in}{2.520525in}}%
\pgfpathlineto{\pgfqpoint{4.496862in}{1.481225in}}%
\pgfpathlineto{\pgfqpoint{4.511678in}{2.006455in}}%
\pgfpathlineto{\pgfqpoint{4.526494in}{1.132976in}}%
\pgfpathlineto{\pgfqpoint{4.541310in}{1.159747in}}%
\pgfpathlineto{\pgfqpoint{4.556127in}{2.388398in}}%
\pgfpathlineto{\pgfqpoint{4.570943in}{1.487798in}}%
\pgfpathlineto{\pgfqpoint{4.585759in}{0.749433in}}%
\pgfpathlineto{\pgfqpoint{4.600575in}{2.418280in}}%
\pgfpathlineto{\pgfqpoint{4.615391in}{1.129325in}}%
\pgfpathlineto{\pgfqpoint{4.630208in}{1.624174in}}%
\pgfpathlineto{\pgfqpoint{4.645024in}{1.661673in}}%
\pgfpathlineto{\pgfqpoint{4.659840in}{0.660912in}}%
\pgfpathlineto{\pgfqpoint{4.674656in}{0.884800in}}%
\pgfpathlineto{\pgfqpoint{4.689472in}{2.201009in}}%
\pgfpathlineto{\pgfqpoint{4.704288in}{2.090529in}}%
\pgfpathlineto{\pgfqpoint{4.719105in}{1.752255in}}%
\pgfpathlineto{\pgfqpoint{4.733921in}{2.380282in}}%
\pgfpathlineto{\pgfqpoint{4.748737in}{2.161480in}}%
\pgfpathlineto{\pgfqpoint{4.763553in}{1.723729in}}%
\pgfpathlineto{\pgfqpoint{4.778369in}{0.812379in}}%
\pgfpathlineto{\pgfqpoint{4.793185in}{2.003746in}}%
\pgfpathlineto{\pgfqpoint{4.808002in}{2.279426in}}%
\pgfpathlineto{\pgfqpoint{4.822818in}{1.259767in}}%
\pgfpathlineto{\pgfqpoint{4.837634in}{1.967784in}}%
\pgfpathlineto{\pgfqpoint{4.852450in}{1.736838in}}%
\pgfpathlineto{\pgfqpoint{4.867266in}{1.420622in}}%
\pgfpathlineto{\pgfqpoint{4.882083in}{1.864175in}}%
\pgfpathlineto{\pgfqpoint{4.896899in}{0.827153in}}%
\pgfpathlineto{\pgfqpoint{4.911715in}{1.930633in}}%
\pgfpathlineto{\pgfqpoint{4.926531in}{1.793208in}}%
\pgfpathlineto{\pgfqpoint{4.941347in}{1.417038in}}%
\pgfpathlineto{\pgfqpoint{4.956163in}{1.314542in}}%
\pgfpathlineto{\pgfqpoint{4.970980in}{1.318315in}}%
\pgfpathlineto{\pgfqpoint{4.985796in}{1.794760in}}%
\pgfpathlineto{\pgfqpoint{5.000612in}{0.668899in}}%
\pgfpathlineto{\pgfqpoint{5.015428in}{2.582122in}}%
\pgfpathlineto{\pgfqpoint{5.030244in}{1.660055in}}%
\pgfpathlineto{\pgfqpoint{5.045060in}{0.825498in}}%
\pgfpathlineto{\pgfqpoint{5.059877in}{1.033115in}}%
\pgfpathlineto{\pgfqpoint{5.074693in}{1.750025in}}%
\pgfpathlineto{\pgfqpoint{5.089509in}{0.922014in}}%
\pgfpathlineto{\pgfqpoint{5.104325in}{2.480590in}}%
\pgfpathlineto{\pgfqpoint{5.119141in}{2.547280in}}%
\pgfpathlineto{\pgfqpoint{5.133958in}{2.175075in}}%
\pgfpathlineto{\pgfqpoint{5.148774in}{1.532640in}}%
\pgfpathlineto{\pgfqpoint{5.163590in}{2.443388in}}%
\pgfpathlineto{\pgfqpoint{5.178406in}{1.585486in}}%
\pgfpathlineto{\pgfqpoint{5.193222in}{0.955123in}}%
\pgfpathlineto{\pgfqpoint{5.208038in}{2.348817in}}%
\pgfpathlineto{\pgfqpoint{5.222855in}{0.715579in}}%
\pgfpathlineto{\pgfqpoint{5.237671in}{2.355849in}}%
\pgfpathlineto{\pgfqpoint{5.252487in}{1.984613in}}%
\pgfpathlineto{\pgfqpoint{5.267303in}{0.759523in}}%
\pgfpathlineto{\pgfqpoint{5.282119in}{0.987513in}}%
\pgfpathlineto{\pgfqpoint{5.296935in}{1.255975in}}%
\pgfpathlineto{\pgfqpoint{5.311752in}{2.027299in}}%
\pgfpathlineto{\pgfqpoint{5.326568in}{0.805824in}}%
\pgfpathlineto{\pgfqpoint{5.341384in}{1.054452in}}%
\pgfpathlineto{\pgfqpoint{5.356200in}{2.105663in}}%
\pgfpathlineto{\pgfqpoint{5.371016in}{1.155631in}}%
\pgfpathlineto{\pgfqpoint{5.385833in}{1.382818in}}%
\pgfpathlineto{\pgfqpoint{5.400649in}{0.716582in}}%
\pgfpathlineto{\pgfqpoint{5.415465in}{2.457784in}}%
\pgfpathlineto{\pgfqpoint{5.430281in}{1.295517in}}%
\pgfpathlineto{\pgfqpoint{5.445097in}{0.858802in}}%
\pgfpathlineto{\pgfqpoint{5.459913in}{1.356959in}}%
\pgfpathlineto{\pgfqpoint{5.474730in}{2.415314in}}%
\pgfpathlineto{\pgfqpoint{5.489546in}{1.026476in}}%
\pgfpathlineto{\pgfqpoint{5.504362in}{1.916624in}}%
\pgfpathlineto{\pgfqpoint{5.519178in}{1.963878in}}%
\pgfpathlineto{\pgfqpoint{5.533994in}{0.967383in}}%
\pgfpathlineto{\pgfqpoint{5.548810in}{0.688011in}}%
\pgfpathlineto{\pgfqpoint{5.563627in}{0.697581in}}%
\pgfpathlineto{\pgfqpoint{5.573627in}{1.277740in}}%
\pgfusepath{stroke}%
\end{pgfscope}%
\begin{pgfscope}%
\pgfpathrectangle{\pgfqpoint{4.082009in}{0.548486in}}{\pgfqpoint{1.481618in}{2.310000in}}%
\pgfusepath{clip}%
\pgfsetrectcap%
\pgfsetroundjoin%
\pgfsetlinewidth{0.501875pt}%
\definecolor{currentstroke}{rgb}{0.000000,0.000000,0.000000}%
\pgfsetstrokecolor{currentstroke}%
\pgfsetdash{}{0pt}%
\pgfpathmoveto{\pgfqpoint{4.082009in}{1.703486in}}%
\pgfpathlineto{\pgfqpoint{5.563627in}{1.703486in}}%
\pgfusepath{stroke}%
\end{pgfscope}%
\begin{pgfscope}%
\definecolor{textcolor}{rgb}{0.000000,0.000000,0.000000}%
\pgfsetstrokecolor{textcolor}%
\pgfsetfillcolor{textcolor}%
\pgftext[x=4.822818in,y=2.941819in,,base]{\color{textcolor}{\rmfamily\fontsize{12.000000}{14.400000}\selectfont\catcode`\^=\active\def^{\ifmmode\sp\else\^{}\fi}\catcode`\%=\active\def%{\%}$dot(a,b)=$2.01}}%
\end{pgfscope}%
\begin{pgfscope}%
\pgfsetbuttcap%
\pgfsetmiterjoin%
\definecolor{currentfill}{rgb}{1.000000,1.000000,1.000000}%
\pgfsetfillcolor{currentfill}%
\pgfsetfillopacity{0.800000}%
\pgfsetlinewidth{1.003750pt}%
\definecolor{currentstroke}{rgb}{0.800000,0.800000,0.800000}%
\pgfsetstrokecolor{currentstroke}%
\pgfsetstrokeopacity{0.800000}%
\pgfsetdash{}{0pt}%
\pgfpathmoveto{\pgfqpoint{4.179231in}{0.617930in}}%
\pgfpathlineto{\pgfqpoint{5.529260in}{0.617930in}}%
\pgfpathquadraticcurveto{\pgfqpoint{5.557038in}{0.617930in}}{\pgfqpoint{5.557038in}{0.645708in}}%
\pgfpathlineto{\pgfqpoint{5.557038in}{1.045366in}}%
\pgfpathquadraticcurveto{\pgfqpoint{5.557038in}{1.073144in}}{\pgfqpoint{5.529260in}{1.073144in}}%
\pgfpathlineto{\pgfqpoint{4.179231in}{1.073144in}}%
\pgfpathquadraticcurveto{\pgfqpoint{4.151453in}{1.073144in}}{\pgfqpoint{4.151453in}{1.045366in}}%
\pgfpathlineto{\pgfqpoint{4.151453in}{0.645708in}}%
\pgfpathquadraticcurveto{\pgfqpoint{4.151453in}{0.617930in}}{\pgfqpoint{4.179231in}{0.617930in}}%
\pgfpathlineto{\pgfqpoint{4.179231in}{0.617930in}}%
\pgfpathclose%
\pgfusepath{stroke,fill}%
\end{pgfscope}%
\begin{pgfscope}%
\pgfsetrectcap%
\pgfsetroundjoin%
\pgfsetlinewidth{1.505625pt}%
\definecolor{currentstroke}{rgb}{0.000000,0.000000,0.000000}%
\pgfsetstrokecolor{currentstroke}%
\pgfsetdash{}{0pt}%
\pgfpathmoveto{\pgfqpoint{4.207009in}{0.960677in}}%
\pgfpathlineto{\pgfqpoint{4.345898in}{0.960677in}}%
\pgfpathlineto{\pgfqpoint{4.484787in}{0.960677in}}%
\pgfusepath{stroke}%
\end{pgfscope}%
\begin{pgfscope}%
\definecolor{textcolor}{rgb}{0.000000,0.000000,0.000000}%
\pgfsetstrokecolor{textcolor}%
\pgfsetfillcolor{textcolor}%
\pgftext[x=4.595898in,y=0.912065in,left,base]{\color{textcolor}{\rmfamily\fontsize{10.000000}{12.000000}\selectfont\catcode`\^=\active\def^{\ifmmode\sp\else\^{}\fi}\catcode`\%=\active\def%{\%}$a = sin(2\pi 50t)$}}%
\end{pgfscope}%
\begin{pgfscope}%
\pgfsetbuttcap%
\pgfsetroundjoin%
\pgfsetlinewidth{1.505625pt}%
\definecolor{currentstroke}{rgb}{1.000000,0.000000,0.000000}%
\pgfsetstrokecolor{currentstroke}%
\pgfsetdash{{5.550000pt}{2.400000pt}}{0.000000pt}%
\pgfpathmoveto{\pgfqpoint{4.207009in}{0.750987in}}%
\pgfpathlineto{\pgfqpoint{4.345898in}{0.750987in}}%
\pgfpathlineto{\pgfqpoint{4.484787in}{0.750987in}}%
\pgfusepath{stroke}%
\end{pgfscope}%
\begin{pgfscope}%
\definecolor{textcolor}{rgb}{0.000000,0.000000,0.000000}%
\pgfsetstrokecolor{textcolor}%
\pgfsetfillcolor{textcolor}%
\pgftext[x=4.595898in,y=0.702376in,left,base]{\color{textcolor}{\rmfamily\fontsize{10.000000}{12.000000}\selectfont\catcode`\^=\active\def^{\ifmmode\sp\else\^{}\fi}\catcode`\%=\active\def%{\%}$b =$ noise}}%
\end{pgfscope}%
\end{pgfpicture}%
\makeatother%
\endgroup%

    \caption{Das Skalare Produkt unterschiedlicher Signale.}
    \label{fig:dotProdCorr}
\end{figure}


}




\subsection*{Kreuzprodukt}
\index{Kreuzprodukt}
Auch 'Vektorprodukt'. 
$$ \vec{c} = \vec{a} \times \vec{b}$$
Nur in $\mathbb{R}^3$ definiert!

Python: \pythoninline{cross(a,b)}

\section{Matrizen}



Eine Matrix ist eine rechteckige Struktur von Zahlen:


\begin{equation}
\mathbf{A} = \begin{bmatrix} a_{11} & a_{12} & \dots & a_{1m} \\
                    a_{21} & a_{22} &       &  \vdots   \\
                    \vdots &        &\ddots &  \vdots \\
                    a_{n 1} & \dots & \dots & a_{n m}
 \end{bmatrix}
\end{equation}



\praxis{
Angenommen wir wollen $4$ Audio Signale, zum Beispiel 4 verschiedene Instrumente auf 6 verschiedene Lautsprecher schicken, unterschiedlicher Lautstärke. Wie könnte man dies Mathematisch formulieren? Man könnte zum Beispiel sagen die 4 Audio Signale seien ein Vektor $\vec{x(t)}$, der die jeweiligen Signale enthält: $\vec{x}(t) = \begin{bmatrix} x_1(t) & x_2(t) & x_3(t) & x_4(t)\end{bmatrix}$. Unsere Output Signale die wir an die 6 Lautsprecher schicken nennen wir vielleicht $y_1(t), \dots, y_6(t)$. Auch diese machen wir zu einem Vektor $\vec{y}(t) = \begin{bmatrix} y_1(t) & y_2(t) & y_3(t) & y_4(t) & y_5(t) & y_6(t)\end{bmatrix}$. Nun können wir unsere Verräumlichung über eine Matrix definieren:

$$ \mathbf{A}\vec{x} = \vec{y}$$

Hier ist zu beachten dass es sich bei $\vec{x}$ um eine $4\times 1$ Matrix, also einen \textbf{Spaltenvektor} mit 4 Reihen handelt, und bei $\vec{y}$ ebenso um einen Spaltenvektor. Die Matrix $\mathbf{A}$ hat die Form $6 \times 4$. zB:

\[
\begin{bmatrix} 1 & 0 & 0 & 0 \\ 0 & 1 & 0 & 0 \\ 0 & 0 & 1 & 0 \\ 0 & 0 & 0 & 1 \\ 1 & 0 & 0 & 0 \\ 1 & 0 & 0 & 0 \end{bmatrix}  
\begin{bmatrix} x_1 \\ x_2 \\ x_3 \\ x_4  \end{bmatrix}  
=
\begin{bmatrix} y_1 \\ y_2 \\ y_3\\ y_4\\ y_5\\ y_6\end{bmatrix}
\]


In diesem Beispiel haben wir beschlossen Instrument $1$ auf Lautsprecher $1$ zu schicken, Instrument $2$ auf Lautsprecher $2$ etc, aber Instrument $1$ auch noch auf Lautsprecher 5 und 6 zu senden. Das heißt: $y_1 = x_1$, $y_2 = x_2$, $y_3 = x_3$, $y_4 = x_4$, $y_5 = x_1$, $y_6 = x_1$.
}




\subsection*{Ein Lineares Gleichungssystem (LGS)}

Gegeben folgendes Gleichungssystem:

\[
\begin{aligned}
2x + 3y &= 8 \\
4x -  y &= 2
\end{aligned}
\]

\subsubsection{Geometrische Interpretation: Zeileninterpretation}
Wollen wir eine Lösung für $x$ und $y$ finden. Aus der Schule kennen wir vielleicht eine Geometrische Interpretation des Problems. Wir können beide Gleichungen für $y$ lösen und bekommen zwei neue Gleichungen. Beide beschreiben eine Linie. Die Lösung besteht in deren Schnittpunkt:

\[
\begin{aligned}
y &= \frac{8}{3} - \frac{2 x}{3} \\
y &= 4 x - 2 
\end{aligned}
\]


\begin{figure}[H]
    \centering
    %% Creator: Matplotlib, PGF backend
%%
%% To include the figure in your LaTeX document, write
%%   \input{<filename>.pgf}
%%
%% Make sure the required packages are loaded in your preamble
%%   \usepackage{pgf}
%%
%% Also ensure that all the required font packages are loaded; for instance,
%% the lmodern package is sometimes necessary when using math font.
%%   \usepackage{lmodern}
%%
%% Figures using additional raster images can only be included by \input if
%% they are in the same directory as the main LaTeX file. For loading figures
%% from other directories you can use the `import` package
%%   \usepackage{import}
%%
%% and then include the figures with
%%   \import{<path to file>}{<filename>.pgf}
%%
%% Matplotlib used the following preamble
%%   \def\mathdefault#1{#1}
%%   \everymath=\expandafter{\the\everymath\displaystyle}
%%   
%%   \usepackage{fontspec}
%%   \setmainfont{VeraSe.ttf}[Path=\detokenize{/usr/share/fonts/TTF/}]
%%   \setsansfont{DejaVuSans.ttf}[Path=\detokenize{/home/pl/miniconda3/lib/python3.12/site-packages/matplotlib/mpl-data/fonts/ttf/}]
%%   \setmonofont{DejaVuSansMono.ttf}[Path=\detokenize{/home/pl/miniconda3/lib/python3.12/site-packages/matplotlib/mpl-data/fonts/ttf/}]
%%   \makeatletter\@ifpackageloaded{underscore}{}{\usepackage[strings]{underscore}}\makeatother
%%
\begingroup%
\makeatletter%
\begin{pgfpicture}%
\pgfpathrectangle{\pgfpointorigin}{\pgfqpoint{5.360463in}{3.011248in}}%
\pgfusepath{use as bounding box, clip}%
\begin{pgfscope}%
\pgfsetbuttcap%
\pgfsetmiterjoin%
\definecolor{currentfill}{rgb}{1.000000,1.000000,1.000000}%
\pgfsetfillcolor{currentfill}%
\pgfsetlinewidth{0.000000pt}%
\definecolor{currentstroke}{rgb}{1.000000,1.000000,1.000000}%
\pgfsetstrokecolor{currentstroke}%
\pgfsetdash{}{0pt}%
\pgfpathmoveto{\pgfqpoint{0.000000in}{0.000000in}}%
\pgfpathlineto{\pgfqpoint{5.360463in}{0.000000in}}%
\pgfpathlineto{\pgfqpoint{5.360463in}{3.011248in}}%
\pgfpathlineto{\pgfqpoint{0.000000in}{3.011248in}}%
\pgfpathlineto{\pgfqpoint{0.000000in}{0.000000in}}%
\pgfpathclose%
\pgfusepath{fill}%
\end{pgfscope}%
\begin{pgfscope}%
\pgfsetbuttcap%
\pgfsetmiterjoin%
\definecolor{currentfill}{rgb}{1.000000,1.000000,1.000000}%
\pgfsetfillcolor{currentfill}%
\pgfsetlinewidth{0.000000pt}%
\definecolor{currentstroke}{rgb}{0.000000,0.000000,0.000000}%
\pgfsetstrokecolor{currentstroke}%
\pgfsetstrokeopacity{0.000000}%
\pgfsetdash{}{0pt}%
\pgfpathmoveto{\pgfqpoint{0.610463in}{0.548486in}}%
\pgfpathlineto{\pgfqpoint{5.260463in}{0.548486in}}%
\pgfpathlineto{\pgfqpoint{5.260463in}{2.858486in}}%
\pgfpathlineto{\pgfqpoint{0.610463in}{2.858486in}}%
\pgfpathlineto{\pgfqpoint{0.610463in}{0.548486in}}%
\pgfpathclose%
\pgfusepath{fill}%
\end{pgfscope}%
\begin{pgfscope}%
\pgfpathrectangle{\pgfqpoint{0.610463in}{0.548486in}}{\pgfqpoint{4.650000in}{2.310000in}}%
\pgfusepath{clip}%
\pgfsetbuttcap%
\pgfsetroundjoin%
\definecolor{currentfill}{rgb}{1.000000,0.000000,0.000000}%
\pgfsetfillcolor{currentfill}%
\pgfsetlinewidth{1.003750pt}%
\definecolor{currentstroke}{rgb}{1.000000,0.000000,0.000000}%
\pgfsetstrokecolor{currentstroke}%
\pgfsetdash{}{0pt}%
\pgfsys@defobject{currentmarker}{\pgfqpoint{-0.041667in}{-0.041667in}}{\pgfqpoint{0.041667in}{0.041667in}}{%
\pgfpathmoveto{\pgfqpoint{0.000000in}{-0.041667in}}%
\pgfpathcurveto{\pgfqpoint{0.011050in}{-0.041667in}}{\pgfqpoint{0.021649in}{-0.037276in}}{\pgfqpoint{0.029463in}{-0.029463in}}%
\pgfpathcurveto{\pgfqpoint{0.037276in}{-0.021649in}}{\pgfqpoint{0.041667in}{-0.011050in}}{\pgfqpoint{0.041667in}{0.000000in}}%
\pgfpathcurveto{\pgfqpoint{0.041667in}{0.011050in}}{\pgfqpoint{0.037276in}{0.021649in}}{\pgfqpoint{0.029463in}{0.029463in}}%
\pgfpathcurveto{\pgfqpoint{0.021649in}{0.037276in}}{\pgfqpoint{0.011050in}{0.041667in}}{\pgfqpoint{0.000000in}{0.041667in}}%
\pgfpathcurveto{\pgfqpoint{-0.011050in}{0.041667in}}{\pgfqpoint{-0.021649in}{0.037276in}}{\pgfqpoint{-0.029463in}{0.029463in}}%
\pgfpathcurveto{\pgfqpoint{-0.037276in}{0.021649in}}{\pgfqpoint{-0.041667in}{0.011050in}}{\pgfqpoint{-0.041667in}{0.000000in}}%
\pgfpathcurveto{\pgfqpoint{-0.041667in}{-0.011050in}}{\pgfqpoint{-0.037276in}{-0.021649in}}{\pgfqpoint{-0.029463in}{-0.029463in}}%
\pgfpathcurveto{\pgfqpoint{-0.021649in}{-0.037276in}}{\pgfqpoint{-0.011050in}{-0.041667in}}{\pgfqpoint{0.000000in}{-0.041667in}}%
\pgfpathlineto{\pgfqpoint{0.000000in}{-0.041667in}}%
\pgfpathclose%
\pgfusepath{stroke,fill}%
}%
\begin{pgfscope}%
\pgfsys@transformshift{3.463873in}{1.831819in}%
\pgfsys@useobject{currentmarker}{}%
\end{pgfscope}%
\end{pgfscope}%
\begin{pgfscope}%
\pgfpathrectangle{\pgfqpoint{0.610463in}{0.548486in}}{\pgfqpoint{4.650000in}{2.310000in}}%
\pgfusepath{clip}%
\pgfsetrectcap%
\pgfsetroundjoin%
\pgfsetlinewidth{0.803000pt}%
\definecolor{currentstroke}{rgb}{0.690196,0.690196,0.690196}%
\pgfsetstrokecolor{currentstroke}%
\pgfsetdash{}{0pt}%
\pgfpathmoveto{\pgfqpoint{0.821827in}{0.548486in}}%
\pgfpathlineto{\pgfqpoint{0.821827in}{2.858486in}}%
\pgfusepath{stroke}%
\end{pgfscope}%
\begin{pgfscope}%
\pgfsetbuttcap%
\pgfsetroundjoin%
\definecolor{currentfill}{rgb}{0.000000,0.000000,0.000000}%
\pgfsetfillcolor{currentfill}%
\pgfsetlinewidth{0.803000pt}%
\definecolor{currentstroke}{rgb}{0.000000,0.000000,0.000000}%
\pgfsetstrokecolor{currentstroke}%
\pgfsetdash{}{0pt}%
\pgfsys@defobject{currentmarker}{\pgfqpoint{0.000000in}{-0.048611in}}{\pgfqpoint{0.000000in}{0.000000in}}{%
\pgfpathmoveto{\pgfqpoint{0.000000in}{0.000000in}}%
\pgfpathlineto{\pgfqpoint{0.000000in}{-0.048611in}}%
\pgfusepath{stroke,fill}%
}%
\begin{pgfscope}%
\pgfsys@transformshift{0.821827in}{0.548486in}%
\pgfsys@useobject{currentmarker}{}%
\end{pgfscope}%
\end{pgfscope}%
\begin{pgfscope}%
\definecolor{textcolor}{rgb}{0.000000,0.000000,0.000000}%
\pgfsetstrokecolor{textcolor}%
\pgfsetfillcolor{textcolor}%
\pgftext[x=0.821827in,y=0.451264in,,top]{\color{textcolor}{\rmfamily\fontsize{10.000000}{12.000000}\selectfont\catcode`\^=\active\def^{\ifmmode\sp\else\^{}\fi}\catcode`\%=\active\def%{\%}\ensuremath{-}4}}%
\end{pgfscope}%
\begin{pgfscope}%
\pgfpathrectangle{\pgfqpoint{0.610463in}{0.548486in}}{\pgfqpoint{4.650000in}{2.310000in}}%
\pgfusepath{clip}%
\pgfsetrectcap%
\pgfsetroundjoin%
\pgfsetlinewidth{0.803000pt}%
\definecolor{currentstroke}{rgb}{0.690196,0.690196,0.690196}%
\pgfsetstrokecolor{currentstroke}%
\pgfsetdash{}{0pt}%
\pgfpathmoveto{\pgfqpoint{1.350236in}{0.548486in}}%
\pgfpathlineto{\pgfqpoint{1.350236in}{2.858486in}}%
\pgfusepath{stroke}%
\end{pgfscope}%
\begin{pgfscope}%
\pgfsetbuttcap%
\pgfsetroundjoin%
\definecolor{currentfill}{rgb}{0.000000,0.000000,0.000000}%
\pgfsetfillcolor{currentfill}%
\pgfsetlinewidth{0.803000pt}%
\definecolor{currentstroke}{rgb}{0.000000,0.000000,0.000000}%
\pgfsetstrokecolor{currentstroke}%
\pgfsetdash{}{0pt}%
\pgfsys@defobject{currentmarker}{\pgfqpoint{0.000000in}{-0.048611in}}{\pgfqpoint{0.000000in}{0.000000in}}{%
\pgfpathmoveto{\pgfqpoint{0.000000in}{0.000000in}}%
\pgfpathlineto{\pgfqpoint{0.000000in}{-0.048611in}}%
\pgfusepath{stroke,fill}%
}%
\begin{pgfscope}%
\pgfsys@transformshift{1.350236in}{0.548486in}%
\pgfsys@useobject{currentmarker}{}%
\end{pgfscope}%
\end{pgfscope}%
\begin{pgfscope}%
\definecolor{textcolor}{rgb}{0.000000,0.000000,0.000000}%
\pgfsetstrokecolor{textcolor}%
\pgfsetfillcolor{textcolor}%
\pgftext[x=1.350236in,y=0.451264in,,top]{\color{textcolor}{\rmfamily\fontsize{10.000000}{12.000000}\selectfont\catcode`\^=\active\def^{\ifmmode\sp\else\^{}\fi}\catcode`\%=\active\def%{\%}\ensuremath{-}3}}%
\end{pgfscope}%
\begin{pgfscope}%
\pgfpathrectangle{\pgfqpoint{0.610463in}{0.548486in}}{\pgfqpoint{4.650000in}{2.310000in}}%
\pgfusepath{clip}%
\pgfsetrectcap%
\pgfsetroundjoin%
\pgfsetlinewidth{0.803000pt}%
\definecolor{currentstroke}{rgb}{0.690196,0.690196,0.690196}%
\pgfsetstrokecolor{currentstroke}%
\pgfsetdash{}{0pt}%
\pgfpathmoveto{\pgfqpoint{1.878645in}{0.548486in}}%
\pgfpathlineto{\pgfqpoint{1.878645in}{2.858486in}}%
\pgfusepath{stroke}%
\end{pgfscope}%
\begin{pgfscope}%
\pgfsetbuttcap%
\pgfsetroundjoin%
\definecolor{currentfill}{rgb}{0.000000,0.000000,0.000000}%
\pgfsetfillcolor{currentfill}%
\pgfsetlinewidth{0.803000pt}%
\definecolor{currentstroke}{rgb}{0.000000,0.000000,0.000000}%
\pgfsetstrokecolor{currentstroke}%
\pgfsetdash{}{0pt}%
\pgfsys@defobject{currentmarker}{\pgfqpoint{0.000000in}{-0.048611in}}{\pgfqpoint{0.000000in}{0.000000in}}{%
\pgfpathmoveto{\pgfqpoint{0.000000in}{0.000000in}}%
\pgfpathlineto{\pgfqpoint{0.000000in}{-0.048611in}}%
\pgfusepath{stroke,fill}%
}%
\begin{pgfscope}%
\pgfsys@transformshift{1.878645in}{0.548486in}%
\pgfsys@useobject{currentmarker}{}%
\end{pgfscope}%
\end{pgfscope}%
\begin{pgfscope}%
\definecolor{textcolor}{rgb}{0.000000,0.000000,0.000000}%
\pgfsetstrokecolor{textcolor}%
\pgfsetfillcolor{textcolor}%
\pgftext[x=1.878645in,y=0.451264in,,top]{\color{textcolor}{\rmfamily\fontsize{10.000000}{12.000000}\selectfont\catcode`\^=\active\def^{\ifmmode\sp\else\^{}\fi}\catcode`\%=\active\def%{\%}\ensuremath{-}2}}%
\end{pgfscope}%
\begin{pgfscope}%
\pgfpathrectangle{\pgfqpoint{0.610463in}{0.548486in}}{\pgfqpoint{4.650000in}{2.310000in}}%
\pgfusepath{clip}%
\pgfsetrectcap%
\pgfsetroundjoin%
\pgfsetlinewidth{0.803000pt}%
\definecolor{currentstroke}{rgb}{0.690196,0.690196,0.690196}%
\pgfsetstrokecolor{currentstroke}%
\pgfsetdash{}{0pt}%
\pgfpathmoveto{\pgfqpoint{2.407054in}{0.548486in}}%
\pgfpathlineto{\pgfqpoint{2.407054in}{2.858486in}}%
\pgfusepath{stroke}%
\end{pgfscope}%
\begin{pgfscope}%
\pgfsetbuttcap%
\pgfsetroundjoin%
\definecolor{currentfill}{rgb}{0.000000,0.000000,0.000000}%
\pgfsetfillcolor{currentfill}%
\pgfsetlinewidth{0.803000pt}%
\definecolor{currentstroke}{rgb}{0.000000,0.000000,0.000000}%
\pgfsetstrokecolor{currentstroke}%
\pgfsetdash{}{0pt}%
\pgfsys@defobject{currentmarker}{\pgfqpoint{0.000000in}{-0.048611in}}{\pgfqpoint{0.000000in}{0.000000in}}{%
\pgfpathmoveto{\pgfqpoint{0.000000in}{0.000000in}}%
\pgfpathlineto{\pgfqpoint{0.000000in}{-0.048611in}}%
\pgfusepath{stroke,fill}%
}%
\begin{pgfscope}%
\pgfsys@transformshift{2.407054in}{0.548486in}%
\pgfsys@useobject{currentmarker}{}%
\end{pgfscope}%
\end{pgfscope}%
\begin{pgfscope}%
\definecolor{textcolor}{rgb}{0.000000,0.000000,0.000000}%
\pgfsetstrokecolor{textcolor}%
\pgfsetfillcolor{textcolor}%
\pgftext[x=2.407054in,y=0.451264in,,top]{\color{textcolor}{\rmfamily\fontsize{10.000000}{12.000000}\selectfont\catcode`\^=\active\def^{\ifmmode\sp\else\^{}\fi}\catcode`\%=\active\def%{\%}\ensuremath{-}1}}%
\end{pgfscope}%
\begin{pgfscope}%
\pgfpathrectangle{\pgfqpoint{0.610463in}{0.548486in}}{\pgfqpoint{4.650000in}{2.310000in}}%
\pgfusepath{clip}%
\pgfsetrectcap%
\pgfsetroundjoin%
\pgfsetlinewidth{0.803000pt}%
\definecolor{currentstroke}{rgb}{0.690196,0.690196,0.690196}%
\pgfsetstrokecolor{currentstroke}%
\pgfsetdash{}{0pt}%
\pgfpathmoveto{\pgfqpoint{2.935463in}{0.548486in}}%
\pgfpathlineto{\pgfqpoint{2.935463in}{2.858486in}}%
\pgfusepath{stroke}%
\end{pgfscope}%
\begin{pgfscope}%
\pgfsetbuttcap%
\pgfsetroundjoin%
\definecolor{currentfill}{rgb}{0.000000,0.000000,0.000000}%
\pgfsetfillcolor{currentfill}%
\pgfsetlinewidth{0.803000pt}%
\definecolor{currentstroke}{rgb}{0.000000,0.000000,0.000000}%
\pgfsetstrokecolor{currentstroke}%
\pgfsetdash{}{0pt}%
\pgfsys@defobject{currentmarker}{\pgfqpoint{0.000000in}{-0.048611in}}{\pgfqpoint{0.000000in}{0.000000in}}{%
\pgfpathmoveto{\pgfqpoint{0.000000in}{0.000000in}}%
\pgfpathlineto{\pgfqpoint{0.000000in}{-0.048611in}}%
\pgfusepath{stroke,fill}%
}%
\begin{pgfscope}%
\pgfsys@transformshift{2.935463in}{0.548486in}%
\pgfsys@useobject{currentmarker}{}%
\end{pgfscope}%
\end{pgfscope}%
\begin{pgfscope}%
\definecolor{textcolor}{rgb}{0.000000,0.000000,0.000000}%
\pgfsetstrokecolor{textcolor}%
\pgfsetfillcolor{textcolor}%
\pgftext[x=2.935463in,y=0.451264in,,top]{\color{textcolor}{\rmfamily\fontsize{10.000000}{12.000000}\selectfont\catcode`\^=\active\def^{\ifmmode\sp\else\^{}\fi}\catcode`\%=\active\def%{\%}0}}%
\end{pgfscope}%
\begin{pgfscope}%
\pgfpathrectangle{\pgfqpoint{0.610463in}{0.548486in}}{\pgfqpoint{4.650000in}{2.310000in}}%
\pgfusepath{clip}%
\pgfsetrectcap%
\pgfsetroundjoin%
\pgfsetlinewidth{0.803000pt}%
\definecolor{currentstroke}{rgb}{0.690196,0.690196,0.690196}%
\pgfsetstrokecolor{currentstroke}%
\pgfsetdash{}{0pt}%
\pgfpathmoveto{\pgfqpoint{3.463873in}{0.548486in}}%
\pgfpathlineto{\pgfqpoint{3.463873in}{2.858486in}}%
\pgfusepath{stroke}%
\end{pgfscope}%
\begin{pgfscope}%
\pgfsetbuttcap%
\pgfsetroundjoin%
\definecolor{currentfill}{rgb}{0.000000,0.000000,0.000000}%
\pgfsetfillcolor{currentfill}%
\pgfsetlinewidth{0.803000pt}%
\definecolor{currentstroke}{rgb}{0.000000,0.000000,0.000000}%
\pgfsetstrokecolor{currentstroke}%
\pgfsetdash{}{0pt}%
\pgfsys@defobject{currentmarker}{\pgfqpoint{0.000000in}{-0.048611in}}{\pgfqpoint{0.000000in}{0.000000in}}{%
\pgfpathmoveto{\pgfqpoint{0.000000in}{0.000000in}}%
\pgfpathlineto{\pgfqpoint{0.000000in}{-0.048611in}}%
\pgfusepath{stroke,fill}%
}%
\begin{pgfscope}%
\pgfsys@transformshift{3.463873in}{0.548486in}%
\pgfsys@useobject{currentmarker}{}%
\end{pgfscope}%
\end{pgfscope}%
\begin{pgfscope}%
\definecolor{textcolor}{rgb}{0.000000,0.000000,0.000000}%
\pgfsetstrokecolor{textcolor}%
\pgfsetfillcolor{textcolor}%
\pgftext[x=3.463873in,y=0.451264in,,top]{\color{textcolor}{\rmfamily\fontsize{10.000000}{12.000000}\selectfont\catcode`\^=\active\def^{\ifmmode\sp\else\^{}\fi}\catcode`\%=\active\def%{\%}1}}%
\end{pgfscope}%
\begin{pgfscope}%
\pgfpathrectangle{\pgfqpoint{0.610463in}{0.548486in}}{\pgfqpoint{4.650000in}{2.310000in}}%
\pgfusepath{clip}%
\pgfsetrectcap%
\pgfsetroundjoin%
\pgfsetlinewidth{0.803000pt}%
\definecolor{currentstroke}{rgb}{0.690196,0.690196,0.690196}%
\pgfsetstrokecolor{currentstroke}%
\pgfsetdash{}{0pt}%
\pgfpathmoveto{\pgfqpoint{3.992282in}{0.548486in}}%
\pgfpathlineto{\pgfqpoint{3.992282in}{2.858486in}}%
\pgfusepath{stroke}%
\end{pgfscope}%
\begin{pgfscope}%
\pgfsetbuttcap%
\pgfsetroundjoin%
\definecolor{currentfill}{rgb}{0.000000,0.000000,0.000000}%
\pgfsetfillcolor{currentfill}%
\pgfsetlinewidth{0.803000pt}%
\definecolor{currentstroke}{rgb}{0.000000,0.000000,0.000000}%
\pgfsetstrokecolor{currentstroke}%
\pgfsetdash{}{0pt}%
\pgfsys@defobject{currentmarker}{\pgfqpoint{0.000000in}{-0.048611in}}{\pgfqpoint{0.000000in}{0.000000in}}{%
\pgfpathmoveto{\pgfqpoint{0.000000in}{0.000000in}}%
\pgfpathlineto{\pgfqpoint{0.000000in}{-0.048611in}}%
\pgfusepath{stroke,fill}%
}%
\begin{pgfscope}%
\pgfsys@transformshift{3.992282in}{0.548486in}%
\pgfsys@useobject{currentmarker}{}%
\end{pgfscope}%
\end{pgfscope}%
\begin{pgfscope}%
\definecolor{textcolor}{rgb}{0.000000,0.000000,0.000000}%
\pgfsetstrokecolor{textcolor}%
\pgfsetfillcolor{textcolor}%
\pgftext[x=3.992282in,y=0.451264in,,top]{\color{textcolor}{\rmfamily\fontsize{10.000000}{12.000000}\selectfont\catcode`\^=\active\def^{\ifmmode\sp\else\^{}\fi}\catcode`\%=\active\def%{\%}2}}%
\end{pgfscope}%
\begin{pgfscope}%
\pgfpathrectangle{\pgfqpoint{0.610463in}{0.548486in}}{\pgfqpoint{4.650000in}{2.310000in}}%
\pgfusepath{clip}%
\pgfsetrectcap%
\pgfsetroundjoin%
\pgfsetlinewidth{0.803000pt}%
\definecolor{currentstroke}{rgb}{0.690196,0.690196,0.690196}%
\pgfsetstrokecolor{currentstroke}%
\pgfsetdash{}{0pt}%
\pgfpathmoveto{\pgfqpoint{4.520691in}{0.548486in}}%
\pgfpathlineto{\pgfqpoint{4.520691in}{2.858486in}}%
\pgfusepath{stroke}%
\end{pgfscope}%
\begin{pgfscope}%
\pgfsetbuttcap%
\pgfsetroundjoin%
\definecolor{currentfill}{rgb}{0.000000,0.000000,0.000000}%
\pgfsetfillcolor{currentfill}%
\pgfsetlinewidth{0.803000pt}%
\definecolor{currentstroke}{rgb}{0.000000,0.000000,0.000000}%
\pgfsetstrokecolor{currentstroke}%
\pgfsetdash{}{0pt}%
\pgfsys@defobject{currentmarker}{\pgfqpoint{0.000000in}{-0.048611in}}{\pgfqpoint{0.000000in}{0.000000in}}{%
\pgfpathmoveto{\pgfqpoint{0.000000in}{0.000000in}}%
\pgfpathlineto{\pgfqpoint{0.000000in}{-0.048611in}}%
\pgfusepath{stroke,fill}%
}%
\begin{pgfscope}%
\pgfsys@transformshift{4.520691in}{0.548486in}%
\pgfsys@useobject{currentmarker}{}%
\end{pgfscope}%
\end{pgfscope}%
\begin{pgfscope}%
\definecolor{textcolor}{rgb}{0.000000,0.000000,0.000000}%
\pgfsetstrokecolor{textcolor}%
\pgfsetfillcolor{textcolor}%
\pgftext[x=4.520691in,y=0.451264in,,top]{\color{textcolor}{\rmfamily\fontsize{10.000000}{12.000000}\selectfont\catcode`\^=\active\def^{\ifmmode\sp\else\^{}\fi}\catcode`\%=\active\def%{\%}3}}%
\end{pgfscope}%
\begin{pgfscope}%
\pgfpathrectangle{\pgfqpoint{0.610463in}{0.548486in}}{\pgfqpoint{4.650000in}{2.310000in}}%
\pgfusepath{clip}%
\pgfsetrectcap%
\pgfsetroundjoin%
\pgfsetlinewidth{0.803000pt}%
\definecolor{currentstroke}{rgb}{0.690196,0.690196,0.690196}%
\pgfsetstrokecolor{currentstroke}%
\pgfsetdash{}{0pt}%
\pgfpathmoveto{\pgfqpoint{5.049100in}{0.548486in}}%
\pgfpathlineto{\pgfqpoint{5.049100in}{2.858486in}}%
\pgfusepath{stroke}%
\end{pgfscope}%
\begin{pgfscope}%
\pgfsetbuttcap%
\pgfsetroundjoin%
\definecolor{currentfill}{rgb}{0.000000,0.000000,0.000000}%
\pgfsetfillcolor{currentfill}%
\pgfsetlinewidth{0.803000pt}%
\definecolor{currentstroke}{rgb}{0.000000,0.000000,0.000000}%
\pgfsetstrokecolor{currentstroke}%
\pgfsetdash{}{0pt}%
\pgfsys@defobject{currentmarker}{\pgfqpoint{0.000000in}{-0.048611in}}{\pgfqpoint{0.000000in}{0.000000in}}{%
\pgfpathmoveto{\pgfqpoint{0.000000in}{0.000000in}}%
\pgfpathlineto{\pgfqpoint{0.000000in}{-0.048611in}}%
\pgfusepath{stroke,fill}%
}%
\begin{pgfscope}%
\pgfsys@transformshift{5.049100in}{0.548486in}%
\pgfsys@useobject{currentmarker}{}%
\end{pgfscope}%
\end{pgfscope}%
\begin{pgfscope}%
\definecolor{textcolor}{rgb}{0.000000,0.000000,0.000000}%
\pgfsetstrokecolor{textcolor}%
\pgfsetfillcolor{textcolor}%
\pgftext[x=5.049100in,y=0.451264in,,top]{\color{textcolor}{\rmfamily\fontsize{10.000000}{12.000000}\selectfont\catcode`\^=\active\def^{\ifmmode\sp\else\^{}\fi}\catcode`\%=\active\def%{\%}4}}%
\end{pgfscope}%
\begin{pgfscope}%
\definecolor{textcolor}{rgb}{0.000000,0.000000,0.000000}%
\pgfsetstrokecolor{textcolor}%
\pgfsetfillcolor{textcolor}%
\pgftext[x=2.935463in,y=0.261295in,,top]{\color{textcolor}{\rmfamily\fontsize{12.000000}{14.400000}\selectfont\catcode`\^=\active\def^{\ifmmode\sp\else\^{}\fi}\catcode`\%=\active\def%{\%}$x$}}%
\end{pgfscope}%
\begin{pgfscope}%
\pgfpathrectangle{\pgfqpoint{0.610463in}{0.548486in}}{\pgfqpoint{4.650000in}{2.310000in}}%
\pgfusepath{clip}%
\pgfsetrectcap%
\pgfsetroundjoin%
\pgfsetlinewidth{0.803000pt}%
\definecolor{currentstroke}{rgb}{0.690196,0.690196,0.690196}%
\pgfsetstrokecolor{currentstroke}%
\pgfsetdash{}{0pt}%
\pgfpathmoveto{\pgfqpoint{0.610463in}{0.805153in}}%
\pgfpathlineto{\pgfqpoint{5.260463in}{0.805153in}}%
\pgfusepath{stroke}%
\end{pgfscope}%
\begin{pgfscope}%
\pgfsetbuttcap%
\pgfsetroundjoin%
\definecolor{currentfill}{rgb}{0.000000,0.000000,0.000000}%
\pgfsetfillcolor{currentfill}%
\pgfsetlinewidth{0.803000pt}%
\definecolor{currentstroke}{rgb}{0.000000,0.000000,0.000000}%
\pgfsetstrokecolor{currentstroke}%
\pgfsetdash{}{0pt}%
\pgfsys@defobject{currentmarker}{\pgfqpoint{-0.048611in}{0.000000in}}{\pgfqpoint{-0.000000in}{0.000000in}}{%
\pgfpathmoveto{\pgfqpoint{-0.000000in}{0.000000in}}%
\pgfpathlineto{\pgfqpoint{-0.048611in}{0.000000in}}%
\pgfusepath{stroke,fill}%
}%
\begin{pgfscope}%
\pgfsys@transformshift{0.610463in}{0.805153in}%
\pgfsys@useobject{currentmarker}{}%
\end{pgfscope}%
\end{pgfscope}%
\begin{pgfscope}%
\definecolor{textcolor}{rgb}{0.000000,0.000000,0.000000}%
\pgfsetstrokecolor{textcolor}%
\pgfsetfillcolor{textcolor}%
\pgftext[x=0.316851in, y=0.752391in, left, base]{\color{textcolor}{\rmfamily\fontsize{10.000000}{12.000000}\selectfont\catcode`\^=\active\def^{\ifmmode\sp\else\^{}\fi}\catcode`\%=\active\def%{\%}\ensuremath{-}2}}%
\end{pgfscope}%
\begin{pgfscope}%
\pgfpathrectangle{\pgfqpoint{0.610463in}{0.548486in}}{\pgfqpoint{4.650000in}{2.310000in}}%
\pgfusepath{clip}%
\pgfsetrectcap%
\pgfsetroundjoin%
\pgfsetlinewidth{0.803000pt}%
\definecolor{currentstroke}{rgb}{0.690196,0.690196,0.690196}%
\pgfsetstrokecolor{currentstroke}%
\pgfsetdash{}{0pt}%
\pgfpathmoveto{\pgfqpoint{0.610463in}{1.318486in}}%
\pgfpathlineto{\pgfqpoint{5.260463in}{1.318486in}}%
\pgfusepath{stroke}%
\end{pgfscope}%
\begin{pgfscope}%
\pgfsetbuttcap%
\pgfsetroundjoin%
\definecolor{currentfill}{rgb}{0.000000,0.000000,0.000000}%
\pgfsetfillcolor{currentfill}%
\pgfsetlinewidth{0.803000pt}%
\definecolor{currentstroke}{rgb}{0.000000,0.000000,0.000000}%
\pgfsetstrokecolor{currentstroke}%
\pgfsetdash{}{0pt}%
\pgfsys@defobject{currentmarker}{\pgfqpoint{-0.048611in}{0.000000in}}{\pgfqpoint{-0.000000in}{0.000000in}}{%
\pgfpathmoveto{\pgfqpoint{-0.000000in}{0.000000in}}%
\pgfpathlineto{\pgfqpoint{-0.048611in}{0.000000in}}%
\pgfusepath{stroke,fill}%
}%
\begin{pgfscope}%
\pgfsys@transformshift{0.610463in}{1.318486in}%
\pgfsys@useobject{currentmarker}{}%
\end{pgfscope}%
\end{pgfscope}%
\begin{pgfscope}%
\definecolor{textcolor}{rgb}{0.000000,0.000000,0.000000}%
\pgfsetstrokecolor{textcolor}%
\pgfsetfillcolor{textcolor}%
\pgftext[x=0.424876in, y=1.265724in, left, base]{\color{textcolor}{\rmfamily\fontsize{10.000000}{12.000000}\selectfont\catcode`\^=\active\def^{\ifmmode\sp\else\^{}\fi}\catcode`\%=\active\def%{\%}0}}%
\end{pgfscope}%
\begin{pgfscope}%
\pgfpathrectangle{\pgfqpoint{0.610463in}{0.548486in}}{\pgfqpoint{4.650000in}{2.310000in}}%
\pgfusepath{clip}%
\pgfsetrectcap%
\pgfsetroundjoin%
\pgfsetlinewidth{0.803000pt}%
\definecolor{currentstroke}{rgb}{0.690196,0.690196,0.690196}%
\pgfsetstrokecolor{currentstroke}%
\pgfsetdash{}{0pt}%
\pgfpathmoveto{\pgfqpoint{0.610463in}{1.831819in}}%
\pgfpathlineto{\pgfqpoint{5.260463in}{1.831819in}}%
\pgfusepath{stroke}%
\end{pgfscope}%
\begin{pgfscope}%
\pgfsetbuttcap%
\pgfsetroundjoin%
\definecolor{currentfill}{rgb}{0.000000,0.000000,0.000000}%
\pgfsetfillcolor{currentfill}%
\pgfsetlinewidth{0.803000pt}%
\definecolor{currentstroke}{rgb}{0.000000,0.000000,0.000000}%
\pgfsetstrokecolor{currentstroke}%
\pgfsetdash{}{0pt}%
\pgfsys@defobject{currentmarker}{\pgfqpoint{-0.048611in}{0.000000in}}{\pgfqpoint{-0.000000in}{0.000000in}}{%
\pgfpathmoveto{\pgfqpoint{-0.000000in}{0.000000in}}%
\pgfpathlineto{\pgfqpoint{-0.048611in}{0.000000in}}%
\pgfusepath{stroke,fill}%
}%
\begin{pgfscope}%
\pgfsys@transformshift{0.610463in}{1.831819in}%
\pgfsys@useobject{currentmarker}{}%
\end{pgfscope}%
\end{pgfscope}%
\begin{pgfscope}%
\definecolor{textcolor}{rgb}{0.000000,0.000000,0.000000}%
\pgfsetstrokecolor{textcolor}%
\pgfsetfillcolor{textcolor}%
\pgftext[x=0.424876in, y=1.779058in, left, base]{\color{textcolor}{\rmfamily\fontsize{10.000000}{12.000000}\selectfont\catcode`\^=\active\def^{\ifmmode\sp\else\^{}\fi}\catcode`\%=\active\def%{\%}2}}%
\end{pgfscope}%
\begin{pgfscope}%
\pgfpathrectangle{\pgfqpoint{0.610463in}{0.548486in}}{\pgfqpoint{4.650000in}{2.310000in}}%
\pgfusepath{clip}%
\pgfsetrectcap%
\pgfsetroundjoin%
\pgfsetlinewidth{0.803000pt}%
\definecolor{currentstroke}{rgb}{0.690196,0.690196,0.690196}%
\pgfsetstrokecolor{currentstroke}%
\pgfsetdash{}{0pt}%
\pgfpathmoveto{\pgfqpoint{0.610463in}{2.345153in}}%
\pgfpathlineto{\pgfqpoint{5.260463in}{2.345153in}}%
\pgfusepath{stroke}%
\end{pgfscope}%
\begin{pgfscope}%
\pgfsetbuttcap%
\pgfsetroundjoin%
\definecolor{currentfill}{rgb}{0.000000,0.000000,0.000000}%
\pgfsetfillcolor{currentfill}%
\pgfsetlinewidth{0.803000pt}%
\definecolor{currentstroke}{rgb}{0.000000,0.000000,0.000000}%
\pgfsetstrokecolor{currentstroke}%
\pgfsetdash{}{0pt}%
\pgfsys@defobject{currentmarker}{\pgfqpoint{-0.048611in}{0.000000in}}{\pgfqpoint{-0.000000in}{0.000000in}}{%
\pgfpathmoveto{\pgfqpoint{-0.000000in}{0.000000in}}%
\pgfpathlineto{\pgfqpoint{-0.048611in}{0.000000in}}%
\pgfusepath{stroke,fill}%
}%
\begin{pgfscope}%
\pgfsys@transformshift{0.610463in}{2.345153in}%
\pgfsys@useobject{currentmarker}{}%
\end{pgfscope}%
\end{pgfscope}%
\begin{pgfscope}%
\definecolor{textcolor}{rgb}{0.000000,0.000000,0.000000}%
\pgfsetstrokecolor{textcolor}%
\pgfsetfillcolor{textcolor}%
\pgftext[x=0.424876in, y=2.292391in, left, base]{\color{textcolor}{\rmfamily\fontsize{10.000000}{12.000000}\selectfont\catcode`\^=\active\def^{\ifmmode\sp\else\^{}\fi}\catcode`\%=\active\def%{\%}4}}%
\end{pgfscope}%
\begin{pgfscope}%
\pgfpathrectangle{\pgfqpoint{0.610463in}{0.548486in}}{\pgfqpoint{4.650000in}{2.310000in}}%
\pgfusepath{clip}%
\pgfsetrectcap%
\pgfsetroundjoin%
\pgfsetlinewidth{0.803000pt}%
\definecolor{currentstroke}{rgb}{0.690196,0.690196,0.690196}%
\pgfsetstrokecolor{currentstroke}%
\pgfsetdash{}{0pt}%
\pgfpathmoveto{\pgfqpoint{0.610463in}{2.858486in}}%
\pgfpathlineto{\pgfqpoint{5.260463in}{2.858486in}}%
\pgfusepath{stroke}%
\end{pgfscope}%
\begin{pgfscope}%
\pgfsetbuttcap%
\pgfsetroundjoin%
\definecolor{currentfill}{rgb}{0.000000,0.000000,0.000000}%
\pgfsetfillcolor{currentfill}%
\pgfsetlinewidth{0.803000pt}%
\definecolor{currentstroke}{rgb}{0.000000,0.000000,0.000000}%
\pgfsetstrokecolor{currentstroke}%
\pgfsetdash{}{0pt}%
\pgfsys@defobject{currentmarker}{\pgfqpoint{-0.048611in}{0.000000in}}{\pgfqpoint{-0.000000in}{0.000000in}}{%
\pgfpathmoveto{\pgfqpoint{-0.000000in}{0.000000in}}%
\pgfpathlineto{\pgfqpoint{-0.048611in}{0.000000in}}%
\pgfusepath{stroke,fill}%
}%
\begin{pgfscope}%
\pgfsys@transformshift{0.610463in}{2.858486in}%
\pgfsys@useobject{currentmarker}{}%
\end{pgfscope}%
\end{pgfscope}%
\begin{pgfscope}%
\definecolor{textcolor}{rgb}{0.000000,0.000000,0.000000}%
\pgfsetstrokecolor{textcolor}%
\pgfsetfillcolor{textcolor}%
\pgftext[x=0.424876in, y=2.805724in, left, base]{\color{textcolor}{\rmfamily\fontsize{10.000000}{12.000000}\selectfont\catcode`\^=\active\def^{\ifmmode\sp\else\^{}\fi}\catcode`\%=\active\def%{\%}6}}%
\end{pgfscope}%
\begin{pgfscope}%
\definecolor{textcolor}{rgb}{0.000000,0.000000,0.000000}%
\pgfsetstrokecolor{textcolor}%
\pgfsetfillcolor{textcolor}%
\pgftext[x=0.261295in,y=1.703486in,,bottom,rotate=90.000000]{\color{textcolor}{\rmfamily\fontsize{12.000000}{14.400000}\selectfont\catcode`\^=\active\def^{\ifmmode\sp\else\^{}\fi}\catcode`\%=\active\def%{\%}$y$}}%
\end{pgfscope}%
\begin{pgfscope}%
\pgfpathrectangle{\pgfqpoint{0.610463in}{0.548486in}}{\pgfqpoint{4.650000in}{2.310000in}}%
\pgfusepath{clip}%
\pgfsetrectcap%
\pgfsetroundjoin%
\pgfsetlinewidth{1.505625pt}%
\definecolor{currentstroke}{rgb}{0.000000,0.000000,0.000000}%
\pgfsetstrokecolor{currentstroke}%
\pgfsetdash{}{0pt}%
\pgfpathmoveto{\pgfqpoint{0.821827in}{2.687375in}}%
\pgfpathlineto{\pgfqpoint{5.049100in}{1.318486in}}%
\pgfpathlineto{\pgfqpoint{5.049100in}{1.318486in}}%
\pgfusepath{stroke}%
\end{pgfscope}%
\begin{pgfscope}%
\pgfpathrectangle{\pgfqpoint{0.610463in}{0.548486in}}{\pgfqpoint{4.650000in}{2.310000in}}%
\pgfusepath{clip}%
\pgfsetbuttcap%
\pgfsetroundjoin%
\pgfsetlinewidth{1.505625pt}%
\definecolor{currentstroke}{rgb}{0.000000,0.000000,0.000000}%
\pgfsetstrokecolor{currentstroke}%
\pgfsetdash{{5.550000pt}{2.400000pt}}{0.000000pt}%
\pgfpathmoveto{\pgfqpoint{2.798214in}{0.538486in}}%
\pgfpathlineto{\pgfqpoint{3.997428in}{2.868486in}}%
\pgfpathlineto{\pgfqpoint{3.997428in}{2.868486in}}%
\pgfusepath{stroke}%
\end{pgfscope}%
\begin{pgfscope}%
\pgfpathrectangle{\pgfqpoint{0.610463in}{0.548486in}}{\pgfqpoint{4.650000in}{2.310000in}}%
\pgfusepath{clip}%
\pgfsetrectcap%
\pgfsetroundjoin%
\pgfsetlinewidth{0.501875pt}%
\definecolor{currentstroke}{rgb}{0.000000,0.000000,0.000000}%
\pgfsetstrokecolor{currentstroke}%
\pgfsetdash{}{0pt}%
\pgfpathmoveto{\pgfqpoint{0.610463in}{1.318486in}}%
\pgfpathlineto{\pgfqpoint{5.260463in}{1.318486in}}%
\pgfusepath{stroke}%
\end{pgfscope}%
\begin{pgfscope}%
\pgfpathrectangle{\pgfqpoint{0.610463in}{0.548486in}}{\pgfqpoint{4.650000in}{2.310000in}}%
\pgfusepath{clip}%
\pgfsetrectcap%
\pgfsetroundjoin%
\pgfsetlinewidth{0.501875pt}%
\definecolor{currentstroke}{rgb}{0.000000,0.000000,0.000000}%
\pgfsetstrokecolor{currentstroke}%
\pgfsetdash{}{0pt}%
\pgfpathmoveto{\pgfqpoint{2.935463in}{0.548486in}}%
\pgfpathlineto{\pgfqpoint{2.935463in}{2.858486in}}%
\pgfusepath{stroke}%
\end{pgfscope}%
\begin{pgfscope}%
\pgfsetbuttcap%
\pgfsetmiterjoin%
\definecolor{currentfill}{rgb}{1.000000,1.000000,1.000000}%
\pgfsetfillcolor{currentfill}%
\pgfsetfillopacity{0.800000}%
\pgfsetlinewidth{1.003750pt}%
\definecolor{currentstroke}{rgb}{0.800000,0.800000,0.800000}%
\pgfsetstrokecolor{currentstroke}%
\pgfsetstrokeopacity{0.800000}%
\pgfsetdash{}{0pt}%
\pgfpathmoveto{\pgfqpoint{0.707686in}{0.617930in}}%
\pgfpathlineto{\pgfqpoint{2.738783in}{0.617930in}}%
\pgfpathquadraticcurveto{\pgfqpoint{2.766560in}{0.617930in}}{\pgfqpoint{2.766560in}{0.645708in}}%
\pgfpathlineto{\pgfqpoint{2.766560in}{1.390563in}}%
\pgfpathquadraticcurveto{\pgfqpoint{2.766560in}{1.418341in}}{\pgfqpoint{2.738783in}{1.418341in}}%
\pgfpathlineto{\pgfqpoint{0.707686in}{1.418341in}}%
\pgfpathquadraticcurveto{\pgfqpoint{0.679908in}{1.418341in}}{\pgfqpoint{0.679908in}{1.390563in}}%
\pgfpathlineto{\pgfqpoint{0.679908in}{0.645708in}}%
\pgfpathquadraticcurveto{\pgfqpoint{0.679908in}{0.617930in}}{\pgfqpoint{0.707686in}{0.617930in}}%
\pgfpathlineto{\pgfqpoint{0.707686in}{0.617930in}}%
\pgfpathclose%
\pgfusepath{stroke,fill}%
\end{pgfscope}%
\begin{pgfscope}%
\pgfsetrectcap%
\pgfsetroundjoin%
\pgfsetlinewidth{1.505625pt}%
\definecolor{currentstroke}{rgb}{0.000000,0.000000,0.000000}%
\pgfsetstrokecolor{currentstroke}%
\pgfsetdash{}{0pt}%
\pgfpathmoveto{\pgfqpoint{0.735463in}{1.227931in}}%
\pgfpathlineto{\pgfqpoint{0.874352in}{1.227931in}}%
\pgfpathlineto{\pgfqpoint{1.013241in}{1.227931in}}%
\pgfusepath{stroke}%
\end{pgfscope}%
\begin{pgfscope}%
\definecolor{textcolor}{rgb}{0.000000,0.000000,0.000000}%
\pgfsetstrokecolor{textcolor}%
\pgfsetfillcolor{textcolor}%
\pgftext[x=1.124352in,y=1.179320in,left,base]{\color{textcolor}{\rmfamily\fontsize{10.000000}{12.000000}\selectfont\catcode`\^=\active\def^{\ifmmode\sp\else\^{}\fi}\catcode`\%=\active\def%{\%}$\frac{8}{3} - \frac{2 x}{3}$}}%
\end{pgfscope}%
\begin{pgfscope}%
\pgfsetbuttcap%
\pgfsetroundjoin%
\pgfsetlinewidth{1.505625pt}%
\definecolor{currentstroke}{rgb}{0.000000,0.000000,0.000000}%
\pgfsetstrokecolor{currentstroke}%
\pgfsetdash{{5.550000pt}{2.400000pt}}{0.000000pt}%
\pgfpathmoveto{\pgfqpoint{0.735463in}{0.957692in}}%
\pgfpathlineto{\pgfqpoint{0.874352in}{0.957692in}}%
\pgfpathlineto{\pgfqpoint{1.013241in}{0.957692in}}%
\pgfusepath{stroke}%
\end{pgfscope}%
\begin{pgfscope}%
\definecolor{textcolor}{rgb}{0.000000,0.000000,0.000000}%
\pgfsetstrokecolor{textcolor}%
\pgfsetfillcolor{textcolor}%
\pgftext[x=1.124352in,y=0.909081in,left,base]{\color{textcolor}{\rmfamily\fontsize{10.000000}{12.000000}\selectfont\catcode`\^=\active\def^{\ifmmode\sp\else\^{}\fi}\catcode`\%=\active\def%{\%}$4 x - 2$}}%
\end{pgfscope}%
\begin{pgfscope}%
\pgfsetbuttcap%
\pgfsetroundjoin%
\definecolor{currentfill}{rgb}{1.000000,0.000000,0.000000}%
\pgfsetfillcolor{currentfill}%
\pgfsetlinewidth{1.003750pt}%
\definecolor{currentstroke}{rgb}{1.000000,0.000000,0.000000}%
\pgfsetstrokecolor{currentstroke}%
\pgfsetdash{}{0pt}%
\pgfsys@defobject{currentmarker}{\pgfqpoint{-0.041667in}{-0.041667in}}{\pgfqpoint{0.041667in}{0.041667in}}{%
\pgfpathmoveto{\pgfqpoint{0.000000in}{-0.041667in}}%
\pgfpathcurveto{\pgfqpoint{0.011050in}{-0.041667in}}{\pgfqpoint{0.021649in}{-0.037276in}}{\pgfqpoint{0.029463in}{-0.029463in}}%
\pgfpathcurveto{\pgfqpoint{0.037276in}{-0.021649in}}{\pgfqpoint{0.041667in}{-0.011050in}}{\pgfqpoint{0.041667in}{0.000000in}}%
\pgfpathcurveto{\pgfqpoint{0.041667in}{0.011050in}}{\pgfqpoint{0.037276in}{0.021649in}}{\pgfqpoint{0.029463in}{0.029463in}}%
\pgfpathcurveto{\pgfqpoint{0.021649in}{0.037276in}}{\pgfqpoint{0.011050in}{0.041667in}}{\pgfqpoint{0.000000in}{0.041667in}}%
\pgfpathcurveto{\pgfqpoint{-0.011050in}{0.041667in}}{\pgfqpoint{-0.021649in}{0.037276in}}{\pgfqpoint{-0.029463in}{0.029463in}}%
\pgfpathcurveto{\pgfqpoint{-0.037276in}{0.021649in}}{\pgfqpoint{-0.041667in}{0.011050in}}{\pgfqpoint{-0.041667in}{0.000000in}}%
\pgfpathcurveto{\pgfqpoint{-0.041667in}{-0.011050in}}{\pgfqpoint{-0.037276in}{-0.021649in}}{\pgfqpoint{-0.029463in}{-0.029463in}}%
\pgfpathcurveto{\pgfqpoint{-0.021649in}{-0.037276in}}{\pgfqpoint{-0.011050in}{-0.041667in}}{\pgfqpoint{0.000000in}{-0.041667in}}%
\pgfpathlineto{\pgfqpoint{0.000000in}{-0.041667in}}%
\pgfpathclose%
\pgfusepath{stroke,fill}%
}%
\begin{pgfscope}%
\pgfsys@transformshift{0.874352in}{0.740801in}%
\pgfsys@useobject{currentmarker}{}%
\end{pgfscope}%
\end{pgfscope}%
\begin{pgfscope}%
\definecolor{textcolor}{rgb}{0.000000,0.000000,0.000000}%
\pgfsetstrokecolor{textcolor}%
\pgfsetfillcolor{textcolor}%
\pgftext[x=1.124352in,y=0.704342in,left,base]{\color{textcolor}{\rmfamily\fontsize{10.000000}{12.000000}\selectfont\catcode`\^=\active\def^{\ifmmode\sp\else\^{}\fi}\catcode`\%=\active\def%{\%}Lösung: $x=1.0$, $y=2.0$}}%
\end{pgfscope}%
\end{pgfpicture}%
\makeatother%
\endgroup%

    \caption{Geometrische Interpretation des Gleichungssystems.}
    \label{fig:lgs_geom}
\end{figure}

Obiger Plot und Lösungen berechnet mit diesem Notebook: \colab{https://colab.research.google.com/github/hrtlacek/matheFuerTonmeisterinnen/blob/master/python/notebooks/lgs\_geo.ipynb}.


\subsubsection{Geometrische Interpretation: 'Spalteninterpretation'}
Wir können das obige System umschreiben zu:
$$
x \color{codeRed} \begin{bmatrix} 2 \\ 4 \end{bmatrix}  
 \color{black} +  y \color{codeGreen} \begin{bmatrix} 3 \\ -1 \end{bmatrix}  
\color{black} =
\color{codeBlue} \begin{bmatrix} 8 \\ 2 \end{bmatrix} $$

Die Frage wird dann zu: 'Wie viel des Vektors $\begin{bmatrix} 2 \\ 4 \end{bmatrix}$ und wie viel des Vektors $\begin{bmatrix} 3 \\ -1 \end{bmatrix}$ muss ich nehmen um den Vektor $\begin{bmatrix} 8 \\ 2 \end{bmatrix}$ zu erhalten?'. Wir kennen die Antwort schon: 1 mal den ersten und zwei mal den zweiten. Dies ist verdeutlicht in Abbildung \ref{fig:lgs_geom_col}. 



\begin{figure}[H]
    \centering
    %% Creator: Matplotlib, PGF backend
%%
%% To include the figure in your LaTeX document, write
%%   \input{<filename>.pgf}
%%
%% Make sure the required packages are loaded in your preamble
%%   \usepackage{pgf}
%%
%% Also ensure that all the required font packages are loaded; for instance,
%% the lmodern package is sometimes necessary when using math font.
%%   \usepackage{lmodern}
%%
%% Figures using additional raster images can only be included by \input if
%% they are in the same directory as the main LaTeX file. For loading figures
%% from other directories you can use the `import` package
%%   \usepackage{import}
%%
%% and then include the figures with
%%   \import{<path to file>}{<filename>.pgf}
%%
%% Matplotlib used the following preamble
%%   \def\mathdefault#1{#1}
%%   \everymath=\expandafter{\the\everymath\displaystyle}
%%   
%%   \usepackage{fontspec}
%%   \setmainfont{VeraSe.ttf}[Path=\detokenize{/usr/share/fonts/TTF/}]
%%   \setsansfont{DejaVuSans.ttf}[Path=\detokenize{/home/pl/miniconda3/lib/python3.12/site-packages/matplotlib/mpl-data/fonts/ttf/}]
%%   \setmonofont{DejaVuSansMono.ttf}[Path=\detokenize{/home/pl/miniconda3/lib/python3.12/site-packages/matplotlib/mpl-data/fonts/ttf/}]
%%   \makeatletter\@ifpackageloaded{underscore}{}{\usepackage[strings]{underscore}}\makeatother
%%
\begingroup%
\makeatletter%
\begin{pgfpicture}%
\pgfpathrectangle{\pgfpointorigin}{\pgfqpoint{3.815041in}{2.794397in}}%
\pgfusepath{use as bounding box, clip}%
\begin{pgfscope}%
\pgfsetbuttcap%
\pgfsetmiterjoin%
\definecolor{currentfill}{rgb}{1.000000,1.000000,1.000000}%
\pgfsetfillcolor{currentfill}%
\pgfsetlinewidth{0.000000pt}%
\definecolor{currentstroke}{rgb}{1.000000,1.000000,1.000000}%
\pgfsetstrokecolor{currentstroke}%
\pgfsetdash{}{0pt}%
\pgfpathmoveto{\pgfqpoint{0.000000in}{0.000000in}}%
\pgfpathlineto{\pgfqpoint{3.815041in}{0.000000in}}%
\pgfpathlineto{\pgfqpoint{3.815041in}{2.794397in}}%
\pgfpathlineto{\pgfqpoint{0.000000in}{2.794397in}}%
\pgfpathlineto{\pgfqpoint{0.000000in}{0.000000in}}%
\pgfpathclose%
\pgfusepath{fill}%
\end{pgfscope}%
\begin{pgfscope}%
\pgfsetbuttcap%
\pgfsetmiterjoin%
\definecolor{currentfill}{rgb}{1.000000,1.000000,1.000000}%
\pgfsetfillcolor{currentfill}%
\pgfsetlinewidth{0.000000pt}%
\definecolor{currentstroke}{rgb}{0.000000,0.000000,0.000000}%
\pgfsetstrokecolor{currentstroke}%
\pgfsetstrokeopacity{0.000000}%
\pgfsetdash{}{0pt}%
\pgfpathmoveto{\pgfqpoint{0.393612in}{0.331635in}}%
\pgfpathlineto{\pgfqpoint{3.715041in}{0.331635in}}%
\pgfpathlineto{\pgfqpoint{3.715041in}{2.641635in}}%
\pgfpathlineto{\pgfqpoint{0.393612in}{2.641635in}}%
\pgfpathlineto{\pgfqpoint{0.393612in}{0.331635in}}%
\pgfpathclose%
\pgfusepath{fill}%
\end{pgfscope}%
\begin{pgfscope}%
\pgfpathrectangle{\pgfqpoint{0.393612in}{0.331635in}}{\pgfqpoint{3.321429in}{2.310000in}}%
\pgfusepath{clip}%
\pgfsetbuttcap%
\pgfsetroundjoin%
\definecolor{currentfill}{rgb}{0.913725,0.309804,0.215686}%
\pgfsetfillcolor{currentfill}%
\pgfsetfillopacity{0.500000}%
\pgfsetlinewidth{0.000000pt}%
\definecolor{currentstroke}{rgb}{0.000000,0.000000,0.000000}%
\pgfsetstrokecolor{currentstroke}%
\pgfsetdash{}{0pt}%
\pgfpathmoveto{\pgfqpoint{0.714629in}{0.997234in}}%
\pgfpathlineto{\pgfqpoint{1.328523in}{2.217101in}}%
\pgfpathlineto{\pgfqpoint{1.300672in}{2.217173in}}%
\pgfpathlineto{\pgfqpoint{1.390041in}{2.311635in}}%
\pgfpathlineto{\pgfqpoint{1.367428in}{2.183579in}}%
\pgfpathlineto{\pgfqpoint{1.350775in}{2.205903in}}%
\pgfpathlineto{\pgfqpoint{0.736881in}{0.986036in}}%
\pgfpathlineto{\pgfqpoint{0.714629in}{0.997234in}}%
\pgfusepath{fill}%
\end{pgfscope}%
\begin{pgfscope}%
\pgfpathrectangle{\pgfqpoint{0.393612in}{0.331635in}}{\pgfqpoint{3.321429in}{2.310000in}}%
\pgfusepath{clip}%
\pgfsetbuttcap%
\pgfsetroundjoin%
\definecolor{currentfill}{rgb}{0.266667,0.733333,0.643137}%
\pgfsetfillcolor{currentfill}%
\pgfsetfillopacity{0.500000}%
\pgfsetlinewidth{0.000000pt}%
\definecolor{currentstroke}{rgb}{0.000000,0.000000,0.000000}%
\pgfsetstrokecolor{currentstroke}%
\pgfsetdash{}{0pt}%
\pgfpathmoveto{\pgfqpoint{0.729671in}{1.003459in}}%
\pgfpathlineto{\pgfqpoint{1.619686in}{0.708701in}}%
\pgfpathlineto{\pgfqpoint{1.615693in}{0.736265in}}%
\pgfpathlineto{\pgfqpoint{1.722184in}{0.661635in}}%
\pgfpathlineto{\pgfqpoint{1.592198in}{0.665322in}}%
\pgfpathlineto{\pgfqpoint{1.611854in}{0.685054in}}%
\pgfpathlineto{\pgfqpoint{0.721840in}{0.979811in}}%
\pgfpathlineto{\pgfqpoint{0.729671in}{1.003459in}}%
\pgfusepath{fill}%
\end{pgfscope}%
\begin{pgfscope}%
\pgfpathrectangle{\pgfqpoint{0.393612in}{0.331635in}}{\pgfqpoint{3.321429in}{2.310000in}}%
\pgfusepath{clip}%
\pgfsetbuttcap%
\pgfsetroundjoin%
\definecolor{currentfill}{rgb}{0.247059,0.533333,0.772549}%
\pgfsetfillcolor{currentfill}%
\pgfsetlinewidth{0.000000pt}%
\definecolor{currentstroke}{rgb}{0.000000,0.000000,0.000000}%
\pgfsetstrokecolor{currentstroke}%
\pgfsetdash{}{0pt}%
\pgfpathmoveto{\pgfqpoint{0.722753in}{1.003723in}}%
\pgfpathlineto{\pgfqpoint{3.271103in}{1.636700in}}%
\pgfpathlineto{\pgfqpoint{3.253010in}{1.657874in}}%
\pgfpathlineto{\pgfqpoint{3.382898in}{1.651635in}}%
\pgfpathlineto{\pgfqpoint{3.271025in}{1.585346in}}%
\pgfpathlineto{\pgfqpoint{3.277108in}{1.612524in}}%
\pgfpathlineto{\pgfqpoint{0.728758in}{0.979547in}}%
\pgfpathlineto{\pgfqpoint{0.722753in}{1.003723in}}%
\pgfusepath{fill}%
\end{pgfscope}%
\begin{pgfscope}%
\pgfpathrectangle{\pgfqpoint{0.393612in}{0.331635in}}{\pgfqpoint{3.321429in}{2.310000in}}%
\pgfusepath{clip}%
\pgfsetbuttcap%
\pgfsetroundjoin%
\definecolor{currentfill}{rgb}{0.266667,0.733333,0.643137}%
\pgfsetfillcolor{currentfill}%
\pgfsetlinewidth{0.000000pt}%
\definecolor{currentstroke}{rgb}{0.000000,0.000000,0.000000}%
\pgfsetstrokecolor{currentstroke}%
\pgfsetdash{}{0pt}%
\pgfpathmoveto{\pgfqpoint{1.393957in}{2.323459in}}%
\pgfpathlineto{\pgfqpoint{3.280400in}{1.698701in}}%
\pgfpathlineto{\pgfqpoint{3.276408in}{1.726265in}}%
\pgfpathlineto{\pgfqpoint{3.382898in}{1.651635in}}%
\pgfpathlineto{\pgfqpoint{3.252913in}{1.655322in}}%
\pgfpathlineto{\pgfqpoint{3.272568in}{1.675054in}}%
\pgfpathlineto{\pgfqpoint{1.386125in}{2.299811in}}%
\pgfpathlineto{\pgfqpoint{1.393957in}{2.323459in}}%
\pgfusepath{fill}%
\end{pgfscope}%
\begin{pgfscope}%
\pgfpathrectangle{\pgfqpoint{0.393612in}{0.331635in}}{\pgfqpoint{3.321429in}{2.310000in}}%
\pgfusepath{clip}%
\pgfsetrectcap%
\pgfsetroundjoin%
\pgfsetlinewidth{0.803000pt}%
\definecolor{currentstroke}{rgb}{0.690196,0.690196,0.690196}%
\pgfsetstrokecolor{currentstroke}%
\pgfsetdash{}{0pt}%
\pgfpathmoveto{\pgfqpoint{0.725755in}{0.331635in}}%
\pgfpathlineto{\pgfqpoint{0.725755in}{2.641635in}}%
\pgfusepath{stroke}%
\end{pgfscope}%
\begin{pgfscope}%
\pgfsetbuttcap%
\pgfsetroundjoin%
\definecolor{currentfill}{rgb}{0.000000,0.000000,0.000000}%
\pgfsetfillcolor{currentfill}%
\pgfsetlinewidth{0.803000pt}%
\definecolor{currentstroke}{rgb}{0.000000,0.000000,0.000000}%
\pgfsetstrokecolor{currentstroke}%
\pgfsetdash{}{0pt}%
\pgfsys@defobject{currentmarker}{\pgfqpoint{0.000000in}{-0.048611in}}{\pgfqpoint{0.000000in}{0.000000in}}{%
\pgfpathmoveto{\pgfqpoint{0.000000in}{0.000000in}}%
\pgfpathlineto{\pgfqpoint{0.000000in}{-0.048611in}}%
\pgfusepath{stroke,fill}%
}%
\begin{pgfscope}%
\pgfsys@transformshift{0.725755in}{0.331635in}%
\pgfsys@useobject{currentmarker}{}%
\end{pgfscope}%
\end{pgfscope}%
\begin{pgfscope}%
\definecolor{textcolor}{rgb}{0.000000,0.000000,0.000000}%
\pgfsetstrokecolor{textcolor}%
\pgfsetfillcolor{textcolor}%
\pgftext[x=0.725755in,y=0.234413in,,top]{\color{textcolor}{\rmfamily\fontsize{10.000000}{12.000000}\selectfont\catcode`\^=\active\def^{\ifmmode\sp\else\^{}\fi}\catcode`\%=\active\def%{\%}0}}%
\end{pgfscope}%
\begin{pgfscope}%
\pgfpathrectangle{\pgfqpoint{0.393612in}{0.331635in}}{\pgfqpoint{3.321429in}{2.310000in}}%
\pgfusepath{clip}%
\pgfsetrectcap%
\pgfsetroundjoin%
\pgfsetlinewidth{0.803000pt}%
\definecolor{currentstroke}{rgb}{0.690196,0.690196,0.690196}%
\pgfsetstrokecolor{currentstroke}%
\pgfsetdash{}{0pt}%
\pgfpathmoveto{\pgfqpoint{1.390041in}{0.331635in}}%
\pgfpathlineto{\pgfqpoint{1.390041in}{2.641635in}}%
\pgfusepath{stroke}%
\end{pgfscope}%
\begin{pgfscope}%
\pgfsetbuttcap%
\pgfsetroundjoin%
\definecolor{currentfill}{rgb}{0.000000,0.000000,0.000000}%
\pgfsetfillcolor{currentfill}%
\pgfsetlinewidth{0.803000pt}%
\definecolor{currentstroke}{rgb}{0.000000,0.000000,0.000000}%
\pgfsetstrokecolor{currentstroke}%
\pgfsetdash{}{0pt}%
\pgfsys@defobject{currentmarker}{\pgfqpoint{0.000000in}{-0.048611in}}{\pgfqpoint{0.000000in}{0.000000in}}{%
\pgfpathmoveto{\pgfqpoint{0.000000in}{0.000000in}}%
\pgfpathlineto{\pgfqpoint{0.000000in}{-0.048611in}}%
\pgfusepath{stroke,fill}%
}%
\begin{pgfscope}%
\pgfsys@transformshift{1.390041in}{0.331635in}%
\pgfsys@useobject{currentmarker}{}%
\end{pgfscope}%
\end{pgfscope}%
\begin{pgfscope}%
\definecolor{textcolor}{rgb}{0.000000,0.000000,0.000000}%
\pgfsetstrokecolor{textcolor}%
\pgfsetfillcolor{textcolor}%
\pgftext[x=1.390041in,y=0.234413in,,top]{\color{textcolor}{\rmfamily\fontsize{10.000000}{12.000000}\selectfont\catcode`\^=\active\def^{\ifmmode\sp\else\^{}\fi}\catcode`\%=\active\def%{\%}2}}%
\end{pgfscope}%
\begin{pgfscope}%
\pgfpathrectangle{\pgfqpoint{0.393612in}{0.331635in}}{\pgfqpoint{3.321429in}{2.310000in}}%
\pgfusepath{clip}%
\pgfsetrectcap%
\pgfsetroundjoin%
\pgfsetlinewidth{0.803000pt}%
\definecolor{currentstroke}{rgb}{0.690196,0.690196,0.690196}%
\pgfsetstrokecolor{currentstroke}%
\pgfsetdash{}{0pt}%
\pgfpathmoveto{\pgfqpoint{2.054327in}{0.331635in}}%
\pgfpathlineto{\pgfqpoint{2.054327in}{2.641635in}}%
\pgfusepath{stroke}%
\end{pgfscope}%
\begin{pgfscope}%
\pgfsetbuttcap%
\pgfsetroundjoin%
\definecolor{currentfill}{rgb}{0.000000,0.000000,0.000000}%
\pgfsetfillcolor{currentfill}%
\pgfsetlinewidth{0.803000pt}%
\definecolor{currentstroke}{rgb}{0.000000,0.000000,0.000000}%
\pgfsetstrokecolor{currentstroke}%
\pgfsetdash{}{0pt}%
\pgfsys@defobject{currentmarker}{\pgfqpoint{0.000000in}{-0.048611in}}{\pgfqpoint{0.000000in}{0.000000in}}{%
\pgfpathmoveto{\pgfqpoint{0.000000in}{0.000000in}}%
\pgfpathlineto{\pgfqpoint{0.000000in}{-0.048611in}}%
\pgfusepath{stroke,fill}%
}%
\begin{pgfscope}%
\pgfsys@transformshift{2.054327in}{0.331635in}%
\pgfsys@useobject{currentmarker}{}%
\end{pgfscope}%
\end{pgfscope}%
\begin{pgfscope}%
\definecolor{textcolor}{rgb}{0.000000,0.000000,0.000000}%
\pgfsetstrokecolor{textcolor}%
\pgfsetfillcolor{textcolor}%
\pgftext[x=2.054327in,y=0.234413in,,top]{\color{textcolor}{\rmfamily\fontsize{10.000000}{12.000000}\selectfont\catcode`\^=\active\def^{\ifmmode\sp\else\^{}\fi}\catcode`\%=\active\def%{\%}4}}%
\end{pgfscope}%
\begin{pgfscope}%
\pgfpathrectangle{\pgfqpoint{0.393612in}{0.331635in}}{\pgfqpoint{3.321429in}{2.310000in}}%
\pgfusepath{clip}%
\pgfsetrectcap%
\pgfsetroundjoin%
\pgfsetlinewidth{0.803000pt}%
\definecolor{currentstroke}{rgb}{0.690196,0.690196,0.690196}%
\pgfsetstrokecolor{currentstroke}%
\pgfsetdash{}{0pt}%
\pgfpathmoveto{\pgfqpoint{2.718612in}{0.331635in}}%
\pgfpathlineto{\pgfqpoint{2.718612in}{2.641635in}}%
\pgfusepath{stroke}%
\end{pgfscope}%
\begin{pgfscope}%
\pgfsetbuttcap%
\pgfsetroundjoin%
\definecolor{currentfill}{rgb}{0.000000,0.000000,0.000000}%
\pgfsetfillcolor{currentfill}%
\pgfsetlinewidth{0.803000pt}%
\definecolor{currentstroke}{rgb}{0.000000,0.000000,0.000000}%
\pgfsetstrokecolor{currentstroke}%
\pgfsetdash{}{0pt}%
\pgfsys@defobject{currentmarker}{\pgfqpoint{0.000000in}{-0.048611in}}{\pgfqpoint{0.000000in}{0.000000in}}{%
\pgfpathmoveto{\pgfqpoint{0.000000in}{0.000000in}}%
\pgfpathlineto{\pgfqpoint{0.000000in}{-0.048611in}}%
\pgfusepath{stroke,fill}%
}%
\begin{pgfscope}%
\pgfsys@transformshift{2.718612in}{0.331635in}%
\pgfsys@useobject{currentmarker}{}%
\end{pgfscope}%
\end{pgfscope}%
\begin{pgfscope}%
\definecolor{textcolor}{rgb}{0.000000,0.000000,0.000000}%
\pgfsetstrokecolor{textcolor}%
\pgfsetfillcolor{textcolor}%
\pgftext[x=2.718612in,y=0.234413in,,top]{\color{textcolor}{\rmfamily\fontsize{10.000000}{12.000000}\selectfont\catcode`\^=\active\def^{\ifmmode\sp\else\^{}\fi}\catcode`\%=\active\def%{\%}6}}%
\end{pgfscope}%
\begin{pgfscope}%
\pgfpathrectangle{\pgfqpoint{0.393612in}{0.331635in}}{\pgfqpoint{3.321429in}{2.310000in}}%
\pgfusepath{clip}%
\pgfsetrectcap%
\pgfsetroundjoin%
\pgfsetlinewidth{0.803000pt}%
\definecolor{currentstroke}{rgb}{0.690196,0.690196,0.690196}%
\pgfsetstrokecolor{currentstroke}%
\pgfsetdash{}{0pt}%
\pgfpathmoveto{\pgfqpoint{3.382898in}{0.331635in}}%
\pgfpathlineto{\pgfqpoint{3.382898in}{2.641635in}}%
\pgfusepath{stroke}%
\end{pgfscope}%
\begin{pgfscope}%
\pgfsetbuttcap%
\pgfsetroundjoin%
\definecolor{currentfill}{rgb}{0.000000,0.000000,0.000000}%
\pgfsetfillcolor{currentfill}%
\pgfsetlinewidth{0.803000pt}%
\definecolor{currentstroke}{rgb}{0.000000,0.000000,0.000000}%
\pgfsetstrokecolor{currentstroke}%
\pgfsetdash{}{0pt}%
\pgfsys@defobject{currentmarker}{\pgfqpoint{0.000000in}{-0.048611in}}{\pgfqpoint{0.000000in}{0.000000in}}{%
\pgfpathmoveto{\pgfqpoint{0.000000in}{0.000000in}}%
\pgfpathlineto{\pgfqpoint{0.000000in}{-0.048611in}}%
\pgfusepath{stroke,fill}%
}%
\begin{pgfscope}%
\pgfsys@transformshift{3.382898in}{0.331635in}%
\pgfsys@useobject{currentmarker}{}%
\end{pgfscope}%
\end{pgfscope}%
\begin{pgfscope}%
\definecolor{textcolor}{rgb}{0.000000,0.000000,0.000000}%
\pgfsetstrokecolor{textcolor}%
\pgfsetfillcolor{textcolor}%
\pgftext[x=3.382898in,y=0.234413in,,top]{\color{textcolor}{\rmfamily\fontsize{10.000000}{12.000000}\selectfont\catcode`\^=\active\def^{\ifmmode\sp\else\^{}\fi}\catcode`\%=\active\def%{\%}8}}%
\end{pgfscope}%
\begin{pgfscope}%
\pgfpathrectangle{\pgfqpoint{0.393612in}{0.331635in}}{\pgfqpoint{3.321429in}{2.310000in}}%
\pgfusepath{clip}%
\pgfsetrectcap%
\pgfsetroundjoin%
\pgfsetlinewidth{0.803000pt}%
\definecolor{currentstroke}{rgb}{0.690196,0.690196,0.690196}%
\pgfsetstrokecolor{currentstroke}%
\pgfsetdash{}{0pt}%
\pgfpathmoveto{\pgfqpoint{0.393612in}{0.331635in}}%
\pgfpathlineto{\pgfqpoint{3.715041in}{0.331635in}}%
\pgfusepath{stroke}%
\end{pgfscope}%
\begin{pgfscope}%
\pgfsetbuttcap%
\pgfsetroundjoin%
\definecolor{currentfill}{rgb}{0.000000,0.000000,0.000000}%
\pgfsetfillcolor{currentfill}%
\pgfsetlinewidth{0.803000pt}%
\definecolor{currentstroke}{rgb}{0.000000,0.000000,0.000000}%
\pgfsetstrokecolor{currentstroke}%
\pgfsetdash{}{0pt}%
\pgfsys@defobject{currentmarker}{\pgfqpoint{-0.048611in}{0.000000in}}{\pgfqpoint{-0.000000in}{0.000000in}}{%
\pgfpathmoveto{\pgfqpoint{-0.000000in}{0.000000in}}%
\pgfpathlineto{\pgfqpoint{-0.048611in}{0.000000in}}%
\pgfusepath{stroke,fill}%
}%
\begin{pgfscope}%
\pgfsys@transformshift{0.393612in}{0.331635in}%
\pgfsys@useobject{currentmarker}{}%
\end{pgfscope}%
\end{pgfscope}%
\begin{pgfscope}%
\definecolor{textcolor}{rgb}{0.000000,0.000000,0.000000}%
\pgfsetstrokecolor{textcolor}%
\pgfsetfillcolor{textcolor}%
\pgftext[x=0.100000in, y=0.278873in, left, base]{\color{textcolor}{\rmfamily\fontsize{10.000000}{12.000000}\selectfont\catcode`\^=\active\def^{\ifmmode\sp\else\^{}\fi}\catcode`\%=\active\def%{\%}\ensuremath{-}2}}%
\end{pgfscope}%
\begin{pgfscope}%
\pgfpathrectangle{\pgfqpoint{0.393612in}{0.331635in}}{\pgfqpoint{3.321429in}{2.310000in}}%
\pgfusepath{clip}%
\pgfsetrectcap%
\pgfsetroundjoin%
\pgfsetlinewidth{0.803000pt}%
\definecolor{currentstroke}{rgb}{0.690196,0.690196,0.690196}%
\pgfsetstrokecolor{currentstroke}%
\pgfsetdash{}{0pt}%
\pgfpathmoveto{\pgfqpoint{0.393612in}{0.661635in}}%
\pgfpathlineto{\pgfqpoint{3.715041in}{0.661635in}}%
\pgfusepath{stroke}%
\end{pgfscope}%
\begin{pgfscope}%
\pgfsetbuttcap%
\pgfsetroundjoin%
\definecolor{currentfill}{rgb}{0.000000,0.000000,0.000000}%
\pgfsetfillcolor{currentfill}%
\pgfsetlinewidth{0.803000pt}%
\definecolor{currentstroke}{rgb}{0.000000,0.000000,0.000000}%
\pgfsetstrokecolor{currentstroke}%
\pgfsetdash{}{0pt}%
\pgfsys@defobject{currentmarker}{\pgfqpoint{-0.048611in}{0.000000in}}{\pgfqpoint{-0.000000in}{0.000000in}}{%
\pgfpathmoveto{\pgfqpoint{-0.000000in}{0.000000in}}%
\pgfpathlineto{\pgfqpoint{-0.048611in}{0.000000in}}%
\pgfusepath{stroke,fill}%
}%
\begin{pgfscope}%
\pgfsys@transformshift{0.393612in}{0.661635in}%
\pgfsys@useobject{currentmarker}{}%
\end{pgfscope}%
\end{pgfscope}%
\begin{pgfscope}%
\definecolor{textcolor}{rgb}{0.000000,0.000000,0.000000}%
\pgfsetstrokecolor{textcolor}%
\pgfsetfillcolor{textcolor}%
\pgftext[x=0.100000in, y=0.608873in, left, base]{\color{textcolor}{\rmfamily\fontsize{10.000000}{12.000000}\selectfont\catcode`\^=\active\def^{\ifmmode\sp\else\^{}\fi}\catcode`\%=\active\def%{\%}\ensuremath{-}1}}%
\end{pgfscope}%
\begin{pgfscope}%
\pgfpathrectangle{\pgfqpoint{0.393612in}{0.331635in}}{\pgfqpoint{3.321429in}{2.310000in}}%
\pgfusepath{clip}%
\pgfsetrectcap%
\pgfsetroundjoin%
\pgfsetlinewidth{0.803000pt}%
\definecolor{currentstroke}{rgb}{0.690196,0.690196,0.690196}%
\pgfsetstrokecolor{currentstroke}%
\pgfsetdash{}{0pt}%
\pgfpathmoveto{\pgfqpoint{0.393612in}{0.991635in}}%
\pgfpathlineto{\pgfqpoint{3.715041in}{0.991635in}}%
\pgfusepath{stroke}%
\end{pgfscope}%
\begin{pgfscope}%
\pgfsetbuttcap%
\pgfsetroundjoin%
\definecolor{currentfill}{rgb}{0.000000,0.000000,0.000000}%
\pgfsetfillcolor{currentfill}%
\pgfsetlinewidth{0.803000pt}%
\definecolor{currentstroke}{rgb}{0.000000,0.000000,0.000000}%
\pgfsetstrokecolor{currentstroke}%
\pgfsetdash{}{0pt}%
\pgfsys@defobject{currentmarker}{\pgfqpoint{-0.048611in}{0.000000in}}{\pgfqpoint{-0.000000in}{0.000000in}}{%
\pgfpathmoveto{\pgfqpoint{-0.000000in}{0.000000in}}%
\pgfpathlineto{\pgfqpoint{-0.048611in}{0.000000in}}%
\pgfusepath{stroke,fill}%
}%
\begin{pgfscope}%
\pgfsys@transformshift{0.393612in}{0.991635in}%
\pgfsys@useobject{currentmarker}{}%
\end{pgfscope}%
\end{pgfscope}%
\begin{pgfscope}%
\definecolor{textcolor}{rgb}{0.000000,0.000000,0.000000}%
\pgfsetstrokecolor{textcolor}%
\pgfsetfillcolor{textcolor}%
\pgftext[x=0.208025in, y=0.938873in, left, base]{\color{textcolor}{\rmfamily\fontsize{10.000000}{12.000000}\selectfont\catcode`\^=\active\def^{\ifmmode\sp\else\^{}\fi}\catcode`\%=\active\def%{\%}0}}%
\end{pgfscope}%
\begin{pgfscope}%
\pgfpathrectangle{\pgfqpoint{0.393612in}{0.331635in}}{\pgfqpoint{3.321429in}{2.310000in}}%
\pgfusepath{clip}%
\pgfsetrectcap%
\pgfsetroundjoin%
\pgfsetlinewidth{0.803000pt}%
\definecolor{currentstroke}{rgb}{0.690196,0.690196,0.690196}%
\pgfsetstrokecolor{currentstroke}%
\pgfsetdash{}{0pt}%
\pgfpathmoveto{\pgfqpoint{0.393612in}{1.321635in}}%
\pgfpathlineto{\pgfqpoint{3.715041in}{1.321635in}}%
\pgfusepath{stroke}%
\end{pgfscope}%
\begin{pgfscope}%
\pgfsetbuttcap%
\pgfsetroundjoin%
\definecolor{currentfill}{rgb}{0.000000,0.000000,0.000000}%
\pgfsetfillcolor{currentfill}%
\pgfsetlinewidth{0.803000pt}%
\definecolor{currentstroke}{rgb}{0.000000,0.000000,0.000000}%
\pgfsetstrokecolor{currentstroke}%
\pgfsetdash{}{0pt}%
\pgfsys@defobject{currentmarker}{\pgfqpoint{-0.048611in}{0.000000in}}{\pgfqpoint{-0.000000in}{0.000000in}}{%
\pgfpathmoveto{\pgfqpoint{-0.000000in}{0.000000in}}%
\pgfpathlineto{\pgfqpoint{-0.048611in}{0.000000in}}%
\pgfusepath{stroke,fill}%
}%
\begin{pgfscope}%
\pgfsys@transformshift{0.393612in}{1.321635in}%
\pgfsys@useobject{currentmarker}{}%
\end{pgfscope}%
\end{pgfscope}%
\begin{pgfscope}%
\definecolor{textcolor}{rgb}{0.000000,0.000000,0.000000}%
\pgfsetstrokecolor{textcolor}%
\pgfsetfillcolor{textcolor}%
\pgftext[x=0.208025in, y=1.268873in, left, base]{\color{textcolor}{\rmfamily\fontsize{10.000000}{12.000000}\selectfont\catcode`\^=\active\def^{\ifmmode\sp\else\^{}\fi}\catcode`\%=\active\def%{\%}1}}%
\end{pgfscope}%
\begin{pgfscope}%
\pgfpathrectangle{\pgfqpoint{0.393612in}{0.331635in}}{\pgfqpoint{3.321429in}{2.310000in}}%
\pgfusepath{clip}%
\pgfsetrectcap%
\pgfsetroundjoin%
\pgfsetlinewidth{0.803000pt}%
\definecolor{currentstroke}{rgb}{0.690196,0.690196,0.690196}%
\pgfsetstrokecolor{currentstroke}%
\pgfsetdash{}{0pt}%
\pgfpathmoveto{\pgfqpoint{0.393612in}{1.651635in}}%
\pgfpathlineto{\pgfqpoint{3.715041in}{1.651635in}}%
\pgfusepath{stroke}%
\end{pgfscope}%
\begin{pgfscope}%
\pgfsetbuttcap%
\pgfsetroundjoin%
\definecolor{currentfill}{rgb}{0.000000,0.000000,0.000000}%
\pgfsetfillcolor{currentfill}%
\pgfsetlinewidth{0.803000pt}%
\definecolor{currentstroke}{rgb}{0.000000,0.000000,0.000000}%
\pgfsetstrokecolor{currentstroke}%
\pgfsetdash{}{0pt}%
\pgfsys@defobject{currentmarker}{\pgfqpoint{-0.048611in}{0.000000in}}{\pgfqpoint{-0.000000in}{0.000000in}}{%
\pgfpathmoveto{\pgfqpoint{-0.000000in}{0.000000in}}%
\pgfpathlineto{\pgfqpoint{-0.048611in}{0.000000in}}%
\pgfusepath{stroke,fill}%
}%
\begin{pgfscope}%
\pgfsys@transformshift{0.393612in}{1.651635in}%
\pgfsys@useobject{currentmarker}{}%
\end{pgfscope}%
\end{pgfscope}%
\begin{pgfscope}%
\definecolor{textcolor}{rgb}{0.000000,0.000000,0.000000}%
\pgfsetstrokecolor{textcolor}%
\pgfsetfillcolor{textcolor}%
\pgftext[x=0.208025in, y=1.598873in, left, base]{\color{textcolor}{\rmfamily\fontsize{10.000000}{12.000000}\selectfont\catcode`\^=\active\def^{\ifmmode\sp\else\^{}\fi}\catcode`\%=\active\def%{\%}2}}%
\end{pgfscope}%
\begin{pgfscope}%
\pgfpathrectangle{\pgfqpoint{0.393612in}{0.331635in}}{\pgfqpoint{3.321429in}{2.310000in}}%
\pgfusepath{clip}%
\pgfsetrectcap%
\pgfsetroundjoin%
\pgfsetlinewidth{0.803000pt}%
\definecolor{currentstroke}{rgb}{0.690196,0.690196,0.690196}%
\pgfsetstrokecolor{currentstroke}%
\pgfsetdash{}{0pt}%
\pgfpathmoveto{\pgfqpoint{0.393612in}{1.981635in}}%
\pgfpathlineto{\pgfqpoint{3.715041in}{1.981635in}}%
\pgfusepath{stroke}%
\end{pgfscope}%
\begin{pgfscope}%
\pgfsetbuttcap%
\pgfsetroundjoin%
\definecolor{currentfill}{rgb}{0.000000,0.000000,0.000000}%
\pgfsetfillcolor{currentfill}%
\pgfsetlinewidth{0.803000pt}%
\definecolor{currentstroke}{rgb}{0.000000,0.000000,0.000000}%
\pgfsetstrokecolor{currentstroke}%
\pgfsetdash{}{0pt}%
\pgfsys@defobject{currentmarker}{\pgfqpoint{-0.048611in}{0.000000in}}{\pgfqpoint{-0.000000in}{0.000000in}}{%
\pgfpathmoveto{\pgfqpoint{-0.000000in}{0.000000in}}%
\pgfpathlineto{\pgfqpoint{-0.048611in}{0.000000in}}%
\pgfusepath{stroke,fill}%
}%
\begin{pgfscope}%
\pgfsys@transformshift{0.393612in}{1.981635in}%
\pgfsys@useobject{currentmarker}{}%
\end{pgfscope}%
\end{pgfscope}%
\begin{pgfscope}%
\definecolor{textcolor}{rgb}{0.000000,0.000000,0.000000}%
\pgfsetstrokecolor{textcolor}%
\pgfsetfillcolor{textcolor}%
\pgftext[x=0.208025in, y=1.928873in, left, base]{\color{textcolor}{\rmfamily\fontsize{10.000000}{12.000000}\selectfont\catcode`\^=\active\def^{\ifmmode\sp\else\^{}\fi}\catcode`\%=\active\def%{\%}3}}%
\end{pgfscope}%
\begin{pgfscope}%
\pgfpathrectangle{\pgfqpoint{0.393612in}{0.331635in}}{\pgfqpoint{3.321429in}{2.310000in}}%
\pgfusepath{clip}%
\pgfsetrectcap%
\pgfsetroundjoin%
\pgfsetlinewidth{0.803000pt}%
\definecolor{currentstroke}{rgb}{0.690196,0.690196,0.690196}%
\pgfsetstrokecolor{currentstroke}%
\pgfsetdash{}{0pt}%
\pgfpathmoveto{\pgfqpoint{0.393612in}{2.311635in}}%
\pgfpathlineto{\pgfqpoint{3.715041in}{2.311635in}}%
\pgfusepath{stroke}%
\end{pgfscope}%
\begin{pgfscope}%
\pgfsetbuttcap%
\pgfsetroundjoin%
\definecolor{currentfill}{rgb}{0.000000,0.000000,0.000000}%
\pgfsetfillcolor{currentfill}%
\pgfsetlinewidth{0.803000pt}%
\definecolor{currentstroke}{rgb}{0.000000,0.000000,0.000000}%
\pgfsetstrokecolor{currentstroke}%
\pgfsetdash{}{0pt}%
\pgfsys@defobject{currentmarker}{\pgfqpoint{-0.048611in}{0.000000in}}{\pgfqpoint{-0.000000in}{0.000000in}}{%
\pgfpathmoveto{\pgfqpoint{-0.000000in}{0.000000in}}%
\pgfpathlineto{\pgfqpoint{-0.048611in}{0.000000in}}%
\pgfusepath{stroke,fill}%
}%
\begin{pgfscope}%
\pgfsys@transformshift{0.393612in}{2.311635in}%
\pgfsys@useobject{currentmarker}{}%
\end{pgfscope}%
\end{pgfscope}%
\begin{pgfscope}%
\definecolor{textcolor}{rgb}{0.000000,0.000000,0.000000}%
\pgfsetstrokecolor{textcolor}%
\pgfsetfillcolor{textcolor}%
\pgftext[x=0.208025in, y=2.258873in, left, base]{\color{textcolor}{\rmfamily\fontsize{10.000000}{12.000000}\selectfont\catcode`\^=\active\def^{\ifmmode\sp\else\^{}\fi}\catcode`\%=\active\def%{\%}4}}%
\end{pgfscope}%
\begin{pgfscope}%
\pgfpathrectangle{\pgfqpoint{0.393612in}{0.331635in}}{\pgfqpoint{3.321429in}{2.310000in}}%
\pgfusepath{clip}%
\pgfsetrectcap%
\pgfsetroundjoin%
\pgfsetlinewidth{0.803000pt}%
\definecolor{currentstroke}{rgb}{0.690196,0.690196,0.690196}%
\pgfsetstrokecolor{currentstroke}%
\pgfsetdash{}{0pt}%
\pgfpathmoveto{\pgfqpoint{0.393612in}{2.641635in}}%
\pgfpathlineto{\pgfqpoint{3.715041in}{2.641635in}}%
\pgfusepath{stroke}%
\end{pgfscope}%
\begin{pgfscope}%
\pgfsetbuttcap%
\pgfsetroundjoin%
\definecolor{currentfill}{rgb}{0.000000,0.000000,0.000000}%
\pgfsetfillcolor{currentfill}%
\pgfsetlinewidth{0.803000pt}%
\definecolor{currentstroke}{rgb}{0.000000,0.000000,0.000000}%
\pgfsetstrokecolor{currentstroke}%
\pgfsetdash{}{0pt}%
\pgfsys@defobject{currentmarker}{\pgfqpoint{-0.048611in}{0.000000in}}{\pgfqpoint{-0.000000in}{0.000000in}}{%
\pgfpathmoveto{\pgfqpoint{-0.000000in}{0.000000in}}%
\pgfpathlineto{\pgfqpoint{-0.048611in}{0.000000in}}%
\pgfusepath{stroke,fill}%
}%
\begin{pgfscope}%
\pgfsys@transformshift{0.393612in}{2.641635in}%
\pgfsys@useobject{currentmarker}{}%
\end{pgfscope}%
\end{pgfscope}%
\begin{pgfscope}%
\definecolor{textcolor}{rgb}{0.000000,0.000000,0.000000}%
\pgfsetstrokecolor{textcolor}%
\pgfsetfillcolor{textcolor}%
\pgftext[x=0.208025in, y=2.588873in, left, base]{\color{textcolor}{\rmfamily\fontsize{10.000000}{12.000000}\selectfont\catcode`\^=\active\def^{\ifmmode\sp\else\^{}\fi}\catcode`\%=\active\def%{\%}5}}%
\end{pgfscope}%
\begin{pgfscope}%
\pgfpathrectangle{\pgfqpoint{0.393612in}{0.331635in}}{\pgfqpoint{3.321429in}{2.310000in}}%
\pgfusepath{clip}%
\pgfsetrectcap%
\pgfsetroundjoin%
\pgfsetlinewidth{0.501875pt}%
\definecolor{currentstroke}{rgb}{0.000000,0.000000,0.000000}%
\pgfsetstrokecolor{currentstroke}%
\pgfsetdash{}{0pt}%
\pgfpathmoveto{\pgfqpoint{0.393612in}{0.991635in}}%
\pgfpathlineto{\pgfqpoint{3.715041in}{0.991635in}}%
\pgfusepath{stroke}%
\end{pgfscope}%
\begin{pgfscope}%
\pgfpathrectangle{\pgfqpoint{0.393612in}{0.331635in}}{\pgfqpoint{3.321429in}{2.310000in}}%
\pgfusepath{clip}%
\pgfsetrectcap%
\pgfsetroundjoin%
\pgfsetlinewidth{0.501875pt}%
\definecolor{currentstroke}{rgb}{0.000000,0.000000,0.000000}%
\pgfsetstrokecolor{currentstroke}%
\pgfsetdash{}{0pt}%
\pgfpathmoveto{\pgfqpoint{0.725755in}{0.331635in}}%
\pgfpathlineto{\pgfqpoint{0.725755in}{2.641635in}}%
\pgfusepath{stroke}%
\end{pgfscope}%
\end{pgfpicture}%
\makeatother%
\endgroup%

    \caption{Geometrische Interpretation des Gleichungssystems: Linearkombination der Spalten.}
    \label{fig:lgs_geom_col}
\end{figure}

Es handelt sich also um eine \emph{Linearkombination der Spalten} der Matrix: $$\mathbf{A} = \begin{bmatrix} 2 & 3 \\ 4 & -1 \end{bmatrix}$$


\subsubsection{Versuch einer Algebraischen Lösung via sympy}

\begin{python}{Lösen eines Gleichungssytems.}
In [1]: import sympy as sp
In [2]: from sympy.abc import x,y,z,t
In [3]: sp.init_printing()
In [4]: g1 = sp.Eq(2*x + 3*y, 8)
In [5]: g2 = sp.Eq(4*x - y, 2)
In [6]: g1
Out[6]: Eq(2*x + 3*y, 8)
In [7]: g1
Out[7]: 2⋅x + 3⋅y = 8
In [8]: g2
Out[8]: 4⋅x - y = 2
In [9]: sp.solve([g1, g2], [x,y])
Out[9]: {x: 1, y: 2}
\end{python}


\subsubsection{LGS in Matrizenschreibweise}


Dies kann in Matrixform geschrieben werden als:

\[
\begin{bmatrix} 2 & 3 \\ 4 & -1 \end{bmatrix}  
\begin{bmatrix} x \\ y \end{bmatrix}  
=
\begin{bmatrix} 8 \\ 2 \end{bmatrix}
\]


\important{Die Situation des Gleichungssystems oben kann allgemeiner als $$\mathbf{A}\vec{x} = \vec{b}$$ zusammengefasst werden. Und das wiederum kann konzeptualisiert werden als Linearkombination der Spalten in $\mathbf{A}$ mit den Faktoren in $\vec{x}$ mit dem Ergebnis $\vec{b}$.
}



\begin{python}{Lösung eines Linearen Gleichungssystems mit Python.}
In [1]: %pylab
In [2]: A = np.array([[2, 3], [4, -1]])
In [3]: b = np.array([8, 2])
In [4]: x = np.linalg.solve(A, b)
In [5]: x
Out[5]: array([1., 2.])
\end{python}








\subsection{Spezielle Matrizen}

\subsubsection*{Null Matrix}
Eine fett notierte 0, $\mathbf{0}$ notiert eine mit nullen gefüllte Matrix.
\begin{equation}
\mathbf{0} =
\begin{bmatrix}
0 & 0 & 0 & \cdots & 0 \\
0 & 0 & 0 & \cdots & 0 \\
0 & 0 & 0 & \cdots & 0 \\
\vdots & \vdots & \vdots & \ddots & \vdots \\
0 & 0 & 0 & \cdots & 0
\end{bmatrix}
\end{equation} 

In python können wir eine solche Matrix leicht erzeugen:
\begin{python}{Nullmatrix in python. 2 Reihen, 3 Spalten.}
In [1]: %pylab
In [2]: zeros([2,3])
Out[2]:
array([[0., 0., 0.],
       [0., 0., 0.]])
\end{python}

Mathematisch wirkt diese Matrix zunächst nicht wahnsinnig hilfreich, in Python kommt dies aber sehr oft vor. Beispielsweise wenn wir einfach einen Array initialisieren wollen um ihn anschließend in einer \texttt{for}-Schleife zu befüllen. \footnote{Das angeführte Beispiel ist ein wenig konstruiert. Wir wissen, wir können das hier erzielte Ergebnis einfacher erreichen: \pythoninline{sin(arange(10))}. Die Struktur des Beispiels wird hilfreicher wenn wir zum Beispiel Filter berechnen oder dynamische Systeme simulieren, siehe Abschnitt \ref{sub:dynamische_systeme}.}

\begin{python}{Array initialisieren und via \texttt{for}-Schleife befüllen.}
In [1]: a = zeros(10)
In [2]: for i in range(10):
   ...:     a[i] = sin(i)
In [3]: a
Out[3]:
array([ 0.        ,  0.84147098,  0.90929743,  0.14112001, -0.7568025 ,
       -0.95892427, -0.2794155 ,  0.6569866 ,  0.98935825,  0.41211849])
\end{python}

\subsubsection*{Einheitsmatrix}

$\mathbf{E}$ in deutschen Texten, in englisch 'Identity Matrix', daher $\mathbf{I}$. Dieser Text hat sich für $\mathbf{I}$ entschieden.

\begin{equation}
\mathbf{I} =
\begin{bmatrix}
1 & 0 & 0 & \cdots & 0 \\
0 & 1 & 0 & \cdots & 0 \\
0 & 0 & 1 & \cdots & 0 \\
\vdots & \vdots & \vdots & \ddots & \vdots \\
0 & 0 & 0 & \cdots & 1
\end{bmatrix}
\end{equation} 


\begin{python}{Numerische $3\times 3$ Einheitsmatrix.}
In [1]: %pylab
In [2]: A = eye(3)
In [3]: A
Out[3]:
array([[1., 0., 0.],
       [0., 1., 0.],
       [0., 0., 1.]])
\end{python}

% \mathbf{I} = \begin{bmatrix} 1 & 0 & \dots & 0 \\
%                     0 & 1 & \dots &  \vdots   \\
%                     \vdots &  \vdots  &1&  0 \\
%                     0 & 0 & \dots & 1
%  \end{bmatrix}

\subsubsection{Diagonalmatrix}

\subsubsection{Symmetrische Matrizen}

$\mathbf{A} \, ist \, symmetrisch \iff für alle i,j, a_{ji} = a_{ij}$

$\mathbf{A} = \mathbf{A}^T$



\subsubsection{Orthogonale matrizen}
(Rotationsmatrizen)

Ist eine Matrix $\mathbf{Q}$ orthogonal dann gilt 
$$ \mathbf{Q}^T\mathbf{Q} = \mathbf{I} $$.

% Hier  https://www.youtube.com/watch?v=j8hEnyOiwhw wird folgendes behauptet:
% $$ \mathbf{Q} \mathbf{Q}^T \neq \mathbf{I} $$.
% Auf wikipedia steht das gegenteil.

\subsection*{Matrixmultiplikation}

\begin{center}
\begin{tikzpicture}

    % First matrix (A) - Governs rows
    \matrix[matrix, 
            nodes={draw, minimum size=1cm, anchor=center}, 
            column sep=2mm, row sep=2mm] (A) {
        x & x & x \\
        x & x & x \\
        x & x & x \\
    };

    % Second matrix (B) - Governs columns
    \matrix[matrix, 
            nodes={draw, minimum size=1cm, anchor=center}, 
            column sep=2mm, row sep=2mm, right=of A] (B) {
        x & x \\
        x & x \\
        x & x \\
    };

    % Resulting matrix (C)
    \matrix[matrix, 
            nodes={draw, minimum size=1cm, anchor=center}, 
            column sep=2mm, row sep=2mm, right=of B] (C) {
        x & x \\
        x & x \\
    };

    % Multiplication sign
    \node at ($(A.east)!0.5!(B.west)$) {$\times$};
    \node at ($(B.east)!0.5!(C.west)$) {$=$};

\end{tikzpicture}
\end{center}

\begin{equation}
{\displaystyle c_{ik}=\sum _{j=1}^{n}a_{ij}\cdot b_{jk}}
\end{equation}


\subsection*{Transposition}
\subsection*{Determinante}
\todo[inline]{A Test for invertibility}
$det \mathbf{A}$ oder auch $|\mathbf{A}|$. 

\begin{itemize}
   \item $det \mathbf{A} = 1$.
   \item Reihen vertauschen kehrt das vorzeichen der Determinante um.
\end{itemize}

$$ \begin{vmatrix} a & b  \\
                    c & d 
 \end{vmatrix} = ad - bc$$

\colab{https://colab.research.google.com/github/hrtlacek/matheFuerTonmeisterinnen/blob/master/python/notebooks/Determinante.ipynb}


\subsection*{Invertierung}

\subsubsection{Diagonalisierung}
 % https://www.youtube.com/watch?v=ZSGrJBS_qtc


\subsection*{Eigenwert und Eigenvektoren}
\index{Eigenwert}\index{Eigenvektor}\index{Zustandsraumdarstellung}\index{State-Space}

\praxis{Eigenwerte der $A$ Matrix in der Zustandsraumdarstellung (engl. 'State-Space') von Systemen sind die Pole des Systems. Eigenvektoren bei der Modalanalyse können Modenformen (die eigentliche Bewegung eines Systems zB.) darstellen und in Eigenwerten sind die entsprechenden Frequenzen ($\Im(\lambda)$) und die Stabilität ($\Re(\lambda)$) enthalten. Eigenvektoren von Matrizen die für Probabilistische Modellierung (siehe Markov Modelle) können Stationäre Wahrscheinlichkeits-Vektoren sein.}


\subsection*{Dynamische Systeme}\label{sub:dynamische_systeme}

\section{Aufgaben}
\begin{enumerate}
\item \todo[inline]{Aufgabe: Stereo/Vectorscope aus Oszilloscop bauen. Rotation d Vektors um 45 grad. }
\end{enumerate}
