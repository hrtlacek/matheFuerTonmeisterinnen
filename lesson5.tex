%!TEX root = main.tex
\chapter{Lineare Algebra, Vektoren und Matrizen}
\todo{Probabilistische system entwicklung, https://www.youtube.com/watch?v=K-8F\_zDMDUI\&list=PLMrJAkhIeNNTYaOnVI3QpH7jgULnAmvPA\&index=3}
\todo[inline]{Digital Audio als vektor, Vector scope/oscilloscope music, stereo sachen, Rotationsmatrix, Hadamardmatrix, Householdermatrix,  state space formulations, modal analyse}

\citep{strang2020linear}

\section{Motivation}
Vektoren und Matrizen findet man sehr häufig in der Audiotechnischen Literatur. Wie bereits in Kapitel \ref{chap:complex} beschrieben, sind komplexe Zahlen allgegenwärtig und deren Interpretation als Vektor bedeutsam und praktisch. Andererseits kann zB. ein Stereosignal als Vektor aufgefasst werden: Zu jedem Zeitpunkt hat es 2 Zahlen, diese können als Vektor im 2-dimensionalen Raum aufgefasst werden, was wiederum diverse Operationen nahelegt. Ebenso scheint es spätestens an diesem Punkt naheliegend ein $n$-Kanal Signal als Vektor mit $n$ Einträgen zu verstehen und womöglich hieraus Schlüsse ziehen zu dürfen. Wenn es um verräumlichung von Signalen (zB. Ambisonics) geht und daher Positionen im Dreidimensionalen Raum ($\mathbb{R}^3$) benötigt werden ist klar, dass wir diese mittels Vektoren beschreiben können. Aber auch wenn es um Direktionalität von Lautsprechern oder Mikrophonen geht werden letztlich 2 bis 3 Dimensionale Felder beschrieben und auch hier kommen Vektoren zum Einsatz. Es sei aber an dieser Stelle betont dass Vektoren und Matrizen nicht ausschließlich für Audioanwendungen im Dreidimensionalen Raum zum Einsatz kommen. Rotations- Hadamard- und Householdermatrizen \citep{PASPWEB2010}\footnote{https://ccrma.stanford.edu/~jos/pasp/Hadamard\_Matrix.html} werden für Hall-Algorithmen und Resonatoren verwendet , Filtersysteme können über Zustandsraumdarstellungen via Vektoren und Matrizen beschrieben werden \citep{bilbao2009numerical} und vieles mehr.       

\section{Vektoren}
Es wird nun einführendes über Vektoren gesagt aber recht kurz und bündig da 1. angenommen wird, dass insbesondere 2 und 3 dimensionale Vektoren zu den weniger abstrakten und aus der Schule bekannten Themen gehören und 2. Vektoren als Spezialfall von Matrizen gesehen werden können und wir unsere Aufmerksamkeit so schnell wie möglich diesen zuwenden wollen.

\important{
Ein Vektor in einem $n$-Dimensionalen Raum ($\mathbb{R}^n$) ist eine geordnete Liste von Zahlen.

\begin{equation}
\vec{a} = \begin{bmatrix}
a_1 & a_2 & \dots & a_n
\end{bmatrix}
\end{equation}


Der Vektor $\vec{a}$ und seine Elemente $a_1, a_2, \dots a_n$ sind hier als \textbf{Reihenvektor} (auch 'Zeilenvektor', engl.'Row vector') dargestellt. Ebenso können wir \textbf{Spaltenvektoren} (engl. 'column vector') darstellen:
\begin{equation}
\vec{a} = \begin{bmatrix}
a_1 \\
a_2 \\
\vdots \\
a_n
\end{bmatrix}
\end{equation}

\textbf{Konventionen} \\
Dieses Dokument wird die Konvention 'Reihen' statt 'Zeilen' verwenden. Vorteilhaft ist daran, dass 'R' vor 'S' im Alphabet kommt und es hilft uns uns die \emph{Reihenfolge} von Indizes von Matrizen zu merken: \textbf{Reihe, Spalte}.

Es gibt unterschiedliche Schreibweisen in Unterschiedlicher Literatur. Dieses Dokument verwendet die Schreibweise $\vec{a}$. Manchmal findet man auch fett gedruckte Buchstaben um zu notieren, dass es sich um einen Vektor handelt: $\mathbf{a}$. Zudem gibt es Literatur die runde Klammern bevorzugt statt eckigen wie in diesem Dokument: $\mathbf{a} = \begin{pmatrix}
a_1 & a_2 & \dots & a_n
\end{pmatrix}$. All das ist lediglich ein Unterschied in Konventionen.
}


\begin{python}{Vektoren: Arrays in Python}
In [1]: %pylab
In [2]: a = array([1,2,3]) #Vektor Definition
   ...: print(a.shape)
   ...: print(a)
(3,)
[1 2 3]
In [3]: b = array([[5, 6, 7]]) # Matrix, 1 Reihenvektor
   ...: print(b.shape)
   ...: print(b)
(1, 3)
[[5 6 7]]
In [4]: c = array([[8, 9, 10]]).T # Matrix, 1 Spaltenvektor
   ...: print(c.shape)
   ...: print(c)
(3, 1)
[[ 8]
 [ 9]
 [10]]
\end{python}

\subsection*{Hadamard\footnote{Jacques Hadamard (Jacques Salomon Hadamard; * 8. Dezember 1865 in Versailles; † 17. Oktober 1963 in Paris) war ein französischer Mathematiker.} Produkt}
\index{Hadamard Produkt}

Auch Elementweises oder Komponentenweises Produkt. Einfache elementweise Multiplikation.
$$ \vec{a}\circ \vec{b} = \begin{bmatrix}a_1b_1 & a_2b_2 & \dots & a_n b_n\end{bmatrix}$$ 

Leider wird üblicherweise die selbe Notation benutzt um die Verkettung von Funktionen zu beschreiben: $(f \circ g)(x) = f(g(x))$. Welche Operation nun gemeint ist, erschließt sich typischerweise aus dem Kontext. Beide Notationen kommen ohnehin nicht wahnsinnig oft vor.

In Python\footnote{Hier ist ein wichtiger unterschied zu \texttt{MATLAB} Code da dort punktweise Multiplikation mit \texttt{.*} ausgeführt wird.} kann elementweise Multiplikation durch einfache Multiplikation ausgeführt werden: \pythoninline{c = a*b}.

\begin{python}{Elementweise Multiplikation von 2 Arrays in Python.}
In [1]: %pylab
In [2]: a = array([1,2,3])
In [3]: b = array([5, 6, 7])
In [4]: a*b
Out[4]: array([ 5, 12, 21])
\end{python}

\begin{question}
In Python, definiere 2 Arrays, $\vec{a}$ und $\vec{b}$. Diese sollen jeweils eine Sinus Schwingung enthalten, aber jeweils mit anderer Frequenz. Multipliziere sie, plotte das Ergebnis und höre dir das Ergebnis an. Befehle um Arrays als Audio file auf die Festplatte zu schreiben sind im Python Cheat-Sheet zu finden in Abschnitt \ref{sec:pythonCheat}. Was passiert hier aus Audio Sicht?
\end{question}

\begin{answer}
Zunächst, was passiert ist eine 'Ringmodulation' (sozusagen eine Variante der Amplitudenmodulation). Um eine Sinus Schwingung mit Frequenz $f_a$ definieren wir eine Funktion $a(t)$:
\begin{equation}
a(t) = sin(2 \pi t f_a)
\end{equation}
Um hier eine Sekunde Sound zu erzeugen mit Samplingrate $f_s = 44100$ benötigen wir einen Vektor $\vec{t}$ der die Zeit in Sekunden enthält: $\vec{t}:= \begin{bmatrix} 0 & \Delta_t& 2 \Delta_t &  \dots 1-\Delta_t \end{bmatrix}$ wobei $\Delta_t$ das Sampling-Intervall $\Delta_t = \frac{1}{f_s}$. Praktisch ist es am einfachsten man erzeugt man einen Array der einen Index enthält: $ns := \begin{bmatrix} 0 & 1 & 2 &  N-1 \end{bmatrix}$. Dies kann man in python via \texttt{arange(N)} erzielen. Diese kann anschliessend durch die samplingrate dividiert werden um $\vec{t}$ zu erhalten. Da wir eine Sekunde Audio erzeugen wollen ist $N=f_s$ die Samplingrate.


\begin{python}{Ringmodulation.}
In [1]: %pylab
In [2]: import soundfile as sf
In [3]: sr= 44100
In [4]: ns = arange(sr)
In [5]: t = ns/sr
In [6]: a = sin(2*pi*t*100)
In [7]: b = sin(2*pi*t*50)
In [8]: c = a*b
In [9]: plot(t,c)
In [10]: sf.write('ringmodTest.wav', c, sr)
\end{python}
\end{answer}


\subsection*{Skalarprodukt}
\index{Skalarprodukt}

Auch 'inneres Produkt' oder 'Punktprodukt', engl. 'Dot product'. Ergibt im fall von Vektoren einen Skalar aber eigentlich ein Spezialfall der Matrixmultiplikation!
$$\vec{a}\cdot \vec{b} = |\vec{a}|\cdot |\vec{b}|\cdot cos(\alpha) = a_x b_x + a_y b_y$$
Siehe \citep[p.~45]{Westermmann2008}  für Herleitung.

Python: \pythoninline{dot(a,b)}

\subsection*{Kreuzprodukt}
\index{Kreuzprodukt}
Auch 'Vektorprodukt'. 
$$ \vec{c} = \vec{a} \times \vec{b}$$
Nur in $\mathbb{R}^3$ definiert!

Python: \pythoninline{cross(a,b)}

\section{Matrizen}

\praxis{
Angenommen wir wollen $4$ Audio signale, zum beispiel 4 verschiedene instrumente auf 6 verschiedene Lautsprecher schicken, unterschiedlicher Lautstärke. Wie könnet man dies Mathematisch formulieren? Man könnte zum Beispiel sagen die 4 audio signale seien ein vektor $\vec{x(t)}$, der die jeweiligen signale enthält: $\vec{x}(t) = \begin{bmatrix} x_1(t) & x_2(t) & x_3(t) & x_4(t)\end{bmatrix}$. Unsere output Signale die wir an die 6 Lautsprecher schicken nennen wir vielleicht $y_1(t), \dots, y_6(t)$. Auch diese machen wir zu einem Vektor $\vec{y}(t) = \begin{bmatrix} y_1(t) & y_2(t) & y_3(t) & y_4(t) & y_5(t) & y_6(t)\end{bmatrix}$. Nun können wir unsere Verräumlichung über eine Matrix definieren:

$$ \mathbf{A}\vec{x} = \vec{y}$$

Hier ist zu beachten dass es sich bei $\vec{x}$ um eine $4\times 1$ Matrix, also einen \textbf{Spaltenvektor} mit 4 Reihen handelt, und bei $\vec{y}$ ebenso um einen Spaltenvektor. Die Matrix $\mathbf{A}$ hat die Form $6 \times 4$. zb:

\[
\begin{bmatrix} 1 & 0 & 0 & 0 \\ 0 & 1 & 0 & 0 \\ 0 & 0 & 1 & 0 \\ 0 & 0 & 0 & 1 \\ 1 & 0 & 0 & 0 \\ 1 & 0 & 0 & 0 \end{bmatrix}  
\begin{bmatrix} x_1 \\ x_2 \\ x_3 \\ x_4  \end{bmatrix}  
=
\begin{bmatrix} y_1 \\ y_2 \\ y_3\\ y_4\\ y_5\\ y_6\end{bmatrix}
\]


In diesem Beispiel haben wir beschlossen instrument $1$ auf lautsprecher $1$ zu schicken, Instrument $2$ auf Lautsprecher $2$ etc, aber Instrument $1$ auch noch auf Lautsprecher 5 und 6 zu senden. Das heisst $y_1 = x_1$, $y_2 = x_2$, $y_3 = x_3$, $y_4 = x_4$, $y_5 = x_1$, $y_6 = x_1$.
}




\subsection*{Ein Lineares Gleichungssystem (LGS)}

\[
\begin{aligned}
2x + 3y &= 8 \\
4x -  y &= 2
\end{aligned}
\]

Dies kann in Matrixform geschrieben werden als:

\[
\begin{bmatrix} 2 & 3 \\ 4 & -1 \end{bmatrix}  
\begin{bmatrix} x \\ y \end{bmatrix}  
=
\begin{bmatrix} 8 \\ 2 \end{bmatrix}
\]

\begin{python}{Lösung eines Linearen Gleichungssytems mit Python.}
In [1]: %pylab
In [2]: A = np.array([[2, 3], [4, -1]])
In [3]: b = np.array([8, 2])
In [4]: x = np.linalg.solve(A, b)
In [5]: x
Out[5]: array([1., 2.])
\end{python}




\begin{equation}
\mathbf{A} = \begin{bmatrix} a_{11} & a_{12} & \dots & a_{1m} \\
                    a_{21} & a_{22} &       &  \vdots   \\
                    \vdots &        &\ddots &  \vdots \\
                    a_{n 1} & \dots & \dots & a_{n m}
 \end{bmatrix}
\end{equation}




\subsection{Spezielle Matrizen}

\subsubsection*{Null Matrix}

\subsubsection*{Einheitsmatrix}

$\mathbf{E}$ in deutschen Texten, in englisch 'Identity Matrix', daher $\mathbf{I}$. 

\begin{equation}
\mathbf{I} =
\begin{bmatrix}
1 & 0 & 0 & \cdots & 0 \\
0 & 1 & 0 & \cdots & 0 \\
0 & 0 & 1 & \cdots & 0 \\
\vdots & \vdots & \vdots & \ddots & \vdots \\
0 & 0 & 0 & \cdots & 1
\end{bmatrix}
\end{equation} 


\begin{python}{Numerische $3\times 3$ Einheitsmatrix.}
In [1]: %pylab
In [2]: A = eye(3)
In [3]: A
Out[3]:
array([[1., 0., 0.],
       [0., 1., 0.],
       [0., 0., 1.]])
\end{python}

% \mathbf{I} = \begin{bmatrix} 1 & 0 & \dots & 0 \\
%                     0 & 1 & \dots &  \vdots   \\
%                     \vdots &  \vdots  &1&  0 \\
%                     0 & 0 & \dots & 1
%  \end{bmatrix}

\subsubsection{Diagonalmatrix}

\subsubsection{Symmetrische Matrizen}

$\mathbf{A} \, ist \, symmetrisch \iff für alle i,j, a_{ji} = a_{ij}$

$\mathbf{A} = \mathbf{A}^T$



\subsubsection{Orthogonale matrizen}
(Rotationsmatrizen)

Ist eine Matrix $\mathbf{Q}$ orthogonal dann gilt 
$$ \mathbf{Q}^T\mathbf{Q} = \mathbf{I} $$.

% Hier  https://www.youtube.com/watch?v=j8hEnyOiwhw wird folgendes behauptet:
% $$ \mathbf{Q} \mathbf{Q}^T \neq \mathbf{I} $$.
% Auf wikipedia steht das gegenteil.

\subsection*{Matrixmultiplikation}

\begin{center}
\begin{tikzpicture}

    % First matrix (A) - Governs rows
    \matrix[matrix, 
            nodes={draw, minimum size=1cm, anchor=center}, 
            column sep=2mm, row sep=2mm] (A) {
        x & x & x \\
        x & x & x \\
        x & x & x \\
    };

    % Second matrix (B) - Governs columns
    \matrix[matrix, 
            nodes={draw, minimum size=1cm, anchor=center}, 
            column sep=2mm, row sep=2mm, right=of A] (B) {
        x & x \\
        x & x \\
        x & x \\
    };

    % Resulting matrix (C)
    \matrix[matrix, 
            nodes={draw, minimum size=1cm, anchor=center}, 
            column sep=2mm, row sep=2mm, right=of B] (C) {
        x & x \\
        x & x \\
    };

    % Multiplication sign
    \node at ($(A.east)!0.5!(B.west)$) {$\times$};
    \node at ($(B.east)!0.5!(C.west)$) {$=$};

\end{tikzpicture}
\end{center}

\begin{equation}
{\displaystyle c_{ik}=\sum _{j=1}^{n}a_{ij}\cdot b_{jk}}
\end{equation}


\subsection*{Transposition}
\subsection*{Determinante}
\todo[inline]{A Test for invertibility}
$det \mathbf{A}$ oder auch $|\mathbf{A}|$. 

\begin{itemize}
   \item $det \mathbf{A} = 1$.
   \item Reihen vertauschen kehrt das vorzeichen der Determinante um.
\end{itemize}

$$ \begin{vmatrix} a & b  \\
                    c & d 
 \end{vmatrix} = ad - bc$$

\colab{https://colab.research.google.com/github/hrtlacek/matheFuerTonmeisterinnen/blob/master/python/notebooks/Determinante.ipynb}


\subsection*{Invertierung}

\subsubsection{Diagonalisierung}
 % https://www.youtube.com/watch?v=ZSGrJBS_qtc


\subsection*{Eigenwert und Eigenvektoren}
\index{Eigenwert}\index{Eigenvektor}\index{Zustandsraumdarstellung}\index{State-Space}

\praxis{Eigenwerte der $A$ Matrix in der Zustandsraumdarstellung (engl. 'State-Space') von Systemen sind die Pole des Systems. Eigenvektoren bei der Modalanalyse können Modenformen (die eigentliche Bewegung eines Systems zB.) darstellen und in Eigenwerten sind die entsprechenden Frequenzen ($\Im(\lambda)$) und die Stabilität ($\Re(\lambda)$) enthalten. Eigenvektoren von Matrizen die für Probabilistische Modellierung (siehe Markov Modelle) können Stationäre Wahrscheinlichkeits-Vektoren sein.}

\section{Aufgaben}
\begin{enumerate}
\item \todo[inline]{Aufgabe: Stereo/Vectorscope aus Oszilloscop bauen. Rotation d Vektors um 45 grad. }
\end{enumerate}
