%!TEX root = main.tex
\chapter{Wichtige Funktionen, Polynome, Python}

% \section{Wiederholung}

% \todo{Uebung: Ausdruck entwerfen der 2er Potenzen beschreibt.}

% \todo[]{Funktionsmanipulation im allgemeinen. }

\section{Funktionen Manipulieren}\label{sec:manipFuncts}
Es ist wichtig ein Gefühl zu entwickeln wie sich einfache Rechenoperationen auf Funktionen auswirken und welche 'Audiointerpretation' hier naheliegt. Siehe Abbildung \ref{fig:functManipul}.
Hier eine Auflistung von Interpretationen unter der Annahme dass die horizontale Achse die Zeit $t$ darstellt:
\begin{itemize}
	\item $f(t) + a$: Gleichspannungsversatz hinzufügen/abziehen, Energie bei $0 Hz$, Verschieben der Welle nach 'oben und unten'
	\item $f(t+a)$: für $a<0$: \emph{Delay}, da Verschiebung auf der Zeitachse nach rechts. Besonders wichtig in der digital Technik, Filter etc. 
	\item $f(t) \cdot a$: Amplitudenänderung
	\item $f(t \cdot a)$: Zeitliche Streckung / Stauchung. Schnelleres oder langsameres abspielen, Änderung der Frequenz.

\end{itemize}

\begin{figure}[H]
	\centering
	%% Creator: Matplotlib, PGF backend
%%
%% To include the figure in your LaTeX document, write
%%   \input{<filename>.pgf}
%%
%% Make sure the required packages are loaded in your preamble
%%   \usepackage{pgf}
%%
%% Also ensure that all the required font packages are loaded; for instance,
%% the lmodern package is sometimes necessary when using math font.
%%   \usepackage{lmodern}
%%
%% Figures using additional raster images can only be included by \input if
%% they are in the same directory as the main LaTeX file. For loading figures
%% from other directories you can use the `import` package
%%   \usepackage{import}
%%
%% and then include the figures with
%%   \import{<path to file>}{<filename>.pgf}
%%
%% Matplotlib used the following preamble
%%   \def\mathdefault#1{#1}
%%   \everymath=\expandafter{\the\everymath\displaystyle}
%%   
%%   \usepackage{fontspec}
%%   \setmainfont{VeraSe.ttf}[Path=\detokenize{/usr/share/fonts/TTF/}]
%%   \setsansfont{DejaVuSans.ttf}[Path=\detokenize{/home/pl/miniconda3/lib/python3.12/site-packages/matplotlib/mpl-data/fonts/ttf/}]
%%   \setmonofont{DejaVuSansMono.ttf}[Path=\detokenize{/home/pl/miniconda3/lib/python3.12/site-packages/matplotlib/mpl-data/fonts/ttf/}]
%%   \makeatletter\@ifpackageloaded{underscore}{}{\usepackage[strings]{underscore}}\makeatother
%%
\begingroup%
\makeatletter%
\begin{pgfpicture}%
\pgfpathrectangle{\pgfpointorigin}{\pgfqpoint{5.254052in}{5.321247in}}%
\pgfusepath{use as bounding box, clip}%
\begin{pgfscope}%
\pgfsetbuttcap%
\pgfsetmiterjoin%
\definecolor{currentfill}{rgb}{1.000000,1.000000,1.000000}%
\pgfsetfillcolor{currentfill}%
\pgfsetlinewidth{0.000000pt}%
\definecolor{currentstroke}{rgb}{1.000000,1.000000,1.000000}%
\pgfsetstrokecolor{currentstroke}%
\pgfsetdash{}{0pt}%
\pgfpathmoveto{\pgfqpoint{0.000000in}{-0.000000in}}%
\pgfpathlineto{\pgfqpoint{5.254052in}{-0.000000in}}%
\pgfpathlineto{\pgfqpoint{5.254052in}{5.321247in}}%
\pgfpathlineto{\pgfqpoint{0.000000in}{5.321247in}}%
\pgfpathlineto{\pgfqpoint{0.000000in}{-0.000000in}}%
\pgfpathclose%
\pgfusepath{fill}%
\end{pgfscope}%
\begin{pgfscope}%
\pgfsetbuttcap%
\pgfsetmiterjoin%
\definecolor{currentfill}{rgb}{1.000000,1.000000,1.000000}%
\pgfsetfillcolor{currentfill}%
\pgfsetlinewidth{0.000000pt}%
\definecolor{currentstroke}{rgb}{0.000000,0.000000,0.000000}%
\pgfsetstrokecolor{currentstroke}%
\pgfsetstrokeopacity{0.000000}%
\pgfsetdash{}{0pt}%
\pgfpathmoveto{\pgfqpoint{0.393613in}{3.068486in}}%
\pgfpathlineto{\pgfqpoint{2.507249in}{3.068486in}}%
\pgfpathlineto{\pgfqpoint{2.507249in}{5.168486in}}%
\pgfpathlineto{\pgfqpoint{0.393613in}{5.168486in}}%
\pgfpathlineto{\pgfqpoint{0.393613in}{3.068486in}}%
\pgfpathclose%
\pgfusepath{fill}%
\end{pgfscope}%
\begin{pgfscope}%
\pgfsetbuttcap%
\pgfsetroundjoin%
\definecolor{currentfill}{rgb}{0.000000,0.000000,0.000000}%
\pgfsetfillcolor{currentfill}%
\pgfsetlinewidth{0.803000pt}%
\definecolor{currentstroke}{rgb}{0.000000,0.000000,0.000000}%
\pgfsetstrokecolor{currentstroke}%
\pgfsetdash{}{0pt}%
\pgfsys@defobject{currentmarker}{\pgfqpoint{0.000000in}{-0.048611in}}{\pgfqpoint{0.000000in}{0.000000in}}{%
\pgfpathmoveto{\pgfqpoint{0.000000in}{0.000000in}}%
\pgfpathlineto{\pgfqpoint{0.000000in}{-0.048611in}}%
\pgfusepath{stroke,fill}%
}%
\begin{pgfscope}%
\pgfsys@transformshift{0.393613in}{3.068486in}%
\pgfsys@useobject{currentmarker}{}%
\end{pgfscope}%
\end{pgfscope}%
\begin{pgfscope}%
\definecolor{textcolor}{rgb}{0.000000,0.000000,0.000000}%
\pgfsetstrokecolor{textcolor}%
\pgfsetfillcolor{textcolor}%
\pgftext[x=0.393613in,y=2.971264in,,top]{\color{textcolor}{\rmfamily\fontsize{10.000000}{12.000000}\selectfont\catcode`\^=\active\def^{\ifmmode\sp\else\^{}\fi}\catcode`\%=\active\def%{\%}\ensuremath{-}1.0}}%
\end{pgfscope}%
\begin{pgfscope}%
\pgfsetbuttcap%
\pgfsetroundjoin%
\definecolor{currentfill}{rgb}{0.000000,0.000000,0.000000}%
\pgfsetfillcolor{currentfill}%
\pgfsetlinewidth{0.803000pt}%
\definecolor{currentstroke}{rgb}{0.000000,0.000000,0.000000}%
\pgfsetstrokecolor{currentstroke}%
\pgfsetdash{}{0pt}%
\pgfsys@defobject{currentmarker}{\pgfqpoint{0.000000in}{-0.048611in}}{\pgfqpoint{0.000000in}{0.000000in}}{%
\pgfpathmoveto{\pgfqpoint{0.000000in}{0.000000in}}%
\pgfpathlineto{\pgfqpoint{0.000000in}{-0.048611in}}%
\pgfusepath{stroke,fill}%
}%
\begin{pgfscope}%
\pgfsys@transformshift{0.922022in}{3.068486in}%
\pgfsys@useobject{currentmarker}{}%
\end{pgfscope}%
\end{pgfscope}%
\begin{pgfscope}%
\definecolor{textcolor}{rgb}{0.000000,0.000000,0.000000}%
\pgfsetstrokecolor{textcolor}%
\pgfsetfillcolor{textcolor}%
\pgftext[x=0.922022in,y=2.971264in,,top]{\color{textcolor}{\rmfamily\fontsize{10.000000}{12.000000}\selectfont\catcode`\^=\active\def^{\ifmmode\sp\else\^{}\fi}\catcode`\%=\active\def%{\%}\ensuremath{-}0.5}}%
\end{pgfscope}%
\begin{pgfscope}%
\pgfsetbuttcap%
\pgfsetroundjoin%
\definecolor{currentfill}{rgb}{0.000000,0.000000,0.000000}%
\pgfsetfillcolor{currentfill}%
\pgfsetlinewidth{0.803000pt}%
\definecolor{currentstroke}{rgb}{0.000000,0.000000,0.000000}%
\pgfsetstrokecolor{currentstroke}%
\pgfsetdash{}{0pt}%
\pgfsys@defobject{currentmarker}{\pgfqpoint{0.000000in}{-0.048611in}}{\pgfqpoint{0.000000in}{0.000000in}}{%
\pgfpathmoveto{\pgfqpoint{0.000000in}{0.000000in}}%
\pgfpathlineto{\pgfqpoint{0.000000in}{-0.048611in}}%
\pgfusepath{stroke,fill}%
}%
\begin{pgfscope}%
\pgfsys@transformshift{1.450431in}{3.068486in}%
\pgfsys@useobject{currentmarker}{}%
\end{pgfscope}%
\end{pgfscope}%
\begin{pgfscope}%
\definecolor{textcolor}{rgb}{0.000000,0.000000,0.000000}%
\pgfsetstrokecolor{textcolor}%
\pgfsetfillcolor{textcolor}%
\pgftext[x=1.450431in,y=2.971264in,,top]{\color{textcolor}{\rmfamily\fontsize{10.000000}{12.000000}\selectfont\catcode`\^=\active\def^{\ifmmode\sp\else\^{}\fi}\catcode`\%=\active\def%{\%}0.0}}%
\end{pgfscope}%
\begin{pgfscope}%
\pgfsetbuttcap%
\pgfsetroundjoin%
\definecolor{currentfill}{rgb}{0.000000,0.000000,0.000000}%
\pgfsetfillcolor{currentfill}%
\pgfsetlinewidth{0.803000pt}%
\definecolor{currentstroke}{rgb}{0.000000,0.000000,0.000000}%
\pgfsetstrokecolor{currentstroke}%
\pgfsetdash{}{0pt}%
\pgfsys@defobject{currentmarker}{\pgfqpoint{0.000000in}{-0.048611in}}{\pgfqpoint{0.000000in}{0.000000in}}{%
\pgfpathmoveto{\pgfqpoint{0.000000in}{0.000000in}}%
\pgfpathlineto{\pgfqpoint{0.000000in}{-0.048611in}}%
\pgfusepath{stroke,fill}%
}%
\begin{pgfscope}%
\pgfsys@transformshift{1.978840in}{3.068486in}%
\pgfsys@useobject{currentmarker}{}%
\end{pgfscope}%
\end{pgfscope}%
\begin{pgfscope}%
\definecolor{textcolor}{rgb}{0.000000,0.000000,0.000000}%
\pgfsetstrokecolor{textcolor}%
\pgfsetfillcolor{textcolor}%
\pgftext[x=1.978840in,y=2.971264in,,top]{\color{textcolor}{\rmfamily\fontsize{10.000000}{12.000000}\selectfont\catcode`\^=\active\def^{\ifmmode\sp\else\^{}\fi}\catcode`\%=\active\def%{\%}0.5}}%
\end{pgfscope}%
\begin{pgfscope}%
\pgfsetbuttcap%
\pgfsetroundjoin%
\definecolor{currentfill}{rgb}{0.000000,0.000000,0.000000}%
\pgfsetfillcolor{currentfill}%
\pgfsetlinewidth{0.803000pt}%
\definecolor{currentstroke}{rgb}{0.000000,0.000000,0.000000}%
\pgfsetstrokecolor{currentstroke}%
\pgfsetdash{}{0pt}%
\pgfsys@defobject{currentmarker}{\pgfqpoint{0.000000in}{-0.048611in}}{\pgfqpoint{0.000000in}{0.000000in}}{%
\pgfpathmoveto{\pgfqpoint{0.000000in}{0.000000in}}%
\pgfpathlineto{\pgfqpoint{0.000000in}{-0.048611in}}%
\pgfusepath{stroke,fill}%
}%
\begin{pgfscope}%
\pgfsys@transformshift{2.507249in}{3.068486in}%
\pgfsys@useobject{currentmarker}{}%
\end{pgfscope}%
\end{pgfscope}%
\begin{pgfscope}%
\definecolor{textcolor}{rgb}{0.000000,0.000000,0.000000}%
\pgfsetstrokecolor{textcolor}%
\pgfsetfillcolor{textcolor}%
\pgftext[x=2.507249in,y=2.971264in,,top]{\color{textcolor}{\rmfamily\fontsize{10.000000}{12.000000}\selectfont\catcode`\^=\active\def^{\ifmmode\sp\else\^{}\fi}\catcode`\%=\active\def%{\%}1.0}}%
\end{pgfscope}%
\begin{pgfscope}%
\definecolor{textcolor}{rgb}{0.000000,0.000000,0.000000}%
\pgfsetstrokecolor{textcolor}%
\pgfsetfillcolor{textcolor}%
\pgftext[x=1.450431in,y=2.781295in,,top]{\color{textcolor}{\rmfamily\fontsize{12.000000}{14.400000}\selectfont\catcode`\^=\active\def^{\ifmmode\sp\else\^{}\fi}\catcode`\%=\active\def%{\%}$t$}}%
\end{pgfscope}%
\begin{pgfscope}%
\pgfsetbuttcap%
\pgfsetroundjoin%
\definecolor{currentfill}{rgb}{0.000000,0.000000,0.000000}%
\pgfsetfillcolor{currentfill}%
\pgfsetlinewidth{0.803000pt}%
\definecolor{currentstroke}{rgb}{0.000000,0.000000,0.000000}%
\pgfsetstrokecolor{currentstroke}%
\pgfsetdash{}{0pt}%
\pgfsys@defobject{currentmarker}{\pgfqpoint{-0.048611in}{0.000000in}}{\pgfqpoint{-0.000000in}{0.000000in}}{%
\pgfpathmoveto{\pgfqpoint{-0.000000in}{0.000000in}}%
\pgfpathlineto{\pgfqpoint{-0.048611in}{0.000000in}}%
\pgfusepath{stroke,fill}%
}%
\begin{pgfscope}%
\pgfsys@transformshift{0.393613in}{3.068486in}%
\pgfsys@useobject{currentmarker}{}%
\end{pgfscope}%
\end{pgfscope}%
\begin{pgfscope}%
\definecolor{textcolor}{rgb}{0.000000,0.000000,0.000000}%
\pgfsetstrokecolor{textcolor}%
\pgfsetfillcolor{textcolor}%
\pgftext[x=0.100000in, y=3.015724in, left, base]{\color{textcolor}{\rmfamily\fontsize{10.000000}{12.000000}\selectfont\catcode`\^=\active\def^{\ifmmode\sp\else\^{}\fi}\catcode`\%=\active\def%{\%}\ensuremath{-}2}}%
\end{pgfscope}%
\begin{pgfscope}%
\pgfsetbuttcap%
\pgfsetroundjoin%
\definecolor{currentfill}{rgb}{0.000000,0.000000,0.000000}%
\pgfsetfillcolor{currentfill}%
\pgfsetlinewidth{0.803000pt}%
\definecolor{currentstroke}{rgb}{0.000000,0.000000,0.000000}%
\pgfsetstrokecolor{currentstroke}%
\pgfsetdash{}{0pt}%
\pgfsys@defobject{currentmarker}{\pgfqpoint{-0.048611in}{0.000000in}}{\pgfqpoint{-0.000000in}{0.000000in}}{%
\pgfpathmoveto{\pgfqpoint{-0.000000in}{0.000000in}}%
\pgfpathlineto{\pgfqpoint{-0.048611in}{0.000000in}}%
\pgfusepath{stroke,fill}%
}%
\begin{pgfscope}%
\pgfsys@transformshift{0.393613in}{3.593486in}%
\pgfsys@useobject{currentmarker}{}%
\end{pgfscope}%
\end{pgfscope}%
\begin{pgfscope}%
\definecolor{textcolor}{rgb}{0.000000,0.000000,0.000000}%
\pgfsetstrokecolor{textcolor}%
\pgfsetfillcolor{textcolor}%
\pgftext[x=0.100000in, y=3.540724in, left, base]{\color{textcolor}{\rmfamily\fontsize{10.000000}{12.000000}\selectfont\catcode`\^=\active\def^{\ifmmode\sp\else\^{}\fi}\catcode`\%=\active\def%{\%}\ensuremath{-}1}}%
\end{pgfscope}%
\begin{pgfscope}%
\pgfsetbuttcap%
\pgfsetroundjoin%
\definecolor{currentfill}{rgb}{0.000000,0.000000,0.000000}%
\pgfsetfillcolor{currentfill}%
\pgfsetlinewidth{0.803000pt}%
\definecolor{currentstroke}{rgb}{0.000000,0.000000,0.000000}%
\pgfsetstrokecolor{currentstroke}%
\pgfsetdash{}{0pt}%
\pgfsys@defobject{currentmarker}{\pgfqpoint{-0.048611in}{0.000000in}}{\pgfqpoint{-0.000000in}{0.000000in}}{%
\pgfpathmoveto{\pgfqpoint{-0.000000in}{0.000000in}}%
\pgfpathlineto{\pgfqpoint{-0.048611in}{0.000000in}}%
\pgfusepath{stroke,fill}%
}%
\begin{pgfscope}%
\pgfsys@transformshift{0.393613in}{4.118486in}%
\pgfsys@useobject{currentmarker}{}%
\end{pgfscope}%
\end{pgfscope}%
\begin{pgfscope}%
\definecolor{textcolor}{rgb}{0.000000,0.000000,0.000000}%
\pgfsetstrokecolor{textcolor}%
\pgfsetfillcolor{textcolor}%
\pgftext[x=0.208025in, y=4.065724in, left, base]{\color{textcolor}{\rmfamily\fontsize{10.000000}{12.000000}\selectfont\catcode`\^=\active\def^{\ifmmode\sp\else\^{}\fi}\catcode`\%=\active\def%{\%}0}}%
\end{pgfscope}%
\begin{pgfscope}%
\pgfsetbuttcap%
\pgfsetroundjoin%
\definecolor{currentfill}{rgb}{0.000000,0.000000,0.000000}%
\pgfsetfillcolor{currentfill}%
\pgfsetlinewidth{0.803000pt}%
\definecolor{currentstroke}{rgb}{0.000000,0.000000,0.000000}%
\pgfsetstrokecolor{currentstroke}%
\pgfsetdash{}{0pt}%
\pgfsys@defobject{currentmarker}{\pgfqpoint{-0.048611in}{0.000000in}}{\pgfqpoint{-0.000000in}{0.000000in}}{%
\pgfpathmoveto{\pgfqpoint{-0.000000in}{0.000000in}}%
\pgfpathlineto{\pgfqpoint{-0.048611in}{0.000000in}}%
\pgfusepath{stroke,fill}%
}%
\begin{pgfscope}%
\pgfsys@transformshift{0.393613in}{4.643486in}%
\pgfsys@useobject{currentmarker}{}%
\end{pgfscope}%
\end{pgfscope}%
\begin{pgfscope}%
\definecolor{textcolor}{rgb}{0.000000,0.000000,0.000000}%
\pgfsetstrokecolor{textcolor}%
\pgfsetfillcolor{textcolor}%
\pgftext[x=0.208025in, y=4.590724in, left, base]{\color{textcolor}{\rmfamily\fontsize{10.000000}{12.000000}\selectfont\catcode`\^=\active\def^{\ifmmode\sp\else\^{}\fi}\catcode`\%=\active\def%{\%}1}}%
\end{pgfscope}%
\begin{pgfscope}%
\pgfsetbuttcap%
\pgfsetroundjoin%
\definecolor{currentfill}{rgb}{0.000000,0.000000,0.000000}%
\pgfsetfillcolor{currentfill}%
\pgfsetlinewidth{0.803000pt}%
\definecolor{currentstroke}{rgb}{0.000000,0.000000,0.000000}%
\pgfsetstrokecolor{currentstroke}%
\pgfsetdash{}{0pt}%
\pgfsys@defobject{currentmarker}{\pgfqpoint{-0.048611in}{0.000000in}}{\pgfqpoint{-0.000000in}{0.000000in}}{%
\pgfpathmoveto{\pgfqpoint{-0.000000in}{0.000000in}}%
\pgfpathlineto{\pgfqpoint{-0.048611in}{0.000000in}}%
\pgfusepath{stroke,fill}%
}%
\begin{pgfscope}%
\pgfsys@transformshift{0.393613in}{5.168486in}%
\pgfsys@useobject{currentmarker}{}%
\end{pgfscope}%
\end{pgfscope}%
\begin{pgfscope}%
\definecolor{textcolor}{rgb}{0.000000,0.000000,0.000000}%
\pgfsetstrokecolor{textcolor}%
\pgfsetfillcolor{textcolor}%
\pgftext[x=0.208025in, y=5.115724in, left, base]{\color{textcolor}{\rmfamily\fontsize{10.000000}{12.000000}\selectfont\catcode`\^=\active\def^{\ifmmode\sp\else\^{}\fi}\catcode`\%=\active\def%{\%}2}}%
\end{pgfscope}%
\begin{pgfscope}%
\pgfpathrectangle{\pgfqpoint{0.393613in}{3.068486in}}{\pgfqpoint{2.113636in}{2.100000in}}%
\pgfusepath{clip}%
\pgfsetbuttcap%
\pgfsetroundjoin%
\pgfsetlinewidth{0.501875pt}%
\definecolor{currentstroke}{rgb}{0.000000,0.000000,0.000000}%
\pgfsetstrokecolor{currentstroke}%
\pgfsetdash{{1.850000pt}{0.800000pt}}{0.000000pt}%
\pgfpathmoveto{\pgfqpoint{0.393613in}{4.118486in}}%
\pgfpathlineto{\pgfqpoint{2.507249in}{4.118486in}}%
\pgfusepath{stroke}%
\end{pgfscope}%
\begin{pgfscope}%
\pgfpathrectangle{\pgfqpoint{0.393613in}{3.068486in}}{\pgfqpoint{2.113636in}{2.100000in}}%
\pgfusepath{clip}%
\pgfsetbuttcap%
\pgfsetroundjoin%
\pgfsetlinewidth{0.501875pt}%
\definecolor{currentstroke}{rgb}{0.000000,0.000000,0.000000}%
\pgfsetstrokecolor{currentstroke}%
\pgfsetdash{{1.850000pt}{0.800000pt}}{0.000000pt}%
\pgfpathmoveto{\pgfqpoint{1.450431in}{3.068486in}}%
\pgfpathlineto{\pgfqpoint{1.450431in}{5.168486in}}%
\pgfusepath{stroke}%
\end{pgfscope}%
\begin{pgfscope}%
\pgfpathrectangle{\pgfqpoint{0.393613in}{3.068486in}}{\pgfqpoint{2.113636in}{2.100000in}}%
\pgfusepath{clip}%
\pgfsetrectcap%
\pgfsetroundjoin%
\pgfsetlinewidth{1.505625pt}%
\definecolor{currentstroke}{rgb}{0.000000,0.000000,0.000000}%
\pgfsetstrokecolor{currentstroke}%
\pgfsetdash{}{0pt}%
\pgfpathmoveto{\pgfqpoint{0.393613in}{3.952409in}}%
\pgfpathlineto{\pgfqpoint{0.438844in}{3.855579in}}%
\pgfpathlineto{\pgfqpoint{0.524235in}{3.672104in}}%
\pgfpathlineto{\pgfqpoint{0.558053in}{3.605241in}}%
\pgfpathlineto{\pgfqpoint{0.586376in}{3.554026in}}%
\pgfpathlineto{\pgfqpoint{0.610894in}{3.514098in}}%
\pgfpathlineto{\pgfqpoint{0.633299in}{3.481760in}}%
\pgfpathlineto{\pgfqpoint{0.653590in}{3.456269in}}%
\pgfpathlineto{\pgfqpoint{0.672190in}{3.436333in}}%
\pgfpathlineto{\pgfqpoint{0.689522in}{3.420888in}}%
\pgfpathlineto{\pgfqpoint{0.706008in}{3.409130in}}%
\pgfpathlineto{\pgfqpoint{0.721226in}{3.400906in}}%
\pgfpathlineto{\pgfqpoint{0.736022in}{3.395395in}}%
\pgfpathlineto{\pgfqpoint{0.750394in}{3.392434in}}%
\pgfpathlineto{\pgfqpoint{0.764344in}{3.391847in}}%
\pgfpathlineto{\pgfqpoint{0.778294in}{3.393533in}}%
\pgfpathlineto{\pgfqpoint{0.792244in}{3.397504in}}%
\pgfpathlineto{\pgfqpoint{0.806617in}{3.403985in}}%
\pgfpathlineto{\pgfqpoint{0.821413in}{3.413180in}}%
\pgfpathlineto{\pgfqpoint{0.836631in}{3.425283in}}%
\pgfpathlineto{\pgfqpoint{0.852694in}{3.440925in}}%
\pgfpathlineto{\pgfqpoint{0.870026in}{3.461035in}}%
\pgfpathlineto{\pgfqpoint{0.888203in}{3.485620in}}%
\pgfpathlineto{\pgfqpoint{0.907649in}{3.515725in}}%
\pgfpathlineto{\pgfqpoint{0.928785in}{3.552670in}}%
\pgfpathlineto{\pgfqpoint{0.951612in}{3.597165in}}%
\pgfpathlineto{\pgfqpoint{0.976976in}{3.651676in}}%
\pgfpathlineto{\pgfqpoint{1.005299in}{3.718049in}}%
\pgfpathlineto{\pgfqpoint{1.038272in}{3.801291in}}%
\pgfpathlineto{\pgfqpoint{1.079699in}{3.912339in}}%
\pgfpathlineto{\pgfqpoint{1.214972in}{4.279868in}}%
\pgfpathlineto{\pgfqpoint{1.247522in}{4.359104in}}%
\pgfpathlineto{\pgfqpoint{1.275422in}{4.421406in}}%
\pgfpathlineto{\pgfqpoint{1.300363in}{4.471896in}}%
\pgfpathlineto{\pgfqpoint{1.323190in}{4.513267in}}%
\pgfpathlineto{\pgfqpoint{1.344326in}{4.547086in}}%
\pgfpathlineto{\pgfqpoint{1.363772in}{4.574144in}}%
\pgfpathlineto{\pgfqpoint{1.381949in}{4.595758in}}%
\pgfpathlineto{\pgfqpoint{1.398858in}{4.612563in}}%
\pgfpathlineto{\pgfqpoint{1.414922in}{4.625511in}}%
\pgfpathlineto{\pgfqpoint{1.418303in}{4.627857in}}%
\pgfpathlineto{\pgfqpoint{1.419572in}{4.786202in}}%
\pgfpathlineto{\pgfqpoint{1.434790in}{4.794887in}}%
\pgfpathlineto{\pgfqpoint{1.449162in}{4.800599in}}%
\pgfpathlineto{\pgfqpoint{1.463113in}{4.803821in}}%
\pgfpathlineto{\pgfqpoint{1.477063in}{4.804762in}}%
\pgfpathlineto{\pgfqpoint{1.481712in}{4.804571in}}%
\pgfpathlineto{\pgfqpoint{1.482981in}{4.646975in}}%
\pgfpathlineto{\pgfqpoint{1.496931in}{4.644692in}}%
\pgfpathlineto{\pgfqpoint{1.511303in}{4.640007in}}%
\pgfpathlineto{\pgfqpoint{1.526099in}{4.632755in}}%
\pgfpathlineto{\pgfqpoint{1.541317in}{4.622783in}}%
\pgfpathlineto{\pgfqpoint{1.557381in}{4.609576in}}%
\pgfpathlineto{\pgfqpoint{1.574712in}{4.592357in}}%
\pgfpathlineto{\pgfqpoint{1.593312in}{4.570617in}}%
\pgfpathlineto{\pgfqpoint{1.613181in}{4.543899in}}%
\pgfpathlineto{\pgfqpoint{1.635162in}{4.510474in}}%
\pgfpathlineto{\pgfqpoint{1.659258in}{4.469669in}}%
\pgfpathlineto{\pgfqpoint{1.686735in}{4.418598in}}%
\pgfpathlineto{\pgfqpoint{1.719285in}{4.353166in}}%
\pgfpathlineto{\pgfqpoint{1.763249in}{4.259329in}}%
\pgfpathlineto{\pgfqpoint{1.854558in}{4.063266in}}%
\pgfpathlineto{\pgfqpoint{1.887953in}{3.997565in}}%
\pgfpathlineto{\pgfqpoint{1.915853in}{3.947401in}}%
\pgfpathlineto{\pgfqpoint{1.940372in}{3.907735in}}%
\pgfpathlineto{\pgfqpoint{1.962353in}{3.876233in}}%
\pgfpathlineto{\pgfqpoint{1.982644in}{3.850932in}}%
\pgfpathlineto{\pgfqpoint{2.001244in}{3.831182in}}%
\pgfpathlineto{\pgfqpoint{2.018576in}{3.815919in}}%
\pgfpathlineto{\pgfqpoint{2.034640in}{3.804602in}}%
\pgfpathlineto{\pgfqpoint{2.049858in}{3.796478in}}%
\pgfpathlineto{\pgfqpoint{2.064653in}{3.791066in}}%
\pgfpathlineto{\pgfqpoint{2.079026in}{3.788204in}}%
\pgfpathlineto{\pgfqpoint{2.092976in}{3.787713in}}%
\pgfpathlineto{\pgfqpoint{2.106926in}{3.789496in}}%
\pgfpathlineto{\pgfqpoint{2.120876in}{3.793565in}}%
\pgfpathlineto{\pgfqpoint{2.135249in}{3.800146in}}%
\pgfpathlineto{\pgfqpoint{2.150044in}{3.809443in}}%
\pgfpathlineto{\pgfqpoint{2.165263in}{3.821650in}}%
\pgfpathlineto{\pgfqpoint{2.181326in}{3.837401in}}%
\pgfpathlineto{\pgfqpoint{2.198658in}{3.857624in}}%
\pgfpathlineto{\pgfqpoint{2.216835in}{3.882325in}}%
\pgfpathlineto{\pgfqpoint{2.236281in}{3.912546in}}%
\pgfpathlineto{\pgfqpoint{2.257417in}{3.949611in}}%
\pgfpathlineto{\pgfqpoint{2.280244in}{3.994225in}}%
\pgfpathlineto{\pgfqpoint{2.305608in}{4.048854in}}%
\pgfpathlineto{\pgfqpoint{2.333931in}{4.115340in}}%
\pgfpathlineto{\pgfqpoint{2.366903in}{4.198683in}}%
\pgfpathlineto{\pgfqpoint{2.408753in}{4.310970in}}%
\pgfpathlineto{\pgfqpoint{2.507249in}{4.582409in}}%
\pgfpathlineto{\pgfqpoint{2.507249in}{4.582409in}}%
\pgfusepath{stroke}%
\end{pgfscope}%
\begin{pgfscope}%
\pgfpathrectangle{\pgfqpoint{0.393613in}{3.068486in}}{\pgfqpoint{2.113636in}{2.100000in}}%
\pgfusepath{clip}%
\pgfsetbuttcap%
\pgfsetroundjoin%
\pgfsetlinewidth{1.505625pt}%
\definecolor{currentstroke}{rgb}{0.000000,0.000000,0.000000}%
\pgfsetstrokecolor{currentstroke}%
\pgfsetdash{{5.550000pt}{2.400000pt}}{0.000000pt}%
\pgfpathmoveto{\pgfqpoint{0.393613in}{4.214909in}}%
\pgfpathlineto{\pgfqpoint{0.438844in}{4.118079in}}%
\pgfpathlineto{\pgfqpoint{0.524235in}{3.934604in}}%
\pgfpathlineto{\pgfqpoint{0.558053in}{3.867741in}}%
\pgfpathlineto{\pgfqpoint{0.586376in}{3.816526in}}%
\pgfpathlineto{\pgfqpoint{0.610894in}{3.776598in}}%
\pgfpathlineto{\pgfqpoint{0.633299in}{3.744260in}}%
\pgfpathlineto{\pgfqpoint{0.653590in}{3.718769in}}%
\pgfpathlineto{\pgfqpoint{0.672190in}{3.698833in}}%
\pgfpathlineto{\pgfqpoint{0.689522in}{3.683388in}}%
\pgfpathlineto{\pgfqpoint{0.706008in}{3.671630in}}%
\pgfpathlineto{\pgfqpoint{0.721226in}{3.663406in}}%
\pgfpathlineto{\pgfqpoint{0.736022in}{3.657895in}}%
\pgfpathlineto{\pgfqpoint{0.750394in}{3.654934in}}%
\pgfpathlineto{\pgfqpoint{0.764344in}{3.654347in}}%
\pgfpathlineto{\pgfqpoint{0.778294in}{3.656033in}}%
\pgfpathlineto{\pgfqpoint{0.792244in}{3.660004in}}%
\pgfpathlineto{\pgfqpoint{0.806617in}{3.666485in}}%
\pgfpathlineto{\pgfqpoint{0.821413in}{3.675680in}}%
\pgfpathlineto{\pgfqpoint{0.836631in}{3.687783in}}%
\pgfpathlineto{\pgfqpoint{0.852694in}{3.703425in}}%
\pgfpathlineto{\pgfqpoint{0.870026in}{3.723535in}}%
\pgfpathlineto{\pgfqpoint{0.888203in}{3.748120in}}%
\pgfpathlineto{\pgfqpoint{0.907649in}{3.778225in}}%
\pgfpathlineto{\pgfqpoint{0.928785in}{3.815170in}}%
\pgfpathlineto{\pgfqpoint{0.951612in}{3.859665in}}%
\pgfpathlineto{\pgfqpoint{0.976976in}{3.914176in}}%
\pgfpathlineto{\pgfqpoint{1.005299in}{3.980549in}}%
\pgfpathlineto{\pgfqpoint{1.038272in}{4.063791in}}%
\pgfpathlineto{\pgfqpoint{1.079699in}{4.174839in}}%
\pgfpathlineto{\pgfqpoint{1.214972in}{4.542368in}}%
\pgfpathlineto{\pgfqpoint{1.247522in}{4.621604in}}%
\pgfpathlineto{\pgfqpoint{1.275422in}{4.683906in}}%
\pgfpathlineto{\pgfqpoint{1.300363in}{4.734396in}}%
\pgfpathlineto{\pgfqpoint{1.323190in}{4.775767in}}%
\pgfpathlineto{\pgfqpoint{1.344326in}{4.809586in}}%
\pgfpathlineto{\pgfqpoint{1.363772in}{4.836644in}}%
\pgfpathlineto{\pgfqpoint{1.381949in}{4.858258in}}%
\pgfpathlineto{\pgfqpoint{1.398858in}{4.875063in}}%
\pgfpathlineto{\pgfqpoint{1.414922in}{4.888011in}}%
\pgfpathlineto{\pgfqpoint{1.418303in}{4.890357in}}%
\pgfpathlineto{\pgfqpoint{1.419572in}{5.048702in}}%
\pgfpathlineto{\pgfqpoint{1.434790in}{5.057387in}}%
\pgfpathlineto{\pgfqpoint{1.449162in}{5.063099in}}%
\pgfpathlineto{\pgfqpoint{1.463113in}{5.066321in}}%
\pgfpathlineto{\pgfqpoint{1.477063in}{5.067262in}}%
\pgfpathlineto{\pgfqpoint{1.481712in}{5.067071in}}%
\pgfpathlineto{\pgfqpoint{1.482981in}{4.909475in}}%
\pgfpathlineto{\pgfqpoint{1.496931in}{4.907192in}}%
\pgfpathlineto{\pgfqpoint{1.511303in}{4.902507in}}%
\pgfpathlineto{\pgfqpoint{1.526099in}{4.895255in}}%
\pgfpathlineto{\pgfqpoint{1.541317in}{4.885283in}}%
\pgfpathlineto{\pgfqpoint{1.557381in}{4.872076in}}%
\pgfpathlineto{\pgfqpoint{1.574712in}{4.854857in}}%
\pgfpathlineto{\pgfqpoint{1.593312in}{4.833117in}}%
\pgfpathlineto{\pgfqpoint{1.613181in}{4.806399in}}%
\pgfpathlineto{\pgfqpoint{1.635162in}{4.772974in}}%
\pgfpathlineto{\pgfqpoint{1.659258in}{4.732169in}}%
\pgfpathlineto{\pgfqpoint{1.686735in}{4.681098in}}%
\pgfpathlineto{\pgfqpoint{1.719285in}{4.615666in}}%
\pgfpathlineto{\pgfqpoint{1.763249in}{4.521829in}}%
\pgfpathlineto{\pgfqpoint{1.854558in}{4.325766in}}%
\pgfpathlineto{\pgfqpoint{1.887953in}{4.260065in}}%
\pgfpathlineto{\pgfqpoint{1.915853in}{4.209901in}}%
\pgfpathlineto{\pgfqpoint{1.940372in}{4.170235in}}%
\pgfpathlineto{\pgfqpoint{1.962353in}{4.138733in}}%
\pgfpathlineto{\pgfqpoint{1.982644in}{4.113432in}}%
\pgfpathlineto{\pgfqpoint{2.001244in}{4.093682in}}%
\pgfpathlineto{\pgfqpoint{2.018576in}{4.078419in}}%
\pgfpathlineto{\pgfqpoint{2.034640in}{4.067102in}}%
\pgfpathlineto{\pgfqpoint{2.049858in}{4.058978in}}%
\pgfpathlineto{\pgfqpoint{2.064653in}{4.053566in}}%
\pgfpathlineto{\pgfqpoint{2.079026in}{4.050704in}}%
\pgfpathlineto{\pgfqpoint{2.092976in}{4.050213in}}%
\pgfpathlineto{\pgfqpoint{2.106926in}{4.051996in}}%
\pgfpathlineto{\pgfqpoint{2.120876in}{4.056065in}}%
\pgfpathlineto{\pgfqpoint{2.135249in}{4.062646in}}%
\pgfpathlineto{\pgfqpoint{2.150044in}{4.071943in}}%
\pgfpathlineto{\pgfqpoint{2.165263in}{4.084150in}}%
\pgfpathlineto{\pgfqpoint{2.181326in}{4.099901in}}%
\pgfpathlineto{\pgfqpoint{2.198658in}{4.120124in}}%
\pgfpathlineto{\pgfqpoint{2.216835in}{4.144825in}}%
\pgfpathlineto{\pgfqpoint{2.236281in}{4.175046in}}%
\pgfpathlineto{\pgfqpoint{2.257417in}{4.212111in}}%
\pgfpathlineto{\pgfqpoint{2.280244in}{4.256725in}}%
\pgfpathlineto{\pgfqpoint{2.305608in}{4.311354in}}%
\pgfpathlineto{\pgfqpoint{2.333931in}{4.377840in}}%
\pgfpathlineto{\pgfqpoint{2.366903in}{4.461183in}}%
\pgfpathlineto{\pgfqpoint{2.408753in}{4.573470in}}%
\pgfpathlineto{\pgfqpoint{2.507249in}{4.844909in}}%
\pgfpathlineto{\pgfqpoint{2.507249in}{4.844909in}}%
\pgfusepath{stroke}%
\end{pgfscope}%
\begin{pgfscope}%
\pgfpathrectangle{\pgfqpoint{0.393613in}{3.068486in}}{\pgfqpoint{2.113636in}{2.100000in}}%
\pgfusepath{clip}%
\pgfsetbuttcap%
\pgfsetroundjoin%
\pgfsetlinewidth{1.505625pt}%
\definecolor{currentstroke}{rgb}{0.000000,0.000000,0.000000}%
\pgfsetstrokecolor{currentstroke}%
\pgfsetdash{{1.500000pt}{2.475000pt}}{0.000000pt}%
\pgfpathmoveto{\pgfqpoint{0.393613in}{3.689909in}}%
\pgfpathlineto{\pgfqpoint{0.438844in}{3.593079in}}%
\pgfpathlineto{\pgfqpoint{0.524235in}{3.409604in}}%
\pgfpathlineto{\pgfqpoint{0.558053in}{3.342741in}}%
\pgfpathlineto{\pgfqpoint{0.586376in}{3.291526in}}%
\pgfpathlineto{\pgfqpoint{0.610894in}{3.251598in}}%
\pgfpathlineto{\pgfqpoint{0.633299in}{3.219260in}}%
\pgfpathlineto{\pgfqpoint{0.653590in}{3.193769in}}%
\pgfpathlineto{\pgfqpoint{0.672190in}{3.173833in}}%
\pgfpathlineto{\pgfqpoint{0.689522in}{3.158388in}}%
\pgfpathlineto{\pgfqpoint{0.706008in}{3.146630in}}%
\pgfpathlineto{\pgfqpoint{0.721226in}{3.138406in}}%
\pgfpathlineto{\pgfqpoint{0.736022in}{3.132895in}}%
\pgfpathlineto{\pgfqpoint{0.750394in}{3.129934in}}%
\pgfpathlineto{\pgfqpoint{0.764344in}{3.129347in}}%
\pgfpathlineto{\pgfqpoint{0.778294in}{3.131033in}}%
\pgfpathlineto{\pgfqpoint{0.792244in}{3.135004in}}%
\pgfpathlineto{\pgfqpoint{0.806617in}{3.141485in}}%
\pgfpathlineto{\pgfqpoint{0.821413in}{3.150680in}}%
\pgfpathlineto{\pgfqpoint{0.836631in}{3.162783in}}%
\pgfpathlineto{\pgfqpoint{0.852694in}{3.178425in}}%
\pgfpathlineto{\pgfqpoint{0.870026in}{3.198535in}}%
\pgfpathlineto{\pgfqpoint{0.888203in}{3.223120in}}%
\pgfpathlineto{\pgfqpoint{0.907649in}{3.253225in}}%
\pgfpathlineto{\pgfqpoint{0.928785in}{3.290170in}}%
\pgfpathlineto{\pgfqpoint{0.951612in}{3.334665in}}%
\pgfpathlineto{\pgfqpoint{0.976976in}{3.389176in}}%
\pgfpathlineto{\pgfqpoint{1.005299in}{3.455549in}}%
\pgfpathlineto{\pgfqpoint{1.038272in}{3.538791in}}%
\pgfpathlineto{\pgfqpoint{1.079699in}{3.649839in}}%
\pgfpathlineto{\pgfqpoint{1.214972in}{4.017368in}}%
\pgfpathlineto{\pgfqpoint{1.247522in}{4.096604in}}%
\pgfpathlineto{\pgfqpoint{1.275422in}{4.158906in}}%
\pgfpathlineto{\pgfqpoint{1.300363in}{4.209396in}}%
\pgfpathlineto{\pgfqpoint{1.323190in}{4.250767in}}%
\pgfpathlineto{\pgfqpoint{1.344326in}{4.284586in}}%
\pgfpathlineto{\pgfqpoint{1.363772in}{4.311644in}}%
\pgfpathlineto{\pgfqpoint{1.381949in}{4.333258in}}%
\pgfpathlineto{\pgfqpoint{1.398858in}{4.350063in}}%
\pgfpathlineto{\pgfqpoint{1.414922in}{4.363011in}}%
\pgfpathlineto{\pgfqpoint{1.418303in}{4.365357in}}%
\pgfpathlineto{\pgfqpoint{1.419572in}{4.523702in}}%
\pgfpathlineto{\pgfqpoint{1.434790in}{4.532387in}}%
\pgfpathlineto{\pgfqpoint{1.449162in}{4.538099in}}%
\pgfpathlineto{\pgfqpoint{1.463113in}{4.541321in}}%
\pgfpathlineto{\pgfqpoint{1.477063in}{4.542262in}}%
\pgfpathlineto{\pgfqpoint{1.481712in}{4.542071in}}%
\pgfpathlineto{\pgfqpoint{1.482981in}{4.384475in}}%
\pgfpathlineto{\pgfqpoint{1.496931in}{4.382192in}}%
\pgfpathlineto{\pgfqpoint{1.511303in}{4.377507in}}%
\pgfpathlineto{\pgfqpoint{1.526099in}{4.370255in}}%
\pgfpathlineto{\pgfqpoint{1.541317in}{4.360283in}}%
\pgfpathlineto{\pgfqpoint{1.557381in}{4.347076in}}%
\pgfpathlineto{\pgfqpoint{1.574712in}{4.329857in}}%
\pgfpathlineto{\pgfqpoint{1.593312in}{4.308117in}}%
\pgfpathlineto{\pgfqpoint{1.613181in}{4.281399in}}%
\pgfpathlineto{\pgfqpoint{1.635162in}{4.247974in}}%
\pgfpathlineto{\pgfqpoint{1.659258in}{4.207169in}}%
\pgfpathlineto{\pgfqpoint{1.686735in}{4.156098in}}%
\pgfpathlineto{\pgfqpoint{1.719285in}{4.090666in}}%
\pgfpathlineto{\pgfqpoint{1.763249in}{3.996829in}}%
\pgfpathlineto{\pgfqpoint{1.854558in}{3.800766in}}%
\pgfpathlineto{\pgfqpoint{1.887953in}{3.735065in}}%
\pgfpathlineto{\pgfqpoint{1.915853in}{3.684901in}}%
\pgfpathlineto{\pgfqpoint{1.940372in}{3.645235in}}%
\pgfpathlineto{\pgfqpoint{1.962353in}{3.613733in}}%
\pgfpathlineto{\pgfqpoint{1.982644in}{3.588432in}}%
\pgfpathlineto{\pgfqpoint{2.001244in}{3.568682in}}%
\pgfpathlineto{\pgfqpoint{2.018576in}{3.553419in}}%
\pgfpathlineto{\pgfqpoint{2.034640in}{3.542102in}}%
\pgfpathlineto{\pgfqpoint{2.049858in}{3.533978in}}%
\pgfpathlineto{\pgfqpoint{2.064653in}{3.528566in}}%
\pgfpathlineto{\pgfqpoint{2.079026in}{3.525704in}}%
\pgfpathlineto{\pgfqpoint{2.092976in}{3.525213in}}%
\pgfpathlineto{\pgfqpoint{2.106926in}{3.526996in}}%
\pgfpathlineto{\pgfqpoint{2.120876in}{3.531065in}}%
\pgfpathlineto{\pgfqpoint{2.135249in}{3.537646in}}%
\pgfpathlineto{\pgfqpoint{2.150044in}{3.546943in}}%
\pgfpathlineto{\pgfqpoint{2.165263in}{3.559150in}}%
\pgfpathlineto{\pgfqpoint{2.181326in}{3.574901in}}%
\pgfpathlineto{\pgfqpoint{2.198658in}{3.595124in}}%
\pgfpathlineto{\pgfqpoint{2.216835in}{3.619825in}}%
\pgfpathlineto{\pgfqpoint{2.236281in}{3.650046in}}%
\pgfpathlineto{\pgfqpoint{2.257417in}{3.687111in}}%
\pgfpathlineto{\pgfqpoint{2.280244in}{3.731725in}}%
\pgfpathlineto{\pgfqpoint{2.305608in}{3.786354in}}%
\pgfpathlineto{\pgfqpoint{2.333931in}{3.852840in}}%
\pgfpathlineto{\pgfqpoint{2.366903in}{3.936183in}}%
\pgfpathlineto{\pgfqpoint{2.408753in}{4.048470in}}%
\pgfpathlineto{\pgfqpoint{2.507249in}{4.319909in}}%
\pgfpathlineto{\pgfqpoint{2.507249in}{4.319909in}}%
\pgfusepath{stroke}%
\end{pgfscope}%
\begin{pgfscope}%
\pgfsetbuttcap%
\pgfsetmiterjoin%
\definecolor{currentfill}{rgb}{1.000000,1.000000,1.000000}%
\pgfsetfillcolor{currentfill}%
\pgfsetfillopacity{0.800000}%
\pgfsetlinewidth{1.003750pt}%
\definecolor{currentstroke}{rgb}{0.800000,0.800000,0.800000}%
\pgfsetstrokecolor{currentstroke}%
\pgfsetstrokeopacity{0.800000}%
\pgfsetdash{}{0pt}%
\pgfpathmoveto{\pgfqpoint{0.934034in}{3.137930in}}%
\pgfpathlineto{\pgfqpoint{1.966827in}{3.137930in}}%
\pgfpathquadraticcurveto{\pgfqpoint{1.994605in}{3.137930in}}{\pgfqpoint{1.994605in}{3.165708in}}%
\pgfpathlineto{\pgfqpoint{1.994605in}{3.780888in}}%
\pgfpathquadraticcurveto{\pgfqpoint{1.994605in}{3.808666in}}{\pgfqpoint{1.966827in}{3.808666in}}%
\pgfpathlineto{\pgfqpoint{0.934034in}{3.808666in}}%
\pgfpathquadraticcurveto{\pgfqpoint{0.906256in}{3.808666in}}{\pgfqpoint{0.906256in}{3.780888in}}%
\pgfpathlineto{\pgfqpoint{0.906256in}{3.165708in}}%
\pgfpathquadraticcurveto{\pgfqpoint{0.906256in}{3.137930in}}{\pgfqpoint{0.934034in}{3.137930in}}%
\pgfpathlineto{\pgfqpoint{0.934034in}{3.137930in}}%
\pgfpathclose%
\pgfusepath{stroke,fill}%
\end{pgfscope}%
\begin{pgfscope}%
\pgfsetrectcap%
\pgfsetroundjoin%
\pgfsetlinewidth{1.505625pt}%
\definecolor{currentstroke}{rgb}{0.000000,0.000000,0.000000}%
\pgfsetstrokecolor{currentstroke}%
\pgfsetdash{}{0pt}%
\pgfpathmoveto{\pgfqpoint{0.961812in}{3.696199in}}%
\pgfpathlineto{\pgfqpoint{1.100701in}{3.696199in}}%
\pgfpathlineto{\pgfqpoint{1.239590in}{3.696199in}}%
\pgfusepath{stroke}%
\end{pgfscope}%
\begin{pgfscope}%
\definecolor{textcolor}{rgb}{0.000000,0.000000,0.000000}%
\pgfsetstrokecolor{textcolor}%
\pgfsetfillcolor{textcolor}%
\pgftext[x=1.350701in,y=3.647588in,left,base]{\color{textcolor}{\rmfamily\fontsize{10.000000}{12.000000}\selectfont\catcode`\^=\active\def^{\ifmmode\sp\else\^{}\fi}\catcode`\%=\active\def%{\%}$f(t)$}}%
\end{pgfscope}%
\begin{pgfscope}%
\pgfsetbuttcap%
\pgfsetroundjoin%
\pgfsetlinewidth{1.505625pt}%
\definecolor{currentstroke}{rgb}{0.000000,0.000000,0.000000}%
\pgfsetstrokecolor{currentstroke}%
\pgfsetdash{{5.550000pt}{2.400000pt}}{0.000000pt}%
\pgfpathmoveto{\pgfqpoint{0.961812in}{3.486509in}}%
\pgfpathlineto{\pgfqpoint{1.100701in}{3.486509in}}%
\pgfpathlineto{\pgfqpoint{1.239590in}{3.486509in}}%
\pgfusepath{stroke}%
\end{pgfscope}%
\begin{pgfscope}%
\definecolor{textcolor}{rgb}{0.000000,0.000000,0.000000}%
\pgfsetstrokecolor{textcolor}%
\pgfsetfillcolor{textcolor}%
\pgftext[x=1.350701in,y=3.437898in,left,base]{\color{textcolor}{\rmfamily\fontsize{10.000000}{12.000000}\selectfont\catcode`\^=\active\def^{\ifmmode\sp\else\^{}\fi}\catcode`\%=\active\def%{\%}$f(t)+0.5$}}%
\end{pgfscope}%
\begin{pgfscope}%
\pgfsetbuttcap%
\pgfsetroundjoin%
\pgfsetlinewidth{1.505625pt}%
\definecolor{currentstroke}{rgb}{0.000000,0.000000,0.000000}%
\pgfsetstrokecolor{currentstroke}%
\pgfsetdash{{1.500000pt}{2.475000pt}}{0.000000pt}%
\pgfpathmoveto{\pgfqpoint{0.961812in}{3.276819in}}%
\pgfpathlineto{\pgfqpoint{1.100701in}{3.276819in}}%
\pgfpathlineto{\pgfqpoint{1.239590in}{3.276819in}}%
\pgfusepath{stroke}%
\end{pgfscope}%
\begin{pgfscope}%
\definecolor{textcolor}{rgb}{0.000000,0.000000,0.000000}%
\pgfsetstrokecolor{textcolor}%
\pgfsetfillcolor{textcolor}%
\pgftext[x=1.350701in,y=3.228208in,left,base]{\color{textcolor}{\rmfamily\fontsize{10.000000}{12.000000}\selectfont\catcode`\^=\active\def^{\ifmmode\sp\else\^{}\fi}\catcode`\%=\active\def%{\%}$f(t)-0.5$}}%
\end{pgfscope}%
\begin{pgfscope}%
\pgfsetbuttcap%
\pgfsetmiterjoin%
\definecolor{currentfill}{rgb}{1.000000,1.000000,1.000000}%
\pgfsetfillcolor{currentfill}%
\pgfsetlinewidth{0.000000pt}%
\definecolor{currentstroke}{rgb}{0.000000,0.000000,0.000000}%
\pgfsetstrokecolor{currentstroke}%
\pgfsetstrokeopacity{0.000000}%
\pgfsetdash{}{0pt}%
\pgfpathmoveto{\pgfqpoint{2.929976in}{3.068486in}}%
\pgfpathlineto{\pgfqpoint{5.043613in}{3.068486in}}%
\pgfpathlineto{\pgfqpoint{5.043613in}{5.168486in}}%
\pgfpathlineto{\pgfqpoint{2.929976in}{5.168486in}}%
\pgfpathlineto{\pgfqpoint{2.929976in}{3.068486in}}%
\pgfpathclose%
\pgfusepath{fill}%
\end{pgfscope}%
\begin{pgfscope}%
\pgfsetbuttcap%
\pgfsetroundjoin%
\definecolor{currentfill}{rgb}{0.000000,0.000000,0.000000}%
\pgfsetfillcolor{currentfill}%
\pgfsetlinewidth{0.803000pt}%
\definecolor{currentstroke}{rgb}{0.000000,0.000000,0.000000}%
\pgfsetstrokecolor{currentstroke}%
\pgfsetdash{}{0pt}%
\pgfsys@defobject{currentmarker}{\pgfqpoint{0.000000in}{-0.048611in}}{\pgfqpoint{0.000000in}{0.000000in}}{%
\pgfpathmoveto{\pgfqpoint{0.000000in}{0.000000in}}%
\pgfpathlineto{\pgfqpoint{0.000000in}{-0.048611in}}%
\pgfusepath{stroke,fill}%
}%
\begin{pgfscope}%
\pgfsys@transformshift{2.929976in}{3.068486in}%
\pgfsys@useobject{currentmarker}{}%
\end{pgfscope}%
\end{pgfscope}%
\begin{pgfscope}%
\definecolor{textcolor}{rgb}{0.000000,0.000000,0.000000}%
\pgfsetstrokecolor{textcolor}%
\pgfsetfillcolor{textcolor}%
\pgftext[x=2.929976in,y=2.971264in,,top]{\color{textcolor}{\rmfamily\fontsize{10.000000}{12.000000}\selectfont\catcode`\^=\active\def^{\ifmmode\sp\else\^{}\fi}\catcode`\%=\active\def%{\%}\ensuremath{-}1.0}}%
\end{pgfscope}%
\begin{pgfscope}%
\pgfsetbuttcap%
\pgfsetroundjoin%
\definecolor{currentfill}{rgb}{0.000000,0.000000,0.000000}%
\pgfsetfillcolor{currentfill}%
\pgfsetlinewidth{0.803000pt}%
\definecolor{currentstroke}{rgb}{0.000000,0.000000,0.000000}%
\pgfsetstrokecolor{currentstroke}%
\pgfsetdash{}{0pt}%
\pgfsys@defobject{currentmarker}{\pgfqpoint{0.000000in}{-0.048611in}}{\pgfqpoint{0.000000in}{0.000000in}}{%
\pgfpathmoveto{\pgfqpoint{0.000000in}{0.000000in}}%
\pgfpathlineto{\pgfqpoint{0.000000in}{-0.048611in}}%
\pgfusepath{stroke,fill}%
}%
\begin{pgfscope}%
\pgfsys@transformshift{3.458385in}{3.068486in}%
\pgfsys@useobject{currentmarker}{}%
\end{pgfscope}%
\end{pgfscope}%
\begin{pgfscope}%
\definecolor{textcolor}{rgb}{0.000000,0.000000,0.000000}%
\pgfsetstrokecolor{textcolor}%
\pgfsetfillcolor{textcolor}%
\pgftext[x=3.458385in,y=2.971264in,,top]{\color{textcolor}{\rmfamily\fontsize{10.000000}{12.000000}\selectfont\catcode`\^=\active\def^{\ifmmode\sp\else\^{}\fi}\catcode`\%=\active\def%{\%}\ensuremath{-}0.5}}%
\end{pgfscope}%
\begin{pgfscope}%
\pgfsetbuttcap%
\pgfsetroundjoin%
\definecolor{currentfill}{rgb}{0.000000,0.000000,0.000000}%
\pgfsetfillcolor{currentfill}%
\pgfsetlinewidth{0.803000pt}%
\definecolor{currentstroke}{rgb}{0.000000,0.000000,0.000000}%
\pgfsetstrokecolor{currentstroke}%
\pgfsetdash{}{0pt}%
\pgfsys@defobject{currentmarker}{\pgfqpoint{0.000000in}{-0.048611in}}{\pgfqpoint{0.000000in}{0.000000in}}{%
\pgfpathmoveto{\pgfqpoint{0.000000in}{0.000000in}}%
\pgfpathlineto{\pgfqpoint{0.000000in}{-0.048611in}}%
\pgfusepath{stroke,fill}%
}%
\begin{pgfscope}%
\pgfsys@transformshift{3.986794in}{3.068486in}%
\pgfsys@useobject{currentmarker}{}%
\end{pgfscope}%
\end{pgfscope}%
\begin{pgfscope}%
\definecolor{textcolor}{rgb}{0.000000,0.000000,0.000000}%
\pgfsetstrokecolor{textcolor}%
\pgfsetfillcolor{textcolor}%
\pgftext[x=3.986794in,y=2.971264in,,top]{\color{textcolor}{\rmfamily\fontsize{10.000000}{12.000000}\selectfont\catcode`\^=\active\def^{\ifmmode\sp\else\^{}\fi}\catcode`\%=\active\def%{\%}0.0}}%
\end{pgfscope}%
\begin{pgfscope}%
\pgfsetbuttcap%
\pgfsetroundjoin%
\definecolor{currentfill}{rgb}{0.000000,0.000000,0.000000}%
\pgfsetfillcolor{currentfill}%
\pgfsetlinewidth{0.803000pt}%
\definecolor{currentstroke}{rgb}{0.000000,0.000000,0.000000}%
\pgfsetstrokecolor{currentstroke}%
\pgfsetdash{}{0pt}%
\pgfsys@defobject{currentmarker}{\pgfqpoint{0.000000in}{-0.048611in}}{\pgfqpoint{0.000000in}{0.000000in}}{%
\pgfpathmoveto{\pgfqpoint{0.000000in}{0.000000in}}%
\pgfpathlineto{\pgfqpoint{0.000000in}{-0.048611in}}%
\pgfusepath{stroke,fill}%
}%
\begin{pgfscope}%
\pgfsys@transformshift{4.515203in}{3.068486in}%
\pgfsys@useobject{currentmarker}{}%
\end{pgfscope}%
\end{pgfscope}%
\begin{pgfscope}%
\definecolor{textcolor}{rgb}{0.000000,0.000000,0.000000}%
\pgfsetstrokecolor{textcolor}%
\pgfsetfillcolor{textcolor}%
\pgftext[x=4.515203in,y=2.971264in,,top]{\color{textcolor}{\rmfamily\fontsize{10.000000}{12.000000}\selectfont\catcode`\^=\active\def^{\ifmmode\sp\else\^{}\fi}\catcode`\%=\active\def%{\%}0.5}}%
\end{pgfscope}%
\begin{pgfscope}%
\pgfsetbuttcap%
\pgfsetroundjoin%
\definecolor{currentfill}{rgb}{0.000000,0.000000,0.000000}%
\pgfsetfillcolor{currentfill}%
\pgfsetlinewidth{0.803000pt}%
\definecolor{currentstroke}{rgb}{0.000000,0.000000,0.000000}%
\pgfsetstrokecolor{currentstroke}%
\pgfsetdash{}{0pt}%
\pgfsys@defobject{currentmarker}{\pgfqpoint{0.000000in}{-0.048611in}}{\pgfqpoint{0.000000in}{0.000000in}}{%
\pgfpathmoveto{\pgfqpoint{0.000000in}{0.000000in}}%
\pgfpathlineto{\pgfqpoint{0.000000in}{-0.048611in}}%
\pgfusepath{stroke,fill}%
}%
\begin{pgfscope}%
\pgfsys@transformshift{5.043613in}{3.068486in}%
\pgfsys@useobject{currentmarker}{}%
\end{pgfscope}%
\end{pgfscope}%
\begin{pgfscope}%
\definecolor{textcolor}{rgb}{0.000000,0.000000,0.000000}%
\pgfsetstrokecolor{textcolor}%
\pgfsetfillcolor{textcolor}%
\pgftext[x=5.043613in,y=2.971264in,,top]{\color{textcolor}{\rmfamily\fontsize{10.000000}{12.000000}\selectfont\catcode`\^=\active\def^{\ifmmode\sp\else\^{}\fi}\catcode`\%=\active\def%{\%}1.0}}%
\end{pgfscope}%
\begin{pgfscope}%
\definecolor{textcolor}{rgb}{0.000000,0.000000,0.000000}%
\pgfsetstrokecolor{textcolor}%
\pgfsetfillcolor{textcolor}%
\pgftext[x=3.986794in,y=2.781295in,,top]{\color{textcolor}{\rmfamily\fontsize{12.000000}{14.400000}\selectfont\catcode`\^=\active\def^{\ifmmode\sp\else\^{}\fi}\catcode`\%=\active\def%{\%}$t$}}%
\end{pgfscope}%
\begin{pgfscope}%
\pgfsetbuttcap%
\pgfsetroundjoin%
\definecolor{currentfill}{rgb}{0.000000,0.000000,0.000000}%
\pgfsetfillcolor{currentfill}%
\pgfsetlinewidth{0.803000pt}%
\definecolor{currentstroke}{rgb}{0.000000,0.000000,0.000000}%
\pgfsetstrokecolor{currentstroke}%
\pgfsetdash{}{0pt}%
\pgfsys@defobject{currentmarker}{\pgfqpoint{-0.048611in}{0.000000in}}{\pgfqpoint{-0.000000in}{0.000000in}}{%
\pgfpathmoveto{\pgfqpoint{-0.000000in}{0.000000in}}%
\pgfpathlineto{\pgfqpoint{-0.048611in}{0.000000in}}%
\pgfusepath{stroke,fill}%
}%
\begin{pgfscope}%
\pgfsys@transformshift{2.929976in}{3.068486in}%
\pgfsys@useobject{currentmarker}{}%
\end{pgfscope}%
\end{pgfscope}%
\begin{pgfscope}%
\definecolor{textcolor}{rgb}{0.000000,0.000000,0.000000}%
\pgfsetstrokecolor{textcolor}%
\pgfsetfillcolor{textcolor}%
\pgftext[x=2.636364in, y=3.015724in, left, base]{\color{textcolor}{\rmfamily\fontsize{10.000000}{12.000000}\selectfont\catcode`\^=\active\def^{\ifmmode\sp\else\^{}\fi}\catcode`\%=\active\def%{\%}\ensuremath{-}2}}%
\end{pgfscope}%
\begin{pgfscope}%
\pgfsetbuttcap%
\pgfsetroundjoin%
\definecolor{currentfill}{rgb}{0.000000,0.000000,0.000000}%
\pgfsetfillcolor{currentfill}%
\pgfsetlinewidth{0.803000pt}%
\definecolor{currentstroke}{rgb}{0.000000,0.000000,0.000000}%
\pgfsetstrokecolor{currentstroke}%
\pgfsetdash{}{0pt}%
\pgfsys@defobject{currentmarker}{\pgfqpoint{-0.048611in}{0.000000in}}{\pgfqpoint{-0.000000in}{0.000000in}}{%
\pgfpathmoveto{\pgfqpoint{-0.000000in}{0.000000in}}%
\pgfpathlineto{\pgfqpoint{-0.048611in}{0.000000in}}%
\pgfusepath{stroke,fill}%
}%
\begin{pgfscope}%
\pgfsys@transformshift{2.929976in}{3.593486in}%
\pgfsys@useobject{currentmarker}{}%
\end{pgfscope}%
\end{pgfscope}%
\begin{pgfscope}%
\definecolor{textcolor}{rgb}{0.000000,0.000000,0.000000}%
\pgfsetstrokecolor{textcolor}%
\pgfsetfillcolor{textcolor}%
\pgftext[x=2.636364in, y=3.540724in, left, base]{\color{textcolor}{\rmfamily\fontsize{10.000000}{12.000000}\selectfont\catcode`\^=\active\def^{\ifmmode\sp\else\^{}\fi}\catcode`\%=\active\def%{\%}\ensuremath{-}1}}%
\end{pgfscope}%
\begin{pgfscope}%
\pgfsetbuttcap%
\pgfsetroundjoin%
\definecolor{currentfill}{rgb}{0.000000,0.000000,0.000000}%
\pgfsetfillcolor{currentfill}%
\pgfsetlinewidth{0.803000pt}%
\definecolor{currentstroke}{rgb}{0.000000,0.000000,0.000000}%
\pgfsetstrokecolor{currentstroke}%
\pgfsetdash{}{0pt}%
\pgfsys@defobject{currentmarker}{\pgfqpoint{-0.048611in}{0.000000in}}{\pgfqpoint{-0.000000in}{0.000000in}}{%
\pgfpathmoveto{\pgfqpoint{-0.000000in}{0.000000in}}%
\pgfpathlineto{\pgfqpoint{-0.048611in}{0.000000in}}%
\pgfusepath{stroke,fill}%
}%
\begin{pgfscope}%
\pgfsys@transformshift{2.929976in}{4.118486in}%
\pgfsys@useobject{currentmarker}{}%
\end{pgfscope}%
\end{pgfscope}%
\begin{pgfscope}%
\definecolor{textcolor}{rgb}{0.000000,0.000000,0.000000}%
\pgfsetstrokecolor{textcolor}%
\pgfsetfillcolor{textcolor}%
\pgftext[x=2.744389in, y=4.065724in, left, base]{\color{textcolor}{\rmfamily\fontsize{10.000000}{12.000000}\selectfont\catcode`\^=\active\def^{\ifmmode\sp\else\^{}\fi}\catcode`\%=\active\def%{\%}0}}%
\end{pgfscope}%
\begin{pgfscope}%
\pgfsetbuttcap%
\pgfsetroundjoin%
\definecolor{currentfill}{rgb}{0.000000,0.000000,0.000000}%
\pgfsetfillcolor{currentfill}%
\pgfsetlinewidth{0.803000pt}%
\definecolor{currentstroke}{rgb}{0.000000,0.000000,0.000000}%
\pgfsetstrokecolor{currentstroke}%
\pgfsetdash{}{0pt}%
\pgfsys@defobject{currentmarker}{\pgfqpoint{-0.048611in}{0.000000in}}{\pgfqpoint{-0.000000in}{0.000000in}}{%
\pgfpathmoveto{\pgfqpoint{-0.000000in}{0.000000in}}%
\pgfpathlineto{\pgfqpoint{-0.048611in}{0.000000in}}%
\pgfusepath{stroke,fill}%
}%
\begin{pgfscope}%
\pgfsys@transformshift{2.929976in}{4.643486in}%
\pgfsys@useobject{currentmarker}{}%
\end{pgfscope}%
\end{pgfscope}%
\begin{pgfscope}%
\definecolor{textcolor}{rgb}{0.000000,0.000000,0.000000}%
\pgfsetstrokecolor{textcolor}%
\pgfsetfillcolor{textcolor}%
\pgftext[x=2.744389in, y=4.590724in, left, base]{\color{textcolor}{\rmfamily\fontsize{10.000000}{12.000000}\selectfont\catcode`\^=\active\def^{\ifmmode\sp\else\^{}\fi}\catcode`\%=\active\def%{\%}1}}%
\end{pgfscope}%
\begin{pgfscope}%
\pgfsetbuttcap%
\pgfsetroundjoin%
\definecolor{currentfill}{rgb}{0.000000,0.000000,0.000000}%
\pgfsetfillcolor{currentfill}%
\pgfsetlinewidth{0.803000pt}%
\definecolor{currentstroke}{rgb}{0.000000,0.000000,0.000000}%
\pgfsetstrokecolor{currentstroke}%
\pgfsetdash{}{0pt}%
\pgfsys@defobject{currentmarker}{\pgfqpoint{-0.048611in}{0.000000in}}{\pgfqpoint{-0.000000in}{0.000000in}}{%
\pgfpathmoveto{\pgfqpoint{-0.000000in}{0.000000in}}%
\pgfpathlineto{\pgfqpoint{-0.048611in}{0.000000in}}%
\pgfusepath{stroke,fill}%
}%
\begin{pgfscope}%
\pgfsys@transformshift{2.929976in}{5.168486in}%
\pgfsys@useobject{currentmarker}{}%
\end{pgfscope}%
\end{pgfscope}%
\begin{pgfscope}%
\definecolor{textcolor}{rgb}{0.000000,0.000000,0.000000}%
\pgfsetstrokecolor{textcolor}%
\pgfsetfillcolor{textcolor}%
\pgftext[x=2.744389in, y=5.115724in, left, base]{\color{textcolor}{\rmfamily\fontsize{10.000000}{12.000000}\selectfont\catcode`\^=\active\def^{\ifmmode\sp\else\^{}\fi}\catcode`\%=\active\def%{\%}2}}%
\end{pgfscope}%
\begin{pgfscope}%
\pgfpathrectangle{\pgfqpoint{2.929976in}{3.068486in}}{\pgfqpoint{2.113636in}{2.100000in}}%
\pgfusepath{clip}%
\pgfsetbuttcap%
\pgfsetroundjoin%
\pgfsetlinewidth{0.501875pt}%
\definecolor{currentstroke}{rgb}{0.000000,0.000000,0.000000}%
\pgfsetstrokecolor{currentstroke}%
\pgfsetdash{{1.850000pt}{0.800000pt}}{0.000000pt}%
\pgfpathmoveto{\pgfqpoint{2.929976in}{4.118486in}}%
\pgfpathlineto{\pgfqpoint{5.043613in}{4.118486in}}%
\pgfusepath{stroke}%
\end{pgfscope}%
\begin{pgfscope}%
\pgfpathrectangle{\pgfqpoint{2.929976in}{3.068486in}}{\pgfqpoint{2.113636in}{2.100000in}}%
\pgfusepath{clip}%
\pgfsetbuttcap%
\pgfsetroundjoin%
\pgfsetlinewidth{0.501875pt}%
\definecolor{currentstroke}{rgb}{0.000000,0.000000,0.000000}%
\pgfsetstrokecolor{currentstroke}%
\pgfsetdash{{1.850000pt}{0.800000pt}}{0.000000pt}%
\pgfpathmoveto{\pgfqpoint{3.986794in}{3.068486in}}%
\pgfpathlineto{\pgfqpoint{3.986794in}{5.168486in}}%
\pgfusepath{stroke}%
\end{pgfscope}%
\begin{pgfscope}%
\pgfpathrectangle{\pgfqpoint{2.929976in}{3.068486in}}{\pgfqpoint{2.113636in}{2.100000in}}%
\pgfusepath{clip}%
\pgfsetrectcap%
\pgfsetroundjoin%
\pgfsetlinewidth{1.505625pt}%
\definecolor{currentstroke}{rgb}{0.000000,0.000000,0.000000}%
\pgfsetstrokecolor{currentstroke}%
\pgfsetdash{}{0pt}%
\pgfpathmoveto{\pgfqpoint{2.929976in}{3.952409in}}%
\pgfpathlineto{\pgfqpoint{2.975208in}{3.855579in}}%
\pgfpathlineto{\pgfqpoint{3.060599in}{3.672104in}}%
\pgfpathlineto{\pgfqpoint{3.094417in}{3.605241in}}%
\pgfpathlineto{\pgfqpoint{3.122740in}{3.554026in}}%
\pgfpathlineto{\pgfqpoint{3.147258in}{3.514098in}}%
\pgfpathlineto{\pgfqpoint{3.169662in}{3.481760in}}%
\pgfpathlineto{\pgfqpoint{3.189953in}{3.456269in}}%
\pgfpathlineto{\pgfqpoint{3.208553in}{3.436333in}}%
\pgfpathlineto{\pgfqpoint{3.225885in}{3.420888in}}%
\pgfpathlineto{\pgfqpoint{3.242372in}{3.409130in}}%
\pgfpathlineto{\pgfqpoint{3.257590in}{3.400906in}}%
\pgfpathlineto{\pgfqpoint{3.272385in}{3.395395in}}%
\pgfpathlineto{\pgfqpoint{3.286758in}{3.392434in}}%
\pgfpathlineto{\pgfqpoint{3.300708in}{3.391847in}}%
\pgfpathlineto{\pgfqpoint{3.314658in}{3.393533in}}%
\pgfpathlineto{\pgfqpoint{3.328608in}{3.397504in}}%
\pgfpathlineto{\pgfqpoint{3.342981in}{3.403985in}}%
\pgfpathlineto{\pgfqpoint{3.357776in}{3.413180in}}%
\pgfpathlineto{\pgfqpoint{3.372994in}{3.425283in}}%
\pgfpathlineto{\pgfqpoint{3.389058in}{3.440925in}}%
\pgfpathlineto{\pgfqpoint{3.406390in}{3.461035in}}%
\pgfpathlineto{\pgfqpoint{3.424567in}{3.485620in}}%
\pgfpathlineto{\pgfqpoint{3.444012in}{3.515725in}}%
\pgfpathlineto{\pgfqpoint{3.465149in}{3.552670in}}%
\pgfpathlineto{\pgfqpoint{3.487976in}{3.597165in}}%
\pgfpathlineto{\pgfqpoint{3.513340in}{3.651676in}}%
\pgfpathlineto{\pgfqpoint{3.541663in}{3.718049in}}%
\pgfpathlineto{\pgfqpoint{3.574635in}{3.801291in}}%
\pgfpathlineto{\pgfqpoint{3.616062in}{3.912339in}}%
\pgfpathlineto{\pgfqpoint{3.751335in}{4.279868in}}%
\pgfpathlineto{\pgfqpoint{3.783885in}{4.359104in}}%
\pgfpathlineto{\pgfqpoint{3.811785in}{4.421406in}}%
\pgfpathlineto{\pgfqpoint{3.836726in}{4.471896in}}%
\pgfpathlineto{\pgfqpoint{3.859553in}{4.513267in}}%
\pgfpathlineto{\pgfqpoint{3.880690in}{4.547086in}}%
\pgfpathlineto{\pgfqpoint{3.900135in}{4.574144in}}%
\pgfpathlineto{\pgfqpoint{3.918312in}{4.595758in}}%
\pgfpathlineto{\pgfqpoint{3.935222in}{4.612563in}}%
\pgfpathlineto{\pgfqpoint{3.951285in}{4.625511in}}%
\pgfpathlineto{\pgfqpoint{3.954667in}{4.627857in}}%
\pgfpathlineto{\pgfqpoint{3.955935in}{4.786202in}}%
\pgfpathlineto{\pgfqpoint{3.971153in}{4.794887in}}%
\pgfpathlineto{\pgfqpoint{3.985526in}{4.800599in}}%
\pgfpathlineto{\pgfqpoint{3.999476in}{4.803821in}}%
\pgfpathlineto{\pgfqpoint{4.013426in}{4.804762in}}%
\pgfpathlineto{\pgfqpoint{4.018076in}{4.804571in}}%
\pgfpathlineto{\pgfqpoint{4.019344in}{4.646975in}}%
\pgfpathlineto{\pgfqpoint{4.033294in}{4.644692in}}%
\pgfpathlineto{\pgfqpoint{4.047667in}{4.640007in}}%
\pgfpathlineto{\pgfqpoint{4.062462in}{4.632755in}}%
\pgfpathlineto{\pgfqpoint{4.077681in}{4.622783in}}%
\pgfpathlineto{\pgfqpoint{4.093744in}{4.609576in}}%
\pgfpathlineto{\pgfqpoint{4.111076in}{4.592357in}}%
\pgfpathlineto{\pgfqpoint{4.129676in}{4.570617in}}%
\pgfpathlineto{\pgfqpoint{4.149544in}{4.543899in}}%
\pgfpathlineto{\pgfqpoint{4.171526in}{4.510474in}}%
\pgfpathlineto{\pgfqpoint{4.195622in}{4.469669in}}%
\pgfpathlineto{\pgfqpoint{4.223099in}{4.418598in}}%
\pgfpathlineto{\pgfqpoint{4.255649in}{4.353166in}}%
\pgfpathlineto{\pgfqpoint{4.299613in}{4.259329in}}%
\pgfpathlineto{\pgfqpoint{4.390922in}{4.063266in}}%
\pgfpathlineto{\pgfqpoint{4.424317in}{3.997565in}}%
\pgfpathlineto{\pgfqpoint{4.452217in}{3.947401in}}%
\pgfpathlineto{\pgfqpoint{4.476735in}{3.907735in}}%
\pgfpathlineto{\pgfqpoint{4.498717in}{3.876233in}}%
\pgfpathlineto{\pgfqpoint{4.519008in}{3.850932in}}%
\pgfpathlineto{\pgfqpoint{4.537608in}{3.831182in}}%
\pgfpathlineto{\pgfqpoint{4.554940in}{3.815919in}}%
\pgfpathlineto{\pgfqpoint{4.571003in}{3.804602in}}%
\pgfpathlineto{\pgfqpoint{4.586222in}{3.796478in}}%
\pgfpathlineto{\pgfqpoint{4.601017in}{3.791066in}}%
\pgfpathlineto{\pgfqpoint{4.615390in}{3.788204in}}%
\pgfpathlineto{\pgfqpoint{4.629340in}{3.787713in}}%
\pgfpathlineto{\pgfqpoint{4.643290in}{3.789496in}}%
\pgfpathlineto{\pgfqpoint{4.657240in}{3.793565in}}%
\pgfpathlineto{\pgfqpoint{4.671613in}{3.800146in}}%
\pgfpathlineto{\pgfqpoint{4.686408in}{3.809443in}}%
\pgfpathlineto{\pgfqpoint{4.701626in}{3.821650in}}%
\pgfpathlineto{\pgfqpoint{4.717690in}{3.837401in}}%
\pgfpathlineto{\pgfqpoint{4.735022in}{3.857624in}}%
\pgfpathlineto{\pgfqpoint{4.753199in}{3.882325in}}%
\pgfpathlineto{\pgfqpoint{4.772644in}{3.912546in}}%
\pgfpathlineto{\pgfqpoint{4.793781in}{3.949611in}}%
\pgfpathlineto{\pgfqpoint{4.816608in}{3.994225in}}%
\pgfpathlineto{\pgfqpoint{4.841972in}{4.048854in}}%
\pgfpathlineto{\pgfqpoint{4.870294in}{4.115340in}}%
\pgfpathlineto{\pgfqpoint{4.903267in}{4.198683in}}%
\pgfpathlineto{\pgfqpoint{4.945117in}{4.310970in}}%
\pgfpathlineto{\pgfqpoint{5.043613in}{4.582409in}}%
\pgfpathlineto{\pgfqpoint{5.043613in}{4.582409in}}%
\pgfusepath{stroke}%
\end{pgfscope}%
\begin{pgfscope}%
\pgfpathrectangle{\pgfqpoint{2.929976in}{3.068486in}}{\pgfqpoint{2.113636in}{2.100000in}}%
\pgfusepath{clip}%
\pgfsetbuttcap%
\pgfsetroundjoin%
\pgfsetlinewidth{1.505625pt}%
\definecolor{currentstroke}{rgb}{0.000000,0.000000,0.000000}%
\pgfsetstrokecolor{currentstroke}%
\pgfsetdash{{5.550000pt}{2.400000pt}}{0.000000pt}%
\pgfpathmoveto{\pgfqpoint{2.929976in}{3.523323in}}%
\pgfpathlineto{\pgfqpoint{2.952803in}{3.489299in}}%
\pgfpathlineto{\pgfqpoint{2.973517in}{3.462286in}}%
\pgfpathlineto{\pgfqpoint{2.992540in}{3.440998in}}%
\pgfpathlineto{\pgfqpoint{3.010294in}{3.424368in}}%
\pgfpathlineto{\pgfqpoint{3.026781in}{3.411866in}}%
\pgfpathlineto{\pgfqpoint{3.042422in}{3.402722in}}%
\pgfpathlineto{\pgfqpoint{3.057217in}{3.396576in}}%
\pgfpathlineto{\pgfqpoint{3.071590in}{3.392986in}}%
\pgfpathlineto{\pgfqpoint{3.085540in}{3.391782in}}%
\pgfpathlineto{\pgfqpoint{3.099490in}{3.392847in}}%
\pgfpathlineto{\pgfqpoint{3.113440in}{3.396194in}}%
\pgfpathlineto{\pgfqpoint{3.127390in}{3.401827in}}%
\pgfpathlineto{\pgfqpoint{3.141763in}{3.410015in}}%
\pgfpathlineto{\pgfqpoint{3.156981in}{3.421302in}}%
\pgfpathlineto{\pgfqpoint{3.173044in}{3.436098in}}%
\pgfpathlineto{\pgfqpoint{3.189953in}{3.454807in}}%
\pgfpathlineto{\pgfqpoint{3.207708in}{3.477818in}}%
\pgfpathlineto{\pgfqpoint{3.226731in}{3.506158in}}%
\pgfpathlineto{\pgfqpoint{3.247444in}{3.541141in}}%
\pgfpathlineto{\pgfqpoint{3.269849in}{3.583508in}}%
\pgfpathlineto{\pgfqpoint{3.294367in}{3.634799in}}%
\pgfpathlineto{\pgfqpoint{3.321422in}{3.696679in}}%
\pgfpathlineto{\pgfqpoint{3.352703in}{3.774009in}}%
\pgfpathlineto{\pgfqpoint{3.390749in}{3.874344in}}%
\pgfpathlineto{\pgfqpoint{3.450353in}{4.038807in}}%
\pgfpathlineto{\pgfqpoint{3.514608in}{4.214238in}}%
\pgfpathlineto{\pgfqpoint{3.551808in}{4.309405in}}%
\pgfpathlineto{\pgfqpoint{3.582667in}{4.382417in}}%
\pgfpathlineto{\pgfqpoint{3.609722in}{4.440844in}}%
\pgfpathlineto{\pgfqpoint{3.633817in}{4.487782in}}%
\pgfpathlineto{\pgfqpoint{3.655799in}{4.525956in}}%
\pgfpathlineto{\pgfqpoint{3.676090in}{4.556949in}}%
\pgfpathlineto{\pgfqpoint{3.695112in}{4.582095in}}%
\pgfpathlineto{\pgfqpoint{3.712867in}{4.602006in}}%
\pgfpathlineto{\pgfqpoint{3.729776in}{4.617678in}}%
\pgfpathlineto{\pgfqpoint{3.743303in}{4.627857in}}%
\pgfpathlineto{\pgfqpoint{3.744572in}{4.786202in}}%
\pgfpathlineto{\pgfqpoint{3.759790in}{4.794887in}}%
\pgfpathlineto{\pgfqpoint{3.774163in}{4.800599in}}%
\pgfpathlineto{\pgfqpoint{3.788112in}{4.803821in}}%
\pgfpathlineto{\pgfqpoint{3.802062in}{4.804762in}}%
\pgfpathlineto{\pgfqpoint{3.806712in}{4.804571in}}%
\pgfpathlineto{\pgfqpoint{3.807981in}{4.646975in}}%
\pgfpathlineto{\pgfqpoint{3.821931in}{4.644692in}}%
\pgfpathlineto{\pgfqpoint{3.836303in}{4.640007in}}%
\pgfpathlineto{\pgfqpoint{3.851099in}{4.632755in}}%
\pgfpathlineto{\pgfqpoint{3.866317in}{4.622783in}}%
\pgfpathlineto{\pgfqpoint{3.882381in}{4.609576in}}%
\pgfpathlineto{\pgfqpoint{3.899713in}{4.592357in}}%
\pgfpathlineto{\pgfqpoint{3.918312in}{4.570617in}}%
\pgfpathlineto{\pgfqpoint{3.938181in}{4.543899in}}%
\pgfpathlineto{\pgfqpoint{3.960162in}{4.510474in}}%
\pgfpathlineto{\pgfqpoint{3.984258in}{4.469669in}}%
\pgfpathlineto{\pgfqpoint{4.011735in}{4.418598in}}%
\pgfpathlineto{\pgfqpoint{4.044285in}{4.353166in}}%
\pgfpathlineto{\pgfqpoint{4.088249in}{4.259329in}}%
\pgfpathlineto{\pgfqpoint{4.179558in}{4.063266in}}%
\pgfpathlineto{\pgfqpoint{4.212953in}{3.997565in}}%
\pgfpathlineto{\pgfqpoint{4.240853in}{3.947401in}}%
\pgfpathlineto{\pgfqpoint{4.265372in}{3.907735in}}%
\pgfpathlineto{\pgfqpoint{4.287353in}{3.876233in}}%
\pgfpathlineto{\pgfqpoint{4.307644in}{3.850932in}}%
\pgfpathlineto{\pgfqpoint{4.326244in}{3.831182in}}%
\pgfpathlineto{\pgfqpoint{4.343576in}{3.815919in}}%
\pgfpathlineto{\pgfqpoint{4.359640in}{3.804602in}}%
\pgfpathlineto{\pgfqpoint{4.374858in}{3.796478in}}%
\pgfpathlineto{\pgfqpoint{4.389653in}{3.791066in}}%
\pgfpathlineto{\pgfqpoint{4.404026in}{3.788204in}}%
\pgfpathlineto{\pgfqpoint{4.417976in}{3.787713in}}%
\pgfpathlineto{\pgfqpoint{4.431926in}{3.789496in}}%
\pgfpathlineto{\pgfqpoint{4.445876in}{3.793565in}}%
\pgfpathlineto{\pgfqpoint{4.460249in}{3.800146in}}%
\pgfpathlineto{\pgfqpoint{4.475044in}{3.809443in}}%
\pgfpathlineto{\pgfqpoint{4.490263in}{3.821650in}}%
\pgfpathlineto{\pgfqpoint{4.506326in}{3.837401in}}%
\pgfpathlineto{\pgfqpoint{4.523658in}{3.857624in}}%
\pgfpathlineto{\pgfqpoint{4.541835in}{3.882325in}}%
\pgfpathlineto{\pgfqpoint{4.561281in}{3.912546in}}%
\pgfpathlineto{\pgfqpoint{4.582417in}{3.949611in}}%
\pgfpathlineto{\pgfqpoint{4.605244in}{3.994225in}}%
\pgfpathlineto{\pgfqpoint{4.630608in}{4.048854in}}%
\pgfpathlineto{\pgfqpoint{4.658931in}{4.115340in}}%
\pgfpathlineto{\pgfqpoint{4.691903in}{4.198683in}}%
\pgfpathlineto{\pgfqpoint{4.733753in}{4.310970in}}%
\pgfpathlineto{\pgfqpoint{4.866490in}{4.671865in}}%
\pgfpathlineto{\pgfqpoint{4.899040in}{4.751394in}}%
\pgfpathlineto{\pgfqpoint{4.927363in}{4.814910in}}%
\pgfpathlineto{\pgfqpoint{4.952726in}{4.866444in}}%
\pgfpathlineto{\pgfqpoint{4.975553in}{4.907958in}}%
\pgfpathlineto{\pgfqpoint{4.996690in}{4.941919in}}%
\pgfpathlineto{\pgfqpoint{5.016135in}{4.969115in}}%
\pgfpathlineto{\pgfqpoint{5.034313in}{4.990863in}}%
\pgfpathlineto{\pgfqpoint{5.043613in}{5.000575in}}%
\pgfpathlineto{\pgfqpoint{5.043613in}{5.000575in}}%
\pgfusepath{stroke}%
\end{pgfscope}%
\begin{pgfscope}%
\pgfpathrectangle{\pgfqpoint{2.929976in}{3.068486in}}{\pgfqpoint{2.113636in}{2.100000in}}%
\pgfusepath{clip}%
\pgfsetbuttcap%
\pgfsetroundjoin%
\pgfsetlinewidth{1.505625pt}%
\definecolor{currentstroke}{rgb}{0.000000,0.000000,0.000000}%
\pgfsetstrokecolor{currentstroke}%
\pgfsetdash{{1.500000pt}{2.475000pt}}{0.000000pt}%
\pgfpathmoveto{\pgfqpoint{2.929976in}{4.244575in}}%
\pgfpathlineto{\pgfqpoint{2.944772in}{4.237491in}}%
\pgfpathlineto{\pgfqpoint{2.959990in}{4.227687in}}%
\pgfpathlineto{\pgfqpoint{2.976053in}{4.214652in}}%
\pgfpathlineto{\pgfqpoint{2.993385in}{4.197610in}}%
\pgfpathlineto{\pgfqpoint{3.011985in}{4.176050in}}%
\pgfpathlineto{\pgfqpoint{3.031853in}{4.149510in}}%
\pgfpathlineto{\pgfqpoint{3.053412in}{4.116940in}}%
\pgfpathlineto{\pgfqpoint{3.077508in}{4.076380in}}%
\pgfpathlineto{\pgfqpoint{3.104562in}{4.026354in}}%
\pgfpathlineto{\pgfqpoint{3.136690in}{3.962061in}}%
\pgfpathlineto{\pgfqpoint{3.179385in}{3.871213in}}%
\pgfpathlineto{\pgfqpoint{3.277035in}{3.661748in}}%
\pgfpathlineto{\pgfqpoint{3.310008in}{3.597289in}}%
\pgfpathlineto{\pgfqpoint{3.337485in}{3.548259in}}%
\pgfpathlineto{\pgfqpoint{3.362003in}{3.508951in}}%
\pgfpathlineto{\pgfqpoint{3.383985in}{3.477809in}}%
\pgfpathlineto{\pgfqpoint{3.403853in}{3.453353in}}%
\pgfpathlineto{\pgfqpoint{3.422453in}{3.433880in}}%
\pgfpathlineto{\pgfqpoint{3.439785in}{3.418890in}}%
\pgfpathlineto{\pgfqpoint{3.455849in}{3.407835in}}%
\pgfpathlineto{\pgfqpoint{3.471067in}{3.399968in}}%
\pgfpathlineto{\pgfqpoint{3.485862in}{3.394811in}}%
\pgfpathlineto{\pgfqpoint{3.499812in}{3.392243in}}%
\pgfpathlineto{\pgfqpoint{3.513762in}{3.391930in}}%
\pgfpathlineto{\pgfqpoint{3.527712in}{3.393893in}}%
\pgfpathlineto{\pgfqpoint{3.541663in}{3.398140in}}%
\pgfpathlineto{\pgfqpoint{3.556035in}{3.404906in}}%
\pgfpathlineto{\pgfqpoint{3.570831in}{3.414392in}}%
\pgfpathlineto{\pgfqpoint{3.586472in}{3.427174in}}%
\pgfpathlineto{\pgfqpoint{3.602958in}{3.443661in}}%
\pgfpathlineto{\pgfqpoint{3.620290in}{3.464253in}}%
\pgfpathlineto{\pgfqpoint{3.638890in}{3.489952in}}%
\pgfpathlineto{\pgfqpoint{3.658758in}{3.521347in}}%
\pgfpathlineto{\pgfqpoint{3.680317in}{3.559765in}}%
\pgfpathlineto{\pgfqpoint{3.703567in}{3.605895in}}%
\pgfpathlineto{\pgfqpoint{3.729353in}{3.662199in}}%
\pgfpathlineto{\pgfqpoint{3.758099in}{3.730476in}}%
\pgfpathlineto{\pgfqpoint{3.791917in}{3.816785in}}%
\pgfpathlineto{\pgfqpoint{3.835881in}{3.935579in}}%
\pgfpathlineto{\pgfqpoint{3.952131in}{4.252880in}}%
\pgfpathlineto{\pgfqpoint{3.986372in}{4.338135in}}%
\pgfpathlineto{\pgfqpoint{4.015540in}{4.404989in}}%
\pgfpathlineto{\pgfqpoint{4.041326in}{4.458727in}}%
\pgfpathlineto{\pgfqpoint{4.064576in}{4.502265in}}%
\pgfpathlineto{\pgfqpoint{4.086135in}{4.538070in}}%
\pgfpathlineto{\pgfqpoint{4.106003in}{4.566903in}}%
\pgfpathlineto{\pgfqpoint{4.124603in}{4.590092in}}%
\pgfpathlineto{\pgfqpoint{4.141935in}{4.608264in}}%
\pgfpathlineto{\pgfqpoint{4.158422in}{4.622391in}}%
\pgfpathlineto{\pgfqpoint{4.166031in}{4.627857in}}%
\pgfpathlineto{\pgfqpoint{4.167299in}{4.786202in}}%
\pgfpathlineto{\pgfqpoint{4.182517in}{4.794887in}}%
\pgfpathlineto{\pgfqpoint{4.196890in}{4.800599in}}%
\pgfpathlineto{\pgfqpoint{4.210840in}{4.803821in}}%
\pgfpathlineto{\pgfqpoint{4.224790in}{4.804762in}}%
\pgfpathlineto{\pgfqpoint{4.229863in}{4.804541in}}%
\pgfpathlineto{\pgfqpoint{4.231131in}{4.646939in}}%
\pgfpathlineto{\pgfqpoint{4.245081in}{4.644588in}}%
\pgfpathlineto{\pgfqpoint{4.259453in}{4.639834in}}%
\pgfpathlineto{\pgfqpoint{4.274249in}{4.632512in}}%
\pgfpathlineto{\pgfqpoint{4.289467in}{4.622470in}}%
\pgfpathlineto{\pgfqpoint{4.305531in}{4.609193in}}%
\pgfpathlineto{\pgfqpoint{4.322863in}{4.591900in}}%
\pgfpathlineto{\pgfqpoint{4.341463in}{4.570085in}}%
\pgfpathlineto{\pgfqpoint{4.361331in}{4.543293in}}%
\pgfpathlineto{\pgfqpoint{4.383313in}{4.509794in}}%
\pgfpathlineto{\pgfqpoint{4.407408in}{4.468918in}}%
\pgfpathlineto{\pgfqpoint{4.434885in}{4.417779in}}%
\pgfpathlineto{\pgfqpoint{4.467435in}{4.352288in}}%
\pgfpathlineto{\pgfqpoint{4.511399in}{4.258409in}}%
\pgfpathlineto{\pgfqpoint{4.602285in}{4.063266in}}%
\pgfpathlineto{\pgfqpoint{4.635681in}{3.997565in}}%
\pgfpathlineto{\pgfqpoint{4.663581in}{3.947401in}}%
\pgfpathlineto{\pgfqpoint{4.688099in}{3.907735in}}%
\pgfpathlineto{\pgfqpoint{4.710081in}{3.876233in}}%
\pgfpathlineto{\pgfqpoint{4.730372in}{3.850932in}}%
\pgfpathlineto{\pgfqpoint{4.748972in}{3.831182in}}%
\pgfpathlineto{\pgfqpoint{4.766303in}{3.815919in}}%
\pgfpathlineto{\pgfqpoint{4.782367in}{3.804602in}}%
\pgfpathlineto{\pgfqpoint{4.797585in}{3.796478in}}%
\pgfpathlineto{\pgfqpoint{4.812381in}{3.791066in}}%
\pgfpathlineto{\pgfqpoint{4.826753in}{3.788204in}}%
\pgfpathlineto{\pgfqpoint{4.840703in}{3.787713in}}%
\pgfpathlineto{\pgfqpoint{4.854653in}{3.789496in}}%
\pgfpathlineto{\pgfqpoint{4.868603in}{3.793565in}}%
\pgfpathlineto{\pgfqpoint{4.882976in}{3.800146in}}%
\pgfpathlineto{\pgfqpoint{4.897772in}{3.809443in}}%
\pgfpathlineto{\pgfqpoint{4.912990in}{3.821650in}}%
\pgfpathlineto{\pgfqpoint{4.929053in}{3.837401in}}%
\pgfpathlineto{\pgfqpoint{4.946385in}{3.857624in}}%
\pgfpathlineto{\pgfqpoint{4.964563in}{3.882325in}}%
\pgfpathlineto{\pgfqpoint{4.984008in}{3.912546in}}%
\pgfpathlineto{\pgfqpoint{5.005144in}{3.949611in}}%
\pgfpathlineto{\pgfqpoint{5.027972in}{3.994225in}}%
\pgfpathlineto{\pgfqpoint{5.043613in}{4.027323in}}%
\pgfpathlineto{\pgfqpoint{5.043613in}{4.027323in}}%
\pgfusepath{stroke}%
\end{pgfscope}%
\begin{pgfscope}%
\pgfsetbuttcap%
\pgfsetmiterjoin%
\definecolor{currentfill}{rgb}{1.000000,1.000000,1.000000}%
\pgfsetfillcolor{currentfill}%
\pgfsetfillopacity{0.800000}%
\pgfsetlinewidth{1.003750pt}%
\definecolor{currentstroke}{rgb}{0.800000,0.800000,0.800000}%
\pgfsetstrokecolor{currentstroke}%
\pgfsetstrokeopacity{0.800000}%
\pgfsetdash{}{0pt}%
\pgfpathmoveto{\pgfqpoint{3.913597in}{3.137930in}}%
\pgfpathlineto{\pgfqpoint{4.946390in}{3.137930in}}%
\pgfpathquadraticcurveto{\pgfqpoint{4.974168in}{3.137930in}}{\pgfqpoint{4.974168in}{3.165708in}}%
\pgfpathlineto{\pgfqpoint{4.974168in}{3.780888in}}%
\pgfpathquadraticcurveto{\pgfqpoint{4.974168in}{3.808666in}}{\pgfqpoint{4.946390in}{3.808666in}}%
\pgfpathlineto{\pgfqpoint{3.913597in}{3.808666in}}%
\pgfpathquadraticcurveto{\pgfqpoint{3.885819in}{3.808666in}}{\pgfqpoint{3.885819in}{3.780888in}}%
\pgfpathlineto{\pgfqpoint{3.885819in}{3.165708in}}%
\pgfpathquadraticcurveto{\pgfqpoint{3.885819in}{3.137930in}}{\pgfqpoint{3.913597in}{3.137930in}}%
\pgfpathlineto{\pgfqpoint{3.913597in}{3.137930in}}%
\pgfpathclose%
\pgfusepath{stroke,fill}%
\end{pgfscope}%
\begin{pgfscope}%
\pgfsetrectcap%
\pgfsetroundjoin%
\pgfsetlinewidth{1.505625pt}%
\definecolor{currentstroke}{rgb}{0.000000,0.000000,0.000000}%
\pgfsetstrokecolor{currentstroke}%
\pgfsetdash{}{0pt}%
\pgfpathmoveto{\pgfqpoint{3.941375in}{3.696199in}}%
\pgfpathlineto{\pgfqpoint{4.080263in}{3.696199in}}%
\pgfpathlineto{\pgfqpoint{4.219152in}{3.696199in}}%
\pgfusepath{stroke}%
\end{pgfscope}%
\begin{pgfscope}%
\definecolor{textcolor}{rgb}{0.000000,0.000000,0.000000}%
\pgfsetstrokecolor{textcolor}%
\pgfsetfillcolor{textcolor}%
\pgftext[x=4.330263in,y=3.647588in,left,base]{\color{textcolor}{\rmfamily\fontsize{10.000000}{12.000000}\selectfont\catcode`\^=\active\def^{\ifmmode\sp\else\^{}\fi}\catcode`\%=\active\def%{\%}$f(t)$}}%
\end{pgfscope}%
\begin{pgfscope}%
\pgfsetbuttcap%
\pgfsetroundjoin%
\pgfsetlinewidth{1.505625pt}%
\definecolor{currentstroke}{rgb}{0.000000,0.000000,0.000000}%
\pgfsetstrokecolor{currentstroke}%
\pgfsetdash{{5.550000pt}{2.400000pt}}{0.000000pt}%
\pgfpathmoveto{\pgfqpoint{3.941375in}{3.486509in}}%
\pgfpathlineto{\pgfqpoint{4.080263in}{3.486509in}}%
\pgfpathlineto{\pgfqpoint{4.219152in}{3.486509in}}%
\pgfusepath{stroke}%
\end{pgfscope}%
\begin{pgfscope}%
\definecolor{textcolor}{rgb}{0.000000,0.000000,0.000000}%
\pgfsetstrokecolor{textcolor}%
\pgfsetfillcolor{textcolor}%
\pgftext[x=4.330263in,y=3.437898in,left,base]{\color{textcolor}{\rmfamily\fontsize{10.000000}{12.000000}\selectfont\catcode`\^=\active\def^{\ifmmode\sp\else\^{}\fi}\catcode`\%=\active\def%{\%}$f(t+0.2)$}}%
\end{pgfscope}%
\begin{pgfscope}%
\pgfsetbuttcap%
\pgfsetroundjoin%
\pgfsetlinewidth{1.505625pt}%
\definecolor{currentstroke}{rgb}{0.000000,0.000000,0.000000}%
\pgfsetstrokecolor{currentstroke}%
\pgfsetdash{{1.500000pt}{2.475000pt}}{0.000000pt}%
\pgfpathmoveto{\pgfqpoint{3.941375in}{3.276819in}}%
\pgfpathlineto{\pgfqpoint{4.080263in}{3.276819in}}%
\pgfpathlineto{\pgfqpoint{4.219152in}{3.276819in}}%
\pgfusepath{stroke}%
\end{pgfscope}%
\begin{pgfscope}%
\definecolor{textcolor}{rgb}{0.000000,0.000000,0.000000}%
\pgfsetstrokecolor{textcolor}%
\pgfsetfillcolor{textcolor}%
\pgftext[x=4.330263in,y=3.228208in,left,base]{\color{textcolor}{\rmfamily\fontsize{10.000000}{12.000000}\selectfont\catcode`\^=\active\def^{\ifmmode\sp\else\^{}\fi}\catcode`\%=\active\def%{\%}$f(t-0.2)$}}%
\end{pgfscope}%
\begin{pgfscope}%
\pgfsetbuttcap%
\pgfsetmiterjoin%
\definecolor{currentfill}{rgb}{1.000000,1.000000,1.000000}%
\pgfsetfillcolor{currentfill}%
\pgfsetlinewidth{0.000000pt}%
\definecolor{currentstroke}{rgb}{0.000000,0.000000,0.000000}%
\pgfsetstrokecolor{currentstroke}%
\pgfsetstrokeopacity{0.000000}%
\pgfsetdash{}{0pt}%
\pgfpathmoveto{\pgfqpoint{0.393613in}{0.548486in}}%
\pgfpathlineto{\pgfqpoint{2.507249in}{0.548486in}}%
\pgfpathlineto{\pgfqpoint{2.507249in}{2.648486in}}%
\pgfpathlineto{\pgfqpoint{0.393613in}{2.648486in}}%
\pgfpathlineto{\pgfqpoint{0.393613in}{0.548486in}}%
\pgfpathclose%
\pgfusepath{fill}%
\end{pgfscope}%
\begin{pgfscope}%
\pgfsetbuttcap%
\pgfsetroundjoin%
\definecolor{currentfill}{rgb}{0.000000,0.000000,0.000000}%
\pgfsetfillcolor{currentfill}%
\pgfsetlinewidth{0.803000pt}%
\definecolor{currentstroke}{rgb}{0.000000,0.000000,0.000000}%
\pgfsetstrokecolor{currentstroke}%
\pgfsetdash{}{0pt}%
\pgfsys@defobject{currentmarker}{\pgfqpoint{0.000000in}{-0.048611in}}{\pgfqpoint{0.000000in}{0.000000in}}{%
\pgfpathmoveto{\pgfqpoint{0.000000in}{0.000000in}}%
\pgfpathlineto{\pgfqpoint{0.000000in}{-0.048611in}}%
\pgfusepath{stroke,fill}%
}%
\begin{pgfscope}%
\pgfsys@transformshift{0.393613in}{0.548486in}%
\pgfsys@useobject{currentmarker}{}%
\end{pgfscope}%
\end{pgfscope}%
\begin{pgfscope}%
\definecolor{textcolor}{rgb}{0.000000,0.000000,0.000000}%
\pgfsetstrokecolor{textcolor}%
\pgfsetfillcolor{textcolor}%
\pgftext[x=0.393613in,y=0.451264in,,top]{\color{textcolor}{\rmfamily\fontsize{10.000000}{12.000000}\selectfont\catcode`\^=\active\def^{\ifmmode\sp\else\^{}\fi}\catcode`\%=\active\def%{\%}\ensuremath{-}1.0}}%
\end{pgfscope}%
\begin{pgfscope}%
\pgfsetbuttcap%
\pgfsetroundjoin%
\definecolor{currentfill}{rgb}{0.000000,0.000000,0.000000}%
\pgfsetfillcolor{currentfill}%
\pgfsetlinewidth{0.803000pt}%
\definecolor{currentstroke}{rgb}{0.000000,0.000000,0.000000}%
\pgfsetstrokecolor{currentstroke}%
\pgfsetdash{}{0pt}%
\pgfsys@defobject{currentmarker}{\pgfqpoint{0.000000in}{-0.048611in}}{\pgfqpoint{0.000000in}{0.000000in}}{%
\pgfpathmoveto{\pgfqpoint{0.000000in}{0.000000in}}%
\pgfpathlineto{\pgfqpoint{0.000000in}{-0.048611in}}%
\pgfusepath{stroke,fill}%
}%
\begin{pgfscope}%
\pgfsys@transformshift{0.922022in}{0.548486in}%
\pgfsys@useobject{currentmarker}{}%
\end{pgfscope}%
\end{pgfscope}%
\begin{pgfscope}%
\definecolor{textcolor}{rgb}{0.000000,0.000000,0.000000}%
\pgfsetstrokecolor{textcolor}%
\pgfsetfillcolor{textcolor}%
\pgftext[x=0.922022in,y=0.451264in,,top]{\color{textcolor}{\rmfamily\fontsize{10.000000}{12.000000}\selectfont\catcode`\^=\active\def^{\ifmmode\sp\else\^{}\fi}\catcode`\%=\active\def%{\%}\ensuremath{-}0.5}}%
\end{pgfscope}%
\begin{pgfscope}%
\pgfsetbuttcap%
\pgfsetroundjoin%
\definecolor{currentfill}{rgb}{0.000000,0.000000,0.000000}%
\pgfsetfillcolor{currentfill}%
\pgfsetlinewidth{0.803000pt}%
\definecolor{currentstroke}{rgb}{0.000000,0.000000,0.000000}%
\pgfsetstrokecolor{currentstroke}%
\pgfsetdash{}{0pt}%
\pgfsys@defobject{currentmarker}{\pgfqpoint{0.000000in}{-0.048611in}}{\pgfqpoint{0.000000in}{0.000000in}}{%
\pgfpathmoveto{\pgfqpoint{0.000000in}{0.000000in}}%
\pgfpathlineto{\pgfqpoint{0.000000in}{-0.048611in}}%
\pgfusepath{stroke,fill}%
}%
\begin{pgfscope}%
\pgfsys@transformshift{1.450431in}{0.548486in}%
\pgfsys@useobject{currentmarker}{}%
\end{pgfscope}%
\end{pgfscope}%
\begin{pgfscope}%
\definecolor{textcolor}{rgb}{0.000000,0.000000,0.000000}%
\pgfsetstrokecolor{textcolor}%
\pgfsetfillcolor{textcolor}%
\pgftext[x=1.450431in,y=0.451264in,,top]{\color{textcolor}{\rmfamily\fontsize{10.000000}{12.000000}\selectfont\catcode`\^=\active\def^{\ifmmode\sp\else\^{}\fi}\catcode`\%=\active\def%{\%}0.0}}%
\end{pgfscope}%
\begin{pgfscope}%
\pgfsetbuttcap%
\pgfsetroundjoin%
\definecolor{currentfill}{rgb}{0.000000,0.000000,0.000000}%
\pgfsetfillcolor{currentfill}%
\pgfsetlinewidth{0.803000pt}%
\definecolor{currentstroke}{rgb}{0.000000,0.000000,0.000000}%
\pgfsetstrokecolor{currentstroke}%
\pgfsetdash{}{0pt}%
\pgfsys@defobject{currentmarker}{\pgfqpoint{0.000000in}{-0.048611in}}{\pgfqpoint{0.000000in}{0.000000in}}{%
\pgfpathmoveto{\pgfqpoint{0.000000in}{0.000000in}}%
\pgfpathlineto{\pgfqpoint{0.000000in}{-0.048611in}}%
\pgfusepath{stroke,fill}%
}%
\begin{pgfscope}%
\pgfsys@transformshift{1.978840in}{0.548486in}%
\pgfsys@useobject{currentmarker}{}%
\end{pgfscope}%
\end{pgfscope}%
\begin{pgfscope}%
\definecolor{textcolor}{rgb}{0.000000,0.000000,0.000000}%
\pgfsetstrokecolor{textcolor}%
\pgfsetfillcolor{textcolor}%
\pgftext[x=1.978840in,y=0.451264in,,top]{\color{textcolor}{\rmfamily\fontsize{10.000000}{12.000000}\selectfont\catcode`\^=\active\def^{\ifmmode\sp\else\^{}\fi}\catcode`\%=\active\def%{\%}0.5}}%
\end{pgfscope}%
\begin{pgfscope}%
\pgfsetbuttcap%
\pgfsetroundjoin%
\definecolor{currentfill}{rgb}{0.000000,0.000000,0.000000}%
\pgfsetfillcolor{currentfill}%
\pgfsetlinewidth{0.803000pt}%
\definecolor{currentstroke}{rgb}{0.000000,0.000000,0.000000}%
\pgfsetstrokecolor{currentstroke}%
\pgfsetdash{}{0pt}%
\pgfsys@defobject{currentmarker}{\pgfqpoint{0.000000in}{-0.048611in}}{\pgfqpoint{0.000000in}{0.000000in}}{%
\pgfpathmoveto{\pgfqpoint{0.000000in}{0.000000in}}%
\pgfpathlineto{\pgfqpoint{0.000000in}{-0.048611in}}%
\pgfusepath{stroke,fill}%
}%
\begin{pgfscope}%
\pgfsys@transformshift{2.507249in}{0.548486in}%
\pgfsys@useobject{currentmarker}{}%
\end{pgfscope}%
\end{pgfscope}%
\begin{pgfscope}%
\definecolor{textcolor}{rgb}{0.000000,0.000000,0.000000}%
\pgfsetstrokecolor{textcolor}%
\pgfsetfillcolor{textcolor}%
\pgftext[x=2.507249in,y=0.451264in,,top]{\color{textcolor}{\rmfamily\fontsize{10.000000}{12.000000}\selectfont\catcode`\^=\active\def^{\ifmmode\sp\else\^{}\fi}\catcode`\%=\active\def%{\%}1.0}}%
\end{pgfscope}%
\begin{pgfscope}%
\definecolor{textcolor}{rgb}{0.000000,0.000000,0.000000}%
\pgfsetstrokecolor{textcolor}%
\pgfsetfillcolor{textcolor}%
\pgftext[x=1.450431in,y=0.261295in,,top]{\color{textcolor}{\rmfamily\fontsize{12.000000}{14.400000}\selectfont\catcode`\^=\active\def^{\ifmmode\sp\else\^{}\fi}\catcode`\%=\active\def%{\%}$t$}}%
\end{pgfscope}%
\begin{pgfscope}%
\pgfsetbuttcap%
\pgfsetroundjoin%
\definecolor{currentfill}{rgb}{0.000000,0.000000,0.000000}%
\pgfsetfillcolor{currentfill}%
\pgfsetlinewidth{0.803000pt}%
\definecolor{currentstroke}{rgb}{0.000000,0.000000,0.000000}%
\pgfsetstrokecolor{currentstroke}%
\pgfsetdash{}{0pt}%
\pgfsys@defobject{currentmarker}{\pgfqpoint{-0.048611in}{0.000000in}}{\pgfqpoint{-0.000000in}{0.000000in}}{%
\pgfpathmoveto{\pgfqpoint{-0.000000in}{0.000000in}}%
\pgfpathlineto{\pgfqpoint{-0.048611in}{0.000000in}}%
\pgfusepath{stroke,fill}%
}%
\begin{pgfscope}%
\pgfsys@transformshift{0.393613in}{0.548486in}%
\pgfsys@useobject{currentmarker}{}%
\end{pgfscope}%
\end{pgfscope}%
\begin{pgfscope}%
\definecolor{textcolor}{rgb}{0.000000,0.000000,0.000000}%
\pgfsetstrokecolor{textcolor}%
\pgfsetfillcolor{textcolor}%
\pgftext[x=0.100000in, y=0.495724in, left, base]{\color{textcolor}{\rmfamily\fontsize{10.000000}{12.000000}\selectfont\catcode`\^=\active\def^{\ifmmode\sp\else\^{}\fi}\catcode`\%=\active\def%{\%}\ensuremath{-}2}}%
\end{pgfscope}%
\begin{pgfscope}%
\pgfsetbuttcap%
\pgfsetroundjoin%
\definecolor{currentfill}{rgb}{0.000000,0.000000,0.000000}%
\pgfsetfillcolor{currentfill}%
\pgfsetlinewidth{0.803000pt}%
\definecolor{currentstroke}{rgb}{0.000000,0.000000,0.000000}%
\pgfsetstrokecolor{currentstroke}%
\pgfsetdash{}{0pt}%
\pgfsys@defobject{currentmarker}{\pgfqpoint{-0.048611in}{0.000000in}}{\pgfqpoint{-0.000000in}{0.000000in}}{%
\pgfpathmoveto{\pgfqpoint{-0.000000in}{0.000000in}}%
\pgfpathlineto{\pgfqpoint{-0.048611in}{0.000000in}}%
\pgfusepath{stroke,fill}%
}%
\begin{pgfscope}%
\pgfsys@transformshift{0.393613in}{1.073486in}%
\pgfsys@useobject{currentmarker}{}%
\end{pgfscope}%
\end{pgfscope}%
\begin{pgfscope}%
\definecolor{textcolor}{rgb}{0.000000,0.000000,0.000000}%
\pgfsetstrokecolor{textcolor}%
\pgfsetfillcolor{textcolor}%
\pgftext[x=0.100000in, y=1.020724in, left, base]{\color{textcolor}{\rmfamily\fontsize{10.000000}{12.000000}\selectfont\catcode`\^=\active\def^{\ifmmode\sp\else\^{}\fi}\catcode`\%=\active\def%{\%}\ensuremath{-}1}}%
\end{pgfscope}%
\begin{pgfscope}%
\pgfsetbuttcap%
\pgfsetroundjoin%
\definecolor{currentfill}{rgb}{0.000000,0.000000,0.000000}%
\pgfsetfillcolor{currentfill}%
\pgfsetlinewidth{0.803000pt}%
\definecolor{currentstroke}{rgb}{0.000000,0.000000,0.000000}%
\pgfsetstrokecolor{currentstroke}%
\pgfsetdash{}{0pt}%
\pgfsys@defobject{currentmarker}{\pgfqpoint{-0.048611in}{0.000000in}}{\pgfqpoint{-0.000000in}{0.000000in}}{%
\pgfpathmoveto{\pgfqpoint{-0.000000in}{0.000000in}}%
\pgfpathlineto{\pgfqpoint{-0.048611in}{0.000000in}}%
\pgfusepath{stroke,fill}%
}%
\begin{pgfscope}%
\pgfsys@transformshift{0.393613in}{1.598486in}%
\pgfsys@useobject{currentmarker}{}%
\end{pgfscope}%
\end{pgfscope}%
\begin{pgfscope}%
\definecolor{textcolor}{rgb}{0.000000,0.000000,0.000000}%
\pgfsetstrokecolor{textcolor}%
\pgfsetfillcolor{textcolor}%
\pgftext[x=0.208025in, y=1.545724in, left, base]{\color{textcolor}{\rmfamily\fontsize{10.000000}{12.000000}\selectfont\catcode`\^=\active\def^{\ifmmode\sp\else\^{}\fi}\catcode`\%=\active\def%{\%}0}}%
\end{pgfscope}%
\begin{pgfscope}%
\pgfsetbuttcap%
\pgfsetroundjoin%
\definecolor{currentfill}{rgb}{0.000000,0.000000,0.000000}%
\pgfsetfillcolor{currentfill}%
\pgfsetlinewidth{0.803000pt}%
\definecolor{currentstroke}{rgb}{0.000000,0.000000,0.000000}%
\pgfsetstrokecolor{currentstroke}%
\pgfsetdash{}{0pt}%
\pgfsys@defobject{currentmarker}{\pgfqpoint{-0.048611in}{0.000000in}}{\pgfqpoint{-0.000000in}{0.000000in}}{%
\pgfpathmoveto{\pgfqpoint{-0.000000in}{0.000000in}}%
\pgfpathlineto{\pgfqpoint{-0.048611in}{0.000000in}}%
\pgfusepath{stroke,fill}%
}%
\begin{pgfscope}%
\pgfsys@transformshift{0.393613in}{2.123486in}%
\pgfsys@useobject{currentmarker}{}%
\end{pgfscope}%
\end{pgfscope}%
\begin{pgfscope}%
\definecolor{textcolor}{rgb}{0.000000,0.000000,0.000000}%
\pgfsetstrokecolor{textcolor}%
\pgfsetfillcolor{textcolor}%
\pgftext[x=0.208025in, y=2.070724in, left, base]{\color{textcolor}{\rmfamily\fontsize{10.000000}{12.000000}\selectfont\catcode`\^=\active\def^{\ifmmode\sp\else\^{}\fi}\catcode`\%=\active\def%{\%}1}}%
\end{pgfscope}%
\begin{pgfscope}%
\pgfsetbuttcap%
\pgfsetroundjoin%
\definecolor{currentfill}{rgb}{0.000000,0.000000,0.000000}%
\pgfsetfillcolor{currentfill}%
\pgfsetlinewidth{0.803000pt}%
\definecolor{currentstroke}{rgb}{0.000000,0.000000,0.000000}%
\pgfsetstrokecolor{currentstroke}%
\pgfsetdash{}{0pt}%
\pgfsys@defobject{currentmarker}{\pgfqpoint{-0.048611in}{0.000000in}}{\pgfqpoint{-0.000000in}{0.000000in}}{%
\pgfpathmoveto{\pgfqpoint{-0.000000in}{0.000000in}}%
\pgfpathlineto{\pgfqpoint{-0.048611in}{0.000000in}}%
\pgfusepath{stroke,fill}%
}%
\begin{pgfscope}%
\pgfsys@transformshift{0.393613in}{2.648486in}%
\pgfsys@useobject{currentmarker}{}%
\end{pgfscope}%
\end{pgfscope}%
\begin{pgfscope}%
\definecolor{textcolor}{rgb}{0.000000,0.000000,0.000000}%
\pgfsetstrokecolor{textcolor}%
\pgfsetfillcolor{textcolor}%
\pgftext[x=0.208025in, y=2.595724in, left, base]{\color{textcolor}{\rmfamily\fontsize{10.000000}{12.000000}\selectfont\catcode`\^=\active\def^{\ifmmode\sp\else\^{}\fi}\catcode`\%=\active\def%{\%}2}}%
\end{pgfscope}%
\begin{pgfscope}%
\pgfpathrectangle{\pgfqpoint{0.393613in}{0.548486in}}{\pgfqpoint{2.113636in}{2.100000in}}%
\pgfusepath{clip}%
\pgfsetbuttcap%
\pgfsetroundjoin%
\pgfsetlinewidth{0.501875pt}%
\definecolor{currentstroke}{rgb}{0.000000,0.000000,0.000000}%
\pgfsetstrokecolor{currentstroke}%
\pgfsetdash{{1.850000pt}{0.800000pt}}{0.000000pt}%
\pgfpathmoveto{\pgfqpoint{0.393613in}{1.598486in}}%
\pgfpathlineto{\pgfqpoint{2.507249in}{1.598486in}}%
\pgfusepath{stroke}%
\end{pgfscope}%
\begin{pgfscope}%
\pgfpathrectangle{\pgfqpoint{0.393613in}{0.548486in}}{\pgfqpoint{2.113636in}{2.100000in}}%
\pgfusepath{clip}%
\pgfsetbuttcap%
\pgfsetroundjoin%
\pgfsetlinewidth{0.501875pt}%
\definecolor{currentstroke}{rgb}{0.000000,0.000000,0.000000}%
\pgfsetstrokecolor{currentstroke}%
\pgfsetdash{{1.850000pt}{0.800000pt}}{0.000000pt}%
\pgfpathmoveto{\pgfqpoint{1.450431in}{0.548486in}}%
\pgfpathlineto{\pgfqpoint{1.450431in}{2.648486in}}%
\pgfusepath{stroke}%
\end{pgfscope}%
\begin{pgfscope}%
\pgfpathrectangle{\pgfqpoint{0.393613in}{0.548486in}}{\pgfqpoint{2.113636in}{2.100000in}}%
\pgfusepath{clip}%
\pgfsetrectcap%
\pgfsetroundjoin%
\pgfsetlinewidth{1.505625pt}%
\definecolor{currentstroke}{rgb}{0.000000,0.000000,0.000000}%
\pgfsetstrokecolor{currentstroke}%
\pgfsetdash{}{0pt}%
\pgfpathmoveto{\pgfqpoint{0.393613in}{1.432409in}}%
\pgfpathlineto{\pgfqpoint{0.438844in}{1.335579in}}%
\pgfpathlineto{\pgfqpoint{0.524235in}{1.152104in}}%
\pgfpathlineto{\pgfqpoint{0.558053in}{1.085241in}}%
\pgfpathlineto{\pgfqpoint{0.586376in}{1.034026in}}%
\pgfpathlineto{\pgfqpoint{0.610894in}{0.994098in}}%
\pgfpathlineto{\pgfqpoint{0.633299in}{0.961760in}}%
\pgfpathlineto{\pgfqpoint{0.653590in}{0.936269in}}%
\pgfpathlineto{\pgfqpoint{0.672190in}{0.916333in}}%
\pgfpathlineto{\pgfqpoint{0.689522in}{0.900888in}}%
\pgfpathlineto{\pgfqpoint{0.706008in}{0.889130in}}%
\pgfpathlineto{\pgfqpoint{0.721226in}{0.880906in}}%
\pgfpathlineto{\pgfqpoint{0.736022in}{0.875395in}}%
\pgfpathlineto{\pgfqpoint{0.750394in}{0.872434in}}%
\pgfpathlineto{\pgfqpoint{0.764344in}{0.871847in}}%
\pgfpathlineto{\pgfqpoint{0.778294in}{0.873533in}}%
\pgfpathlineto{\pgfqpoint{0.792244in}{0.877504in}}%
\pgfpathlineto{\pgfqpoint{0.806617in}{0.883985in}}%
\pgfpathlineto{\pgfqpoint{0.821413in}{0.893180in}}%
\pgfpathlineto{\pgfqpoint{0.836631in}{0.905283in}}%
\pgfpathlineto{\pgfqpoint{0.852694in}{0.920925in}}%
\pgfpathlineto{\pgfqpoint{0.870026in}{0.941035in}}%
\pgfpathlineto{\pgfqpoint{0.888203in}{0.965620in}}%
\pgfpathlineto{\pgfqpoint{0.907649in}{0.995725in}}%
\pgfpathlineto{\pgfqpoint{0.928785in}{1.032670in}}%
\pgfpathlineto{\pgfqpoint{0.951612in}{1.077165in}}%
\pgfpathlineto{\pgfqpoint{0.976976in}{1.131676in}}%
\pgfpathlineto{\pgfqpoint{1.005299in}{1.198049in}}%
\pgfpathlineto{\pgfqpoint{1.038272in}{1.281291in}}%
\pgfpathlineto{\pgfqpoint{1.079699in}{1.392339in}}%
\pgfpathlineto{\pgfqpoint{1.214972in}{1.759868in}}%
\pgfpathlineto{\pgfqpoint{1.247522in}{1.839104in}}%
\pgfpathlineto{\pgfqpoint{1.275422in}{1.901406in}}%
\pgfpathlineto{\pgfqpoint{1.300363in}{1.951896in}}%
\pgfpathlineto{\pgfqpoint{1.323190in}{1.993267in}}%
\pgfpathlineto{\pgfqpoint{1.344326in}{2.027086in}}%
\pgfpathlineto{\pgfqpoint{1.363772in}{2.054144in}}%
\pgfpathlineto{\pgfqpoint{1.381949in}{2.075758in}}%
\pgfpathlineto{\pgfqpoint{1.398858in}{2.092563in}}%
\pgfpathlineto{\pgfqpoint{1.414922in}{2.105511in}}%
\pgfpathlineto{\pgfqpoint{1.418303in}{2.107857in}}%
\pgfpathlineto{\pgfqpoint{1.419572in}{2.266202in}}%
\pgfpathlineto{\pgfqpoint{1.434790in}{2.274887in}}%
\pgfpathlineto{\pgfqpoint{1.449162in}{2.280599in}}%
\pgfpathlineto{\pgfqpoint{1.463113in}{2.283821in}}%
\pgfpathlineto{\pgfqpoint{1.477063in}{2.284762in}}%
\pgfpathlineto{\pgfqpoint{1.481712in}{2.284571in}}%
\pgfpathlineto{\pgfqpoint{1.482981in}{2.126975in}}%
\pgfpathlineto{\pgfqpoint{1.496931in}{2.124692in}}%
\pgfpathlineto{\pgfqpoint{1.511303in}{2.120007in}}%
\pgfpathlineto{\pgfqpoint{1.526099in}{2.112755in}}%
\pgfpathlineto{\pgfqpoint{1.541317in}{2.102783in}}%
\pgfpathlineto{\pgfqpoint{1.557381in}{2.089576in}}%
\pgfpathlineto{\pgfqpoint{1.574712in}{2.072357in}}%
\pgfpathlineto{\pgfqpoint{1.593312in}{2.050617in}}%
\pgfpathlineto{\pgfqpoint{1.613181in}{2.023899in}}%
\pgfpathlineto{\pgfqpoint{1.635162in}{1.990474in}}%
\pgfpathlineto{\pgfqpoint{1.659258in}{1.949669in}}%
\pgfpathlineto{\pgfqpoint{1.686735in}{1.898598in}}%
\pgfpathlineto{\pgfqpoint{1.719285in}{1.833166in}}%
\pgfpathlineto{\pgfqpoint{1.763249in}{1.739329in}}%
\pgfpathlineto{\pgfqpoint{1.854558in}{1.543266in}}%
\pgfpathlineto{\pgfqpoint{1.887953in}{1.477565in}}%
\pgfpathlineto{\pgfqpoint{1.915853in}{1.427401in}}%
\pgfpathlineto{\pgfqpoint{1.940372in}{1.387735in}}%
\pgfpathlineto{\pgfqpoint{1.962353in}{1.356233in}}%
\pgfpathlineto{\pgfqpoint{1.982644in}{1.330932in}}%
\pgfpathlineto{\pgfqpoint{2.001244in}{1.311182in}}%
\pgfpathlineto{\pgfqpoint{2.018576in}{1.295919in}}%
\pgfpathlineto{\pgfqpoint{2.034640in}{1.284602in}}%
\pgfpathlineto{\pgfqpoint{2.049858in}{1.276478in}}%
\pgfpathlineto{\pgfqpoint{2.064653in}{1.271066in}}%
\pgfpathlineto{\pgfqpoint{2.079026in}{1.268204in}}%
\pgfpathlineto{\pgfqpoint{2.092976in}{1.267713in}}%
\pgfpathlineto{\pgfqpoint{2.106926in}{1.269496in}}%
\pgfpathlineto{\pgfqpoint{2.120876in}{1.273565in}}%
\pgfpathlineto{\pgfqpoint{2.135249in}{1.280146in}}%
\pgfpathlineto{\pgfqpoint{2.150044in}{1.289443in}}%
\pgfpathlineto{\pgfqpoint{2.165263in}{1.301650in}}%
\pgfpathlineto{\pgfqpoint{2.181326in}{1.317401in}}%
\pgfpathlineto{\pgfqpoint{2.198658in}{1.337624in}}%
\pgfpathlineto{\pgfqpoint{2.216835in}{1.362325in}}%
\pgfpathlineto{\pgfqpoint{2.236281in}{1.392546in}}%
\pgfpathlineto{\pgfqpoint{2.257417in}{1.429611in}}%
\pgfpathlineto{\pgfqpoint{2.280244in}{1.474225in}}%
\pgfpathlineto{\pgfqpoint{2.305608in}{1.528854in}}%
\pgfpathlineto{\pgfqpoint{2.333931in}{1.595340in}}%
\pgfpathlineto{\pgfqpoint{2.366903in}{1.678683in}}%
\pgfpathlineto{\pgfqpoint{2.408753in}{1.790970in}}%
\pgfpathlineto{\pgfqpoint{2.507249in}{2.062409in}}%
\pgfpathlineto{\pgfqpoint{2.507249in}{2.062409in}}%
\pgfusepath{stroke}%
\end{pgfscope}%
\begin{pgfscope}%
\pgfpathrectangle{\pgfqpoint{0.393613in}{0.548486in}}{\pgfqpoint{2.113636in}{2.100000in}}%
\pgfusepath{clip}%
\pgfsetbuttcap%
\pgfsetroundjoin%
\pgfsetlinewidth{1.505625pt}%
\definecolor{currentstroke}{rgb}{0.000000,0.000000,0.000000}%
\pgfsetstrokecolor{currentstroke}%
\pgfsetdash{{5.550000pt}{2.400000pt}}{0.000000pt}%
\pgfpathmoveto{\pgfqpoint{0.393613in}{1.349370in}}%
\pgfpathlineto{\pgfqpoint{0.437576in}{1.208269in}}%
\pgfpathlineto{\pgfqpoint{0.524235in}{0.928913in}}%
\pgfpathlineto{\pgfqpoint{0.557208in}{0.831023in}}%
\pgfpathlineto{\pgfqpoint{0.584685in}{0.756166in}}%
\pgfpathlineto{\pgfqpoint{0.608781in}{0.696799in}}%
\pgfpathlineto{\pgfqpoint{0.630340in}{0.649437in}}%
\pgfpathlineto{\pgfqpoint{0.649785in}{0.611896in}}%
\pgfpathlineto{\pgfqpoint{0.667540in}{0.582254in}}%
\pgfpathlineto{\pgfqpoint{0.684026in}{0.558930in}}%
\pgfpathlineto{\pgfqpoint{0.699244in}{0.541158in}}%
\pgfpathlineto{\pgfqpoint{0.701851in}{0.538486in}}%
\pgfpathmoveto{\pgfqpoint{0.819458in}{0.538486in}}%
\pgfpathlineto{\pgfqpoint{0.832826in}{0.553768in}}%
\pgfpathlineto{\pgfqpoint{0.847199in}{0.573626in}}%
\pgfpathlineto{\pgfqpoint{0.862840in}{0.599200in}}%
\pgfpathlineto{\pgfqpoint{0.879326in}{0.630518in}}%
\pgfpathlineto{\pgfqpoint{0.897081in}{0.669088in}}%
\pgfpathlineto{\pgfqpoint{0.916526in}{0.716839in}}%
\pgfpathlineto{\pgfqpoint{0.937663in}{0.774899in}}%
\pgfpathlineto{\pgfqpoint{0.960490in}{0.844258in}}%
\pgfpathlineto{\pgfqpoint{0.985853in}{0.928605in}}%
\pgfpathlineto{\pgfqpoint{1.014599in}{1.032173in}}%
\pgfpathlineto{\pgfqpoint{1.048417in}{1.162667in}}%
\pgfpathlineto{\pgfqpoint{1.093226in}{1.345130in}}%
\pgfpathlineto{\pgfqpoint{1.198908in}{1.778713in}}%
\pgfpathlineto{\pgfqpoint{1.233572in}{1.909646in}}%
\pgfpathlineto{\pgfqpoint{1.262740in}{2.011455in}}%
\pgfpathlineto{\pgfqpoint{1.288526in}{2.093642in}}%
\pgfpathlineto{\pgfqpoint{1.311776in}{2.160539in}}%
\pgfpathlineto{\pgfqpoint{1.332912in}{2.214825in}}%
\pgfpathlineto{\pgfqpoint{1.352358in}{2.258875in}}%
\pgfpathlineto{\pgfqpoint{1.370535in}{2.294671in}}%
\pgfpathlineto{\pgfqpoint{1.387022in}{2.322463in}}%
\pgfpathlineto{\pgfqpoint{1.402663in}{2.344603in}}%
\pgfpathlineto{\pgfqpoint{1.417035in}{2.361246in}}%
\pgfpathlineto{\pgfqpoint{1.418303in}{2.362542in}}%
\pgfpathlineto{\pgfqpoint{1.419572in}{2.600061in}}%
\pgfpathlineto{\pgfqpoint{1.433099in}{2.611841in}}%
\pgfpathlineto{\pgfqpoint{1.445781in}{2.619966in}}%
\pgfpathlineto{\pgfqpoint{1.457617in}{2.624994in}}%
\pgfpathlineto{\pgfqpoint{1.469031in}{2.627505in}}%
\pgfpathlineto{\pgfqpoint{1.480444in}{2.627729in}}%
\pgfpathlineto{\pgfqpoint{1.481712in}{2.627613in}}%
\pgfpathlineto{\pgfqpoint{1.482981in}{2.391219in}}%
\pgfpathlineto{\pgfqpoint{1.494394in}{2.388668in}}%
\pgfpathlineto{\pgfqpoint{1.506231in}{2.383650in}}%
\pgfpathlineto{\pgfqpoint{1.518490in}{2.375941in}}%
\pgfpathlineto{\pgfqpoint{1.531594in}{2.364923in}}%
\pgfpathlineto{\pgfqpoint{1.545544in}{2.350113in}}%
\pgfpathlineto{\pgfqpoint{1.560340in}{2.331034in}}%
\pgfpathlineto{\pgfqpoint{1.576403in}{2.306533in}}%
\pgfpathlineto{\pgfqpoint{1.593735in}{2.275885in}}%
\pgfpathlineto{\pgfqpoint{1.612758in}{2.237512in}}%
\pgfpathlineto{\pgfqpoint{1.633472in}{2.190527in}}%
\pgfpathlineto{\pgfqpoint{1.656299in}{2.133100in}}%
\pgfpathlineto{\pgfqpoint{1.682085in}{2.062069in}}%
\pgfpathlineto{\pgfqpoint{1.712522in}{1.971443in}}%
\pgfpathlineto{\pgfqpoint{1.751835in}{1.846872in}}%
\pgfpathlineto{\pgfqpoint{1.864703in}{1.484888in}}%
\pgfpathlineto{\pgfqpoint{1.895140in}{1.397032in}}%
\pgfpathlineto{\pgfqpoint{1.921349in}{1.327944in}}%
\pgfpathlineto{\pgfqpoint{1.944176in}{1.273749in}}%
\pgfpathlineto{\pgfqpoint{1.964890in}{1.230055in}}%
\pgfpathlineto{\pgfqpoint{1.983912in}{1.194975in}}%
\pgfpathlineto{\pgfqpoint{2.001244in}{1.167529in}}%
\pgfpathlineto{\pgfqpoint{2.017308in}{1.146151in}}%
\pgfpathlineto{\pgfqpoint{2.032103in}{1.130062in}}%
\pgfpathlineto{\pgfqpoint{2.045631in}{1.118472in}}%
\pgfpathlineto{\pgfqpoint{2.058312in}{1.110380in}}%
\pgfpathlineto{\pgfqpoint{2.070572in}{1.105157in}}%
\pgfpathlineto{\pgfqpoint{2.081985in}{1.102623in}}%
\pgfpathlineto{\pgfqpoint{2.093399in}{1.102357in}}%
\pgfpathlineto{\pgfqpoint{2.104812in}{1.104376in}}%
\pgfpathlineto{\pgfqpoint{2.116226in}{1.108689in}}%
\pgfpathlineto{\pgfqpoint{2.128062in}{1.115586in}}%
\pgfpathlineto{\pgfqpoint{2.140322in}{1.125326in}}%
\pgfpathlineto{\pgfqpoint{2.153426in}{1.138645in}}%
\pgfpathlineto{\pgfqpoint{2.167376in}{1.156091in}}%
\pgfpathlineto{\pgfqpoint{2.182172in}{1.178223in}}%
\pgfpathlineto{\pgfqpoint{2.198235in}{1.206394in}}%
\pgfpathlineto{\pgfqpoint{2.215567in}{1.241489in}}%
\pgfpathlineto{\pgfqpoint{2.234167in}{1.284373in}}%
\pgfpathlineto{\pgfqpoint{2.254458in}{1.337009in}}%
\pgfpathlineto{\pgfqpoint{2.276440in}{1.400471in}}%
\pgfpathlineto{\pgfqpoint{2.300535in}{1.477056in}}%
\pgfpathlineto{\pgfqpoint{2.327590in}{1.570759in}}%
\pgfpathlineto{\pgfqpoint{2.358449in}{1.685945in}}%
\pgfpathlineto{\pgfqpoint{2.396494in}{1.837041in}}%
\pgfpathlineto{\pgfqpoint{2.458635in}{2.094595in}}%
\pgfpathlineto{\pgfqpoint{2.507249in}{2.294370in}}%
\pgfpathlineto{\pgfqpoint{2.507249in}{2.294370in}}%
\pgfusepath{stroke}%
\end{pgfscope}%
\begin{pgfscope}%
\pgfpathrectangle{\pgfqpoint{0.393613in}{0.548486in}}{\pgfqpoint{2.113636in}{2.100000in}}%
\pgfusepath{clip}%
\pgfsetbuttcap%
\pgfsetroundjoin%
\pgfsetlinewidth{1.505625pt}%
\definecolor{currentstroke}{rgb}{0.000000,0.000000,0.000000}%
\pgfsetstrokecolor{currentstroke}%
\pgfsetdash{{1.500000pt}{2.475000pt}}{0.000000pt}%
\pgfpathmoveto{\pgfqpoint{0.393613in}{1.515447in}}%
\pgfpathlineto{\pgfqpoint{0.445608in}{1.459655in}}%
\pgfpathlineto{\pgfqpoint{0.526349in}{1.373131in}}%
\pgfpathlineto{\pgfqpoint{0.563549in}{1.336704in}}%
\pgfpathlineto{\pgfqpoint{0.594408in}{1.309473in}}%
\pgfpathlineto{\pgfqpoint{0.621885in}{1.288097in}}%
\pgfpathlineto{\pgfqpoint{0.646826in}{1.271416in}}%
\pgfpathlineto{\pgfqpoint{0.670076in}{1.258456in}}%
\pgfpathlineto{\pgfqpoint{0.692058in}{1.248688in}}%
\pgfpathlineto{\pgfqpoint{0.713194in}{1.241707in}}%
\pgfpathlineto{\pgfqpoint{0.733485in}{1.237325in}}%
\pgfpathlineto{\pgfqpoint{0.753353in}{1.235303in}}%
\pgfpathlineto{\pgfqpoint{0.772799in}{1.235541in}}%
\pgfpathlineto{\pgfqpoint{0.792244in}{1.237995in}}%
\pgfpathlineto{\pgfqpoint{0.811690in}{1.242668in}}%
\pgfpathlineto{\pgfqpoint{0.831558in}{1.249719in}}%
\pgfpathlineto{\pgfqpoint{0.852272in}{1.259481in}}%
\pgfpathlineto{\pgfqpoint{0.873408in}{1.271914in}}%
\pgfpathlineto{\pgfqpoint{0.895813in}{1.287714in}}%
\pgfpathlineto{\pgfqpoint{0.919485in}{1.307188in}}%
\pgfpathlineto{\pgfqpoint{0.944849in}{1.330997in}}%
\pgfpathlineto{\pgfqpoint{0.972326in}{1.359898in}}%
\pgfpathlineto{\pgfqpoint{1.003185in}{1.395702in}}%
\pgfpathlineto{\pgfqpoint{1.038694in}{1.440439in}}%
\pgfpathlineto{\pgfqpoint{1.084349in}{1.501795in}}%
\pgfpathlineto{\pgfqpoint{1.211590in}{1.674889in}}%
\pgfpathlineto{\pgfqpoint{1.247522in}{1.718795in}}%
\pgfpathlineto{\pgfqpoint{1.278381in}{1.753076in}}%
\pgfpathlineto{\pgfqpoint{1.305858in}{1.780391in}}%
\pgfpathlineto{\pgfqpoint{1.331222in}{1.802565in}}%
\pgfpathlineto{\pgfqpoint{1.354894in}{1.820387in}}%
\pgfpathlineto{\pgfqpoint{1.377299in}{1.834530in}}%
\pgfpathlineto{\pgfqpoint{1.398858in}{1.845525in}}%
\pgfpathlineto{\pgfqpoint{1.418303in}{1.853171in}}%
\pgfpathlineto{\pgfqpoint{1.419572in}{1.932344in}}%
\pgfpathlineto{\pgfqpoint{1.439440in}{1.937743in}}%
\pgfpathlineto{\pgfqpoint{1.459308in}{1.940827in}}%
\pgfpathlineto{\pgfqpoint{1.478753in}{1.941604in}}%
\pgfpathlineto{\pgfqpoint{1.481712in}{1.941528in}}%
\pgfpathlineto{\pgfqpoint{1.482981in}{1.862730in}}%
\pgfpathlineto{\pgfqpoint{1.502426in}{1.860832in}}%
\pgfpathlineto{\pgfqpoint{1.522294in}{1.856669in}}%
\pgfpathlineto{\pgfqpoint{1.543008in}{1.850003in}}%
\pgfpathlineto{\pgfqpoint{1.564144in}{1.840851in}}%
\pgfpathlineto{\pgfqpoint{1.586549in}{1.828693in}}%
\pgfpathlineto{\pgfqpoint{1.610222in}{1.813291in}}%
\pgfpathlineto{\pgfqpoint{1.636008in}{1.793799in}}%
\pgfpathlineto{\pgfqpoint{1.664331in}{1.769533in}}%
\pgfpathlineto{\pgfqpoint{1.696881in}{1.738596in}}%
\pgfpathlineto{\pgfqpoint{1.737040in}{1.697151in}}%
\pgfpathlineto{\pgfqpoint{1.815667in}{1.611822in}}%
\pgfpathlineto{\pgfqpoint{1.864703in}{1.560620in}}%
\pgfpathlineto{\pgfqpoint{1.899790in}{1.527087in}}%
\pgfpathlineto{\pgfqpoint{1.929803in}{1.501383in}}%
\pgfpathlineto{\pgfqpoint{1.956435in}{1.481398in}}%
\pgfpathlineto{\pgfqpoint{1.980953in}{1.465690in}}%
\pgfpathlineto{\pgfqpoint{2.003781in}{1.453621in}}%
\pgfpathlineto{\pgfqpoint{2.025340in}{1.444651in}}%
\pgfpathlineto{\pgfqpoint{2.046053in}{1.438377in}}%
\pgfpathlineto{\pgfqpoint{2.066344in}{1.434546in}}%
\pgfpathlineto{\pgfqpoint{2.086212in}{1.433077in}}%
\pgfpathlineto{\pgfqpoint{2.105658in}{1.433863in}}%
\pgfpathlineto{\pgfqpoint{2.125103in}{1.436867in}}%
\pgfpathlineto{\pgfqpoint{2.144972in}{1.442227in}}%
\pgfpathlineto{\pgfqpoint{2.165263in}{1.450068in}}%
\pgfpathlineto{\pgfqpoint{2.185976in}{1.460494in}}%
\pgfpathlineto{\pgfqpoint{2.207535in}{1.473867in}}%
\pgfpathlineto{\pgfqpoint{2.230362in}{1.490715in}}%
\pgfpathlineto{\pgfqpoint{2.254458in}{1.511327in}}%
\pgfpathlineto{\pgfqpoint{2.280244in}{1.536356in}}%
\pgfpathlineto{\pgfqpoint{2.308567in}{1.567016in}}%
\pgfpathlineto{\pgfqpoint{2.340272in}{1.604702in}}%
\pgfpathlineto{\pgfqpoint{2.377894in}{1.653041in}}%
\pgfpathlineto{\pgfqpoint{2.429044in}{1.722728in}}%
\pgfpathlineto{\pgfqpoint{2.507249in}{1.830447in}}%
\pgfpathlineto{\pgfqpoint{2.507249in}{1.830447in}}%
\pgfusepath{stroke}%
\end{pgfscope}%
\begin{pgfscope}%
\pgfsetbuttcap%
\pgfsetmiterjoin%
\definecolor{currentfill}{rgb}{1.000000,1.000000,1.000000}%
\pgfsetfillcolor{currentfill}%
\pgfsetfillopacity{0.800000}%
\pgfsetlinewidth{1.003750pt}%
\definecolor{currentstroke}{rgb}{0.800000,0.800000,0.800000}%
\pgfsetstrokecolor{currentstroke}%
\pgfsetstrokeopacity{0.800000}%
\pgfsetdash{}{0pt}%
\pgfpathmoveto{\pgfqpoint{0.968756in}{0.617930in}}%
\pgfpathlineto{\pgfqpoint{1.932105in}{0.617930in}}%
\pgfpathquadraticcurveto{\pgfqpoint{1.959883in}{0.617930in}}{\pgfqpoint{1.959883in}{0.645708in}}%
\pgfpathlineto{\pgfqpoint{1.959883in}{1.260888in}}%
\pgfpathquadraticcurveto{\pgfqpoint{1.959883in}{1.288666in}}{\pgfqpoint{1.932105in}{1.288666in}}%
\pgfpathlineto{\pgfqpoint{0.968756in}{1.288666in}}%
\pgfpathquadraticcurveto{\pgfqpoint{0.940979in}{1.288666in}}{\pgfqpoint{0.940979in}{1.260888in}}%
\pgfpathlineto{\pgfqpoint{0.940979in}{0.645708in}}%
\pgfpathquadraticcurveto{\pgfqpoint{0.940979in}{0.617930in}}{\pgfqpoint{0.968756in}{0.617930in}}%
\pgfpathlineto{\pgfqpoint{0.968756in}{0.617930in}}%
\pgfpathclose%
\pgfusepath{stroke,fill}%
\end{pgfscope}%
\begin{pgfscope}%
\pgfsetrectcap%
\pgfsetroundjoin%
\pgfsetlinewidth{1.505625pt}%
\definecolor{currentstroke}{rgb}{0.000000,0.000000,0.000000}%
\pgfsetstrokecolor{currentstroke}%
\pgfsetdash{}{0pt}%
\pgfpathmoveto{\pgfqpoint{0.996534in}{1.176199in}}%
\pgfpathlineto{\pgfqpoint{1.135423in}{1.176199in}}%
\pgfpathlineto{\pgfqpoint{1.274312in}{1.176199in}}%
\pgfusepath{stroke}%
\end{pgfscope}%
\begin{pgfscope}%
\definecolor{textcolor}{rgb}{0.000000,0.000000,0.000000}%
\pgfsetstrokecolor{textcolor}%
\pgfsetfillcolor{textcolor}%
\pgftext[x=1.385423in,y=1.127588in,left,base]{\color{textcolor}{\rmfamily\fontsize{10.000000}{12.000000}\selectfont\catcode`\^=\active\def^{\ifmmode\sp\else\^{}\fi}\catcode`\%=\active\def%{\%}$f(t)$}}%
\end{pgfscope}%
\begin{pgfscope}%
\pgfsetbuttcap%
\pgfsetroundjoin%
\pgfsetlinewidth{1.505625pt}%
\definecolor{currentstroke}{rgb}{0.000000,0.000000,0.000000}%
\pgfsetstrokecolor{currentstroke}%
\pgfsetdash{{5.550000pt}{2.400000pt}}{0.000000pt}%
\pgfpathmoveto{\pgfqpoint{0.996534in}{0.966509in}}%
\pgfpathlineto{\pgfqpoint{1.135423in}{0.966509in}}%
\pgfpathlineto{\pgfqpoint{1.274312in}{0.966509in}}%
\pgfusepath{stroke}%
\end{pgfscope}%
\begin{pgfscope}%
\definecolor{textcolor}{rgb}{0.000000,0.000000,0.000000}%
\pgfsetstrokecolor{textcolor}%
\pgfsetfillcolor{textcolor}%
\pgftext[x=1.385423in,y=0.917898in,left,base]{\color{textcolor}{\rmfamily\fontsize{10.000000}{12.000000}\selectfont\catcode`\^=\active\def^{\ifmmode\sp\else\^{}\fi}\catcode`\%=\active\def%{\%}$f(t) \cdot 1.5$}}%
\end{pgfscope}%
\begin{pgfscope}%
\pgfsetbuttcap%
\pgfsetroundjoin%
\pgfsetlinewidth{1.505625pt}%
\definecolor{currentstroke}{rgb}{0.000000,0.000000,0.000000}%
\pgfsetstrokecolor{currentstroke}%
\pgfsetdash{{1.500000pt}{2.475000pt}}{0.000000pt}%
\pgfpathmoveto{\pgfqpoint{0.996534in}{0.756819in}}%
\pgfpathlineto{\pgfqpoint{1.135423in}{0.756819in}}%
\pgfpathlineto{\pgfqpoint{1.274312in}{0.756819in}}%
\pgfusepath{stroke}%
\end{pgfscope}%
\begin{pgfscope}%
\definecolor{textcolor}{rgb}{0.000000,0.000000,0.000000}%
\pgfsetstrokecolor{textcolor}%
\pgfsetfillcolor{textcolor}%
\pgftext[x=1.385423in,y=0.708208in,left,base]{\color{textcolor}{\rmfamily\fontsize{10.000000}{12.000000}\selectfont\catcode`\^=\active\def^{\ifmmode\sp\else\^{}\fi}\catcode`\%=\active\def%{\%}$f(t) \cdot 0.5$}}%
\end{pgfscope}%
\begin{pgfscope}%
\pgfsetbuttcap%
\pgfsetmiterjoin%
\definecolor{currentfill}{rgb}{1.000000,1.000000,1.000000}%
\pgfsetfillcolor{currentfill}%
\pgfsetlinewidth{0.000000pt}%
\definecolor{currentstroke}{rgb}{0.000000,0.000000,0.000000}%
\pgfsetstrokecolor{currentstroke}%
\pgfsetstrokeopacity{0.000000}%
\pgfsetdash{}{0pt}%
\pgfpathmoveto{\pgfqpoint{2.929976in}{0.548486in}}%
\pgfpathlineto{\pgfqpoint{5.043613in}{0.548486in}}%
\pgfpathlineto{\pgfqpoint{5.043613in}{2.648486in}}%
\pgfpathlineto{\pgfqpoint{2.929976in}{2.648486in}}%
\pgfpathlineto{\pgfqpoint{2.929976in}{0.548486in}}%
\pgfpathclose%
\pgfusepath{fill}%
\end{pgfscope}%
\begin{pgfscope}%
\pgfsetbuttcap%
\pgfsetroundjoin%
\definecolor{currentfill}{rgb}{0.000000,0.000000,0.000000}%
\pgfsetfillcolor{currentfill}%
\pgfsetlinewidth{0.803000pt}%
\definecolor{currentstroke}{rgb}{0.000000,0.000000,0.000000}%
\pgfsetstrokecolor{currentstroke}%
\pgfsetdash{}{0pt}%
\pgfsys@defobject{currentmarker}{\pgfqpoint{0.000000in}{-0.048611in}}{\pgfqpoint{0.000000in}{0.000000in}}{%
\pgfpathmoveto{\pgfqpoint{0.000000in}{0.000000in}}%
\pgfpathlineto{\pgfqpoint{0.000000in}{-0.048611in}}%
\pgfusepath{stroke,fill}%
}%
\begin{pgfscope}%
\pgfsys@transformshift{2.929976in}{0.548486in}%
\pgfsys@useobject{currentmarker}{}%
\end{pgfscope}%
\end{pgfscope}%
\begin{pgfscope}%
\definecolor{textcolor}{rgb}{0.000000,0.000000,0.000000}%
\pgfsetstrokecolor{textcolor}%
\pgfsetfillcolor{textcolor}%
\pgftext[x=2.929976in,y=0.451264in,,top]{\color{textcolor}{\rmfamily\fontsize{10.000000}{12.000000}\selectfont\catcode`\^=\active\def^{\ifmmode\sp\else\^{}\fi}\catcode`\%=\active\def%{\%}\ensuremath{-}1.0}}%
\end{pgfscope}%
\begin{pgfscope}%
\pgfsetbuttcap%
\pgfsetroundjoin%
\definecolor{currentfill}{rgb}{0.000000,0.000000,0.000000}%
\pgfsetfillcolor{currentfill}%
\pgfsetlinewidth{0.803000pt}%
\definecolor{currentstroke}{rgb}{0.000000,0.000000,0.000000}%
\pgfsetstrokecolor{currentstroke}%
\pgfsetdash{}{0pt}%
\pgfsys@defobject{currentmarker}{\pgfqpoint{0.000000in}{-0.048611in}}{\pgfqpoint{0.000000in}{0.000000in}}{%
\pgfpathmoveto{\pgfqpoint{0.000000in}{0.000000in}}%
\pgfpathlineto{\pgfqpoint{0.000000in}{-0.048611in}}%
\pgfusepath{stroke,fill}%
}%
\begin{pgfscope}%
\pgfsys@transformshift{3.458385in}{0.548486in}%
\pgfsys@useobject{currentmarker}{}%
\end{pgfscope}%
\end{pgfscope}%
\begin{pgfscope}%
\definecolor{textcolor}{rgb}{0.000000,0.000000,0.000000}%
\pgfsetstrokecolor{textcolor}%
\pgfsetfillcolor{textcolor}%
\pgftext[x=3.458385in,y=0.451264in,,top]{\color{textcolor}{\rmfamily\fontsize{10.000000}{12.000000}\selectfont\catcode`\^=\active\def^{\ifmmode\sp\else\^{}\fi}\catcode`\%=\active\def%{\%}\ensuremath{-}0.5}}%
\end{pgfscope}%
\begin{pgfscope}%
\pgfsetbuttcap%
\pgfsetroundjoin%
\definecolor{currentfill}{rgb}{0.000000,0.000000,0.000000}%
\pgfsetfillcolor{currentfill}%
\pgfsetlinewidth{0.803000pt}%
\definecolor{currentstroke}{rgb}{0.000000,0.000000,0.000000}%
\pgfsetstrokecolor{currentstroke}%
\pgfsetdash{}{0pt}%
\pgfsys@defobject{currentmarker}{\pgfqpoint{0.000000in}{-0.048611in}}{\pgfqpoint{0.000000in}{0.000000in}}{%
\pgfpathmoveto{\pgfqpoint{0.000000in}{0.000000in}}%
\pgfpathlineto{\pgfqpoint{0.000000in}{-0.048611in}}%
\pgfusepath{stroke,fill}%
}%
\begin{pgfscope}%
\pgfsys@transformshift{3.986794in}{0.548486in}%
\pgfsys@useobject{currentmarker}{}%
\end{pgfscope}%
\end{pgfscope}%
\begin{pgfscope}%
\definecolor{textcolor}{rgb}{0.000000,0.000000,0.000000}%
\pgfsetstrokecolor{textcolor}%
\pgfsetfillcolor{textcolor}%
\pgftext[x=3.986794in,y=0.451264in,,top]{\color{textcolor}{\rmfamily\fontsize{10.000000}{12.000000}\selectfont\catcode`\^=\active\def^{\ifmmode\sp\else\^{}\fi}\catcode`\%=\active\def%{\%}0.0}}%
\end{pgfscope}%
\begin{pgfscope}%
\pgfsetbuttcap%
\pgfsetroundjoin%
\definecolor{currentfill}{rgb}{0.000000,0.000000,0.000000}%
\pgfsetfillcolor{currentfill}%
\pgfsetlinewidth{0.803000pt}%
\definecolor{currentstroke}{rgb}{0.000000,0.000000,0.000000}%
\pgfsetstrokecolor{currentstroke}%
\pgfsetdash{}{0pt}%
\pgfsys@defobject{currentmarker}{\pgfqpoint{0.000000in}{-0.048611in}}{\pgfqpoint{0.000000in}{0.000000in}}{%
\pgfpathmoveto{\pgfqpoint{0.000000in}{0.000000in}}%
\pgfpathlineto{\pgfqpoint{0.000000in}{-0.048611in}}%
\pgfusepath{stroke,fill}%
}%
\begin{pgfscope}%
\pgfsys@transformshift{4.515203in}{0.548486in}%
\pgfsys@useobject{currentmarker}{}%
\end{pgfscope}%
\end{pgfscope}%
\begin{pgfscope}%
\definecolor{textcolor}{rgb}{0.000000,0.000000,0.000000}%
\pgfsetstrokecolor{textcolor}%
\pgfsetfillcolor{textcolor}%
\pgftext[x=4.515203in,y=0.451264in,,top]{\color{textcolor}{\rmfamily\fontsize{10.000000}{12.000000}\selectfont\catcode`\^=\active\def^{\ifmmode\sp\else\^{}\fi}\catcode`\%=\active\def%{\%}0.5}}%
\end{pgfscope}%
\begin{pgfscope}%
\pgfsetbuttcap%
\pgfsetroundjoin%
\definecolor{currentfill}{rgb}{0.000000,0.000000,0.000000}%
\pgfsetfillcolor{currentfill}%
\pgfsetlinewidth{0.803000pt}%
\definecolor{currentstroke}{rgb}{0.000000,0.000000,0.000000}%
\pgfsetstrokecolor{currentstroke}%
\pgfsetdash{}{0pt}%
\pgfsys@defobject{currentmarker}{\pgfqpoint{0.000000in}{-0.048611in}}{\pgfqpoint{0.000000in}{0.000000in}}{%
\pgfpathmoveto{\pgfqpoint{0.000000in}{0.000000in}}%
\pgfpathlineto{\pgfqpoint{0.000000in}{-0.048611in}}%
\pgfusepath{stroke,fill}%
}%
\begin{pgfscope}%
\pgfsys@transformshift{5.043613in}{0.548486in}%
\pgfsys@useobject{currentmarker}{}%
\end{pgfscope}%
\end{pgfscope}%
\begin{pgfscope}%
\definecolor{textcolor}{rgb}{0.000000,0.000000,0.000000}%
\pgfsetstrokecolor{textcolor}%
\pgfsetfillcolor{textcolor}%
\pgftext[x=5.043613in,y=0.451264in,,top]{\color{textcolor}{\rmfamily\fontsize{10.000000}{12.000000}\selectfont\catcode`\^=\active\def^{\ifmmode\sp\else\^{}\fi}\catcode`\%=\active\def%{\%}1.0}}%
\end{pgfscope}%
\begin{pgfscope}%
\definecolor{textcolor}{rgb}{0.000000,0.000000,0.000000}%
\pgfsetstrokecolor{textcolor}%
\pgfsetfillcolor{textcolor}%
\pgftext[x=3.986794in,y=0.261295in,,top]{\color{textcolor}{\rmfamily\fontsize{12.000000}{14.400000}\selectfont\catcode`\^=\active\def^{\ifmmode\sp\else\^{}\fi}\catcode`\%=\active\def%{\%}$t$}}%
\end{pgfscope}%
\begin{pgfscope}%
\pgfsetbuttcap%
\pgfsetroundjoin%
\definecolor{currentfill}{rgb}{0.000000,0.000000,0.000000}%
\pgfsetfillcolor{currentfill}%
\pgfsetlinewidth{0.803000pt}%
\definecolor{currentstroke}{rgb}{0.000000,0.000000,0.000000}%
\pgfsetstrokecolor{currentstroke}%
\pgfsetdash{}{0pt}%
\pgfsys@defobject{currentmarker}{\pgfqpoint{-0.048611in}{0.000000in}}{\pgfqpoint{-0.000000in}{0.000000in}}{%
\pgfpathmoveto{\pgfqpoint{-0.000000in}{0.000000in}}%
\pgfpathlineto{\pgfqpoint{-0.048611in}{0.000000in}}%
\pgfusepath{stroke,fill}%
}%
\begin{pgfscope}%
\pgfsys@transformshift{2.929976in}{0.548486in}%
\pgfsys@useobject{currentmarker}{}%
\end{pgfscope}%
\end{pgfscope}%
\begin{pgfscope}%
\definecolor{textcolor}{rgb}{0.000000,0.000000,0.000000}%
\pgfsetstrokecolor{textcolor}%
\pgfsetfillcolor{textcolor}%
\pgftext[x=2.636364in, y=0.495724in, left, base]{\color{textcolor}{\rmfamily\fontsize{10.000000}{12.000000}\selectfont\catcode`\^=\active\def^{\ifmmode\sp\else\^{}\fi}\catcode`\%=\active\def%{\%}\ensuremath{-}2}}%
\end{pgfscope}%
\begin{pgfscope}%
\pgfsetbuttcap%
\pgfsetroundjoin%
\definecolor{currentfill}{rgb}{0.000000,0.000000,0.000000}%
\pgfsetfillcolor{currentfill}%
\pgfsetlinewidth{0.803000pt}%
\definecolor{currentstroke}{rgb}{0.000000,0.000000,0.000000}%
\pgfsetstrokecolor{currentstroke}%
\pgfsetdash{}{0pt}%
\pgfsys@defobject{currentmarker}{\pgfqpoint{-0.048611in}{0.000000in}}{\pgfqpoint{-0.000000in}{0.000000in}}{%
\pgfpathmoveto{\pgfqpoint{-0.000000in}{0.000000in}}%
\pgfpathlineto{\pgfqpoint{-0.048611in}{0.000000in}}%
\pgfusepath{stroke,fill}%
}%
\begin{pgfscope}%
\pgfsys@transformshift{2.929976in}{1.073486in}%
\pgfsys@useobject{currentmarker}{}%
\end{pgfscope}%
\end{pgfscope}%
\begin{pgfscope}%
\definecolor{textcolor}{rgb}{0.000000,0.000000,0.000000}%
\pgfsetstrokecolor{textcolor}%
\pgfsetfillcolor{textcolor}%
\pgftext[x=2.636364in, y=1.020724in, left, base]{\color{textcolor}{\rmfamily\fontsize{10.000000}{12.000000}\selectfont\catcode`\^=\active\def^{\ifmmode\sp\else\^{}\fi}\catcode`\%=\active\def%{\%}\ensuremath{-}1}}%
\end{pgfscope}%
\begin{pgfscope}%
\pgfsetbuttcap%
\pgfsetroundjoin%
\definecolor{currentfill}{rgb}{0.000000,0.000000,0.000000}%
\pgfsetfillcolor{currentfill}%
\pgfsetlinewidth{0.803000pt}%
\definecolor{currentstroke}{rgb}{0.000000,0.000000,0.000000}%
\pgfsetstrokecolor{currentstroke}%
\pgfsetdash{}{0pt}%
\pgfsys@defobject{currentmarker}{\pgfqpoint{-0.048611in}{0.000000in}}{\pgfqpoint{-0.000000in}{0.000000in}}{%
\pgfpathmoveto{\pgfqpoint{-0.000000in}{0.000000in}}%
\pgfpathlineto{\pgfqpoint{-0.048611in}{0.000000in}}%
\pgfusepath{stroke,fill}%
}%
\begin{pgfscope}%
\pgfsys@transformshift{2.929976in}{1.598486in}%
\pgfsys@useobject{currentmarker}{}%
\end{pgfscope}%
\end{pgfscope}%
\begin{pgfscope}%
\definecolor{textcolor}{rgb}{0.000000,0.000000,0.000000}%
\pgfsetstrokecolor{textcolor}%
\pgfsetfillcolor{textcolor}%
\pgftext[x=2.744389in, y=1.545724in, left, base]{\color{textcolor}{\rmfamily\fontsize{10.000000}{12.000000}\selectfont\catcode`\^=\active\def^{\ifmmode\sp\else\^{}\fi}\catcode`\%=\active\def%{\%}0}}%
\end{pgfscope}%
\begin{pgfscope}%
\pgfsetbuttcap%
\pgfsetroundjoin%
\definecolor{currentfill}{rgb}{0.000000,0.000000,0.000000}%
\pgfsetfillcolor{currentfill}%
\pgfsetlinewidth{0.803000pt}%
\definecolor{currentstroke}{rgb}{0.000000,0.000000,0.000000}%
\pgfsetstrokecolor{currentstroke}%
\pgfsetdash{}{0pt}%
\pgfsys@defobject{currentmarker}{\pgfqpoint{-0.048611in}{0.000000in}}{\pgfqpoint{-0.000000in}{0.000000in}}{%
\pgfpathmoveto{\pgfqpoint{-0.000000in}{0.000000in}}%
\pgfpathlineto{\pgfqpoint{-0.048611in}{0.000000in}}%
\pgfusepath{stroke,fill}%
}%
\begin{pgfscope}%
\pgfsys@transformshift{2.929976in}{2.123486in}%
\pgfsys@useobject{currentmarker}{}%
\end{pgfscope}%
\end{pgfscope}%
\begin{pgfscope}%
\definecolor{textcolor}{rgb}{0.000000,0.000000,0.000000}%
\pgfsetstrokecolor{textcolor}%
\pgfsetfillcolor{textcolor}%
\pgftext[x=2.744389in, y=2.070724in, left, base]{\color{textcolor}{\rmfamily\fontsize{10.000000}{12.000000}\selectfont\catcode`\^=\active\def^{\ifmmode\sp\else\^{}\fi}\catcode`\%=\active\def%{\%}1}}%
\end{pgfscope}%
\begin{pgfscope}%
\pgfsetbuttcap%
\pgfsetroundjoin%
\definecolor{currentfill}{rgb}{0.000000,0.000000,0.000000}%
\pgfsetfillcolor{currentfill}%
\pgfsetlinewidth{0.803000pt}%
\definecolor{currentstroke}{rgb}{0.000000,0.000000,0.000000}%
\pgfsetstrokecolor{currentstroke}%
\pgfsetdash{}{0pt}%
\pgfsys@defobject{currentmarker}{\pgfqpoint{-0.048611in}{0.000000in}}{\pgfqpoint{-0.000000in}{0.000000in}}{%
\pgfpathmoveto{\pgfqpoint{-0.000000in}{0.000000in}}%
\pgfpathlineto{\pgfqpoint{-0.048611in}{0.000000in}}%
\pgfusepath{stroke,fill}%
}%
\begin{pgfscope}%
\pgfsys@transformshift{2.929976in}{2.648486in}%
\pgfsys@useobject{currentmarker}{}%
\end{pgfscope}%
\end{pgfscope}%
\begin{pgfscope}%
\definecolor{textcolor}{rgb}{0.000000,0.000000,0.000000}%
\pgfsetstrokecolor{textcolor}%
\pgfsetfillcolor{textcolor}%
\pgftext[x=2.744389in, y=2.595724in, left, base]{\color{textcolor}{\rmfamily\fontsize{10.000000}{12.000000}\selectfont\catcode`\^=\active\def^{\ifmmode\sp\else\^{}\fi}\catcode`\%=\active\def%{\%}2}}%
\end{pgfscope}%
\begin{pgfscope}%
\pgfpathrectangle{\pgfqpoint{2.929976in}{0.548486in}}{\pgfqpoint{2.113636in}{2.100000in}}%
\pgfusepath{clip}%
\pgfsetbuttcap%
\pgfsetroundjoin%
\pgfsetlinewidth{0.501875pt}%
\definecolor{currentstroke}{rgb}{0.000000,0.000000,0.000000}%
\pgfsetstrokecolor{currentstroke}%
\pgfsetdash{{1.850000pt}{0.800000pt}}{0.000000pt}%
\pgfpathmoveto{\pgfqpoint{2.929976in}{1.598486in}}%
\pgfpathlineto{\pgfqpoint{5.043613in}{1.598486in}}%
\pgfusepath{stroke}%
\end{pgfscope}%
\begin{pgfscope}%
\pgfpathrectangle{\pgfqpoint{2.929976in}{0.548486in}}{\pgfqpoint{2.113636in}{2.100000in}}%
\pgfusepath{clip}%
\pgfsetbuttcap%
\pgfsetroundjoin%
\pgfsetlinewidth{0.501875pt}%
\definecolor{currentstroke}{rgb}{0.000000,0.000000,0.000000}%
\pgfsetstrokecolor{currentstroke}%
\pgfsetdash{{1.850000pt}{0.800000pt}}{0.000000pt}%
\pgfpathmoveto{\pgfqpoint{3.986794in}{0.548486in}}%
\pgfpathlineto{\pgfqpoint{3.986794in}{2.648486in}}%
\pgfusepath{stroke}%
\end{pgfscope}%
\begin{pgfscope}%
\pgfpathrectangle{\pgfqpoint{2.929976in}{0.548486in}}{\pgfqpoint{2.113636in}{2.100000in}}%
\pgfusepath{clip}%
\pgfsetrectcap%
\pgfsetroundjoin%
\pgfsetlinewidth{1.505625pt}%
\definecolor{currentstroke}{rgb}{0.000000,0.000000,0.000000}%
\pgfsetstrokecolor{currentstroke}%
\pgfsetdash{}{0pt}%
\pgfpathmoveto{\pgfqpoint{2.929976in}{1.432409in}}%
\pgfpathlineto{\pgfqpoint{2.975208in}{1.335579in}}%
\pgfpathlineto{\pgfqpoint{3.060599in}{1.152104in}}%
\pgfpathlineto{\pgfqpoint{3.094417in}{1.085241in}}%
\pgfpathlineto{\pgfqpoint{3.122740in}{1.034026in}}%
\pgfpathlineto{\pgfqpoint{3.147258in}{0.994098in}}%
\pgfpathlineto{\pgfqpoint{3.169662in}{0.961760in}}%
\pgfpathlineto{\pgfqpoint{3.189953in}{0.936269in}}%
\pgfpathlineto{\pgfqpoint{3.208553in}{0.916333in}}%
\pgfpathlineto{\pgfqpoint{3.225885in}{0.900888in}}%
\pgfpathlineto{\pgfqpoint{3.242372in}{0.889130in}}%
\pgfpathlineto{\pgfqpoint{3.257590in}{0.880906in}}%
\pgfpathlineto{\pgfqpoint{3.272385in}{0.875395in}}%
\pgfpathlineto{\pgfqpoint{3.286758in}{0.872434in}}%
\pgfpathlineto{\pgfqpoint{3.300708in}{0.871847in}}%
\pgfpathlineto{\pgfqpoint{3.314658in}{0.873533in}}%
\pgfpathlineto{\pgfqpoint{3.328608in}{0.877504in}}%
\pgfpathlineto{\pgfqpoint{3.342981in}{0.883985in}}%
\pgfpathlineto{\pgfqpoint{3.357776in}{0.893180in}}%
\pgfpathlineto{\pgfqpoint{3.372994in}{0.905283in}}%
\pgfpathlineto{\pgfqpoint{3.389058in}{0.920925in}}%
\pgfpathlineto{\pgfqpoint{3.406390in}{0.941035in}}%
\pgfpathlineto{\pgfqpoint{3.424567in}{0.965620in}}%
\pgfpathlineto{\pgfqpoint{3.444012in}{0.995725in}}%
\pgfpathlineto{\pgfqpoint{3.465149in}{1.032670in}}%
\pgfpathlineto{\pgfqpoint{3.487976in}{1.077165in}}%
\pgfpathlineto{\pgfqpoint{3.513340in}{1.131676in}}%
\pgfpathlineto{\pgfqpoint{3.541663in}{1.198049in}}%
\pgfpathlineto{\pgfqpoint{3.574635in}{1.281291in}}%
\pgfpathlineto{\pgfqpoint{3.616062in}{1.392339in}}%
\pgfpathlineto{\pgfqpoint{3.751335in}{1.759868in}}%
\pgfpathlineto{\pgfqpoint{3.783885in}{1.839104in}}%
\pgfpathlineto{\pgfqpoint{3.811785in}{1.901406in}}%
\pgfpathlineto{\pgfqpoint{3.836726in}{1.951896in}}%
\pgfpathlineto{\pgfqpoint{3.859553in}{1.993267in}}%
\pgfpathlineto{\pgfqpoint{3.880690in}{2.027086in}}%
\pgfpathlineto{\pgfqpoint{3.900135in}{2.054144in}}%
\pgfpathlineto{\pgfqpoint{3.918312in}{2.075758in}}%
\pgfpathlineto{\pgfqpoint{3.935222in}{2.092563in}}%
\pgfpathlineto{\pgfqpoint{3.951285in}{2.105511in}}%
\pgfpathlineto{\pgfqpoint{3.954667in}{2.107857in}}%
\pgfpathlineto{\pgfqpoint{3.955935in}{2.266202in}}%
\pgfpathlineto{\pgfqpoint{3.971153in}{2.274887in}}%
\pgfpathlineto{\pgfqpoint{3.985526in}{2.280599in}}%
\pgfpathlineto{\pgfqpoint{3.999476in}{2.283821in}}%
\pgfpathlineto{\pgfqpoint{4.013426in}{2.284762in}}%
\pgfpathlineto{\pgfqpoint{4.018076in}{2.284571in}}%
\pgfpathlineto{\pgfqpoint{4.019344in}{2.126975in}}%
\pgfpathlineto{\pgfqpoint{4.033294in}{2.124692in}}%
\pgfpathlineto{\pgfqpoint{4.047667in}{2.120007in}}%
\pgfpathlineto{\pgfqpoint{4.062462in}{2.112755in}}%
\pgfpathlineto{\pgfqpoint{4.077681in}{2.102783in}}%
\pgfpathlineto{\pgfqpoint{4.093744in}{2.089576in}}%
\pgfpathlineto{\pgfqpoint{4.111076in}{2.072357in}}%
\pgfpathlineto{\pgfqpoint{4.129676in}{2.050617in}}%
\pgfpathlineto{\pgfqpoint{4.149544in}{2.023899in}}%
\pgfpathlineto{\pgfqpoint{4.171526in}{1.990474in}}%
\pgfpathlineto{\pgfqpoint{4.195622in}{1.949669in}}%
\pgfpathlineto{\pgfqpoint{4.223099in}{1.898598in}}%
\pgfpathlineto{\pgfqpoint{4.255649in}{1.833166in}}%
\pgfpathlineto{\pgfqpoint{4.299613in}{1.739329in}}%
\pgfpathlineto{\pgfqpoint{4.390922in}{1.543266in}}%
\pgfpathlineto{\pgfqpoint{4.424317in}{1.477565in}}%
\pgfpathlineto{\pgfqpoint{4.452217in}{1.427401in}}%
\pgfpathlineto{\pgfqpoint{4.476735in}{1.387735in}}%
\pgfpathlineto{\pgfqpoint{4.498717in}{1.356233in}}%
\pgfpathlineto{\pgfqpoint{4.519008in}{1.330932in}}%
\pgfpathlineto{\pgfqpoint{4.537608in}{1.311182in}}%
\pgfpathlineto{\pgfqpoint{4.554940in}{1.295919in}}%
\pgfpathlineto{\pgfqpoint{4.571003in}{1.284602in}}%
\pgfpathlineto{\pgfqpoint{4.586222in}{1.276478in}}%
\pgfpathlineto{\pgfqpoint{4.601017in}{1.271066in}}%
\pgfpathlineto{\pgfqpoint{4.615390in}{1.268204in}}%
\pgfpathlineto{\pgfqpoint{4.629340in}{1.267713in}}%
\pgfpathlineto{\pgfqpoint{4.643290in}{1.269496in}}%
\pgfpathlineto{\pgfqpoint{4.657240in}{1.273565in}}%
\pgfpathlineto{\pgfqpoint{4.671613in}{1.280146in}}%
\pgfpathlineto{\pgfqpoint{4.686408in}{1.289443in}}%
\pgfpathlineto{\pgfqpoint{4.701626in}{1.301650in}}%
\pgfpathlineto{\pgfqpoint{4.717690in}{1.317401in}}%
\pgfpathlineto{\pgfqpoint{4.735022in}{1.337624in}}%
\pgfpathlineto{\pgfqpoint{4.753199in}{1.362325in}}%
\pgfpathlineto{\pgfqpoint{4.772644in}{1.392546in}}%
\pgfpathlineto{\pgfqpoint{4.793781in}{1.429611in}}%
\pgfpathlineto{\pgfqpoint{4.816608in}{1.474225in}}%
\pgfpathlineto{\pgfqpoint{4.841972in}{1.528854in}}%
\pgfpathlineto{\pgfqpoint{4.870294in}{1.595340in}}%
\pgfpathlineto{\pgfqpoint{4.903267in}{1.678683in}}%
\pgfpathlineto{\pgfqpoint{4.945117in}{1.790970in}}%
\pgfpathlineto{\pgfqpoint{5.043613in}{2.062409in}}%
\pgfpathlineto{\pgfqpoint{5.043613in}{2.062409in}}%
\pgfusepath{stroke}%
\end{pgfscope}%
\begin{pgfscope}%
\pgfpathrectangle{\pgfqpoint{2.929976in}{0.548486in}}{\pgfqpoint{2.113636in}{2.100000in}}%
\pgfusepath{clip}%
\pgfsetbuttcap%
\pgfsetroundjoin%
\pgfsetlinewidth{1.505625pt}%
\definecolor{currentstroke}{rgb}{0.000000,0.000000,0.000000}%
\pgfsetstrokecolor{currentstroke}%
\pgfsetdash{{5.550000pt}{2.400000pt}}{0.000000pt}%
\pgfpathmoveto{\pgfqpoint{2.929976in}{1.307970in}}%
\pgfpathlineto{\pgfqpoint{2.958722in}{1.416812in}}%
\pgfpathlineto{\pgfqpoint{2.981972in}{1.496585in}}%
\pgfpathlineto{\pgfqpoint{3.002262in}{1.558554in}}%
\pgfpathlineto{\pgfqpoint{3.020017in}{1.606014in}}%
\pgfpathlineto{\pgfqpoint{3.036081in}{1.642960in}}%
\pgfpathlineto{\pgfqpoint{3.050876in}{1.671605in}}%
\pgfpathlineto{\pgfqpoint{3.064403in}{1.693061in}}%
\pgfpathlineto{\pgfqpoint{3.076662in}{1.708476in}}%
\pgfpathlineto{\pgfqpoint{3.088076in}{1.719314in}}%
\pgfpathlineto{\pgfqpoint{3.098644in}{1.726294in}}%
\pgfpathlineto{\pgfqpoint{3.108367in}{1.730109in}}%
\pgfpathlineto{\pgfqpoint{3.117667in}{1.731423in}}%
\pgfpathlineto{\pgfqpoint{3.126967in}{1.730466in}}%
\pgfpathlineto{\pgfqpoint{3.136267in}{1.727261in}}%
\pgfpathlineto{\pgfqpoint{3.145990in}{1.721540in}}%
\pgfpathlineto{\pgfqpoint{3.156558in}{1.712630in}}%
\pgfpathlineto{\pgfqpoint{3.167972in}{1.699946in}}%
\pgfpathlineto{\pgfqpoint{3.180231in}{1.682910in}}%
\pgfpathlineto{\pgfqpoint{3.193758in}{1.660213in}}%
\pgfpathlineto{\pgfqpoint{3.208976in}{1.630111in}}%
\pgfpathlineto{\pgfqpoint{3.225885in}{1.591500in}}%
\pgfpathlineto{\pgfqpoint{3.245331in}{1.541217in}}%
\pgfpathlineto{\pgfqpoint{3.268158in}{1.475624in}}%
\pgfpathlineto{\pgfqpoint{3.297326in}{1.384533in}}%
\pgfpathlineto{\pgfqpoint{3.387790in}{1.096947in}}%
\pgfpathlineto{\pgfqpoint{3.410617in}{1.034389in}}%
\pgfpathlineto{\pgfqpoint{3.430062in}{0.987361in}}%
\pgfpathlineto{\pgfqpoint{3.446972in}{0.952031in}}%
\pgfpathlineto{\pgfqpoint{3.462190in}{0.925199in}}%
\pgfpathlineto{\pgfqpoint{3.475717in}{0.905630in}}%
\pgfpathlineto{\pgfqpoint{3.487976in}{0.891584in}}%
\pgfpathlineto{\pgfqpoint{3.499390in}{0.881794in}}%
\pgfpathlineto{\pgfqpoint{3.509958in}{0.875643in}}%
\pgfpathlineto{\pgfqpoint{3.519681in}{0.872515in}}%
\pgfpathlineto{\pgfqpoint{3.528981in}{0.871824in}}%
\pgfpathlineto{\pgfqpoint{3.538281in}{0.873407in}}%
\pgfpathlineto{\pgfqpoint{3.547581in}{0.877274in}}%
\pgfpathlineto{\pgfqpoint{3.557303in}{0.883760in}}%
\pgfpathlineto{\pgfqpoint{3.567872in}{0.893630in}}%
\pgfpathlineto{\pgfqpoint{3.579285in}{0.907558in}}%
\pgfpathlineto{\pgfqpoint{3.591544in}{0.926231in}}%
\pgfpathlineto{\pgfqpoint{3.604649in}{0.950334in}}%
\pgfpathlineto{\pgfqpoint{3.619022in}{0.981506in}}%
\pgfpathlineto{\pgfqpoint{3.635085in}{1.021898in}}%
\pgfpathlineto{\pgfqpoint{3.652840in}{1.072855in}}%
\pgfpathlineto{\pgfqpoint{3.672708in}{1.136918in}}%
\pgfpathlineto{\pgfqpoint{3.695535in}{1.218317in}}%
\pgfpathlineto{\pgfqpoint{3.723435in}{1.326492in}}%
\pgfpathlineto{\pgfqpoint{3.765708in}{1.500582in}}%
\pgfpathlineto{\pgfqpoint{3.818972in}{1.718087in}}%
\pgfpathlineto{\pgfqpoint{3.846872in}{1.822671in}}%
\pgfpathlineto{\pgfqpoint{3.869699in}{1.900055in}}%
\pgfpathlineto{\pgfqpoint{3.889567in}{1.959920in}}%
\pgfpathlineto{\pgfqpoint{3.907322in}{2.006644in}}%
\pgfpathlineto{\pgfqpoint{3.923385in}{2.042871in}}%
\pgfpathlineto{\pgfqpoint{3.937758in}{2.070092in}}%
\pgfpathlineto{\pgfqpoint{3.950862in}{2.090444in}}%
\pgfpathlineto{\pgfqpoint{3.963122in}{2.105511in}}%
\pgfpathlineto{\pgfqpoint{3.965235in}{2.107714in}}%
\pgfpathlineto{\pgfqpoint{3.966503in}{2.266480in}}%
\pgfpathlineto{\pgfqpoint{3.977494in}{2.275685in}}%
\pgfpathlineto{\pgfqpoint{3.987640in}{2.281355in}}%
\pgfpathlineto{\pgfqpoint{3.997362in}{2.284235in}}%
\pgfpathlineto{\pgfqpoint{4.006663in}{2.284659in}}%
\pgfpathlineto{\pgfqpoint{4.007508in}{2.284585in}}%
\pgfpathlineto{\pgfqpoint{4.008776in}{2.126939in}}%
\pgfpathlineto{\pgfqpoint{4.018076in}{2.124588in}}%
\pgfpathlineto{\pgfqpoint{4.027799in}{2.119747in}}%
\pgfpathlineto{\pgfqpoint{4.037944in}{2.112144in}}%
\pgfpathlineto{\pgfqpoint{4.048935in}{2.101040in}}%
\pgfpathlineto{\pgfqpoint{4.060772in}{2.085856in}}%
\pgfpathlineto{\pgfqpoint{4.073876in}{2.065315in}}%
\pgfpathlineto{\pgfqpoint{4.088249in}{2.038544in}}%
\pgfpathlineto{\pgfqpoint{4.104312in}{2.003785in}}%
\pgfpathlineto{\pgfqpoint{4.122490in}{1.958961in}}%
\pgfpathlineto{\pgfqpoint{4.143626in}{1.900640in}}%
\pgfpathlineto{\pgfqpoint{4.169412in}{1.822593in}}%
\pgfpathlineto{\pgfqpoint{4.208726in}{1.695504in}}%
\pgfpathlineto{\pgfqpoint{4.257763in}{1.538527in}}%
\pgfpathlineto{\pgfqpoint{4.283549in}{1.463405in}}%
\pgfpathlineto{\pgfqpoint{4.304685in}{1.408378in}}%
\pgfpathlineto{\pgfqpoint{4.322863in}{1.366972in}}%
\pgfpathlineto{\pgfqpoint{4.338926in}{1.335637in}}%
\pgfpathlineto{\pgfqpoint{4.353299in}{1.312212in}}%
\pgfpathlineto{\pgfqpoint{4.366403in}{1.294925in}}%
\pgfpathlineto{\pgfqpoint{4.378240in}{1.282822in}}%
\pgfpathlineto{\pgfqpoint{4.389231in}{1.274679in}}%
\pgfpathlineto{\pgfqpoint{4.399376in}{1.269880in}}%
\pgfpathlineto{\pgfqpoint{4.408676in}{1.267816in}}%
\pgfpathlineto{\pgfqpoint{4.417976in}{1.268013in}}%
\pgfpathlineto{\pgfqpoint{4.427276in}{1.270488in}}%
\pgfpathlineto{\pgfqpoint{4.436576in}{1.275249in}}%
\pgfpathlineto{\pgfqpoint{4.446722in}{1.283046in}}%
\pgfpathlineto{\pgfqpoint{4.457290in}{1.294039in}}%
\pgfpathlineto{\pgfqpoint{4.468703in}{1.309161in}}%
\pgfpathlineto{\pgfqpoint{4.481385in}{1.329842in}}%
\pgfpathlineto{\pgfqpoint{4.494913in}{1.356268in}}%
\pgfpathlineto{\pgfqpoint{4.509708in}{1.390113in}}%
\pgfpathlineto{\pgfqpoint{4.526194in}{1.433548in}}%
\pgfpathlineto{\pgfqpoint{4.544372in}{1.487852in}}%
\pgfpathlineto{\pgfqpoint{4.565085in}{1.556976in}}%
\pgfpathlineto{\pgfqpoint{4.589181in}{1.645434in}}%
\pgfpathlineto{\pgfqpoint{4.619617in}{1.766137in}}%
\pgfpathlineto{\pgfqpoint{4.726144in}{2.196538in}}%
\pgfpathlineto{\pgfqpoint{4.749817in}{2.278893in}}%
\pgfpathlineto{\pgfqpoint{4.770531in}{2.343179in}}%
\pgfpathlineto{\pgfqpoint{4.788708in}{2.392623in}}%
\pgfpathlineto{\pgfqpoint{4.805194in}{2.431224in}}%
\pgfpathlineto{\pgfqpoint{4.819990in}{2.460442in}}%
\pgfpathlineto{\pgfqpoint{4.833517in}{2.482442in}}%
\pgfpathlineto{\pgfqpoint{4.845776in}{2.498363in}}%
\pgfpathlineto{\pgfqpoint{4.857190in}{2.509680in}}%
\pgfpathlineto{\pgfqpoint{4.867758in}{2.517107in}}%
\pgfpathlineto{\pgfqpoint{4.877481in}{2.521334in}}%
\pgfpathlineto{\pgfqpoint{4.886781in}{2.523040in}}%
\pgfpathlineto{\pgfqpoint{4.896081in}{2.522473in}}%
\pgfpathlineto{\pgfqpoint{4.905381in}{2.519653in}}%
\pgfpathlineto{\pgfqpoint{4.915103in}{2.514329in}}%
\pgfpathlineto{\pgfqpoint{4.925672in}{2.505840in}}%
\pgfpathlineto{\pgfqpoint{4.937085in}{2.493595in}}%
\pgfpathlineto{\pgfqpoint{4.949344in}{2.477013in}}%
\pgfpathlineto{\pgfqpoint{4.962872in}{2.454788in}}%
\pgfpathlineto{\pgfqpoint{4.977667in}{2.426063in}}%
\pgfpathlineto{\pgfqpoint{4.994153in}{2.389083in}}%
\pgfpathlineto{\pgfqpoint{5.013176in}{2.340677in}}%
\pgfpathlineto{\pgfqpoint{5.035158in}{2.278391in}}%
\pgfpathlineto{\pgfqpoint{5.043613in}{2.252970in}}%
\pgfpathlineto{\pgfqpoint{5.043613in}{2.252970in}}%
\pgfusepath{stroke}%
\end{pgfscope}%
\begin{pgfscope}%
\pgfpathrectangle{\pgfqpoint{2.929976in}{0.548486in}}{\pgfqpoint{2.113636in}{2.100000in}}%
\pgfusepath{clip}%
\pgfsetbuttcap%
\pgfsetroundjoin%
\pgfsetlinewidth{1.505625pt}%
\definecolor{currentstroke}{rgb}{0.000000,0.000000,0.000000}%
\pgfsetstrokecolor{currentstroke}%
\pgfsetdash{{1.500000pt}{2.475000pt}}{0.000000pt}%
\pgfpathmoveto{\pgfqpoint{2.929976in}{1.020386in}}%
\pgfpathlineto{\pgfqpoint{2.965908in}{1.053934in}}%
\pgfpathlineto{\pgfqpoint{3.003953in}{1.092539in}}%
\pgfpathlineto{\pgfqpoint{3.044958in}{1.137397in}}%
\pgfpathlineto{\pgfqpoint{3.089344in}{1.189341in}}%
\pgfpathlineto{\pgfqpoint{3.138381in}{1.250232in}}%
\pgfpathlineto{\pgfqpoint{3.195026in}{1.324234in}}%
\pgfpathlineto{\pgfqpoint{3.267312in}{1.422576in}}%
\pgfpathlineto{\pgfqpoint{3.472335in}{1.703707in}}%
\pgfpathlineto{\pgfqpoint{3.528981in}{1.776314in}}%
\pgfpathlineto{\pgfqpoint{3.578017in}{1.835645in}}%
\pgfpathlineto{\pgfqpoint{3.622403in}{1.885913in}}%
\pgfpathlineto{\pgfqpoint{3.662985in}{1.928590in}}%
\pgfpathlineto{\pgfqpoint{3.701031in}{1.965441in}}%
\pgfpathlineto{\pgfqpoint{3.736962in}{1.997205in}}%
\pgfpathlineto{\pgfqpoint{3.771203in}{2.024547in}}%
\pgfpathlineto{\pgfqpoint{3.803753in}{2.047755in}}%
\pgfpathlineto{\pgfqpoint{3.835035in}{2.067400in}}%
\pgfpathlineto{\pgfqpoint{3.865472in}{2.083931in}}%
\pgfpathlineto{\pgfqpoint{3.895062in}{2.097503in}}%
\pgfpathlineto{\pgfqpoint{3.922963in}{2.107999in}}%
\pgfpathlineto{\pgfqpoint{3.924231in}{2.265923in}}%
\pgfpathlineto{\pgfqpoint{3.952553in}{2.274162in}}%
\pgfpathlineto{\pgfqpoint{3.980453in}{2.279982in}}%
\pgfpathlineto{\pgfqpoint{4.008353in}{2.283516in}}%
\pgfpathlineto{\pgfqpoint{4.035831in}{2.284766in}}%
\pgfpathlineto{\pgfqpoint{4.049781in}{2.284556in}}%
\pgfpathlineto{\pgfqpoint{4.051049in}{2.127009in}}%
\pgfpathlineto{\pgfqpoint{4.078949in}{2.124794in}}%
\pgfpathlineto{\pgfqpoint{4.106849in}{2.120348in}}%
\pgfpathlineto{\pgfqpoint{4.135172in}{2.113589in}}%
\pgfpathlineto{\pgfqpoint{4.164340in}{2.104318in}}%
\pgfpathlineto{\pgfqpoint{4.194353in}{2.092396in}}%
\pgfpathlineto{\pgfqpoint{4.225213in}{2.077708in}}%
\pgfpathlineto{\pgfqpoint{4.257340in}{2.059906in}}%
\pgfpathlineto{\pgfqpoint{4.290735in}{2.038827in}}%
\pgfpathlineto{\pgfqpoint{4.325822in}{2.014029in}}%
\pgfpathlineto{\pgfqpoint{4.363444in}{1.984652in}}%
\pgfpathlineto{\pgfqpoint{4.403603in}{1.950419in}}%
\pgfpathlineto{\pgfqpoint{4.447567in}{1.909961in}}%
\pgfpathlineto{\pgfqpoint{4.497026in}{1.861335in}}%
\pgfpathlineto{\pgfqpoint{4.555785in}{1.800311in}}%
\pgfpathlineto{\pgfqpoint{4.639485in}{1.709824in}}%
\pgfpathlineto{\pgfqpoint{4.764190in}{1.575218in}}%
\pgfpathlineto{\pgfqpoint{4.823794in}{1.514340in}}%
\pgfpathlineto{\pgfqpoint{4.873253in}{1.466914in}}%
\pgfpathlineto{\pgfqpoint{4.917217in}{1.427762in}}%
\pgfpathlineto{\pgfqpoint{4.957376in}{1.394908in}}%
\pgfpathlineto{\pgfqpoint{4.994999in}{1.366972in}}%
\pgfpathlineto{\pgfqpoint{5.030085in}{1.343638in}}%
\pgfpathlineto{\pgfqpoint{5.043613in}{1.335386in}}%
\pgfpathlineto{\pgfqpoint{5.043613in}{1.335386in}}%
\pgfusepath{stroke}%
\end{pgfscope}%
\begin{pgfscope}%
\pgfsetbuttcap%
\pgfsetmiterjoin%
\definecolor{currentfill}{rgb}{1.000000,1.000000,1.000000}%
\pgfsetfillcolor{currentfill}%
\pgfsetfillopacity{0.800000}%
\pgfsetlinewidth{1.003750pt}%
\definecolor{currentstroke}{rgb}{0.800000,0.800000,0.800000}%
\pgfsetstrokecolor{currentstroke}%
\pgfsetstrokeopacity{0.800000}%
\pgfsetdash{}{0pt}%
\pgfpathmoveto{\pgfqpoint{3.983042in}{0.617930in}}%
\pgfpathlineto{\pgfqpoint{4.946390in}{0.617930in}}%
\pgfpathquadraticcurveto{\pgfqpoint{4.974168in}{0.617930in}}{\pgfqpoint{4.974168in}{0.645708in}}%
\pgfpathlineto{\pgfqpoint{4.974168in}{1.260888in}}%
\pgfpathquadraticcurveto{\pgfqpoint{4.974168in}{1.288666in}}{\pgfqpoint{4.946390in}{1.288666in}}%
\pgfpathlineto{\pgfqpoint{3.983042in}{1.288666in}}%
\pgfpathquadraticcurveto{\pgfqpoint{3.955264in}{1.288666in}}{\pgfqpoint{3.955264in}{1.260888in}}%
\pgfpathlineto{\pgfqpoint{3.955264in}{0.645708in}}%
\pgfpathquadraticcurveto{\pgfqpoint{3.955264in}{0.617930in}}{\pgfqpoint{3.983042in}{0.617930in}}%
\pgfpathlineto{\pgfqpoint{3.983042in}{0.617930in}}%
\pgfpathclose%
\pgfusepath{stroke,fill}%
\end{pgfscope}%
\begin{pgfscope}%
\pgfsetrectcap%
\pgfsetroundjoin%
\pgfsetlinewidth{1.505625pt}%
\definecolor{currentstroke}{rgb}{0.000000,0.000000,0.000000}%
\pgfsetstrokecolor{currentstroke}%
\pgfsetdash{}{0pt}%
\pgfpathmoveto{\pgfqpoint{4.010819in}{1.176199in}}%
\pgfpathlineto{\pgfqpoint{4.149708in}{1.176199in}}%
\pgfpathlineto{\pgfqpoint{4.288597in}{1.176199in}}%
\pgfusepath{stroke}%
\end{pgfscope}%
\begin{pgfscope}%
\definecolor{textcolor}{rgb}{0.000000,0.000000,0.000000}%
\pgfsetstrokecolor{textcolor}%
\pgfsetfillcolor{textcolor}%
\pgftext[x=4.399708in,y=1.127588in,left,base]{\color{textcolor}{\rmfamily\fontsize{10.000000}{12.000000}\selectfont\catcode`\^=\active\def^{\ifmmode\sp\else\^{}\fi}\catcode`\%=\active\def%{\%}$f(t)$}}%
\end{pgfscope}%
\begin{pgfscope}%
\pgfsetbuttcap%
\pgfsetroundjoin%
\pgfsetlinewidth{1.505625pt}%
\definecolor{currentstroke}{rgb}{0.000000,0.000000,0.000000}%
\pgfsetstrokecolor{currentstroke}%
\pgfsetdash{{5.550000pt}{2.400000pt}}{0.000000pt}%
\pgfpathmoveto{\pgfqpoint{4.010819in}{0.966509in}}%
\pgfpathlineto{\pgfqpoint{4.149708in}{0.966509in}}%
\pgfpathlineto{\pgfqpoint{4.288597in}{0.966509in}}%
\pgfusepath{stroke}%
\end{pgfscope}%
\begin{pgfscope}%
\definecolor{textcolor}{rgb}{0.000000,0.000000,0.000000}%
\pgfsetstrokecolor{textcolor}%
\pgfsetfillcolor{textcolor}%
\pgftext[x=4.399708in,y=0.917898in,left,base]{\color{textcolor}{\rmfamily\fontsize{10.000000}{12.000000}\selectfont\catcode`\^=\active\def^{\ifmmode\sp\else\^{}\fi}\catcode`\%=\active\def%{\%}$f(t\cdot1.5)$}}%
\end{pgfscope}%
\begin{pgfscope}%
\pgfsetbuttcap%
\pgfsetroundjoin%
\pgfsetlinewidth{1.505625pt}%
\definecolor{currentstroke}{rgb}{0.000000,0.000000,0.000000}%
\pgfsetstrokecolor{currentstroke}%
\pgfsetdash{{1.500000pt}{2.475000pt}}{0.000000pt}%
\pgfpathmoveto{\pgfqpoint{4.010819in}{0.756819in}}%
\pgfpathlineto{\pgfqpoint{4.149708in}{0.756819in}}%
\pgfpathlineto{\pgfqpoint{4.288597in}{0.756819in}}%
\pgfusepath{stroke}%
\end{pgfscope}%
\begin{pgfscope}%
\definecolor{textcolor}{rgb}{0.000000,0.000000,0.000000}%
\pgfsetstrokecolor{textcolor}%
\pgfsetfillcolor{textcolor}%
\pgftext[x=4.399708in,y=0.708208in,left,base]{\color{textcolor}{\rmfamily\fontsize{10.000000}{12.000000}\selectfont\catcode`\^=\active\def^{\ifmmode\sp\else\^{}\fi}\catcode`\%=\active\def%{\%}$f(t\cdot0.5)$}}%
\end{pgfscope}%
\end{pgfpicture}%
\makeatother%
\endgroup%

	\caption{Generelle Manipulation von Funktionen.}
	\label{fig:functManipul}
\end{figure}


% \section{Motivation}

% \section{Wichtige Funktionen im Audio Bereich}

\section{Impuls, $\delta(t)$}
Impuls, delta funktion, dirac delta funktion, kronecker delta, (Engl. 'unit impulse'). Alle diese Worte bezeichnen ähnliche Ideen. Je nachdem ob im 'analogen' oder 'digitalen' wird der Impuls unterschiedlich definiert, hat aber im wesentlichen die gleiche Funktion. Die analoge Idee ist ein wenig abstrakt: Ein unendlich kurzes unendlich hohes Signal mit der Fläche 1 ('Dirac Delta Funktion'):

\begin{equation}
 \delta (x)={\begin{cases}0,&x\neq 0\\{\infty },&x=0\end{cases}}
\end{equation}

und 
\begin{equation}
\int _{-\infty }^{\infty }\delta (x)=1.
\end{equation}


Im digitalen ist die Idee recht simpel, ein Signal dass an der stelle $n=0$ den Wert $1$ hat und sonst null ist:
\begin{equation}
	\delta(n) =
	\begin{cases} 
		1, & n = 0 \\ 
		0, &  n \neq 0 
\end{cases}
	%  \case 
\end{equation}

% \begin{figure}[htbp]
% 	\centering
% 	%% Creator: Matplotlib, PGF backend
%%
%% To include the figure in your LaTeX document, write
%%   \input{<filename>.pgf}
%%
%% Make sure the required packages are loaded in your preamble
%%   \usepackage{pgf}
%%
%% Also ensure that all the required font packages are loaded; for instance,
%% the lmodern package is sometimes necessary when using math font.
%%   \usepackage{lmodern}
%%
%% Figures using additional raster images can only be included by \input if
%% they are in the same directory as the main LaTeX file. For loading figures
%% from other directories you can use the `import` package
%%   \usepackage{import}
%%
%% and then include the figures with
%%   \import{<path to file>}{<filename>.pgf}
%%
%% Matplotlib used the following preamble
%%   \def\mathdefault#1{#1}
%%   \everymath=\expandafter{\the\everymath\displaystyle}
%%   
%%   \usepackage{fontspec}
%%   \setmainfont{VeraSe.ttf}[Path=\detokenize{/usr/share/fonts/TTF/}]
%%   \setsansfont{DejaVuSans.ttf}[Path=\detokenize{/home/pl/miniconda3/lib/python3.12/site-packages/matplotlib/mpl-data/fonts/ttf/}]
%%   \setmonofont{DejaVuSansMono.ttf}[Path=\detokenize{/home/pl/miniconda3/lib/python3.12/site-packages/matplotlib/mpl-data/fonts/ttf/}]
%%   \makeatletter\@ifpackageloaded{underscore}{}{\usepackage[strings]{underscore}}\makeatother
%%
\begingroup%
\makeatletter%
\begin{pgfpicture}%
\pgfpathrectangle{\pgfpointorigin}{\pgfqpoint{3.070082in}{2.958486in}}%
\pgfusepath{use as bounding box, clip}%
\begin{pgfscope}%
\pgfsetbuttcap%
\pgfsetmiterjoin%
\definecolor{currentfill}{rgb}{1.000000,1.000000,1.000000}%
\pgfsetfillcolor{currentfill}%
\pgfsetlinewidth{0.000000pt}%
\definecolor{currentstroke}{rgb}{1.000000,1.000000,1.000000}%
\pgfsetstrokecolor{currentstroke}%
\pgfsetdash{}{0pt}%
\pgfpathmoveto{\pgfqpoint{0.000000in}{0.000000in}}%
\pgfpathlineto{\pgfqpoint{3.070082in}{0.000000in}}%
\pgfpathlineto{\pgfqpoint{3.070082in}{2.958486in}}%
\pgfpathlineto{\pgfqpoint{0.000000in}{2.958486in}}%
\pgfpathlineto{\pgfqpoint{0.000000in}{0.000000in}}%
\pgfpathclose%
\pgfusepath{fill}%
\end{pgfscope}%
\begin{pgfscope}%
\pgfsetbuttcap%
\pgfsetmiterjoin%
\definecolor{currentfill}{rgb}{1.000000,1.000000,1.000000}%
\pgfsetfillcolor{currentfill}%
\pgfsetlinewidth{0.000000pt}%
\definecolor{currentstroke}{rgb}{0.000000,0.000000,0.000000}%
\pgfsetstrokecolor{currentstroke}%
\pgfsetstrokeopacity{0.000000}%
\pgfsetdash{}{0pt}%
\pgfpathmoveto{\pgfqpoint{0.640324in}{0.548486in}}%
\pgfpathlineto{\pgfqpoint{2.965324in}{0.548486in}}%
\pgfpathlineto{\pgfqpoint{2.965324in}{2.858486in}}%
\pgfpathlineto{\pgfqpoint{0.640324in}{2.858486in}}%
\pgfpathlineto{\pgfqpoint{0.640324in}{0.548486in}}%
\pgfpathclose%
\pgfusepath{fill}%
\end{pgfscope}%
\begin{pgfscope}%
\pgfsetbuttcap%
\pgfsetroundjoin%
\definecolor{currentfill}{rgb}{0.000000,0.000000,0.000000}%
\pgfsetfillcolor{currentfill}%
\pgfsetlinewidth{0.803000pt}%
\definecolor{currentstroke}{rgb}{0.000000,0.000000,0.000000}%
\pgfsetstrokecolor{currentstroke}%
\pgfsetdash{}{0pt}%
\pgfsys@defobject{currentmarker}{\pgfqpoint{0.000000in}{-0.048611in}}{\pgfqpoint{0.000000in}{0.000000in}}{%
\pgfpathmoveto{\pgfqpoint{0.000000in}{0.000000in}}%
\pgfpathlineto{\pgfqpoint{0.000000in}{-0.048611in}}%
\pgfusepath{stroke,fill}%
}%
\begin{pgfscope}%
\pgfsys@transformshift{0.746006in}{0.548486in}%
\pgfsys@useobject{currentmarker}{}%
\end{pgfscope}%
\end{pgfscope}%
\begin{pgfscope}%
\definecolor{textcolor}{rgb}{0.000000,0.000000,0.000000}%
\pgfsetstrokecolor{textcolor}%
\pgfsetfillcolor{textcolor}%
\pgftext[x=0.746006in,y=0.451264in,,top]{\color{textcolor}{\rmfamily\fontsize{10.000000}{12.000000}\selectfont\catcode`\^=\active\def^{\ifmmode\sp\else\^{}\fi}\catcode`\%=\active\def%{\%}\ensuremath{-}1.0}}%
\end{pgfscope}%
\begin{pgfscope}%
\pgfsetbuttcap%
\pgfsetroundjoin%
\definecolor{currentfill}{rgb}{0.000000,0.000000,0.000000}%
\pgfsetfillcolor{currentfill}%
\pgfsetlinewidth{0.803000pt}%
\definecolor{currentstroke}{rgb}{0.000000,0.000000,0.000000}%
\pgfsetstrokecolor{currentstroke}%
\pgfsetdash{}{0pt}%
\pgfsys@defobject{currentmarker}{\pgfqpoint{0.000000in}{-0.048611in}}{\pgfqpoint{0.000000in}{0.000000in}}{%
\pgfpathmoveto{\pgfqpoint{0.000000in}{0.000000in}}%
\pgfpathlineto{\pgfqpoint{0.000000in}{-0.048611in}}%
\pgfusepath{stroke,fill}%
}%
\begin{pgfscope}%
\pgfsys@transformshift{1.274415in}{0.548486in}%
\pgfsys@useobject{currentmarker}{}%
\end{pgfscope}%
\end{pgfscope}%
\begin{pgfscope}%
\definecolor{textcolor}{rgb}{0.000000,0.000000,0.000000}%
\pgfsetstrokecolor{textcolor}%
\pgfsetfillcolor{textcolor}%
\pgftext[x=1.274415in,y=0.451264in,,top]{\color{textcolor}{\rmfamily\fontsize{10.000000}{12.000000}\selectfont\catcode`\^=\active\def^{\ifmmode\sp\else\^{}\fi}\catcode`\%=\active\def%{\%}\ensuremath{-}0.5}}%
\end{pgfscope}%
\begin{pgfscope}%
\pgfsetbuttcap%
\pgfsetroundjoin%
\definecolor{currentfill}{rgb}{0.000000,0.000000,0.000000}%
\pgfsetfillcolor{currentfill}%
\pgfsetlinewidth{0.803000pt}%
\definecolor{currentstroke}{rgb}{0.000000,0.000000,0.000000}%
\pgfsetstrokecolor{currentstroke}%
\pgfsetdash{}{0pt}%
\pgfsys@defobject{currentmarker}{\pgfqpoint{0.000000in}{-0.048611in}}{\pgfqpoint{0.000000in}{0.000000in}}{%
\pgfpathmoveto{\pgfqpoint{0.000000in}{0.000000in}}%
\pgfpathlineto{\pgfqpoint{0.000000in}{-0.048611in}}%
\pgfusepath{stroke,fill}%
}%
\begin{pgfscope}%
\pgfsys@transformshift{1.802824in}{0.548486in}%
\pgfsys@useobject{currentmarker}{}%
\end{pgfscope}%
\end{pgfscope}%
\begin{pgfscope}%
\definecolor{textcolor}{rgb}{0.000000,0.000000,0.000000}%
\pgfsetstrokecolor{textcolor}%
\pgfsetfillcolor{textcolor}%
\pgftext[x=1.802824in,y=0.451264in,,top]{\color{textcolor}{\rmfamily\fontsize{10.000000}{12.000000}\selectfont\catcode`\^=\active\def^{\ifmmode\sp\else\^{}\fi}\catcode`\%=\active\def%{\%}0.0}}%
\end{pgfscope}%
\begin{pgfscope}%
\pgfsetbuttcap%
\pgfsetroundjoin%
\definecolor{currentfill}{rgb}{0.000000,0.000000,0.000000}%
\pgfsetfillcolor{currentfill}%
\pgfsetlinewidth{0.803000pt}%
\definecolor{currentstroke}{rgb}{0.000000,0.000000,0.000000}%
\pgfsetstrokecolor{currentstroke}%
\pgfsetdash{}{0pt}%
\pgfsys@defobject{currentmarker}{\pgfqpoint{0.000000in}{-0.048611in}}{\pgfqpoint{0.000000in}{0.000000in}}{%
\pgfpathmoveto{\pgfqpoint{0.000000in}{0.000000in}}%
\pgfpathlineto{\pgfqpoint{0.000000in}{-0.048611in}}%
\pgfusepath{stroke,fill}%
}%
\begin{pgfscope}%
\pgfsys@transformshift{2.331233in}{0.548486in}%
\pgfsys@useobject{currentmarker}{}%
\end{pgfscope}%
\end{pgfscope}%
\begin{pgfscope}%
\definecolor{textcolor}{rgb}{0.000000,0.000000,0.000000}%
\pgfsetstrokecolor{textcolor}%
\pgfsetfillcolor{textcolor}%
\pgftext[x=2.331233in,y=0.451264in,,top]{\color{textcolor}{\rmfamily\fontsize{10.000000}{12.000000}\selectfont\catcode`\^=\active\def^{\ifmmode\sp\else\^{}\fi}\catcode`\%=\active\def%{\%}0.5}}%
\end{pgfscope}%
\begin{pgfscope}%
\pgfsetbuttcap%
\pgfsetroundjoin%
\definecolor{currentfill}{rgb}{0.000000,0.000000,0.000000}%
\pgfsetfillcolor{currentfill}%
\pgfsetlinewidth{0.803000pt}%
\definecolor{currentstroke}{rgb}{0.000000,0.000000,0.000000}%
\pgfsetstrokecolor{currentstroke}%
\pgfsetdash{}{0pt}%
\pgfsys@defobject{currentmarker}{\pgfqpoint{0.000000in}{-0.048611in}}{\pgfqpoint{0.000000in}{0.000000in}}{%
\pgfpathmoveto{\pgfqpoint{0.000000in}{0.000000in}}%
\pgfpathlineto{\pgfqpoint{0.000000in}{-0.048611in}}%
\pgfusepath{stroke,fill}%
}%
\begin{pgfscope}%
\pgfsys@transformshift{2.859642in}{0.548486in}%
\pgfsys@useobject{currentmarker}{}%
\end{pgfscope}%
\end{pgfscope}%
\begin{pgfscope}%
\definecolor{textcolor}{rgb}{0.000000,0.000000,0.000000}%
\pgfsetstrokecolor{textcolor}%
\pgfsetfillcolor{textcolor}%
\pgftext[x=2.859642in,y=0.451264in,,top]{\color{textcolor}{\rmfamily\fontsize{10.000000}{12.000000}\selectfont\catcode`\^=\active\def^{\ifmmode\sp\else\^{}\fi}\catcode`\%=\active\def%{\%}1.0}}%
\end{pgfscope}%
\begin{pgfscope}%
\definecolor{textcolor}{rgb}{0.000000,0.000000,0.000000}%
\pgfsetstrokecolor{textcolor}%
\pgfsetfillcolor{textcolor}%
\pgftext[x=1.802824in,y=0.261295in,,top]{\color{textcolor}{\rmfamily\fontsize{12.000000}{14.400000}\selectfont\catcode`\^=\active\def^{\ifmmode\sp\else\^{}\fi}\catcode`\%=\active\def%{\%}$t$}}%
\end{pgfscope}%
\begin{pgfscope}%
\pgfsetbuttcap%
\pgfsetroundjoin%
\definecolor{currentfill}{rgb}{0.000000,0.000000,0.000000}%
\pgfsetfillcolor{currentfill}%
\pgfsetlinewidth{0.803000pt}%
\definecolor{currentstroke}{rgb}{0.000000,0.000000,0.000000}%
\pgfsetstrokecolor{currentstroke}%
\pgfsetdash{}{0pt}%
\pgfsys@defobject{currentmarker}{\pgfqpoint{-0.048611in}{0.000000in}}{\pgfqpoint{-0.000000in}{0.000000in}}{%
\pgfpathmoveto{\pgfqpoint{-0.000000in}{0.000000in}}%
\pgfpathlineto{\pgfqpoint{-0.048611in}{0.000000in}}%
\pgfusepath{stroke,fill}%
}%
\begin{pgfscope}%
\pgfsys@transformshift{0.640324in}{0.653486in}%
\pgfsys@useobject{currentmarker}{}%
\end{pgfscope}%
\end{pgfscope}%
\begin{pgfscope}%
\definecolor{textcolor}{rgb}{0.000000,0.000000,0.000000}%
\pgfsetstrokecolor{textcolor}%
\pgfsetfillcolor{textcolor}%
\pgftext[x=0.322222in, y=0.600724in, left, base]{\color{textcolor}{\rmfamily\fontsize{10.000000}{12.000000}\selectfont\catcode`\^=\active\def^{\ifmmode\sp\else\^{}\fi}\catcode`\%=\active\def%{\%}0.0}}%
\end{pgfscope}%
\begin{pgfscope}%
\pgfsetbuttcap%
\pgfsetroundjoin%
\definecolor{currentfill}{rgb}{0.000000,0.000000,0.000000}%
\pgfsetfillcolor{currentfill}%
\pgfsetlinewidth{0.803000pt}%
\definecolor{currentstroke}{rgb}{0.000000,0.000000,0.000000}%
\pgfsetstrokecolor{currentstroke}%
\pgfsetdash{}{0pt}%
\pgfsys@defobject{currentmarker}{\pgfqpoint{-0.048611in}{0.000000in}}{\pgfqpoint{-0.000000in}{0.000000in}}{%
\pgfpathmoveto{\pgfqpoint{-0.000000in}{0.000000in}}%
\pgfpathlineto{\pgfqpoint{-0.048611in}{0.000000in}}%
\pgfusepath{stroke,fill}%
}%
\begin{pgfscope}%
\pgfsys@transformshift{0.640324in}{1.073486in}%
\pgfsys@useobject{currentmarker}{}%
\end{pgfscope}%
\end{pgfscope}%
\begin{pgfscope}%
\definecolor{textcolor}{rgb}{0.000000,0.000000,0.000000}%
\pgfsetstrokecolor{textcolor}%
\pgfsetfillcolor{textcolor}%
\pgftext[x=0.322222in, y=1.020724in, left, base]{\color{textcolor}{\rmfamily\fontsize{10.000000}{12.000000}\selectfont\catcode`\^=\active\def^{\ifmmode\sp\else\^{}\fi}\catcode`\%=\active\def%{\%}0.2}}%
\end{pgfscope}%
\begin{pgfscope}%
\pgfsetbuttcap%
\pgfsetroundjoin%
\definecolor{currentfill}{rgb}{0.000000,0.000000,0.000000}%
\pgfsetfillcolor{currentfill}%
\pgfsetlinewidth{0.803000pt}%
\definecolor{currentstroke}{rgb}{0.000000,0.000000,0.000000}%
\pgfsetstrokecolor{currentstroke}%
\pgfsetdash{}{0pt}%
\pgfsys@defobject{currentmarker}{\pgfqpoint{-0.048611in}{0.000000in}}{\pgfqpoint{-0.000000in}{0.000000in}}{%
\pgfpathmoveto{\pgfqpoint{-0.000000in}{0.000000in}}%
\pgfpathlineto{\pgfqpoint{-0.048611in}{0.000000in}}%
\pgfusepath{stroke,fill}%
}%
\begin{pgfscope}%
\pgfsys@transformshift{0.640324in}{1.493486in}%
\pgfsys@useobject{currentmarker}{}%
\end{pgfscope}%
\end{pgfscope}%
\begin{pgfscope}%
\definecolor{textcolor}{rgb}{0.000000,0.000000,0.000000}%
\pgfsetstrokecolor{textcolor}%
\pgfsetfillcolor{textcolor}%
\pgftext[x=0.322222in, y=1.440724in, left, base]{\color{textcolor}{\rmfamily\fontsize{10.000000}{12.000000}\selectfont\catcode`\^=\active\def^{\ifmmode\sp\else\^{}\fi}\catcode`\%=\active\def%{\%}0.4}}%
\end{pgfscope}%
\begin{pgfscope}%
\pgfsetbuttcap%
\pgfsetroundjoin%
\definecolor{currentfill}{rgb}{0.000000,0.000000,0.000000}%
\pgfsetfillcolor{currentfill}%
\pgfsetlinewidth{0.803000pt}%
\definecolor{currentstroke}{rgb}{0.000000,0.000000,0.000000}%
\pgfsetstrokecolor{currentstroke}%
\pgfsetdash{}{0pt}%
\pgfsys@defobject{currentmarker}{\pgfqpoint{-0.048611in}{0.000000in}}{\pgfqpoint{-0.000000in}{0.000000in}}{%
\pgfpathmoveto{\pgfqpoint{-0.000000in}{0.000000in}}%
\pgfpathlineto{\pgfqpoint{-0.048611in}{0.000000in}}%
\pgfusepath{stroke,fill}%
}%
\begin{pgfscope}%
\pgfsys@transformshift{0.640324in}{1.913486in}%
\pgfsys@useobject{currentmarker}{}%
\end{pgfscope}%
\end{pgfscope}%
\begin{pgfscope}%
\definecolor{textcolor}{rgb}{0.000000,0.000000,0.000000}%
\pgfsetstrokecolor{textcolor}%
\pgfsetfillcolor{textcolor}%
\pgftext[x=0.322222in, y=1.860724in, left, base]{\color{textcolor}{\rmfamily\fontsize{10.000000}{12.000000}\selectfont\catcode`\^=\active\def^{\ifmmode\sp\else\^{}\fi}\catcode`\%=\active\def%{\%}0.6}}%
\end{pgfscope}%
\begin{pgfscope}%
\pgfsetbuttcap%
\pgfsetroundjoin%
\definecolor{currentfill}{rgb}{0.000000,0.000000,0.000000}%
\pgfsetfillcolor{currentfill}%
\pgfsetlinewidth{0.803000pt}%
\definecolor{currentstroke}{rgb}{0.000000,0.000000,0.000000}%
\pgfsetstrokecolor{currentstroke}%
\pgfsetdash{}{0pt}%
\pgfsys@defobject{currentmarker}{\pgfqpoint{-0.048611in}{0.000000in}}{\pgfqpoint{-0.000000in}{0.000000in}}{%
\pgfpathmoveto{\pgfqpoint{-0.000000in}{0.000000in}}%
\pgfpathlineto{\pgfqpoint{-0.048611in}{0.000000in}}%
\pgfusepath{stroke,fill}%
}%
\begin{pgfscope}%
\pgfsys@transformshift{0.640324in}{2.333486in}%
\pgfsys@useobject{currentmarker}{}%
\end{pgfscope}%
\end{pgfscope}%
\begin{pgfscope}%
\definecolor{textcolor}{rgb}{0.000000,0.000000,0.000000}%
\pgfsetstrokecolor{textcolor}%
\pgfsetfillcolor{textcolor}%
\pgftext[x=0.322222in, y=2.280724in, left, base]{\color{textcolor}{\rmfamily\fontsize{10.000000}{12.000000}\selectfont\catcode`\^=\active\def^{\ifmmode\sp\else\^{}\fi}\catcode`\%=\active\def%{\%}0.8}}%
\end{pgfscope}%
\begin{pgfscope}%
\pgfsetbuttcap%
\pgfsetroundjoin%
\definecolor{currentfill}{rgb}{0.000000,0.000000,0.000000}%
\pgfsetfillcolor{currentfill}%
\pgfsetlinewidth{0.803000pt}%
\definecolor{currentstroke}{rgb}{0.000000,0.000000,0.000000}%
\pgfsetstrokecolor{currentstroke}%
\pgfsetdash{}{0pt}%
\pgfsys@defobject{currentmarker}{\pgfqpoint{-0.048611in}{0.000000in}}{\pgfqpoint{-0.000000in}{0.000000in}}{%
\pgfpathmoveto{\pgfqpoint{-0.000000in}{0.000000in}}%
\pgfpathlineto{\pgfqpoint{-0.048611in}{0.000000in}}%
\pgfusepath{stroke,fill}%
}%
\begin{pgfscope}%
\pgfsys@transformshift{0.640324in}{2.753486in}%
\pgfsys@useobject{currentmarker}{}%
\end{pgfscope}%
\end{pgfscope}%
\begin{pgfscope}%
\definecolor{textcolor}{rgb}{0.000000,0.000000,0.000000}%
\pgfsetstrokecolor{textcolor}%
\pgfsetfillcolor{textcolor}%
\pgftext[x=0.322222in, y=2.700724in, left, base]{\color{textcolor}{\rmfamily\fontsize{10.000000}{12.000000}\selectfont\catcode`\^=\active\def^{\ifmmode\sp\else\^{}\fi}\catcode`\%=\active\def%{\%}1.0}}%
\end{pgfscope}%
\begin{pgfscope}%
\definecolor{textcolor}{rgb}{0.000000,0.000000,0.000000}%
\pgfsetstrokecolor{textcolor}%
\pgfsetfillcolor{textcolor}%
\pgftext[x=0.266667in,y=1.703486in,,bottom,rotate=90.000000]{\color{textcolor}{\rmfamily\fontsize{12.000000}{14.400000}\selectfont\catcode`\^=\active\def^{\ifmmode\sp\else\^{}\fi}\catcode`\%=\active\def%{\%}$\delta(t)$}}%
\end{pgfscope}%
\begin{pgfscope}%
\pgfpathrectangle{\pgfqpoint{0.640324in}{0.548486in}}{\pgfqpoint{2.325000in}{2.310000in}}%
\pgfusepath{clip}%
\pgfsetrectcap%
\pgfsetroundjoin%
\pgfsetlinewidth{1.505625pt}%
\definecolor{currentstroke}{rgb}{0.000000,0.000000,0.000000}%
\pgfsetstrokecolor{currentstroke}%
\pgfsetdash{}{0pt}%
\pgfpathmoveto{\pgfqpoint{0.746006in}{0.653486in}}%
\pgfpathlineto{\pgfqpoint{1.800710in}{0.653486in}}%
\pgfpathlineto{\pgfqpoint{1.802824in}{2.753486in}}%
\pgfpathlineto{\pgfqpoint{1.804938in}{0.653486in}}%
\pgfpathlineto{\pgfqpoint{2.859642in}{0.653486in}}%
\pgfpathlineto{\pgfqpoint{2.859642in}{0.653486in}}%
\pgfusepath{stroke}%
\end{pgfscope}%
\end{pgfpicture}%
\makeatother%
\endgroup%

% 	\caption{Impuls.}
% 	\label{fig:delta}
% \end{figure}


\begin{figure}[H]
    \centering
    \subfigure[Impuls]{
        %% Creator: Matplotlib, PGF backend
%%
%% To include the figure in your LaTeX document, write
%%   \input{<filename>.pgf}
%%
%% Make sure the required packages are loaded in your preamble
%%   \usepackage{pgf}
%%
%% Also ensure that all the required font packages are loaded; for instance,
%% the lmodern package is sometimes necessary when using math font.
%%   \usepackage{lmodern}
%%
%% Figures using additional raster images can only be included by \input if
%% they are in the same directory as the main LaTeX file. For loading figures
%% from other directories you can use the `import` package
%%   \usepackage{import}
%%
%% and then include the figures with
%%   \import{<path to file>}{<filename>.pgf}
%%
%% Matplotlib used the following preamble
%%   \def\mathdefault#1{#1}
%%   \everymath=\expandafter{\the\everymath\displaystyle}
%%   
%%   \usepackage{fontspec}
%%   \setmainfont{VeraSe.ttf}[Path=\detokenize{/usr/share/fonts/TTF/}]
%%   \setsansfont{DejaVuSans.ttf}[Path=\detokenize{/home/pl/miniconda3/lib/python3.12/site-packages/matplotlib/mpl-data/fonts/ttf/}]
%%   \setmonofont{DejaVuSansMono.ttf}[Path=\detokenize{/home/pl/miniconda3/lib/python3.12/site-packages/matplotlib/mpl-data/fonts/ttf/}]
%%   \makeatletter\@ifpackageloaded{underscore}{}{\usepackage[strings]{underscore}}\makeatother
%%
\begingroup%
\makeatletter%
\begin{pgfpicture}%
\pgfpathrectangle{\pgfpointorigin}{\pgfqpoint{3.070082in}{2.958486in}}%
\pgfusepath{use as bounding box, clip}%
\begin{pgfscope}%
\pgfsetbuttcap%
\pgfsetmiterjoin%
\definecolor{currentfill}{rgb}{1.000000,1.000000,1.000000}%
\pgfsetfillcolor{currentfill}%
\pgfsetlinewidth{0.000000pt}%
\definecolor{currentstroke}{rgb}{1.000000,1.000000,1.000000}%
\pgfsetstrokecolor{currentstroke}%
\pgfsetdash{}{0pt}%
\pgfpathmoveto{\pgfqpoint{0.000000in}{0.000000in}}%
\pgfpathlineto{\pgfqpoint{3.070082in}{0.000000in}}%
\pgfpathlineto{\pgfqpoint{3.070082in}{2.958486in}}%
\pgfpathlineto{\pgfqpoint{0.000000in}{2.958486in}}%
\pgfpathlineto{\pgfqpoint{0.000000in}{0.000000in}}%
\pgfpathclose%
\pgfusepath{fill}%
\end{pgfscope}%
\begin{pgfscope}%
\pgfsetbuttcap%
\pgfsetmiterjoin%
\definecolor{currentfill}{rgb}{1.000000,1.000000,1.000000}%
\pgfsetfillcolor{currentfill}%
\pgfsetlinewidth{0.000000pt}%
\definecolor{currentstroke}{rgb}{0.000000,0.000000,0.000000}%
\pgfsetstrokecolor{currentstroke}%
\pgfsetstrokeopacity{0.000000}%
\pgfsetdash{}{0pt}%
\pgfpathmoveto{\pgfqpoint{0.640324in}{0.548486in}}%
\pgfpathlineto{\pgfqpoint{2.965324in}{0.548486in}}%
\pgfpathlineto{\pgfqpoint{2.965324in}{2.858486in}}%
\pgfpathlineto{\pgfqpoint{0.640324in}{2.858486in}}%
\pgfpathlineto{\pgfqpoint{0.640324in}{0.548486in}}%
\pgfpathclose%
\pgfusepath{fill}%
\end{pgfscope}%
\begin{pgfscope}%
\pgfsetbuttcap%
\pgfsetroundjoin%
\definecolor{currentfill}{rgb}{0.000000,0.000000,0.000000}%
\pgfsetfillcolor{currentfill}%
\pgfsetlinewidth{0.803000pt}%
\definecolor{currentstroke}{rgb}{0.000000,0.000000,0.000000}%
\pgfsetstrokecolor{currentstroke}%
\pgfsetdash{}{0pt}%
\pgfsys@defobject{currentmarker}{\pgfqpoint{0.000000in}{-0.048611in}}{\pgfqpoint{0.000000in}{0.000000in}}{%
\pgfpathmoveto{\pgfqpoint{0.000000in}{0.000000in}}%
\pgfpathlineto{\pgfqpoint{0.000000in}{-0.048611in}}%
\pgfusepath{stroke,fill}%
}%
\begin{pgfscope}%
\pgfsys@transformshift{0.746006in}{0.548486in}%
\pgfsys@useobject{currentmarker}{}%
\end{pgfscope}%
\end{pgfscope}%
\begin{pgfscope}%
\definecolor{textcolor}{rgb}{0.000000,0.000000,0.000000}%
\pgfsetstrokecolor{textcolor}%
\pgfsetfillcolor{textcolor}%
\pgftext[x=0.746006in,y=0.451264in,,top]{\color{textcolor}{\rmfamily\fontsize{10.000000}{12.000000}\selectfont\catcode`\^=\active\def^{\ifmmode\sp\else\^{}\fi}\catcode`\%=\active\def%{\%}\ensuremath{-}1.0}}%
\end{pgfscope}%
\begin{pgfscope}%
\pgfsetbuttcap%
\pgfsetroundjoin%
\definecolor{currentfill}{rgb}{0.000000,0.000000,0.000000}%
\pgfsetfillcolor{currentfill}%
\pgfsetlinewidth{0.803000pt}%
\definecolor{currentstroke}{rgb}{0.000000,0.000000,0.000000}%
\pgfsetstrokecolor{currentstroke}%
\pgfsetdash{}{0pt}%
\pgfsys@defobject{currentmarker}{\pgfqpoint{0.000000in}{-0.048611in}}{\pgfqpoint{0.000000in}{0.000000in}}{%
\pgfpathmoveto{\pgfqpoint{0.000000in}{0.000000in}}%
\pgfpathlineto{\pgfqpoint{0.000000in}{-0.048611in}}%
\pgfusepath{stroke,fill}%
}%
\begin{pgfscope}%
\pgfsys@transformshift{1.274415in}{0.548486in}%
\pgfsys@useobject{currentmarker}{}%
\end{pgfscope}%
\end{pgfscope}%
\begin{pgfscope}%
\definecolor{textcolor}{rgb}{0.000000,0.000000,0.000000}%
\pgfsetstrokecolor{textcolor}%
\pgfsetfillcolor{textcolor}%
\pgftext[x=1.274415in,y=0.451264in,,top]{\color{textcolor}{\rmfamily\fontsize{10.000000}{12.000000}\selectfont\catcode`\^=\active\def^{\ifmmode\sp\else\^{}\fi}\catcode`\%=\active\def%{\%}\ensuremath{-}0.5}}%
\end{pgfscope}%
\begin{pgfscope}%
\pgfsetbuttcap%
\pgfsetroundjoin%
\definecolor{currentfill}{rgb}{0.000000,0.000000,0.000000}%
\pgfsetfillcolor{currentfill}%
\pgfsetlinewidth{0.803000pt}%
\definecolor{currentstroke}{rgb}{0.000000,0.000000,0.000000}%
\pgfsetstrokecolor{currentstroke}%
\pgfsetdash{}{0pt}%
\pgfsys@defobject{currentmarker}{\pgfqpoint{0.000000in}{-0.048611in}}{\pgfqpoint{0.000000in}{0.000000in}}{%
\pgfpathmoveto{\pgfqpoint{0.000000in}{0.000000in}}%
\pgfpathlineto{\pgfqpoint{0.000000in}{-0.048611in}}%
\pgfusepath{stroke,fill}%
}%
\begin{pgfscope}%
\pgfsys@transformshift{1.802824in}{0.548486in}%
\pgfsys@useobject{currentmarker}{}%
\end{pgfscope}%
\end{pgfscope}%
\begin{pgfscope}%
\definecolor{textcolor}{rgb}{0.000000,0.000000,0.000000}%
\pgfsetstrokecolor{textcolor}%
\pgfsetfillcolor{textcolor}%
\pgftext[x=1.802824in,y=0.451264in,,top]{\color{textcolor}{\rmfamily\fontsize{10.000000}{12.000000}\selectfont\catcode`\^=\active\def^{\ifmmode\sp\else\^{}\fi}\catcode`\%=\active\def%{\%}0.0}}%
\end{pgfscope}%
\begin{pgfscope}%
\pgfsetbuttcap%
\pgfsetroundjoin%
\definecolor{currentfill}{rgb}{0.000000,0.000000,0.000000}%
\pgfsetfillcolor{currentfill}%
\pgfsetlinewidth{0.803000pt}%
\definecolor{currentstroke}{rgb}{0.000000,0.000000,0.000000}%
\pgfsetstrokecolor{currentstroke}%
\pgfsetdash{}{0pt}%
\pgfsys@defobject{currentmarker}{\pgfqpoint{0.000000in}{-0.048611in}}{\pgfqpoint{0.000000in}{0.000000in}}{%
\pgfpathmoveto{\pgfqpoint{0.000000in}{0.000000in}}%
\pgfpathlineto{\pgfqpoint{0.000000in}{-0.048611in}}%
\pgfusepath{stroke,fill}%
}%
\begin{pgfscope}%
\pgfsys@transformshift{2.331233in}{0.548486in}%
\pgfsys@useobject{currentmarker}{}%
\end{pgfscope}%
\end{pgfscope}%
\begin{pgfscope}%
\definecolor{textcolor}{rgb}{0.000000,0.000000,0.000000}%
\pgfsetstrokecolor{textcolor}%
\pgfsetfillcolor{textcolor}%
\pgftext[x=2.331233in,y=0.451264in,,top]{\color{textcolor}{\rmfamily\fontsize{10.000000}{12.000000}\selectfont\catcode`\^=\active\def^{\ifmmode\sp\else\^{}\fi}\catcode`\%=\active\def%{\%}0.5}}%
\end{pgfscope}%
\begin{pgfscope}%
\pgfsetbuttcap%
\pgfsetroundjoin%
\definecolor{currentfill}{rgb}{0.000000,0.000000,0.000000}%
\pgfsetfillcolor{currentfill}%
\pgfsetlinewidth{0.803000pt}%
\definecolor{currentstroke}{rgb}{0.000000,0.000000,0.000000}%
\pgfsetstrokecolor{currentstroke}%
\pgfsetdash{}{0pt}%
\pgfsys@defobject{currentmarker}{\pgfqpoint{0.000000in}{-0.048611in}}{\pgfqpoint{0.000000in}{0.000000in}}{%
\pgfpathmoveto{\pgfqpoint{0.000000in}{0.000000in}}%
\pgfpathlineto{\pgfqpoint{0.000000in}{-0.048611in}}%
\pgfusepath{stroke,fill}%
}%
\begin{pgfscope}%
\pgfsys@transformshift{2.859642in}{0.548486in}%
\pgfsys@useobject{currentmarker}{}%
\end{pgfscope}%
\end{pgfscope}%
\begin{pgfscope}%
\definecolor{textcolor}{rgb}{0.000000,0.000000,0.000000}%
\pgfsetstrokecolor{textcolor}%
\pgfsetfillcolor{textcolor}%
\pgftext[x=2.859642in,y=0.451264in,,top]{\color{textcolor}{\rmfamily\fontsize{10.000000}{12.000000}\selectfont\catcode`\^=\active\def^{\ifmmode\sp\else\^{}\fi}\catcode`\%=\active\def%{\%}1.0}}%
\end{pgfscope}%
\begin{pgfscope}%
\definecolor{textcolor}{rgb}{0.000000,0.000000,0.000000}%
\pgfsetstrokecolor{textcolor}%
\pgfsetfillcolor{textcolor}%
\pgftext[x=1.802824in,y=0.261295in,,top]{\color{textcolor}{\rmfamily\fontsize{12.000000}{14.400000}\selectfont\catcode`\^=\active\def^{\ifmmode\sp\else\^{}\fi}\catcode`\%=\active\def%{\%}$t$}}%
\end{pgfscope}%
\begin{pgfscope}%
\pgfsetbuttcap%
\pgfsetroundjoin%
\definecolor{currentfill}{rgb}{0.000000,0.000000,0.000000}%
\pgfsetfillcolor{currentfill}%
\pgfsetlinewidth{0.803000pt}%
\definecolor{currentstroke}{rgb}{0.000000,0.000000,0.000000}%
\pgfsetstrokecolor{currentstroke}%
\pgfsetdash{}{0pt}%
\pgfsys@defobject{currentmarker}{\pgfqpoint{-0.048611in}{0.000000in}}{\pgfqpoint{-0.000000in}{0.000000in}}{%
\pgfpathmoveto{\pgfqpoint{-0.000000in}{0.000000in}}%
\pgfpathlineto{\pgfqpoint{-0.048611in}{0.000000in}}%
\pgfusepath{stroke,fill}%
}%
\begin{pgfscope}%
\pgfsys@transformshift{0.640324in}{0.653486in}%
\pgfsys@useobject{currentmarker}{}%
\end{pgfscope}%
\end{pgfscope}%
\begin{pgfscope}%
\definecolor{textcolor}{rgb}{0.000000,0.000000,0.000000}%
\pgfsetstrokecolor{textcolor}%
\pgfsetfillcolor{textcolor}%
\pgftext[x=0.322222in, y=0.600724in, left, base]{\color{textcolor}{\rmfamily\fontsize{10.000000}{12.000000}\selectfont\catcode`\^=\active\def^{\ifmmode\sp\else\^{}\fi}\catcode`\%=\active\def%{\%}0.0}}%
\end{pgfscope}%
\begin{pgfscope}%
\pgfsetbuttcap%
\pgfsetroundjoin%
\definecolor{currentfill}{rgb}{0.000000,0.000000,0.000000}%
\pgfsetfillcolor{currentfill}%
\pgfsetlinewidth{0.803000pt}%
\definecolor{currentstroke}{rgb}{0.000000,0.000000,0.000000}%
\pgfsetstrokecolor{currentstroke}%
\pgfsetdash{}{0pt}%
\pgfsys@defobject{currentmarker}{\pgfqpoint{-0.048611in}{0.000000in}}{\pgfqpoint{-0.000000in}{0.000000in}}{%
\pgfpathmoveto{\pgfqpoint{-0.000000in}{0.000000in}}%
\pgfpathlineto{\pgfqpoint{-0.048611in}{0.000000in}}%
\pgfusepath{stroke,fill}%
}%
\begin{pgfscope}%
\pgfsys@transformshift{0.640324in}{1.073486in}%
\pgfsys@useobject{currentmarker}{}%
\end{pgfscope}%
\end{pgfscope}%
\begin{pgfscope}%
\definecolor{textcolor}{rgb}{0.000000,0.000000,0.000000}%
\pgfsetstrokecolor{textcolor}%
\pgfsetfillcolor{textcolor}%
\pgftext[x=0.322222in, y=1.020724in, left, base]{\color{textcolor}{\rmfamily\fontsize{10.000000}{12.000000}\selectfont\catcode`\^=\active\def^{\ifmmode\sp\else\^{}\fi}\catcode`\%=\active\def%{\%}0.2}}%
\end{pgfscope}%
\begin{pgfscope}%
\pgfsetbuttcap%
\pgfsetroundjoin%
\definecolor{currentfill}{rgb}{0.000000,0.000000,0.000000}%
\pgfsetfillcolor{currentfill}%
\pgfsetlinewidth{0.803000pt}%
\definecolor{currentstroke}{rgb}{0.000000,0.000000,0.000000}%
\pgfsetstrokecolor{currentstroke}%
\pgfsetdash{}{0pt}%
\pgfsys@defobject{currentmarker}{\pgfqpoint{-0.048611in}{0.000000in}}{\pgfqpoint{-0.000000in}{0.000000in}}{%
\pgfpathmoveto{\pgfqpoint{-0.000000in}{0.000000in}}%
\pgfpathlineto{\pgfqpoint{-0.048611in}{0.000000in}}%
\pgfusepath{stroke,fill}%
}%
\begin{pgfscope}%
\pgfsys@transformshift{0.640324in}{1.493486in}%
\pgfsys@useobject{currentmarker}{}%
\end{pgfscope}%
\end{pgfscope}%
\begin{pgfscope}%
\definecolor{textcolor}{rgb}{0.000000,0.000000,0.000000}%
\pgfsetstrokecolor{textcolor}%
\pgfsetfillcolor{textcolor}%
\pgftext[x=0.322222in, y=1.440724in, left, base]{\color{textcolor}{\rmfamily\fontsize{10.000000}{12.000000}\selectfont\catcode`\^=\active\def^{\ifmmode\sp\else\^{}\fi}\catcode`\%=\active\def%{\%}0.4}}%
\end{pgfscope}%
\begin{pgfscope}%
\pgfsetbuttcap%
\pgfsetroundjoin%
\definecolor{currentfill}{rgb}{0.000000,0.000000,0.000000}%
\pgfsetfillcolor{currentfill}%
\pgfsetlinewidth{0.803000pt}%
\definecolor{currentstroke}{rgb}{0.000000,0.000000,0.000000}%
\pgfsetstrokecolor{currentstroke}%
\pgfsetdash{}{0pt}%
\pgfsys@defobject{currentmarker}{\pgfqpoint{-0.048611in}{0.000000in}}{\pgfqpoint{-0.000000in}{0.000000in}}{%
\pgfpathmoveto{\pgfqpoint{-0.000000in}{0.000000in}}%
\pgfpathlineto{\pgfqpoint{-0.048611in}{0.000000in}}%
\pgfusepath{stroke,fill}%
}%
\begin{pgfscope}%
\pgfsys@transformshift{0.640324in}{1.913486in}%
\pgfsys@useobject{currentmarker}{}%
\end{pgfscope}%
\end{pgfscope}%
\begin{pgfscope}%
\definecolor{textcolor}{rgb}{0.000000,0.000000,0.000000}%
\pgfsetstrokecolor{textcolor}%
\pgfsetfillcolor{textcolor}%
\pgftext[x=0.322222in, y=1.860724in, left, base]{\color{textcolor}{\rmfamily\fontsize{10.000000}{12.000000}\selectfont\catcode`\^=\active\def^{\ifmmode\sp\else\^{}\fi}\catcode`\%=\active\def%{\%}0.6}}%
\end{pgfscope}%
\begin{pgfscope}%
\pgfsetbuttcap%
\pgfsetroundjoin%
\definecolor{currentfill}{rgb}{0.000000,0.000000,0.000000}%
\pgfsetfillcolor{currentfill}%
\pgfsetlinewidth{0.803000pt}%
\definecolor{currentstroke}{rgb}{0.000000,0.000000,0.000000}%
\pgfsetstrokecolor{currentstroke}%
\pgfsetdash{}{0pt}%
\pgfsys@defobject{currentmarker}{\pgfqpoint{-0.048611in}{0.000000in}}{\pgfqpoint{-0.000000in}{0.000000in}}{%
\pgfpathmoveto{\pgfqpoint{-0.000000in}{0.000000in}}%
\pgfpathlineto{\pgfqpoint{-0.048611in}{0.000000in}}%
\pgfusepath{stroke,fill}%
}%
\begin{pgfscope}%
\pgfsys@transformshift{0.640324in}{2.333486in}%
\pgfsys@useobject{currentmarker}{}%
\end{pgfscope}%
\end{pgfscope}%
\begin{pgfscope}%
\definecolor{textcolor}{rgb}{0.000000,0.000000,0.000000}%
\pgfsetstrokecolor{textcolor}%
\pgfsetfillcolor{textcolor}%
\pgftext[x=0.322222in, y=2.280724in, left, base]{\color{textcolor}{\rmfamily\fontsize{10.000000}{12.000000}\selectfont\catcode`\^=\active\def^{\ifmmode\sp\else\^{}\fi}\catcode`\%=\active\def%{\%}0.8}}%
\end{pgfscope}%
\begin{pgfscope}%
\pgfsetbuttcap%
\pgfsetroundjoin%
\definecolor{currentfill}{rgb}{0.000000,0.000000,0.000000}%
\pgfsetfillcolor{currentfill}%
\pgfsetlinewidth{0.803000pt}%
\definecolor{currentstroke}{rgb}{0.000000,0.000000,0.000000}%
\pgfsetstrokecolor{currentstroke}%
\pgfsetdash{}{0pt}%
\pgfsys@defobject{currentmarker}{\pgfqpoint{-0.048611in}{0.000000in}}{\pgfqpoint{-0.000000in}{0.000000in}}{%
\pgfpathmoveto{\pgfqpoint{-0.000000in}{0.000000in}}%
\pgfpathlineto{\pgfqpoint{-0.048611in}{0.000000in}}%
\pgfusepath{stroke,fill}%
}%
\begin{pgfscope}%
\pgfsys@transformshift{0.640324in}{2.753486in}%
\pgfsys@useobject{currentmarker}{}%
\end{pgfscope}%
\end{pgfscope}%
\begin{pgfscope}%
\definecolor{textcolor}{rgb}{0.000000,0.000000,0.000000}%
\pgfsetstrokecolor{textcolor}%
\pgfsetfillcolor{textcolor}%
\pgftext[x=0.322222in, y=2.700724in, left, base]{\color{textcolor}{\rmfamily\fontsize{10.000000}{12.000000}\selectfont\catcode`\^=\active\def^{\ifmmode\sp\else\^{}\fi}\catcode`\%=\active\def%{\%}1.0}}%
\end{pgfscope}%
\begin{pgfscope}%
\definecolor{textcolor}{rgb}{0.000000,0.000000,0.000000}%
\pgfsetstrokecolor{textcolor}%
\pgfsetfillcolor{textcolor}%
\pgftext[x=0.266667in,y=1.703486in,,bottom,rotate=90.000000]{\color{textcolor}{\rmfamily\fontsize{12.000000}{14.400000}\selectfont\catcode`\^=\active\def^{\ifmmode\sp\else\^{}\fi}\catcode`\%=\active\def%{\%}$\delta(t)$}}%
\end{pgfscope}%
\begin{pgfscope}%
\pgfpathrectangle{\pgfqpoint{0.640324in}{0.548486in}}{\pgfqpoint{2.325000in}{2.310000in}}%
\pgfusepath{clip}%
\pgfsetrectcap%
\pgfsetroundjoin%
\pgfsetlinewidth{1.505625pt}%
\definecolor{currentstroke}{rgb}{0.000000,0.000000,0.000000}%
\pgfsetstrokecolor{currentstroke}%
\pgfsetdash{}{0pt}%
\pgfpathmoveto{\pgfqpoint{0.746006in}{0.653486in}}%
\pgfpathlineto{\pgfqpoint{1.800710in}{0.653486in}}%
\pgfpathlineto{\pgfqpoint{1.802824in}{2.753486in}}%
\pgfpathlineto{\pgfqpoint{1.804938in}{0.653486in}}%
\pgfpathlineto{\pgfqpoint{2.859642in}{0.653486in}}%
\pgfpathlineto{\pgfqpoint{2.859642in}{0.653486in}}%
\pgfusepath{stroke}%
\end{pgfscope}%
\end{pgfpicture}%
\makeatother%
\endgroup%

        \label{fig:delta}
    }
    % \hfill
    \subfigure[Alternative Darstellung.]{
        %% Creator: Matplotlib, PGF backend
%%
%% To include the figure in your LaTeX document, write
%%   \input{<filename>.pgf}
%%
%% Make sure the required packages are loaded in your preamble
%%   \usepackage{pgf}
%%
%% Also ensure that all the required font packages are loaded; for instance,
%% the lmodern package is sometimes necessary when using math font.
%%   \usepackage{lmodern}
%%
%% Figures using additional raster images can only be included by \input if
%% they are in the same directory as the main LaTeX file. For loading figures
%% from other directories you can use the `import` package
%%   \usepackage{import}
%%
%% and then include the figures with
%%   \import{<path to file>}{<filename>.pgf}
%%
%% Matplotlib used the following preamble
%%   \def\mathdefault#1{#1}
%%   \everymath=\expandafter{\the\everymath\displaystyle}
%%   
%%   \usepackage{fontspec}
%%   \setmainfont{VeraSe.ttf}[Path=\detokenize{/usr/share/fonts/TTF/}]
%%   \setsansfont{DejaVuSans.ttf}[Path=\detokenize{/home/pl/miniconda3/lib/python3.12/site-packages/matplotlib/mpl-data/fonts/ttf/}]
%%   \setmonofont{DejaVuSansMono.ttf}[Path=\detokenize{/home/pl/miniconda3/lib/python3.12/site-packages/matplotlib/mpl-data/fonts/ttf/}]
%%   \makeatletter\@ifpackageloaded{underscore}{}{\usepackage[strings]{underscore}}\makeatother
%%
\begingroup%
\makeatletter%
\begin{pgfpicture}%
\pgfpathrectangle{\pgfpointorigin}{\pgfqpoint{2.847860in}{2.958486in}}%
\pgfusepath{use as bounding box, clip}%
\begin{pgfscope}%
\pgfsetbuttcap%
\pgfsetmiterjoin%
\definecolor{currentfill}{rgb}{1.000000,1.000000,1.000000}%
\pgfsetfillcolor{currentfill}%
\pgfsetlinewidth{0.000000pt}%
\definecolor{currentstroke}{rgb}{1.000000,1.000000,1.000000}%
\pgfsetstrokecolor{currentstroke}%
\pgfsetdash{}{0pt}%
\pgfpathmoveto{\pgfqpoint{0.000000in}{0.000000in}}%
\pgfpathlineto{\pgfqpoint{2.847860in}{0.000000in}}%
\pgfpathlineto{\pgfqpoint{2.847860in}{2.958486in}}%
\pgfpathlineto{\pgfqpoint{0.000000in}{2.958486in}}%
\pgfpathlineto{\pgfqpoint{0.000000in}{0.000000in}}%
\pgfpathclose%
\pgfusepath{fill}%
\end{pgfscope}%
\begin{pgfscope}%
\pgfsetbuttcap%
\pgfsetmiterjoin%
\definecolor{currentfill}{rgb}{1.000000,1.000000,1.000000}%
\pgfsetfillcolor{currentfill}%
\pgfsetlinewidth{0.000000pt}%
\definecolor{currentstroke}{rgb}{0.000000,0.000000,0.000000}%
\pgfsetstrokecolor{currentstroke}%
\pgfsetstrokeopacity{0.000000}%
\pgfsetdash{}{0pt}%
\pgfpathmoveto{\pgfqpoint{0.418102in}{0.548486in}}%
\pgfpathlineto{\pgfqpoint{2.743102in}{0.548486in}}%
\pgfpathlineto{\pgfqpoint{2.743102in}{2.858486in}}%
\pgfpathlineto{\pgfqpoint{0.418102in}{2.858486in}}%
\pgfpathlineto{\pgfqpoint{0.418102in}{0.548486in}}%
\pgfpathclose%
\pgfusepath{fill}%
\end{pgfscope}%
\begin{pgfscope}%
\pgfsetbuttcap%
\pgfsetroundjoin%
\definecolor{currentfill}{rgb}{0.000000,0.000000,0.000000}%
\pgfsetfillcolor{currentfill}%
\pgfsetlinewidth{0.803000pt}%
\definecolor{currentstroke}{rgb}{0.000000,0.000000,0.000000}%
\pgfsetstrokecolor{currentstroke}%
\pgfsetdash{}{0pt}%
\pgfsys@defobject{currentmarker}{\pgfqpoint{0.000000in}{-0.048611in}}{\pgfqpoint{0.000000in}{0.000000in}}{%
\pgfpathmoveto{\pgfqpoint{0.000000in}{0.000000in}}%
\pgfpathlineto{\pgfqpoint{0.000000in}{-0.048611in}}%
\pgfusepath{stroke,fill}%
}%
\begin{pgfscope}%
\pgfsys@transformshift{0.523783in}{0.548486in}%
\pgfsys@useobject{currentmarker}{}%
\end{pgfscope}%
\end{pgfscope}%
\begin{pgfscope}%
\definecolor{textcolor}{rgb}{0.000000,0.000000,0.000000}%
\pgfsetstrokecolor{textcolor}%
\pgfsetfillcolor{textcolor}%
\pgftext[x=0.523783in,y=0.451264in,,top]{\color{textcolor}{\rmfamily\fontsize{10.000000}{12.000000}\selectfont\catcode`\^=\active\def^{\ifmmode\sp\else\^{}\fi}\catcode`\%=\active\def%{\%}\ensuremath{-}1.0}}%
\end{pgfscope}%
\begin{pgfscope}%
\pgfsetbuttcap%
\pgfsetroundjoin%
\definecolor{currentfill}{rgb}{0.000000,0.000000,0.000000}%
\pgfsetfillcolor{currentfill}%
\pgfsetlinewidth{0.803000pt}%
\definecolor{currentstroke}{rgb}{0.000000,0.000000,0.000000}%
\pgfsetstrokecolor{currentstroke}%
\pgfsetdash{}{0pt}%
\pgfsys@defobject{currentmarker}{\pgfqpoint{0.000000in}{-0.048611in}}{\pgfqpoint{0.000000in}{0.000000in}}{%
\pgfpathmoveto{\pgfqpoint{0.000000in}{0.000000in}}%
\pgfpathlineto{\pgfqpoint{0.000000in}{-0.048611in}}%
\pgfusepath{stroke,fill}%
}%
\begin{pgfscope}%
\pgfsys@transformshift{1.052193in}{0.548486in}%
\pgfsys@useobject{currentmarker}{}%
\end{pgfscope}%
\end{pgfscope}%
\begin{pgfscope}%
\definecolor{textcolor}{rgb}{0.000000,0.000000,0.000000}%
\pgfsetstrokecolor{textcolor}%
\pgfsetfillcolor{textcolor}%
\pgftext[x=1.052193in,y=0.451264in,,top]{\color{textcolor}{\rmfamily\fontsize{10.000000}{12.000000}\selectfont\catcode`\^=\active\def^{\ifmmode\sp\else\^{}\fi}\catcode`\%=\active\def%{\%}\ensuremath{-}0.5}}%
\end{pgfscope}%
\begin{pgfscope}%
\pgfsetbuttcap%
\pgfsetroundjoin%
\definecolor{currentfill}{rgb}{0.000000,0.000000,0.000000}%
\pgfsetfillcolor{currentfill}%
\pgfsetlinewidth{0.803000pt}%
\definecolor{currentstroke}{rgb}{0.000000,0.000000,0.000000}%
\pgfsetstrokecolor{currentstroke}%
\pgfsetdash{}{0pt}%
\pgfsys@defobject{currentmarker}{\pgfqpoint{0.000000in}{-0.048611in}}{\pgfqpoint{0.000000in}{0.000000in}}{%
\pgfpathmoveto{\pgfqpoint{0.000000in}{0.000000in}}%
\pgfpathlineto{\pgfqpoint{0.000000in}{-0.048611in}}%
\pgfusepath{stroke,fill}%
}%
\begin{pgfscope}%
\pgfsys@transformshift{1.580602in}{0.548486in}%
\pgfsys@useobject{currentmarker}{}%
\end{pgfscope}%
\end{pgfscope}%
\begin{pgfscope}%
\definecolor{textcolor}{rgb}{0.000000,0.000000,0.000000}%
\pgfsetstrokecolor{textcolor}%
\pgfsetfillcolor{textcolor}%
\pgftext[x=1.580602in,y=0.451264in,,top]{\color{textcolor}{\rmfamily\fontsize{10.000000}{12.000000}\selectfont\catcode`\^=\active\def^{\ifmmode\sp\else\^{}\fi}\catcode`\%=\active\def%{\%}0.0}}%
\end{pgfscope}%
\begin{pgfscope}%
\pgfsetbuttcap%
\pgfsetroundjoin%
\definecolor{currentfill}{rgb}{0.000000,0.000000,0.000000}%
\pgfsetfillcolor{currentfill}%
\pgfsetlinewidth{0.803000pt}%
\definecolor{currentstroke}{rgb}{0.000000,0.000000,0.000000}%
\pgfsetstrokecolor{currentstroke}%
\pgfsetdash{}{0pt}%
\pgfsys@defobject{currentmarker}{\pgfqpoint{0.000000in}{-0.048611in}}{\pgfqpoint{0.000000in}{0.000000in}}{%
\pgfpathmoveto{\pgfqpoint{0.000000in}{0.000000in}}%
\pgfpathlineto{\pgfqpoint{0.000000in}{-0.048611in}}%
\pgfusepath{stroke,fill}%
}%
\begin{pgfscope}%
\pgfsys@transformshift{2.109011in}{0.548486in}%
\pgfsys@useobject{currentmarker}{}%
\end{pgfscope}%
\end{pgfscope}%
\begin{pgfscope}%
\definecolor{textcolor}{rgb}{0.000000,0.000000,0.000000}%
\pgfsetstrokecolor{textcolor}%
\pgfsetfillcolor{textcolor}%
\pgftext[x=2.109011in,y=0.451264in,,top]{\color{textcolor}{\rmfamily\fontsize{10.000000}{12.000000}\selectfont\catcode`\^=\active\def^{\ifmmode\sp\else\^{}\fi}\catcode`\%=\active\def%{\%}0.5}}%
\end{pgfscope}%
\begin{pgfscope}%
\pgfsetbuttcap%
\pgfsetroundjoin%
\definecolor{currentfill}{rgb}{0.000000,0.000000,0.000000}%
\pgfsetfillcolor{currentfill}%
\pgfsetlinewidth{0.803000pt}%
\definecolor{currentstroke}{rgb}{0.000000,0.000000,0.000000}%
\pgfsetstrokecolor{currentstroke}%
\pgfsetdash{}{0pt}%
\pgfsys@defobject{currentmarker}{\pgfqpoint{0.000000in}{-0.048611in}}{\pgfqpoint{0.000000in}{0.000000in}}{%
\pgfpathmoveto{\pgfqpoint{0.000000in}{0.000000in}}%
\pgfpathlineto{\pgfqpoint{0.000000in}{-0.048611in}}%
\pgfusepath{stroke,fill}%
}%
\begin{pgfscope}%
\pgfsys@transformshift{2.637420in}{0.548486in}%
\pgfsys@useobject{currentmarker}{}%
\end{pgfscope}%
\end{pgfscope}%
\begin{pgfscope}%
\definecolor{textcolor}{rgb}{0.000000,0.000000,0.000000}%
\pgfsetstrokecolor{textcolor}%
\pgfsetfillcolor{textcolor}%
\pgftext[x=2.637420in,y=0.451264in,,top]{\color{textcolor}{\rmfamily\fontsize{10.000000}{12.000000}\selectfont\catcode`\^=\active\def^{\ifmmode\sp\else\^{}\fi}\catcode`\%=\active\def%{\%}1.0}}%
\end{pgfscope}%
\begin{pgfscope}%
\definecolor{textcolor}{rgb}{0.000000,0.000000,0.000000}%
\pgfsetstrokecolor{textcolor}%
\pgfsetfillcolor{textcolor}%
\pgftext[x=1.580602in,y=0.261295in,,top]{\color{textcolor}{\rmfamily\fontsize{12.000000}{14.400000}\selectfont\catcode`\^=\active\def^{\ifmmode\sp\else\^{}\fi}\catcode`\%=\active\def%{\%}$t$}}%
\end{pgfscope}%
\begin{pgfscope}%
\pgfsetbuttcap%
\pgfsetroundjoin%
\definecolor{currentfill}{rgb}{0.000000,0.000000,0.000000}%
\pgfsetfillcolor{currentfill}%
\pgfsetlinewidth{0.803000pt}%
\definecolor{currentstroke}{rgb}{0.000000,0.000000,0.000000}%
\pgfsetstrokecolor{currentstroke}%
\pgfsetdash{}{0pt}%
\pgfsys@defobject{currentmarker}{\pgfqpoint{-0.048611in}{0.000000in}}{\pgfqpoint{-0.000000in}{0.000000in}}{%
\pgfpathmoveto{\pgfqpoint{-0.000000in}{0.000000in}}%
\pgfpathlineto{\pgfqpoint{-0.048611in}{0.000000in}}%
\pgfusepath{stroke,fill}%
}%
\begin{pgfscope}%
\pgfsys@transformshift{0.418102in}{0.653486in}%
\pgfsys@useobject{currentmarker}{}%
\end{pgfscope}%
\end{pgfscope}%
\begin{pgfscope}%
\definecolor{textcolor}{rgb}{0.000000,0.000000,0.000000}%
\pgfsetstrokecolor{textcolor}%
\pgfsetfillcolor{textcolor}%
\pgftext[x=0.100000in, y=0.600724in, left, base]{\color{textcolor}{\rmfamily\fontsize{10.000000}{12.000000}\selectfont\catcode`\^=\active\def^{\ifmmode\sp\else\^{}\fi}\catcode`\%=\active\def%{\%}0.0}}%
\end{pgfscope}%
\begin{pgfscope}%
\pgfsetbuttcap%
\pgfsetroundjoin%
\definecolor{currentfill}{rgb}{0.000000,0.000000,0.000000}%
\pgfsetfillcolor{currentfill}%
\pgfsetlinewidth{0.803000pt}%
\definecolor{currentstroke}{rgb}{0.000000,0.000000,0.000000}%
\pgfsetstrokecolor{currentstroke}%
\pgfsetdash{}{0pt}%
\pgfsys@defobject{currentmarker}{\pgfqpoint{-0.048611in}{0.000000in}}{\pgfqpoint{-0.000000in}{0.000000in}}{%
\pgfpathmoveto{\pgfqpoint{-0.000000in}{0.000000in}}%
\pgfpathlineto{\pgfqpoint{-0.048611in}{0.000000in}}%
\pgfusepath{stroke,fill}%
}%
\begin{pgfscope}%
\pgfsys@transformshift{0.418102in}{1.073486in}%
\pgfsys@useobject{currentmarker}{}%
\end{pgfscope}%
\end{pgfscope}%
\begin{pgfscope}%
\definecolor{textcolor}{rgb}{0.000000,0.000000,0.000000}%
\pgfsetstrokecolor{textcolor}%
\pgfsetfillcolor{textcolor}%
\pgftext[x=0.100000in, y=1.020724in, left, base]{\color{textcolor}{\rmfamily\fontsize{10.000000}{12.000000}\selectfont\catcode`\^=\active\def^{\ifmmode\sp\else\^{}\fi}\catcode`\%=\active\def%{\%}0.2}}%
\end{pgfscope}%
\begin{pgfscope}%
\pgfsetbuttcap%
\pgfsetroundjoin%
\definecolor{currentfill}{rgb}{0.000000,0.000000,0.000000}%
\pgfsetfillcolor{currentfill}%
\pgfsetlinewidth{0.803000pt}%
\definecolor{currentstroke}{rgb}{0.000000,0.000000,0.000000}%
\pgfsetstrokecolor{currentstroke}%
\pgfsetdash{}{0pt}%
\pgfsys@defobject{currentmarker}{\pgfqpoint{-0.048611in}{0.000000in}}{\pgfqpoint{-0.000000in}{0.000000in}}{%
\pgfpathmoveto{\pgfqpoint{-0.000000in}{0.000000in}}%
\pgfpathlineto{\pgfqpoint{-0.048611in}{0.000000in}}%
\pgfusepath{stroke,fill}%
}%
\begin{pgfscope}%
\pgfsys@transformshift{0.418102in}{1.493486in}%
\pgfsys@useobject{currentmarker}{}%
\end{pgfscope}%
\end{pgfscope}%
\begin{pgfscope}%
\definecolor{textcolor}{rgb}{0.000000,0.000000,0.000000}%
\pgfsetstrokecolor{textcolor}%
\pgfsetfillcolor{textcolor}%
\pgftext[x=0.100000in, y=1.440724in, left, base]{\color{textcolor}{\rmfamily\fontsize{10.000000}{12.000000}\selectfont\catcode`\^=\active\def^{\ifmmode\sp\else\^{}\fi}\catcode`\%=\active\def%{\%}0.4}}%
\end{pgfscope}%
\begin{pgfscope}%
\pgfsetbuttcap%
\pgfsetroundjoin%
\definecolor{currentfill}{rgb}{0.000000,0.000000,0.000000}%
\pgfsetfillcolor{currentfill}%
\pgfsetlinewidth{0.803000pt}%
\definecolor{currentstroke}{rgb}{0.000000,0.000000,0.000000}%
\pgfsetstrokecolor{currentstroke}%
\pgfsetdash{}{0pt}%
\pgfsys@defobject{currentmarker}{\pgfqpoint{-0.048611in}{0.000000in}}{\pgfqpoint{-0.000000in}{0.000000in}}{%
\pgfpathmoveto{\pgfqpoint{-0.000000in}{0.000000in}}%
\pgfpathlineto{\pgfqpoint{-0.048611in}{0.000000in}}%
\pgfusepath{stroke,fill}%
}%
\begin{pgfscope}%
\pgfsys@transformshift{0.418102in}{1.913486in}%
\pgfsys@useobject{currentmarker}{}%
\end{pgfscope}%
\end{pgfscope}%
\begin{pgfscope}%
\definecolor{textcolor}{rgb}{0.000000,0.000000,0.000000}%
\pgfsetstrokecolor{textcolor}%
\pgfsetfillcolor{textcolor}%
\pgftext[x=0.100000in, y=1.860724in, left, base]{\color{textcolor}{\rmfamily\fontsize{10.000000}{12.000000}\selectfont\catcode`\^=\active\def^{\ifmmode\sp\else\^{}\fi}\catcode`\%=\active\def%{\%}0.6}}%
\end{pgfscope}%
\begin{pgfscope}%
\pgfsetbuttcap%
\pgfsetroundjoin%
\definecolor{currentfill}{rgb}{0.000000,0.000000,0.000000}%
\pgfsetfillcolor{currentfill}%
\pgfsetlinewidth{0.803000pt}%
\definecolor{currentstroke}{rgb}{0.000000,0.000000,0.000000}%
\pgfsetstrokecolor{currentstroke}%
\pgfsetdash{}{0pt}%
\pgfsys@defobject{currentmarker}{\pgfqpoint{-0.048611in}{0.000000in}}{\pgfqpoint{-0.000000in}{0.000000in}}{%
\pgfpathmoveto{\pgfqpoint{-0.000000in}{0.000000in}}%
\pgfpathlineto{\pgfqpoint{-0.048611in}{0.000000in}}%
\pgfusepath{stroke,fill}%
}%
\begin{pgfscope}%
\pgfsys@transformshift{0.418102in}{2.333486in}%
\pgfsys@useobject{currentmarker}{}%
\end{pgfscope}%
\end{pgfscope}%
\begin{pgfscope}%
\definecolor{textcolor}{rgb}{0.000000,0.000000,0.000000}%
\pgfsetstrokecolor{textcolor}%
\pgfsetfillcolor{textcolor}%
\pgftext[x=0.100000in, y=2.280724in, left, base]{\color{textcolor}{\rmfamily\fontsize{10.000000}{12.000000}\selectfont\catcode`\^=\active\def^{\ifmmode\sp\else\^{}\fi}\catcode`\%=\active\def%{\%}0.8}}%
\end{pgfscope}%
\begin{pgfscope}%
\pgfsetbuttcap%
\pgfsetroundjoin%
\definecolor{currentfill}{rgb}{0.000000,0.000000,0.000000}%
\pgfsetfillcolor{currentfill}%
\pgfsetlinewidth{0.803000pt}%
\definecolor{currentstroke}{rgb}{0.000000,0.000000,0.000000}%
\pgfsetstrokecolor{currentstroke}%
\pgfsetdash{}{0pt}%
\pgfsys@defobject{currentmarker}{\pgfqpoint{-0.048611in}{0.000000in}}{\pgfqpoint{-0.000000in}{0.000000in}}{%
\pgfpathmoveto{\pgfqpoint{-0.000000in}{0.000000in}}%
\pgfpathlineto{\pgfqpoint{-0.048611in}{0.000000in}}%
\pgfusepath{stroke,fill}%
}%
\begin{pgfscope}%
\pgfsys@transformshift{0.418102in}{2.753486in}%
\pgfsys@useobject{currentmarker}{}%
\end{pgfscope}%
\end{pgfscope}%
\begin{pgfscope}%
\definecolor{textcolor}{rgb}{0.000000,0.000000,0.000000}%
\pgfsetstrokecolor{textcolor}%
\pgfsetfillcolor{textcolor}%
\pgftext[x=0.100000in, y=2.700724in, left, base]{\color{textcolor}{\rmfamily\fontsize{10.000000}{12.000000}\selectfont\catcode`\^=\active\def^{\ifmmode\sp\else\^{}\fi}\catcode`\%=\active\def%{\%}1.0}}%
\end{pgfscope}%
\begin{pgfscope}%
\pgfpathrectangle{\pgfqpoint{0.418102in}{0.548486in}}{\pgfqpoint{2.325000in}{2.310000in}}%
\pgfusepath{clip}%
\pgfsetbuttcap%
\pgfsetroundjoin%
\pgfsetlinewidth{1.505625pt}%
\definecolor{currentstroke}{rgb}{0.000000,0.000000,0.000000}%
\pgfsetstrokecolor{currentstroke}%
\pgfsetdash{}{0pt}%
\pgfpathmoveto{\pgfqpoint{0.523783in}{0.653486in}}%
\pgfpathlineto{\pgfqpoint{0.523783in}{0.653486in}}%
\pgfusepath{stroke}%
\end{pgfscope}%
\begin{pgfscope}%
\pgfpathrectangle{\pgfqpoint{0.418102in}{0.548486in}}{\pgfqpoint{2.325000in}{2.310000in}}%
\pgfusepath{clip}%
\pgfsetbuttcap%
\pgfsetroundjoin%
\pgfsetlinewidth{1.505625pt}%
\definecolor{currentstroke}{rgb}{0.000000,0.000000,0.000000}%
\pgfsetstrokecolor{currentstroke}%
\pgfsetdash{}{0pt}%
\pgfpathmoveto{\pgfqpoint{0.735147in}{0.653486in}}%
\pgfpathlineto{\pgfqpoint{0.735147in}{0.653486in}}%
\pgfusepath{stroke}%
\end{pgfscope}%
\begin{pgfscope}%
\pgfpathrectangle{\pgfqpoint{0.418102in}{0.548486in}}{\pgfqpoint{2.325000in}{2.310000in}}%
\pgfusepath{clip}%
\pgfsetbuttcap%
\pgfsetroundjoin%
\pgfsetlinewidth{1.505625pt}%
\definecolor{currentstroke}{rgb}{0.000000,0.000000,0.000000}%
\pgfsetstrokecolor{currentstroke}%
\pgfsetdash{}{0pt}%
\pgfpathmoveto{\pgfqpoint{0.946511in}{0.653486in}}%
\pgfpathlineto{\pgfqpoint{0.946511in}{0.653486in}}%
\pgfusepath{stroke}%
\end{pgfscope}%
\begin{pgfscope}%
\pgfpathrectangle{\pgfqpoint{0.418102in}{0.548486in}}{\pgfqpoint{2.325000in}{2.310000in}}%
\pgfusepath{clip}%
\pgfsetbuttcap%
\pgfsetroundjoin%
\pgfsetlinewidth{1.505625pt}%
\definecolor{currentstroke}{rgb}{0.000000,0.000000,0.000000}%
\pgfsetstrokecolor{currentstroke}%
\pgfsetdash{}{0pt}%
\pgfpathmoveto{\pgfqpoint{1.157874in}{0.653486in}}%
\pgfpathlineto{\pgfqpoint{1.157874in}{0.653486in}}%
\pgfusepath{stroke}%
\end{pgfscope}%
\begin{pgfscope}%
\pgfpathrectangle{\pgfqpoint{0.418102in}{0.548486in}}{\pgfqpoint{2.325000in}{2.310000in}}%
\pgfusepath{clip}%
\pgfsetbuttcap%
\pgfsetroundjoin%
\pgfsetlinewidth{1.505625pt}%
\definecolor{currentstroke}{rgb}{0.000000,0.000000,0.000000}%
\pgfsetstrokecolor{currentstroke}%
\pgfsetdash{}{0pt}%
\pgfpathmoveto{\pgfqpoint{1.369238in}{0.653486in}}%
\pgfpathlineto{\pgfqpoint{1.369238in}{0.653486in}}%
\pgfusepath{stroke}%
\end{pgfscope}%
\begin{pgfscope}%
\pgfpathrectangle{\pgfqpoint{0.418102in}{0.548486in}}{\pgfqpoint{2.325000in}{2.310000in}}%
\pgfusepath{clip}%
\pgfsetbuttcap%
\pgfsetroundjoin%
\pgfsetlinewidth{1.505625pt}%
\definecolor{currentstroke}{rgb}{0.000000,0.000000,0.000000}%
\pgfsetstrokecolor{currentstroke}%
\pgfsetdash{}{0pt}%
\pgfpathmoveto{\pgfqpoint{1.580602in}{0.653486in}}%
\pgfpathlineto{\pgfqpoint{1.580602in}{2.753486in}}%
\pgfusepath{stroke}%
\end{pgfscope}%
\begin{pgfscope}%
\pgfpathrectangle{\pgfqpoint{0.418102in}{0.548486in}}{\pgfqpoint{2.325000in}{2.310000in}}%
\pgfusepath{clip}%
\pgfsetbuttcap%
\pgfsetroundjoin%
\pgfsetlinewidth{1.505625pt}%
\definecolor{currentstroke}{rgb}{0.000000,0.000000,0.000000}%
\pgfsetstrokecolor{currentstroke}%
\pgfsetdash{}{0pt}%
\pgfpathmoveto{\pgfqpoint{1.791965in}{0.653486in}}%
\pgfpathlineto{\pgfqpoint{1.791965in}{0.653486in}}%
\pgfusepath{stroke}%
\end{pgfscope}%
\begin{pgfscope}%
\pgfpathrectangle{\pgfqpoint{0.418102in}{0.548486in}}{\pgfqpoint{2.325000in}{2.310000in}}%
\pgfusepath{clip}%
\pgfsetbuttcap%
\pgfsetroundjoin%
\pgfsetlinewidth{1.505625pt}%
\definecolor{currentstroke}{rgb}{0.000000,0.000000,0.000000}%
\pgfsetstrokecolor{currentstroke}%
\pgfsetdash{}{0pt}%
\pgfpathmoveto{\pgfqpoint{2.003329in}{0.653486in}}%
\pgfpathlineto{\pgfqpoint{2.003329in}{0.653486in}}%
\pgfusepath{stroke}%
\end{pgfscope}%
\begin{pgfscope}%
\pgfpathrectangle{\pgfqpoint{0.418102in}{0.548486in}}{\pgfqpoint{2.325000in}{2.310000in}}%
\pgfusepath{clip}%
\pgfsetbuttcap%
\pgfsetroundjoin%
\pgfsetlinewidth{1.505625pt}%
\definecolor{currentstroke}{rgb}{0.000000,0.000000,0.000000}%
\pgfsetstrokecolor{currentstroke}%
\pgfsetdash{}{0pt}%
\pgfpathmoveto{\pgfqpoint{2.214693in}{0.653486in}}%
\pgfpathlineto{\pgfqpoint{2.214693in}{0.653486in}}%
\pgfusepath{stroke}%
\end{pgfscope}%
\begin{pgfscope}%
\pgfpathrectangle{\pgfqpoint{0.418102in}{0.548486in}}{\pgfqpoint{2.325000in}{2.310000in}}%
\pgfusepath{clip}%
\pgfsetbuttcap%
\pgfsetroundjoin%
\pgfsetlinewidth{1.505625pt}%
\definecolor{currentstroke}{rgb}{0.000000,0.000000,0.000000}%
\pgfsetstrokecolor{currentstroke}%
\pgfsetdash{}{0pt}%
\pgfpathmoveto{\pgfqpoint{2.426056in}{0.653486in}}%
\pgfpathlineto{\pgfqpoint{2.426056in}{0.653486in}}%
\pgfusepath{stroke}%
\end{pgfscope}%
\begin{pgfscope}%
\pgfpathrectangle{\pgfqpoint{0.418102in}{0.548486in}}{\pgfqpoint{2.325000in}{2.310000in}}%
\pgfusepath{clip}%
\pgfsetbuttcap%
\pgfsetroundjoin%
\pgfsetlinewidth{1.505625pt}%
\definecolor{currentstroke}{rgb}{0.000000,0.000000,0.000000}%
\pgfsetstrokecolor{currentstroke}%
\pgfsetdash{}{0pt}%
\pgfpathmoveto{\pgfqpoint{2.637420in}{0.653486in}}%
\pgfpathlineto{\pgfqpoint{2.637420in}{0.653486in}}%
\pgfusepath{stroke}%
\end{pgfscope}%
\begin{pgfscope}%
\pgfpathrectangle{\pgfqpoint{0.418102in}{0.548486in}}{\pgfqpoint{2.325000in}{2.310000in}}%
\pgfusepath{clip}%
\pgfsetbuttcap%
\pgfsetroundjoin%
\definecolor{currentfill}{rgb}{0.000000,0.000000,0.000000}%
\pgfsetfillcolor{currentfill}%
\pgfsetlinewidth{1.003750pt}%
\definecolor{currentstroke}{rgb}{0.000000,0.000000,0.000000}%
\pgfsetstrokecolor{currentstroke}%
\pgfsetdash{}{0pt}%
\pgfsys@defobject{currentmarker}{\pgfqpoint{-0.041667in}{-0.041667in}}{\pgfqpoint{0.041667in}{0.041667in}}{%
\pgfpathmoveto{\pgfqpoint{0.000000in}{-0.041667in}}%
\pgfpathcurveto{\pgfqpoint{0.011050in}{-0.041667in}}{\pgfqpoint{0.021649in}{-0.037276in}}{\pgfqpoint{0.029463in}{-0.029463in}}%
\pgfpathcurveto{\pgfqpoint{0.037276in}{-0.021649in}}{\pgfqpoint{0.041667in}{-0.011050in}}{\pgfqpoint{0.041667in}{0.000000in}}%
\pgfpathcurveto{\pgfqpoint{0.041667in}{0.011050in}}{\pgfqpoint{0.037276in}{0.021649in}}{\pgfqpoint{0.029463in}{0.029463in}}%
\pgfpathcurveto{\pgfqpoint{0.021649in}{0.037276in}}{\pgfqpoint{0.011050in}{0.041667in}}{\pgfqpoint{0.000000in}{0.041667in}}%
\pgfpathcurveto{\pgfqpoint{-0.011050in}{0.041667in}}{\pgfqpoint{-0.021649in}{0.037276in}}{\pgfqpoint{-0.029463in}{0.029463in}}%
\pgfpathcurveto{\pgfqpoint{-0.037276in}{0.021649in}}{\pgfqpoint{-0.041667in}{0.011050in}}{\pgfqpoint{-0.041667in}{0.000000in}}%
\pgfpathcurveto{\pgfqpoint{-0.041667in}{-0.011050in}}{\pgfqpoint{-0.037276in}{-0.021649in}}{\pgfqpoint{-0.029463in}{-0.029463in}}%
\pgfpathcurveto{\pgfqpoint{-0.021649in}{-0.037276in}}{\pgfqpoint{-0.011050in}{-0.041667in}}{\pgfqpoint{0.000000in}{-0.041667in}}%
\pgfpathlineto{\pgfqpoint{0.000000in}{-0.041667in}}%
\pgfpathclose%
\pgfusepath{stroke,fill}%
}%
\begin{pgfscope}%
\pgfsys@transformshift{0.523783in}{0.653486in}%
\pgfsys@useobject{currentmarker}{}%
\end{pgfscope}%
\begin{pgfscope}%
\pgfsys@transformshift{0.735147in}{0.653486in}%
\pgfsys@useobject{currentmarker}{}%
\end{pgfscope}%
\begin{pgfscope}%
\pgfsys@transformshift{0.946511in}{0.653486in}%
\pgfsys@useobject{currentmarker}{}%
\end{pgfscope}%
\begin{pgfscope}%
\pgfsys@transformshift{1.157874in}{0.653486in}%
\pgfsys@useobject{currentmarker}{}%
\end{pgfscope}%
\begin{pgfscope}%
\pgfsys@transformshift{1.369238in}{0.653486in}%
\pgfsys@useobject{currentmarker}{}%
\end{pgfscope}%
\begin{pgfscope}%
\pgfsys@transformshift{1.580602in}{2.753486in}%
\pgfsys@useobject{currentmarker}{}%
\end{pgfscope}%
\begin{pgfscope}%
\pgfsys@transformshift{1.791965in}{0.653486in}%
\pgfsys@useobject{currentmarker}{}%
\end{pgfscope}%
\begin{pgfscope}%
\pgfsys@transformshift{2.003329in}{0.653486in}%
\pgfsys@useobject{currentmarker}{}%
\end{pgfscope}%
\begin{pgfscope}%
\pgfsys@transformshift{2.214693in}{0.653486in}%
\pgfsys@useobject{currentmarker}{}%
\end{pgfscope}%
\begin{pgfscope}%
\pgfsys@transformshift{2.426056in}{0.653486in}%
\pgfsys@useobject{currentmarker}{}%
\end{pgfscope}%
\begin{pgfscope}%
\pgfsys@transformshift{2.637420in}{0.653486in}%
\pgfsys@useobject{currentmarker}{}%
\end{pgfscope}%
\end{pgfscope}%
\begin{pgfscope}%
\pgfpathrectangle{\pgfqpoint{0.418102in}{0.548486in}}{\pgfqpoint{2.325000in}{2.310000in}}%
\pgfusepath{clip}%
\pgfsetrectcap%
\pgfsetroundjoin%
\pgfsetlinewidth{1.505625pt}%
\definecolor{currentstroke}{rgb}{1.000000,0.000000,0.000000}%
\pgfsetstrokecolor{currentstroke}%
\pgfsetdash{}{0pt}%
\pgfpathmoveto{\pgfqpoint{0.523783in}{0.653486in}}%
\pgfpathlineto{\pgfqpoint{2.637420in}{0.653486in}}%
\pgfusepath{stroke}%
\end{pgfscope}%
\end{pgfpicture}%
\makeatother%
\endgroup%

        \label{fig:varImp}
    }
    \caption{Impuls funktion.}
    \label{fig:deltaVersions}
\end{figure}
\todo{Possibly change x achsis in figure \ref{fig:deltaVersions} to $n$.}



\section{Heaviside-Funktion, $u(t)$}

Engl. 'unit step' auch oft Heaviside-Funktion genannt. Typischerweise $u(t)$ aber in manchen Kontexten (vor allem Elektrotechnik, Steuerungstechnik) wird $u(t)$ oft schon benutzt um den Input in ein System zu benennen (statt $x(t)$). $u(t)$ ist das Integral des Impulses und damit eine Funktion die $0$ ist, bei $t=0$ bzw. $n=0$ auf $1$ springt und auf diesem Wert verweilt. Siehe Abb \ref{fig:heavyside}. Wird oft Verwendet um Kausale Abläufe zu beschreiben bzw. schlicht eine gegebene Funktion erst bei 0 starten zu lassen, zb eine Exponentialfunktion, siehe Abb \ref{fig:unitstepUsed} .
% \todo{plot}



\begin{figure}[H]
    \centering
    \subfigure[Einheitsschrittfunktion]{
        %% Creator: Matplotlib, PGF backend
%%
%% To include the figure in your LaTeX document, write
%%   \input{<filename>.pgf}
%%
%% Make sure the required packages are loaded in your preamble
%%   \usepackage{pgf}
%%
%% Also ensure that all the required font packages are loaded; for instance,
%% the lmodern package is sometimes necessary when using math font.
%%   \usepackage{lmodern}
%%
%% Figures using additional raster images can only be included by \input if
%% they are in the same directory as the main LaTeX file. For loading figures
%% from other directories you can use the `import` package
%%   \usepackage{import}
%%
%% and then include the figures with
%%   \import{<path to file>}{<filename>.pgf}
%%
%% Matplotlib used the following preamble
%%   \def\mathdefault#1{#1}
%%   \everymath=\expandafter{\the\everymath\displaystyle}
%%   
%%   \usepackage{fontspec}
%%   \setmainfont{VeraSe.ttf}[Path=\detokenize{/usr/share/fonts/TTF/}]
%%   \setsansfont{DejaVuSans.ttf}[Path=\detokenize{/home/pl/miniconda3/lib/python3.12/site-packages/matplotlib/mpl-data/fonts/ttf/}]
%%   \setmonofont{DejaVuSansMono.ttf}[Path=\detokenize{/home/pl/miniconda3/lib/python3.12/site-packages/matplotlib/mpl-data/fonts/ttf/}]
%%   \makeatletter\@ifpackageloaded{underscore}{}{\usepackage[strings]{underscore}}\makeatother
%%
\begingroup%
\makeatletter%
\begin{pgfpicture}%
\pgfpathrectangle{\pgfpointorigin}{\pgfqpoint{3.070082in}{2.958486in}}%
\pgfusepath{use as bounding box, clip}%
\begin{pgfscope}%
\pgfsetbuttcap%
\pgfsetmiterjoin%
\definecolor{currentfill}{rgb}{1.000000,1.000000,1.000000}%
\pgfsetfillcolor{currentfill}%
\pgfsetlinewidth{0.000000pt}%
\definecolor{currentstroke}{rgb}{1.000000,1.000000,1.000000}%
\pgfsetstrokecolor{currentstroke}%
\pgfsetdash{}{0pt}%
\pgfpathmoveto{\pgfqpoint{0.000000in}{0.000000in}}%
\pgfpathlineto{\pgfqpoint{3.070082in}{0.000000in}}%
\pgfpathlineto{\pgfqpoint{3.070082in}{2.958486in}}%
\pgfpathlineto{\pgfqpoint{0.000000in}{2.958486in}}%
\pgfpathlineto{\pgfqpoint{0.000000in}{0.000000in}}%
\pgfpathclose%
\pgfusepath{fill}%
\end{pgfscope}%
\begin{pgfscope}%
\pgfsetbuttcap%
\pgfsetmiterjoin%
\definecolor{currentfill}{rgb}{1.000000,1.000000,1.000000}%
\pgfsetfillcolor{currentfill}%
\pgfsetlinewidth{0.000000pt}%
\definecolor{currentstroke}{rgb}{0.000000,0.000000,0.000000}%
\pgfsetstrokecolor{currentstroke}%
\pgfsetstrokeopacity{0.000000}%
\pgfsetdash{}{0pt}%
\pgfpathmoveto{\pgfqpoint{0.640324in}{0.548486in}}%
\pgfpathlineto{\pgfqpoint{2.965324in}{0.548486in}}%
\pgfpathlineto{\pgfqpoint{2.965324in}{2.858486in}}%
\pgfpathlineto{\pgfqpoint{0.640324in}{2.858486in}}%
\pgfpathlineto{\pgfqpoint{0.640324in}{0.548486in}}%
\pgfpathclose%
\pgfusepath{fill}%
\end{pgfscope}%
\begin{pgfscope}%
\pgfsetbuttcap%
\pgfsetroundjoin%
\definecolor{currentfill}{rgb}{0.000000,0.000000,0.000000}%
\pgfsetfillcolor{currentfill}%
\pgfsetlinewidth{0.803000pt}%
\definecolor{currentstroke}{rgb}{0.000000,0.000000,0.000000}%
\pgfsetstrokecolor{currentstroke}%
\pgfsetdash{}{0pt}%
\pgfsys@defobject{currentmarker}{\pgfqpoint{0.000000in}{-0.048611in}}{\pgfqpoint{0.000000in}{0.000000in}}{%
\pgfpathmoveto{\pgfqpoint{0.000000in}{0.000000in}}%
\pgfpathlineto{\pgfqpoint{0.000000in}{-0.048611in}}%
\pgfusepath{stroke,fill}%
}%
\begin{pgfscope}%
\pgfsys@transformshift{0.746006in}{0.548486in}%
\pgfsys@useobject{currentmarker}{}%
\end{pgfscope}%
\end{pgfscope}%
\begin{pgfscope}%
\definecolor{textcolor}{rgb}{0.000000,0.000000,0.000000}%
\pgfsetstrokecolor{textcolor}%
\pgfsetfillcolor{textcolor}%
\pgftext[x=0.746006in,y=0.451264in,,top]{\color{textcolor}{\rmfamily\fontsize{10.000000}{12.000000}\selectfont\catcode`\^=\active\def^{\ifmmode\sp\else\^{}\fi}\catcode`\%=\active\def%{\%}\ensuremath{-}1.0}}%
\end{pgfscope}%
\begin{pgfscope}%
\pgfsetbuttcap%
\pgfsetroundjoin%
\definecolor{currentfill}{rgb}{0.000000,0.000000,0.000000}%
\pgfsetfillcolor{currentfill}%
\pgfsetlinewidth{0.803000pt}%
\definecolor{currentstroke}{rgb}{0.000000,0.000000,0.000000}%
\pgfsetstrokecolor{currentstroke}%
\pgfsetdash{}{0pt}%
\pgfsys@defobject{currentmarker}{\pgfqpoint{0.000000in}{-0.048611in}}{\pgfqpoint{0.000000in}{0.000000in}}{%
\pgfpathmoveto{\pgfqpoint{0.000000in}{0.000000in}}%
\pgfpathlineto{\pgfqpoint{0.000000in}{-0.048611in}}%
\pgfusepath{stroke,fill}%
}%
\begin{pgfscope}%
\pgfsys@transformshift{1.274415in}{0.548486in}%
\pgfsys@useobject{currentmarker}{}%
\end{pgfscope}%
\end{pgfscope}%
\begin{pgfscope}%
\definecolor{textcolor}{rgb}{0.000000,0.000000,0.000000}%
\pgfsetstrokecolor{textcolor}%
\pgfsetfillcolor{textcolor}%
\pgftext[x=1.274415in,y=0.451264in,,top]{\color{textcolor}{\rmfamily\fontsize{10.000000}{12.000000}\selectfont\catcode`\^=\active\def^{\ifmmode\sp\else\^{}\fi}\catcode`\%=\active\def%{\%}\ensuremath{-}0.5}}%
\end{pgfscope}%
\begin{pgfscope}%
\pgfsetbuttcap%
\pgfsetroundjoin%
\definecolor{currentfill}{rgb}{0.000000,0.000000,0.000000}%
\pgfsetfillcolor{currentfill}%
\pgfsetlinewidth{0.803000pt}%
\definecolor{currentstroke}{rgb}{0.000000,0.000000,0.000000}%
\pgfsetstrokecolor{currentstroke}%
\pgfsetdash{}{0pt}%
\pgfsys@defobject{currentmarker}{\pgfqpoint{0.000000in}{-0.048611in}}{\pgfqpoint{0.000000in}{0.000000in}}{%
\pgfpathmoveto{\pgfqpoint{0.000000in}{0.000000in}}%
\pgfpathlineto{\pgfqpoint{0.000000in}{-0.048611in}}%
\pgfusepath{stroke,fill}%
}%
\begin{pgfscope}%
\pgfsys@transformshift{1.802824in}{0.548486in}%
\pgfsys@useobject{currentmarker}{}%
\end{pgfscope}%
\end{pgfscope}%
\begin{pgfscope}%
\definecolor{textcolor}{rgb}{0.000000,0.000000,0.000000}%
\pgfsetstrokecolor{textcolor}%
\pgfsetfillcolor{textcolor}%
\pgftext[x=1.802824in,y=0.451264in,,top]{\color{textcolor}{\rmfamily\fontsize{10.000000}{12.000000}\selectfont\catcode`\^=\active\def^{\ifmmode\sp\else\^{}\fi}\catcode`\%=\active\def%{\%}0.0}}%
\end{pgfscope}%
\begin{pgfscope}%
\pgfsetbuttcap%
\pgfsetroundjoin%
\definecolor{currentfill}{rgb}{0.000000,0.000000,0.000000}%
\pgfsetfillcolor{currentfill}%
\pgfsetlinewidth{0.803000pt}%
\definecolor{currentstroke}{rgb}{0.000000,0.000000,0.000000}%
\pgfsetstrokecolor{currentstroke}%
\pgfsetdash{}{0pt}%
\pgfsys@defobject{currentmarker}{\pgfqpoint{0.000000in}{-0.048611in}}{\pgfqpoint{0.000000in}{0.000000in}}{%
\pgfpathmoveto{\pgfqpoint{0.000000in}{0.000000in}}%
\pgfpathlineto{\pgfqpoint{0.000000in}{-0.048611in}}%
\pgfusepath{stroke,fill}%
}%
\begin{pgfscope}%
\pgfsys@transformshift{2.331233in}{0.548486in}%
\pgfsys@useobject{currentmarker}{}%
\end{pgfscope}%
\end{pgfscope}%
\begin{pgfscope}%
\definecolor{textcolor}{rgb}{0.000000,0.000000,0.000000}%
\pgfsetstrokecolor{textcolor}%
\pgfsetfillcolor{textcolor}%
\pgftext[x=2.331233in,y=0.451264in,,top]{\color{textcolor}{\rmfamily\fontsize{10.000000}{12.000000}\selectfont\catcode`\^=\active\def^{\ifmmode\sp\else\^{}\fi}\catcode`\%=\active\def%{\%}0.5}}%
\end{pgfscope}%
\begin{pgfscope}%
\pgfsetbuttcap%
\pgfsetroundjoin%
\definecolor{currentfill}{rgb}{0.000000,0.000000,0.000000}%
\pgfsetfillcolor{currentfill}%
\pgfsetlinewidth{0.803000pt}%
\definecolor{currentstroke}{rgb}{0.000000,0.000000,0.000000}%
\pgfsetstrokecolor{currentstroke}%
\pgfsetdash{}{0pt}%
\pgfsys@defobject{currentmarker}{\pgfqpoint{0.000000in}{-0.048611in}}{\pgfqpoint{0.000000in}{0.000000in}}{%
\pgfpathmoveto{\pgfqpoint{0.000000in}{0.000000in}}%
\pgfpathlineto{\pgfqpoint{0.000000in}{-0.048611in}}%
\pgfusepath{stroke,fill}%
}%
\begin{pgfscope}%
\pgfsys@transformshift{2.859642in}{0.548486in}%
\pgfsys@useobject{currentmarker}{}%
\end{pgfscope}%
\end{pgfscope}%
\begin{pgfscope}%
\definecolor{textcolor}{rgb}{0.000000,0.000000,0.000000}%
\pgfsetstrokecolor{textcolor}%
\pgfsetfillcolor{textcolor}%
\pgftext[x=2.859642in,y=0.451264in,,top]{\color{textcolor}{\rmfamily\fontsize{10.000000}{12.000000}\selectfont\catcode`\^=\active\def^{\ifmmode\sp\else\^{}\fi}\catcode`\%=\active\def%{\%}1.0}}%
\end{pgfscope}%
\begin{pgfscope}%
\definecolor{textcolor}{rgb}{0.000000,0.000000,0.000000}%
\pgfsetstrokecolor{textcolor}%
\pgfsetfillcolor{textcolor}%
\pgftext[x=1.802824in,y=0.261295in,,top]{\color{textcolor}{\rmfamily\fontsize{12.000000}{14.400000}\selectfont\catcode`\^=\active\def^{\ifmmode\sp\else\^{}\fi}\catcode`\%=\active\def%{\%}$t$}}%
\end{pgfscope}%
\begin{pgfscope}%
\pgfsetbuttcap%
\pgfsetroundjoin%
\definecolor{currentfill}{rgb}{0.000000,0.000000,0.000000}%
\pgfsetfillcolor{currentfill}%
\pgfsetlinewidth{0.803000pt}%
\definecolor{currentstroke}{rgb}{0.000000,0.000000,0.000000}%
\pgfsetstrokecolor{currentstroke}%
\pgfsetdash{}{0pt}%
\pgfsys@defobject{currentmarker}{\pgfqpoint{-0.048611in}{0.000000in}}{\pgfqpoint{-0.000000in}{0.000000in}}{%
\pgfpathmoveto{\pgfqpoint{-0.000000in}{0.000000in}}%
\pgfpathlineto{\pgfqpoint{-0.048611in}{0.000000in}}%
\pgfusepath{stroke,fill}%
}%
\begin{pgfscope}%
\pgfsys@transformshift{0.640324in}{0.653486in}%
\pgfsys@useobject{currentmarker}{}%
\end{pgfscope}%
\end{pgfscope}%
\begin{pgfscope}%
\definecolor{textcolor}{rgb}{0.000000,0.000000,0.000000}%
\pgfsetstrokecolor{textcolor}%
\pgfsetfillcolor{textcolor}%
\pgftext[x=0.322222in, y=0.600724in, left, base]{\color{textcolor}{\rmfamily\fontsize{10.000000}{12.000000}\selectfont\catcode`\^=\active\def^{\ifmmode\sp\else\^{}\fi}\catcode`\%=\active\def%{\%}0.0}}%
\end{pgfscope}%
\begin{pgfscope}%
\pgfsetbuttcap%
\pgfsetroundjoin%
\definecolor{currentfill}{rgb}{0.000000,0.000000,0.000000}%
\pgfsetfillcolor{currentfill}%
\pgfsetlinewidth{0.803000pt}%
\definecolor{currentstroke}{rgb}{0.000000,0.000000,0.000000}%
\pgfsetstrokecolor{currentstroke}%
\pgfsetdash{}{0pt}%
\pgfsys@defobject{currentmarker}{\pgfqpoint{-0.048611in}{0.000000in}}{\pgfqpoint{-0.000000in}{0.000000in}}{%
\pgfpathmoveto{\pgfqpoint{-0.000000in}{0.000000in}}%
\pgfpathlineto{\pgfqpoint{-0.048611in}{0.000000in}}%
\pgfusepath{stroke,fill}%
}%
\begin{pgfscope}%
\pgfsys@transformshift{0.640324in}{1.073486in}%
\pgfsys@useobject{currentmarker}{}%
\end{pgfscope}%
\end{pgfscope}%
\begin{pgfscope}%
\definecolor{textcolor}{rgb}{0.000000,0.000000,0.000000}%
\pgfsetstrokecolor{textcolor}%
\pgfsetfillcolor{textcolor}%
\pgftext[x=0.322222in, y=1.020724in, left, base]{\color{textcolor}{\rmfamily\fontsize{10.000000}{12.000000}\selectfont\catcode`\^=\active\def^{\ifmmode\sp\else\^{}\fi}\catcode`\%=\active\def%{\%}0.2}}%
\end{pgfscope}%
\begin{pgfscope}%
\pgfsetbuttcap%
\pgfsetroundjoin%
\definecolor{currentfill}{rgb}{0.000000,0.000000,0.000000}%
\pgfsetfillcolor{currentfill}%
\pgfsetlinewidth{0.803000pt}%
\definecolor{currentstroke}{rgb}{0.000000,0.000000,0.000000}%
\pgfsetstrokecolor{currentstroke}%
\pgfsetdash{}{0pt}%
\pgfsys@defobject{currentmarker}{\pgfqpoint{-0.048611in}{0.000000in}}{\pgfqpoint{-0.000000in}{0.000000in}}{%
\pgfpathmoveto{\pgfqpoint{-0.000000in}{0.000000in}}%
\pgfpathlineto{\pgfqpoint{-0.048611in}{0.000000in}}%
\pgfusepath{stroke,fill}%
}%
\begin{pgfscope}%
\pgfsys@transformshift{0.640324in}{1.493486in}%
\pgfsys@useobject{currentmarker}{}%
\end{pgfscope}%
\end{pgfscope}%
\begin{pgfscope}%
\definecolor{textcolor}{rgb}{0.000000,0.000000,0.000000}%
\pgfsetstrokecolor{textcolor}%
\pgfsetfillcolor{textcolor}%
\pgftext[x=0.322222in, y=1.440724in, left, base]{\color{textcolor}{\rmfamily\fontsize{10.000000}{12.000000}\selectfont\catcode`\^=\active\def^{\ifmmode\sp\else\^{}\fi}\catcode`\%=\active\def%{\%}0.4}}%
\end{pgfscope}%
\begin{pgfscope}%
\pgfsetbuttcap%
\pgfsetroundjoin%
\definecolor{currentfill}{rgb}{0.000000,0.000000,0.000000}%
\pgfsetfillcolor{currentfill}%
\pgfsetlinewidth{0.803000pt}%
\definecolor{currentstroke}{rgb}{0.000000,0.000000,0.000000}%
\pgfsetstrokecolor{currentstroke}%
\pgfsetdash{}{0pt}%
\pgfsys@defobject{currentmarker}{\pgfqpoint{-0.048611in}{0.000000in}}{\pgfqpoint{-0.000000in}{0.000000in}}{%
\pgfpathmoveto{\pgfqpoint{-0.000000in}{0.000000in}}%
\pgfpathlineto{\pgfqpoint{-0.048611in}{0.000000in}}%
\pgfusepath{stroke,fill}%
}%
\begin{pgfscope}%
\pgfsys@transformshift{0.640324in}{1.913486in}%
\pgfsys@useobject{currentmarker}{}%
\end{pgfscope}%
\end{pgfscope}%
\begin{pgfscope}%
\definecolor{textcolor}{rgb}{0.000000,0.000000,0.000000}%
\pgfsetstrokecolor{textcolor}%
\pgfsetfillcolor{textcolor}%
\pgftext[x=0.322222in, y=1.860724in, left, base]{\color{textcolor}{\rmfamily\fontsize{10.000000}{12.000000}\selectfont\catcode`\^=\active\def^{\ifmmode\sp\else\^{}\fi}\catcode`\%=\active\def%{\%}0.6}}%
\end{pgfscope}%
\begin{pgfscope}%
\pgfsetbuttcap%
\pgfsetroundjoin%
\definecolor{currentfill}{rgb}{0.000000,0.000000,0.000000}%
\pgfsetfillcolor{currentfill}%
\pgfsetlinewidth{0.803000pt}%
\definecolor{currentstroke}{rgb}{0.000000,0.000000,0.000000}%
\pgfsetstrokecolor{currentstroke}%
\pgfsetdash{}{0pt}%
\pgfsys@defobject{currentmarker}{\pgfqpoint{-0.048611in}{0.000000in}}{\pgfqpoint{-0.000000in}{0.000000in}}{%
\pgfpathmoveto{\pgfqpoint{-0.000000in}{0.000000in}}%
\pgfpathlineto{\pgfqpoint{-0.048611in}{0.000000in}}%
\pgfusepath{stroke,fill}%
}%
\begin{pgfscope}%
\pgfsys@transformshift{0.640324in}{2.333486in}%
\pgfsys@useobject{currentmarker}{}%
\end{pgfscope}%
\end{pgfscope}%
\begin{pgfscope}%
\definecolor{textcolor}{rgb}{0.000000,0.000000,0.000000}%
\pgfsetstrokecolor{textcolor}%
\pgfsetfillcolor{textcolor}%
\pgftext[x=0.322222in, y=2.280724in, left, base]{\color{textcolor}{\rmfamily\fontsize{10.000000}{12.000000}\selectfont\catcode`\^=\active\def^{\ifmmode\sp\else\^{}\fi}\catcode`\%=\active\def%{\%}0.8}}%
\end{pgfscope}%
\begin{pgfscope}%
\pgfsetbuttcap%
\pgfsetroundjoin%
\definecolor{currentfill}{rgb}{0.000000,0.000000,0.000000}%
\pgfsetfillcolor{currentfill}%
\pgfsetlinewidth{0.803000pt}%
\definecolor{currentstroke}{rgb}{0.000000,0.000000,0.000000}%
\pgfsetstrokecolor{currentstroke}%
\pgfsetdash{}{0pt}%
\pgfsys@defobject{currentmarker}{\pgfqpoint{-0.048611in}{0.000000in}}{\pgfqpoint{-0.000000in}{0.000000in}}{%
\pgfpathmoveto{\pgfqpoint{-0.000000in}{0.000000in}}%
\pgfpathlineto{\pgfqpoint{-0.048611in}{0.000000in}}%
\pgfusepath{stroke,fill}%
}%
\begin{pgfscope}%
\pgfsys@transformshift{0.640324in}{2.753486in}%
\pgfsys@useobject{currentmarker}{}%
\end{pgfscope}%
\end{pgfscope}%
\begin{pgfscope}%
\definecolor{textcolor}{rgb}{0.000000,0.000000,0.000000}%
\pgfsetstrokecolor{textcolor}%
\pgfsetfillcolor{textcolor}%
\pgftext[x=0.322222in, y=2.700724in, left, base]{\color{textcolor}{\rmfamily\fontsize{10.000000}{12.000000}\selectfont\catcode`\^=\active\def^{\ifmmode\sp\else\^{}\fi}\catcode`\%=\active\def%{\%}1.0}}%
\end{pgfscope}%
\begin{pgfscope}%
\definecolor{textcolor}{rgb}{0.000000,0.000000,0.000000}%
\pgfsetstrokecolor{textcolor}%
\pgfsetfillcolor{textcolor}%
\pgftext[x=0.266667in,y=1.703486in,,bottom,rotate=90.000000]{\color{textcolor}{\rmfamily\fontsize{12.000000}{14.400000}\selectfont\catcode`\^=\active\def^{\ifmmode\sp\else\^{}\fi}\catcode`\%=\active\def%{\%}$u(t)$}}%
\end{pgfscope}%
\begin{pgfscope}%
\pgfpathrectangle{\pgfqpoint{0.640324in}{0.548486in}}{\pgfqpoint{2.325000in}{2.310000in}}%
\pgfusepath{clip}%
\pgfsetrectcap%
\pgfsetroundjoin%
\pgfsetlinewidth{1.505625pt}%
\definecolor{currentstroke}{rgb}{0.000000,0.000000,0.000000}%
\pgfsetstrokecolor{currentstroke}%
\pgfsetdash{}{0pt}%
\pgfpathmoveto{\pgfqpoint{0.746006in}{0.653486in}}%
\pgfpathlineto{\pgfqpoint{1.801766in}{0.653486in}}%
\pgfpathlineto{\pgfqpoint{1.803882in}{2.753486in}}%
\pgfpathlineto{\pgfqpoint{2.859642in}{2.753486in}}%
\pgfpathlineto{\pgfqpoint{2.859642in}{2.753486in}}%
\pgfusepath{stroke}%
\end{pgfscope}%
\end{pgfpicture}%
\makeatother%
\endgroup%

        \label{fig:heavyside}
    }
    \hfill
    \subfigure[Verwendung der Einheitsschrittfunktion zur Konstruktion eines Signals das bei $0$ beginnt und abklingt.]{
        %% Creator: Matplotlib, PGF backend
%%
%% To include the figure in your LaTeX document, write
%%   \input{<filename>.pgf}
%%
%% Make sure the required packages are loaded in your preamble
%%   \usepackage{pgf}
%%
%% Also ensure that all the required font packages are loaded; for instance,
%% the lmodern package is sometimes necessary when using math font.
%%   \usepackage{lmodern}
%%
%% Figures using additional raster images can only be included by \input if
%% they are in the same directory as the main LaTeX file. For loading figures
%% from other directories you can use the `import` package
%%   \usepackage{import}
%%
%% and then include the figures with
%%   \import{<path to file>}{<filename>.pgf}
%%
%% Matplotlib used the following preamble
%%   \def\mathdefault#1{#1}
%%   \everymath=\expandafter{\the\everymath\displaystyle}
%%   
%%   \usepackage{fontspec}
%%   \setmainfont{VeraSe.ttf}[Path=\detokenize{/usr/share/fonts/TTF/}]
%%   \setsansfont{DejaVuSans.ttf}[Path=\detokenize{/home/pl/miniconda3/lib/python3.12/site-packages/matplotlib/mpl-data/fonts/ttf/}]
%%   \setmonofont{DejaVuSansMono.ttf}[Path=\detokenize{/home/pl/miniconda3/lib/python3.12/site-packages/matplotlib/mpl-data/fonts/ttf/}]
%%   \makeatletter\@ifpackageloaded{underscore}{}{\usepackage[strings]{underscore}}\makeatother
%%
\begingroup%
\makeatletter%
\begin{pgfpicture}%
\pgfpathrectangle{\pgfpointorigin}{\pgfqpoint{2.847860in}{3.011248in}}%
\pgfusepath{use as bounding box, clip}%
\begin{pgfscope}%
\pgfsetbuttcap%
\pgfsetmiterjoin%
\definecolor{currentfill}{rgb}{1.000000,1.000000,1.000000}%
\pgfsetfillcolor{currentfill}%
\pgfsetlinewidth{0.000000pt}%
\definecolor{currentstroke}{rgb}{1.000000,1.000000,1.000000}%
\pgfsetstrokecolor{currentstroke}%
\pgfsetdash{}{0pt}%
\pgfpathmoveto{\pgfqpoint{0.000000in}{0.000000in}}%
\pgfpathlineto{\pgfqpoint{2.847860in}{0.000000in}}%
\pgfpathlineto{\pgfqpoint{2.847860in}{3.011248in}}%
\pgfpathlineto{\pgfqpoint{0.000000in}{3.011248in}}%
\pgfpathlineto{\pgfqpoint{0.000000in}{0.000000in}}%
\pgfpathclose%
\pgfusepath{fill}%
\end{pgfscope}%
\begin{pgfscope}%
\pgfsetbuttcap%
\pgfsetmiterjoin%
\definecolor{currentfill}{rgb}{1.000000,1.000000,1.000000}%
\pgfsetfillcolor{currentfill}%
\pgfsetlinewidth{0.000000pt}%
\definecolor{currentstroke}{rgb}{0.000000,0.000000,0.000000}%
\pgfsetstrokecolor{currentstroke}%
\pgfsetstrokeopacity{0.000000}%
\pgfsetdash{}{0pt}%
\pgfpathmoveto{\pgfqpoint{0.418102in}{0.548486in}}%
\pgfpathlineto{\pgfqpoint{2.743102in}{0.548486in}}%
\pgfpathlineto{\pgfqpoint{2.743102in}{2.858486in}}%
\pgfpathlineto{\pgfqpoint{0.418102in}{2.858486in}}%
\pgfpathlineto{\pgfqpoint{0.418102in}{0.548486in}}%
\pgfpathclose%
\pgfusepath{fill}%
\end{pgfscope}%
\begin{pgfscope}%
\pgfsetbuttcap%
\pgfsetroundjoin%
\definecolor{currentfill}{rgb}{0.000000,0.000000,0.000000}%
\pgfsetfillcolor{currentfill}%
\pgfsetlinewidth{0.803000pt}%
\definecolor{currentstroke}{rgb}{0.000000,0.000000,0.000000}%
\pgfsetstrokecolor{currentstroke}%
\pgfsetdash{}{0pt}%
\pgfsys@defobject{currentmarker}{\pgfqpoint{0.000000in}{-0.048611in}}{\pgfqpoint{0.000000in}{0.000000in}}{%
\pgfpathmoveto{\pgfqpoint{0.000000in}{0.000000in}}%
\pgfpathlineto{\pgfqpoint{0.000000in}{-0.048611in}}%
\pgfusepath{stroke,fill}%
}%
\begin{pgfscope}%
\pgfsys@transformshift{0.523783in}{0.548486in}%
\pgfsys@useobject{currentmarker}{}%
\end{pgfscope}%
\end{pgfscope}%
\begin{pgfscope}%
\definecolor{textcolor}{rgb}{0.000000,0.000000,0.000000}%
\pgfsetstrokecolor{textcolor}%
\pgfsetfillcolor{textcolor}%
\pgftext[x=0.523783in,y=0.451264in,,top]{\color{textcolor}{\rmfamily\fontsize{10.000000}{12.000000}\selectfont\catcode`\^=\active\def^{\ifmmode\sp\else\^{}\fi}\catcode`\%=\active\def%{\%}\ensuremath{-}1.0}}%
\end{pgfscope}%
\begin{pgfscope}%
\pgfsetbuttcap%
\pgfsetroundjoin%
\definecolor{currentfill}{rgb}{0.000000,0.000000,0.000000}%
\pgfsetfillcolor{currentfill}%
\pgfsetlinewidth{0.803000pt}%
\definecolor{currentstroke}{rgb}{0.000000,0.000000,0.000000}%
\pgfsetstrokecolor{currentstroke}%
\pgfsetdash{}{0pt}%
\pgfsys@defobject{currentmarker}{\pgfqpoint{0.000000in}{-0.048611in}}{\pgfqpoint{0.000000in}{0.000000in}}{%
\pgfpathmoveto{\pgfqpoint{0.000000in}{0.000000in}}%
\pgfpathlineto{\pgfqpoint{0.000000in}{-0.048611in}}%
\pgfusepath{stroke,fill}%
}%
\begin{pgfscope}%
\pgfsys@transformshift{1.052193in}{0.548486in}%
\pgfsys@useobject{currentmarker}{}%
\end{pgfscope}%
\end{pgfscope}%
\begin{pgfscope}%
\definecolor{textcolor}{rgb}{0.000000,0.000000,0.000000}%
\pgfsetstrokecolor{textcolor}%
\pgfsetfillcolor{textcolor}%
\pgftext[x=1.052193in,y=0.451264in,,top]{\color{textcolor}{\rmfamily\fontsize{10.000000}{12.000000}\selectfont\catcode`\^=\active\def^{\ifmmode\sp\else\^{}\fi}\catcode`\%=\active\def%{\%}\ensuremath{-}0.5}}%
\end{pgfscope}%
\begin{pgfscope}%
\pgfsetbuttcap%
\pgfsetroundjoin%
\definecolor{currentfill}{rgb}{0.000000,0.000000,0.000000}%
\pgfsetfillcolor{currentfill}%
\pgfsetlinewidth{0.803000pt}%
\definecolor{currentstroke}{rgb}{0.000000,0.000000,0.000000}%
\pgfsetstrokecolor{currentstroke}%
\pgfsetdash{}{0pt}%
\pgfsys@defobject{currentmarker}{\pgfqpoint{0.000000in}{-0.048611in}}{\pgfqpoint{0.000000in}{0.000000in}}{%
\pgfpathmoveto{\pgfqpoint{0.000000in}{0.000000in}}%
\pgfpathlineto{\pgfqpoint{0.000000in}{-0.048611in}}%
\pgfusepath{stroke,fill}%
}%
\begin{pgfscope}%
\pgfsys@transformshift{1.580602in}{0.548486in}%
\pgfsys@useobject{currentmarker}{}%
\end{pgfscope}%
\end{pgfscope}%
\begin{pgfscope}%
\definecolor{textcolor}{rgb}{0.000000,0.000000,0.000000}%
\pgfsetstrokecolor{textcolor}%
\pgfsetfillcolor{textcolor}%
\pgftext[x=1.580602in,y=0.451264in,,top]{\color{textcolor}{\rmfamily\fontsize{10.000000}{12.000000}\selectfont\catcode`\^=\active\def^{\ifmmode\sp\else\^{}\fi}\catcode`\%=\active\def%{\%}0.0}}%
\end{pgfscope}%
\begin{pgfscope}%
\pgfsetbuttcap%
\pgfsetroundjoin%
\definecolor{currentfill}{rgb}{0.000000,0.000000,0.000000}%
\pgfsetfillcolor{currentfill}%
\pgfsetlinewidth{0.803000pt}%
\definecolor{currentstroke}{rgb}{0.000000,0.000000,0.000000}%
\pgfsetstrokecolor{currentstroke}%
\pgfsetdash{}{0pt}%
\pgfsys@defobject{currentmarker}{\pgfqpoint{0.000000in}{-0.048611in}}{\pgfqpoint{0.000000in}{0.000000in}}{%
\pgfpathmoveto{\pgfqpoint{0.000000in}{0.000000in}}%
\pgfpathlineto{\pgfqpoint{0.000000in}{-0.048611in}}%
\pgfusepath{stroke,fill}%
}%
\begin{pgfscope}%
\pgfsys@transformshift{2.109011in}{0.548486in}%
\pgfsys@useobject{currentmarker}{}%
\end{pgfscope}%
\end{pgfscope}%
\begin{pgfscope}%
\definecolor{textcolor}{rgb}{0.000000,0.000000,0.000000}%
\pgfsetstrokecolor{textcolor}%
\pgfsetfillcolor{textcolor}%
\pgftext[x=2.109011in,y=0.451264in,,top]{\color{textcolor}{\rmfamily\fontsize{10.000000}{12.000000}\selectfont\catcode`\^=\active\def^{\ifmmode\sp\else\^{}\fi}\catcode`\%=\active\def%{\%}0.5}}%
\end{pgfscope}%
\begin{pgfscope}%
\pgfsetbuttcap%
\pgfsetroundjoin%
\definecolor{currentfill}{rgb}{0.000000,0.000000,0.000000}%
\pgfsetfillcolor{currentfill}%
\pgfsetlinewidth{0.803000pt}%
\definecolor{currentstroke}{rgb}{0.000000,0.000000,0.000000}%
\pgfsetstrokecolor{currentstroke}%
\pgfsetdash{}{0pt}%
\pgfsys@defobject{currentmarker}{\pgfqpoint{0.000000in}{-0.048611in}}{\pgfqpoint{0.000000in}{0.000000in}}{%
\pgfpathmoveto{\pgfqpoint{0.000000in}{0.000000in}}%
\pgfpathlineto{\pgfqpoint{0.000000in}{-0.048611in}}%
\pgfusepath{stroke,fill}%
}%
\begin{pgfscope}%
\pgfsys@transformshift{2.637420in}{0.548486in}%
\pgfsys@useobject{currentmarker}{}%
\end{pgfscope}%
\end{pgfscope}%
\begin{pgfscope}%
\definecolor{textcolor}{rgb}{0.000000,0.000000,0.000000}%
\pgfsetstrokecolor{textcolor}%
\pgfsetfillcolor{textcolor}%
\pgftext[x=2.637420in,y=0.451264in,,top]{\color{textcolor}{\rmfamily\fontsize{10.000000}{12.000000}\selectfont\catcode`\^=\active\def^{\ifmmode\sp\else\^{}\fi}\catcode`\%=\active\def%{\%}1.0}}%
\end{pgfscope}%
\begin{pgfscope}%
\definecolor{textcolor}{rgb}{0.000000,0.000000,0.000000}%
\pgfsetstrokecolor{textcolor}%
\pgfsetfillcolor{textcolor}%
\pgftext[x=1.580602in,y=0.261295in,,top]{\color{textcolor}{\rmfamily\fontsize{12.000000}{14.400000}\selectfont\catcode`\^=\active\def^{\ifmmode\sp\else\^{}\fi}\catcode`\%=\active\def%{\%}$t$}}%
\end{pgfscope}%
\begin{pgfscope}%
\pgfsetbuttcap%
\pgfsetroundjoin%
\definecolor{currentfill}{rgb}{0.000000,0.000000,0.000000}%
\pgfsetfillcolor{currentfill}%
\pgfsetlinewidth{0.803000pt}%
\definecolor{currentstroke}{rgb}{0.000000,0.000000,0.000000}%
\pgfsetstrokecolor{currentstroke}%
\pgfsetdash{}{0pt}%
\pgfsys@defobject{currentmarker}{\pgfqpoint{-0.048611in}{0.000000in}}{\pgfqpoint{-0.000000in}{0.000000in}}{%
\pgfpathmoveto{\pgfqpoint{-0.000000in}{0.000000in}}%
\pgfpathlineto{\pgfqpoint{-0.048611in}{0.000000in}}%
\pgfusepath{stroke,fill}%
}%
\begin{pgfscope}%
\pgfsys@transformshift{0.418102in}{0.658486in}%
\pgfsys@useobject{currentmarker}{}%
\end{pgfscope}%
\end{pgfscope}%
\begin{pgfscope}%
\definecolor{textcolor}{rgb}{0.000000,0.000000,0.000000}%
\pgfsetstrokecolor{textcolor}%
\pgfsetfillcolor{textcolor}%
\pgftext[x=0.100000in, y=0.605724in, left, base]{\color{textcolor}{\rmfamily\fontsize{10.000000}{12.000000}\selectfont\catcode`\^=\active\def^{\ifmmode\sp\else\^{}\fi}\catcode`\%=\active\def%{\%}0.0}}%
\end{pgfscope}%
\begin{pgfscope}%
\pgfsetbuttcap%
\pgfsetroundjoin%
\definecolor{currentfill}{rgb}{0.000000,0.000000,0.000000}%
\pgfsetfillcolor{currentfill}%
\pgfsetlinewidth{0.803000pt}%
\definecolor{currentstroke}{rgb}{0.000000,0.000000,0.000000}%
\pgfsetstrokecolor{currentstroke}%
\pgfsetdash{}{0pt}%
\pgfsys@defobject{currentmarker}{\pgfqpoint{-0.048611in}{0.000000in}}{\pgfqpoint{-0.000000in}{0.000000in}}{%
\pgfpathmoveto{\pgfqpoint{-0.000000in}{0.000000in}}%
\pgfpathlineto{\pgfqpoint{-0.048611in}{0.000000in}}%
\pgfusepath{stroke,fill}%
}%
\begin{pgfscope}%
\pgfsys@transformshift{0.418102in}{1.208486in}%
\pgfsys@useobject{currentmarker}{}%
\end{pgfscope}%
\end{pgfscope}%
\begin{pgfscope}%
\definecolor{textcolor}{rgb}{0.000000,0.000000,0.000000}%
\pgfsetstrokecolor{textcolor}%
\pgfsetfillcolor{textcolor}%
\pgftext[x=0.100000in, y=1.155724in, left, base]{\color{textcolor}{\rmfamily\fontsize{10.000000}{12.000000}\selectfont\catcode`\^=\active\def^{\ifmmode\sp\else\^{}\fi}\catcode`\%=\active\def%{\%}0.5}}%
\end{pgfscope}%
\begin{pgfscope}%
\pgfsetbuttcap%
\pgfsetroundjoin%
\definecolor{currentfill}{rgb}{0.000000,0.000000,0.000000}%
\pgfsetfillcolor{currentfill}%
\pgfsetlinewidth{0.803000pt}%
\definecolor{currentstroke}{rgb}{0.000000,0.000000,0.000000}%
\pgfsetstrokecolor{currentstroke}%
\pgfsetdash{}{0pt}%
\pgfsys@defobject{currentmarker}{\pgfqpoint{-0.048611in}{0.000000in}}{\pgfqpoint{-0.000000in}{0.000000in}}{%
\pgfpathmoveto{\pgfqpoint{-0.000000in}{0.000000in}}%
\pgfpathlineto{\pgfqpoint{-0.048611in}{0.000000in}}%
\pgfusepath{stroke,fill}%
}%
\begin{pgfscope}%
\pgfsys@transformshift{0.418102in}{1.758486in}%
\pgfsys@useobject{currentmarker}{}%
\end{pgfscope}%
\end{pgfscope}%
\begin{pgfscope}%
\definecolor{textcolor}{rgb}{0.000000,0.000000,0.000000}%
\pgfsetstrokecolor{textcolor}%
\pgfsetfillcolor{textcolor}%
\pgftext[x=0.100000in, y=1.705724in, left, base]{\color{textcolor}{\rmfamily\fontsize{10.000000}{12.000000}\selectfont\catcode`\^=\active\def^{\ifmmode\sp\else\^{}\fi}\catcode`\%=\active\def%{\%}1.0}}%
\end{pgfscope}%
\begin{pgfscope}%
\pgfsetbuttcap%
\pgfsetroundjoin%
\definecolor{currentfill}{rgb}{0.000000,0.000000,0.000000}%
\pgfsetfillcolor{currentfill}%
\pgfsetlinewidth{0.803000pt}%
\definecolor{currentstroke}{rgb}{0.000000,0.000000,0.000000}%
\pgfsetstrokecolor{currentstroke}%
\pgfsetdash{}{0pt}%
\pgfsys@defobject{currentmarker}{\pgfqpoint{-0.048611in}{0.000000in}}{\pgfqpoint{-0.000000in}{0.000000in}}{%
\pgfpathmoveto{\pgfqpoint{-0.000000in}{0.000000in}}%
\pgfpathlineto{\pgfqpoint{-0.048611in}{0.000000in}}%
\pgfusepath{stroke,fill}%
}%
\begin{pgfscope}%
\pgfsys@transformshift{0.418102in}{2.308486in}%
\pgfsys@useobject{currentmarker}{}%
\end{pgfscope}%
\end{pgfscope}%
\begin{pgfscope}%
\definecolor{textcolor}{rgb}{0.000000,0.000000,0.000000}%
\pgfsetstrokecolor{textcolor}%
\pgfsetfillcolor{textcolor}%
\pgftext[x=0.100000in, y=2.255724in, left, base]{\color{textcolor}{\rmfamily\fontsize{10.000000}{12.000000}\selectfont\catcode`\^=\active\def^{\ifmmode\sp\else\^{}\fi}\catcode`\%=\active\def%{\%}1.5}}%
\end{pgfscope}%
\begin{pgfscope}%
\pgfsetbuttcap%
\pgfsetroundjoin%
\definecolor{currentfill}{rgb}{0.000000,0.000000,0.000000}%
\pgfsetfillcolor{currentfill}%
\pgfsetlinewidth{0.803000pt}%
\definecolor{currentstroke}{rgb}{0.000000,0.000000,0.000000}%
\pgfsetstrokecolor{currentstroke}%
\pgfsetdash{}{0pt}%
\pgfsys@defobject{currentmarker}{\pgfqpoint{-0.048611in}{0.000000in}}{\pgfqpoint{-0.000000in}{0.000000in}}{%
\pgfpathmoveto{\pgfqpoint{-0.000000in}{0.000000in}}%
\pgfpathlineto{\pgfqpoint{-0.048611in}{0.000000in}}%
\pgfusepath{stroke,fill}%
}%
\begin{pgfscope}%
\pgfsys@transformshift{0.418102in}{2.858486in}%
\pgfsys@useobject{currentmarker}{}%
\end{pgfscope}%
\end{pgfscope}%
\begin{pgfscope}%
\definecolor{textcolor}{rgb}{0.000000,0.000000,0.000000}%
\pgfsetstrokecolor{textcolor}%
\pgfsetfillcolor{textcolor}%
\pgftext[x=0.100000in, y=2.805724in, left, base]{\color{textcolor}{\rmfamily\fontsize{10.000000}{12.000000}\selectfont\catcode`\^=\active\def^{\ifmmode\sp\else\^{}\fi}\catcode`\%=\active\def%{\%}2.0}}%
\end{pgfscope}%
\begin{pgfscope}%
\pgfpathrectangle{\pgfqpoint{0.418102in}{0.548486in}}{\pgfqpoint{2.325000in}{2.310000in}}%
\pgfusepath{clip}%
\pgfsetrectcap%
\pgfsetroundjoin%
\pgfsetlinewidth{1.505625pt}%
\definecolor{currentstroke}{rgb}{0.000000,0.000000,0.000000}%
\pgfsetstrokecolor{currentstroke}%
\pgfsetdash{}{0pt}%
\pgfpathmoveto{\pgfqpoint{0.523783in}{0.658486in}}%
\pgfpathlineto{\pgfqpoint{1.579544in}{0.658486in}}%
\pgfpathlineto{\pgfqpoint{1.581660in}{1.758486in}}%
\pgfpathlineto{\pgfqpoint{2.637420in}{1.758486in}}%
\pgfpathlineto{\pgfqpoint{2.637420in}{1.758486in}}%
\pgfusepath{stroke}%
\end{pgfscope}%
\begin{pgfscope}%
\pgfpathrectangle{\pgfqpoint{0.418102in}{0.548486in}}{\pgfqpoint{2.325000in}{2.310000in}}%
\pgfusepath{clip}%
\pgfsetbuttcap%
\pgfsetroundjoin%
\pgfsetlinewidth{1.505625pt}%
\definecolor{currentstroke}{rgb}{0.000000,0.000000,0.000000}%
\pgfsetstrokecolor{currentstroke}%
\pgfsetdash{{5.550000pt}{2.400000pt}}{0.000000pt}%
\pgfpathmoveto{\pgfqpoint{1.334829in}{2.868486in}}%
\pgfpathlineto{\pgfqpoint{1.361621in}{2.706625in}}%
\pgfpathlineto{\pgfqpoint{1.391242in}{2.541451in}}%
\pgfpathlineto{\pgfqpoint{1.420862in}{2.389597in}}%
\pgfpathlineto{\pgfqpoint{1.450483in}{2.249990in}}%
\pgfpathlineto{\pgfqpoint{1.480103in}{2.121642in}}%
\pgfpathlineto{\pgfqpoint{1.509724in}{2.003644in}}%
\pgfpathlineto{\pgfqpoint{1.539345in}{1.895162in}}%
\pgfpathlineto{\pgfqpoint{1.568965in}{1.795429in}}%
\pgfpathlineto{\pgfqpoint{1.598586in}{1.703739in}}%
\pgfpathlineto{\pgfqpoint{1.628206in}{1.619444in}}%
\pgfpathlineto{\pgfqpoint{1.657827in}{1.541946in}}%
\pgfpathlineto{\pgfqpoint{1.687447in}{1.470699in}}%
\pgfpathlineto{\pgfqpoint{1.717068in}{1.405197in}}%
\pgfpathlineto{\pgfqpoint{1.746688in}{1.344978in}}%
\pgfpathlineto{\pgfqpoint{1.776309in}{1.289615in}}%
\pgfpathlineto{\pgfqpoint{1.805929in}{1.238717in}}%
\pgfpathlineto{\pgfqpoint{1.837666in}{1.188729in}}%
\pgfpathlineto{\pgfqpoint{1.869402in}{1.143048in}}%
\pgfpathlineto{\pgfqpoint{1.901138in}{1.101303in}}%
\pgfpathlineto{\pgfqpoint{1.932874in}{1.063153in}}%
\pgfpathlineto{\pgfqpoint{1.964611in}{1.028291in}}%
\pgfpathlineto{\pgfqpoint{1.998463in}{0.994408in}}%
\pgfpathlineto{\pgfqpoint{2.032315in}{0.963630in}}%
\pgfpathlineto{\pgfqpoint{2.068283in}{0.934012in}}%
\pgfpathlineto{\pgfqpoint{2.104250in}{0.907268in}}%
\pgfpathlineto{\pgfqpoint{2.142334in}{0.881776in}}%
\pgfpathlineto{\pgfqpoint{2.180417in}{0.858895in}}%
\pgfpathlineto{\pgfqpoint{2.220617in}{0.837283in}}%
\pgfpathlineto{\pgfqpoint{2.262932in}{0.817045in}}%
\pgfpathlineto{\pgfqpoint{2.307363in}{0.798257in}}%
\pgfpathlineto{\pgfqpoint{2.353909in}{0.780956in}}%
\pgfpathlineto{\pgfqpoint{2.404687in}{0.764516in}}%
\pgfpathlineto{\pgfqpoint{2.457581in}{0.749734in}}%
\pgfpathlineto{\pgfqpoint{2.514706in}{0.736074in}}%
\pgfpathlineto{\pgfqpoint{2.576063in}{0.723672in}}%
\pgfpathlineto{\pgfqpoint{2.637420in}{0.713252in}}%
\pgfpathlineto{\pgfqpoint{2.637420in}{0.713252in}}%
\pgfusepath{stroke}%
\end{pgfscope}%
\begin{pgfscope}%
\pgfpathrectangle{\pgfqpoint{0.418102in}{0.548486in}}{\pgfqpoint{2.325000in}{2.310000in}}%
\pgfusepath{clip}%
\pgfsetbuttcap%
\pgfsetroundjoin%
\pgfsetlinewidth{1.505625pt}%
\definecolor{currentstroke}{rgb}{1.000000,0.000000,0.000000}%
\pgfsetstrokecolor{currentstroke}%
\pgfsetdash{{5.550000pt}{2.400000pt}}{0.000000pt}%
\pgfpathmoveto{\pgfqpoint{0.523783in}{0.658486in}}%
\pgfpathlineto{\pgfqpoint{1.579544in}{0.658486in}}%
\pgfpathlineto{\pgfqpoint{1.581660in}{1.755188in}}%
\pgfpathlineto{\pgfqpoint{1.611280in}{1.666743in}}%
\pgfpathlineto{\pgfqpoint{1.640901in}{1.585431in}}%
\pgfpathlineto{\pgfqpoint{1.670521in}{1.510677in}}%
\pgfpathlineto{\pgfqpoint{1.700142in}{1.441951in}}%
\pgfpathlineto{\pgfqpoint{1.729762in}{1.378767in}}%
\pgfpathlineto{\pgfqpoint{1.759383in}{1.320680in}}%
\pgfpathlineto{\pgfqpoint{1.789003in}{1.267276in}}%
\pgfpathlineto{\pgfqpoint{1.818624in}{1.218180in}}%
\pgfpathlineto{\pgfqpoint{1.850360in}{1.169961in}}%
\pgfpathlineto{\pgfqpoint{1.882096in}{1.125897in}}%
\pgfpathlineto{\pgfqpoint{1.913833in}{1.085629in}}%
\pgfpathlineto{\pgfqpoint{1.945569in}{1.048830in}}%
\pgfpathlineto{\pgfqpoint{1.979421in}{1.013066in}}%
\pgfpathlineto{\pgfqpoint{2.013273in}{0.980578in}}%
\pgfpathlineto{\pgfqpoint{2.047125in}{0.951067in}}%
\pgfpathlineto{\pgfqpoint{2.083093in}{0.922668in}}%
\pgfpathlineto{\pgfqpoint{2.119061in}{0.897026in}}%
\pgfpathlineto{\pgfqpoint{2.157144in}{0.872583in}}%
\pgfpathlineto{\pgfqpoint{2.197343in}{0.849494in}}%
\pgfpathlineto{\pgfqpoint{2.239658in}{0.827874in}}%
\pgfpathlineto{\pgfqpoint{2.284089in}{0.807803in}}%
\pgfpathlineto{\pgfqpoint{2.330636in}{0.789321in}}%
\pgfpathlineto{\pgfqpoint{2.379298in}{0.772440in}}%
\pgfpathlineto{\pgfqpoint{2.430076in}{0.757143in}}%
\pgfpathlineto{\pgfqpoint{2.485086in}{0.742880in}}%
\pgfpathlineto{\pgfqpoint{2.544327in}{0.729817in}}%
\pgfpathlineto{\pgfqpoint{2.609915in}{0.717699in}}%
\pgfpathlineto{\pgfqpoint{2.637420in}{0.713252in}}%
\pgfpathlineto{\pgfqpoint{2.637420in}{0.713252in}}%
\pgfusepath{stroke}%
\end{pgfscope}%
\begin{pgfscope}%
\pgfsetbuttcap%
\pgfsetmiterjoin%
\definecolor{currentfill}{rgb}{1.000000,1.000000,1.000000}%
\pgfsetfillcolor{currentfill}%
\pgfsetfillopacity{0.800000}%
\pgfsetlinewidth{1.003750pt}%
\definecolor{currentstroke}{rgb}{0.800000,0.800000,0.800000}%
\pgfsetstrokecolor{currentstroke}%
\pgfsetstrokeopacity{0.800000}%
\pgfsetdash{}{0pt}%
\pgfpathmoveto{\pgfqpoint{1.607701in}{2.109661in}}%
\pgfpathlineto{\pgfqpoint{2.645879in}{2.109661in}}%
\pgfpathquadraticcurveto{\pgfqpoint{2.673657in}{2.109661in}}{\pgfqpoint{2.673657in}{2.137439in}}%
\pgfpathlineto{\pgfqpoint{2.673657in}{2.761264in}}%
\pgfpathquadraticcurveto{\pgfqpoint{2.673657in}{2.789042in}}{\pgfqpoint{2.645879in}{2.789042in}}%
\pgfpathlineto{\pgfqpoint{1.607701in}{2.789042in}}%
\pgfpathquadraticcurveto{\pgfqpoint{1.579923in}{2.789042in}}{\pgfqpoint{1.579923in}{2.761264in}}%
\pgfpathlineto{\pgfqpoint{1.579923in}{2.137439in}}%
\pgfpathquadraticcurveto{\pgfqpoint{1.579923in}{2.109661in}}{\pgfqpoint{1.607701in}{2.109661in}}%
\pgfpathlineto{\pgfqpoint{1.607701in}{2.109661in}}%
\pgfpathclose%
\pgfusepath{stroke,fill}%
\end{pgfscope}%
\begin{pgfscope}%
\pgfsetrectcap%
\pgfsetroundjoin%
\pgfsetlinewidth{1.505625pt}%
\definecolor{currentstroke}{rgb}{0.000000,0.000000,0.000000}%
\pgfsetstrokecolor{currentstroke}%
\pgfsetdash{}{0pt}%
\pgfpathmoveto{\pgfqpoint{1.635478in}{2.676574in}}%
\pgfpathlineto{\pgfqpoint{1.774367in}{2.676574in}}%
\pgfpathlineto{\pgfqpoint{1.913256in}{2.676574in}}%
\pgfusepath{stroke}%
\end{pgfscope}%
\begin{pgfscope}%
\definecolor{textcolor}{rgb}{0.000000,0.000000,0.000000}%
\pgfsetstrokecolor{textcolor}%
\pgfsetfillcolor{textcolor}%
\pgftext[x=2.024367in,y=2.627963in,left,base]{\color{textcolor}{\rmfamily\fontsize{10.000000}{12.000000}\selectfont\catcode`\^=\active\def^{\ifmmode\sp\else\^{}\fi}\catcode`\%=\active\def%{\%}$u(t)$}}%
\end{pgfscope}%
\begin{pgfscope}%
\pgfsetbuttcap%
\pgfsetroundjoin%
\pgfsetlinewidth{1.505625pt}%
\definecolor{currentstroke}{rgb}{0.000000,0.000000,0.000000}%
\pgfsetstrokecolor{currentstroke}%
\pgfsetdash{{5.550000pt}{2.400000pt}}{0.000000pt}%
\pgfpathmoveto{\pgfqpoint{1.635478in}{2.466884in}}%
\pgfpathlineto{\pgfqpoint{1.774367in}{2.466884in}}%
\pgfpathlineto{\pgfqpoint{1.913256in}{2.466884in}}%
\pgfusepath{stroke}%
\end{pgfscope}%
\begin{pgfscope}%
\definecolor{textcolor}{rgb}{0.000000,0.000000,0.000000}%
\pgfsetstrokecolor{textcolor}%
\pgfsetfillcolor{textcolor}%
\pgftext[x=2.024367in,y=2.418273in,left,base]{\color{textcolor}{\rmfamily\fontsize{10.000000}{12.000000}\selectfont\catcode`\^=\active\def^{\ifmmode\sp\else\^{}\fi}\catcode`\%=\active\def%{\%}$e^{-3t}$}}%
\end{pgfscope}%
\begin{pgfscope}%
\pgfsetbuttcap%
\pgfsetroundjoin%
\pgfsetlinewidth{1.505625pt}%
\definecolor{currentstroke}{rgb}{1.000000,0.000000,0.000000}%
\pgfsetstrokecolor{currentstroke}%
\pgfsetdash{{5.550000pt}{2.400000pt}}{0.000000pt}%
\pgfpathmoveto{\pgfqpoint{1.635478in}{2.248550in}}%
\pgfpathlineto{\pgfqpoint{1.774367in}{2.248550in}}%
\pgfpathlineto{\pgfqpoint{1.913256in}{2.248550in}}%
\pgfusepath{stroke}%
\end{pgfscope}%
\begin{pgfscope}%
\definecolor{textcolor}{rgb}{0.000000,0.000000,0.000000}%
\pgfsetstrokecolor{textcolor}%
\pgfsetfillcolor{textcolor}%
\pgftext[x=2.024367in,y=2.199939in,left,base]{\color{textcolor}{\rmfamily\fontsize{10.000000}{12.000000}\selectfont\catcode`\^=\active\def^{\ifmmode\sp\else\^{}\fi}\catcode`\%=\active\def%{\%}$u(t) \cdot e^{-3t}$}}%
\end{pgfscope}%
\end{pgfpicture}%
\makeatother%
\endgroup%

        \label{fig:unitstepUsed}
    }
    \caption{Heavyside Funktion und eine beispielhafte Anwendung.}
    \label{fig:unitStepComparison}
\end{figure}

\section{Sinus und Kosinus}
\todo[inline]{Todo. Not urgent.}
\begin{figure}[H]
	\centering
	%% Creator: Matplotlib, PGF backend
%%
%% To include the figure in your LaTeX document, write
%%   \input{<filename>.pgf}
%%
%% Make sure the required packages are loaded in your preamble
%%   \usepackage{pgf}
%%
%% Also ensure that all the required font packages are loaded; for instance,
%% the lmodern package is sometimes necessary when using math font.
%%   \usepackage{lmodern}
%%
%% Figures using additional raster images can only be included by \input if
%% they are in the same directory as the main LaTeX file. For loading figures
%% from other directories you can use the `import` package
%%   \usepackage{import}
%%
%% and then include the figures with
%%   \import{<path to file>}{<filename>.pgf}
%%
%% Matplotlib used the following preamble
%%   \def\mathdefault#1{#1}
%%   \everymath=\expandafter{\the\everymath\displaystyle}
%%   
%%   \usepackage{fontspec}
%%   \setmainfont{VeraSe.ttf}[Path=\detokenize{/usr/share/fonts/TTF/}]
%%   \setsansfont{DejaVuSans.ttf}[Path=\detokenize{/home/pl/miniconda3/lib/python3.12/site-packages/matplotlib/mpl-data/fonts/ttf/}]
%%   \setmonofont{DejaVuSansMono.ttf}[Path=\detokenize{/home/pl/miniconda3/lib/python3.12/site-packages/matplotlib/mpl-data/fonts/ttf/}]
%%   \makeatletter\@ifpackageloaded{underscore}{}{\usepackage[strings]{underscore}}\makeatother
%%
\begingroup%
\makeatletter%
\begin{pgfpicture}%
\pgfpathrectangle{\pgfpointorigin}{\pgfqpoint{5.276127in}{2.885959in}}%
\pgfusepath{use as bounding box, clip}%
\begin{pgfscope}%
\pgfsetbuttcap%
\pgfsetmiterjoin%
\definecolor{currentfill}{rgb}{1.000000,1.000000,1.000000}%
\pgfsetfillcolor{currentfill}%
\pgfsetlinewidth{0.000000pt}%
\definecolor{currentstroke}{rgb}{1.000000,1.000000,1.000000}%
\pgfsetstrokecolor{currentstroke}%
\pgfsetdash{}{0pt}%
\pgfpathmoveto{\pgfqpoint{0.000000in}{0.000000in}}%
\pgfpathlineto{\pgfqpoint{5.276127in}{0.000000in}}%
\pgfpathlineto{\pgfqpoint{5.276127in}{2.885959in}}%
\pgfpathlineto{\pgfqpoint{0.000000in}{2.885959in}}%
\pgfpathlineto{\pgfqpoint{0.000000in}{0.000000in}}%
\pgfpathclose%
\pgfusepath{fill}%
\end{pgfscope}%
\begin{pgfscope}%
\pgfsetbuttcap%
\pgfsetmiterjoin%
\definecolor{currentfill}{rgb}{1.000000,1.000000,1.000000}%
\pgfsetfillcolor{currentfill}%
\pgfsetlinewidth{0.000000pt}%
\definecolor{currentstroke}{rgb}{0.000000,0.000000,0.000000}%
\pgfsetstrokecolor{currentstroke}%
\pgfsetstrokeopacity{0.000000}%
\pgfsetdash{}{0pt}%
\pgfpathmoveto{\pgfqpoint{0.526127in}{0.475959in}}%
\pgfpathlineto{\pgfqpoint{5.176127in}{0.475959in}}%
\pgfpathlineto{\pgfqpoint{5.176127in}{2.785959in}}%
\pgfpathlineto{\pgfqpoint{0.526127in}{2.785959in}}%
\pgfpathlineto{\pgfqpoint{0.526127in}{0.475959in}}%
\pgfpathclose%
\pgfusepath{fill}%
\end{pgfscope}%
\begin{pgfscope}%
\pgfsetbuttcap%
\pgfsetroundjoin%
\definecolor{currentfill}{rgb}{0.000000,0.000000,0.000000}%
\pgfsetfillcolor{currentfill}%
\pgfsetlinewidth{0.803000pt}%
\definecolor{currentstroke}{rgb}{0.000000,0.000000,0.000000}%
\pgfsetstrokecolor{currentstroke}%
\pgfsetdash{}{0pt}%
\pgfsys@defobject{currentmarker}{\pgfqpoint{0.000000in}{-0.048611in}}{\pgfqpoint{0.000000in}{0.000000in}}{%
\pgfpathmoveto{\pgfqpoint{0.000000in}{0.000000in}}%
\pgfpathlineto{\pgfqpoint{0.000000in}{-0.048611in}}%
\pgfusepath{stroke,fill}%
}%
\begin{pgfscope}%
\pgfsys@transformshift{0.953931in}{0.475959in}%
\pgfsys@useobject{currentmarker}{}%
\end{pgfscope}%
\end{pgfscope}%
\begin{pgfscope}%
\definecolor{textcolor}{rgb}{0.000000,0.000000,0.000000}%
\pgfsetstrokecolor{textcolor}%
\pgfsetfillcolor{textcolor}%
\pgftext[x=0.953931in,y=0.378737in,,top]{\color{textcolor}{\rmfamily\fontsize{10.000000}{12.000000}\selectfont\catcode`\^=\active\def^{\ifmmode\sp\else\^{}\fi}\catcode`\%=\active\def%{\%}$-2\pi$}}%
\end{pgfscope}%
\begin{pgfscope}%
\pgfsetbuttcap%
\pgfsetroundjoin%
\definecolor{currentfill}{rgb}{0.000000,0.000000,0.000000}%
\pgfsetfillcolor{currentfill}%
\pgfsetlinewidth{0.803000pt}%
\definecolor{currentstroke}{rgb}{0.000000,0.000000,0.000000}%
\pgfsetstrokecolor{currentstroke}%
\pgfsetdash{}{0pt}%
\pgfsys@defobject{currentmarker}{\pgfqpoint{0.000000in}{-0.048611in}}{\pgfqpoint{0.000000in}{0.000000in}}{%
\pgfpathmoveto{\pgfqpoint{0.000000in}{0.000000in}}%
\pgfpathlineto{\pgfqpoint{0.000000in}{-0.048611in}}%
\pgfusepath{stroke,fill}%
}%
\begin{pgfscope}%
\pgfsys@transformshift{1.428230in}{0.475959in}%
\pgfsys@useobject{currentmarker}{}%
\end{pgfscope}%
\end{pgfscope}%
\begin{pgfscope}%
\definecolor{textcolor}{rgb}{0.000000,0.000000,0.000000}%
\pgfsetstrokecolor{textcolor}%
\pgfsetfillcolor{textcolor}%
\pgftext[x=1.428230in,y=0.378737in,,top]{\color{textcolor}{\rmfamily\fontsize{10.000000}{12.000000}\selectfont\catcode`\^=\active\def^{\ifmmode\sp\else\^{}\fi}\catcode`\%=\active\def%{\%}$-\pi \cdot \frac{3}{2}$}}%
\end{pgfscope}%
\begin{pgfscope}%
\pgfsetbuttcap%
\pgfsetroundjoin%
\definecolor{currentfill}{rgb}{0.000000,0.000000,0.000000}%
\pgfsetfillcolor{currentfill}%
\pgfsetlinewidth{0.803000pt}%
\definecolor{currentstroke}{rgb}{0.000000,0.000000,0.000000}%
\pgfsetstrokecolor{currentstroke}%
\pgfsetdash{}{0pt}%
\pgfsys@defobject{currentmarker}{\pgfqpoint{0.000000in}{-0.048611in}}{\pgfqpoint{0.000000in}{0.000000in}}{%
\pgfpathmoveto{\pgfqpoint{0.000000in}{0.000000in}}%
\pgfpathlineto{\pgfqpoint{0.000000in}{-0.048611in}}%
\pgfusepath{stroke,fill}%
}%
\begin{pgfscope}%
\pgfsys@transformshift{1.902529in}{0.475959in}%
\pgfsys@useobject{currentmarker}{}%
\end{pgfscope}%
\end{pgfscope}%
\begin{pgfscope}%
\definecolor{textcolor}{rgb}{0.000000,0.000000,0.000000}%
\pgfsetstrokecolor{textcolor}%
\pgfsetfillcolor{textcolor}%
\pgftext[x=1.902529in,y=0.378737in,,top]{\color{textcolor}{\rmfamily\fontsize{10.000000}{12.000000}\selectfont\catcode`\^=\active\def^{\ifmmode\sp\else\^{}\fi}\catcode`\%=\active\def%{\%}$\pi$}}%
\end{pgfscope}%
\begin{pgfscope}%
\pgfsetbuttcap%
\pgfsetroundjoin%
\definecolor{currentfill}{rgb}{0.000000,0.000000,0.000000}%
\pgfsetfillcolor{currentfill}%
\pgfsetlinewidth{0.803000pt}%
\definecolor{currentstroke}{rgb}{0.000000,0.000000,0.000000}%
\pgfsetstrokecolor{currentstroke}%
\pgfsetdash{}{0pt}%
\pgfsys@defobject{currentmarker}{\pgfqpoint{0.000000in}{-0.048611in}}{\pgfqpoint{0.000000in}{0.000000in}}{%
\pgfpathmoveto{\pgfqpoint{0.000000in}{0.000000in}}%
\pgfpathlineto{\pgfqpoint{0.000000in}{-0.048611in}}%
\pgfusepath{stroke,fill}%
}%
\begin{pgfscope}%
\pgfsys@transformshift{2.376828in}{0.475959in}%
\pgfsys@useobject{currentmarker}{}%
\end{pgfscope}%
\end{pgfscope}%
\begin{pgfscope}%
\definecolor{textcolor}{rgb}{0.000000,0.000000,0.000000}%
\pgfsetstrokecolor{textcolor}%
\pgfsetfillcolor{textcolor}%
\pgftext[x=2.376828in,y=0.378737in,,top]{\color{textcolor}{\rmfamily\fontsize{10.000000}{12.000000}\selectfont\catcode`\^=\active\def^{\ifmmode\sp\else\^{}\fi}\catcode`\%=\active\def%{\%}$-\frac{\pi}{2}$}}%
\end{pgfscope}%
\begin{pgfscope}%
\pgfsetbuttcap%
\pgfsetroundjoin%
\definecolor{currentfill}{rgb}{0.000000,0.000000,0.000000}%
\pgfsetfillcolor{currentfill}%
\pgfsetlinewidth{0.803000pt}%
\definecolor{currentstroke}{rgb}{0.000000,0.000000,0.000000}%
\pgfsetstrokecolor{currentstroke}%
\pgfsetdash{}{0pt}%
\pgfsys@defobject{currentmarker}{\pgfqpoint{0.000000in}{-0.048611in}}{\pgfqpoint{0.000000in}{0.000000in}}{%
\pgfpathmoveto{\pgfqpoint{0.000000in}{0.000000in}}%
\pgfpathlineto{\pgfqpoint{0.000000in}{-0.048611in}}%
\pgfusepath{stroke,fill}%
}%
\begin{pgfscope}%
\pgfsys@transformshift{2.851127in}{0.475959in}%
\pgfsys@useobject{currentmarker}{}%
\end{pgfscope}%
\end{pgfscope}%
\begin{pgfscope}%
\definecolor{textcolor}{rgb}{0.000000,0.000000,0.000000}%
\pgfsetstrokecolor{textcolor}%
\pgfsetfillcolor{textcolor}%
\pgftext[x=2.851127in,y=0.378737in,,top]{\color{textcolor}{\rmfamily\fontsize{10.000000}{12.000000}\selectfont\catcode`\^=\active\def^{\ifmmode\sp\else\^{}\fi}\catcode`\%=\active\def%{\%}0}}%
\end{pgfscope}%
\begin{pgfscope}%
\pgfsetbuttcap%
\pgfsetroundjoin%
\definecolor{currentfill}{rgb}{0.000000,0.000000,0.000000}%
\pgfsetfillcolor{currentfill}%
\pgfsetlinewidth{0.803000pt}%
\definecolor{currentstroke}{rgb}{0.000000,0.000000,0.000000}%
\pgfsetstrokecolor{currentstroke}%
\pgfsetdash{}{0pt}%
\pgfsys@defobject{currentmarker}{\pgfqpoint{0.000000in}{-0.048611in}}{\pgfqpoint{0.000000in}{0.000000in}}{%
\pgfpathmoveto{\pgfqpoint{0.000000in}{0.000000in}}%
\pgfpathlineto{\pgfqpoint{0.000000in}{-0.048611in}}%
\pgfusepath{stroke,fill}%
}%
\begin{pgfscope}%
\pgfsys@transformshift{3.325426in}{0.475959in}%
\pgfsys@useobject{currentmarker}{}%
\end{pgfscope}%
\end{pgfscope}%
\begin{pgfscope}%
\definecolor{textcolor}{rgb}{0.000000,0.000000,0.000000}%
\pgfsetstrokecolor{textcolor}%
\pgfsetfillcolor{textcolor}%
\pgftext[x=3.325426in,y=0.378737in,,top]{\color{textcolor}{\rmfamily\fontsize{10.000000}{12.000000}\selectfont\catcode`\^=\active\def^{\ifmmode\sp\else\^{}\fi}\catcode`\%=\active\def%{\%}$\frac{\pi}{2}$}}%
\end{pgfscope}%
\begin{pgfscope}%
\pgfsetbuttcap%
\pgfsetroundjoin%
\definecolor{currentfill}{rgb}{0.000000,0.000000,0.000000}%
\pgfsetfillcolor{currentfill}%
\pgfsetlinewidth{0.803000pt}%
\definecolor{currentstroke}{rgb}{0.000000,0.000000,0.000000}%
\pgfsetstrokecolor{currentstroke}%
\pgfsetdash{}{0pt}%
\pgfsys@defobject{currentmarker}{\pgfqpoint{0.000000in}{-0.048611in}}{\pgfqpoint{0.000000in}{0.000000in}}{%
\pgfpathmoveto{\pgfqpoint{0.000000in}{0.000000in}}%
\pgfpathlineto{\pgfqpoint{0.000000in}{-0.048611in}}%
\pgfusepath{stroke,fill}%
}%
\begin{pgfscope}%
\pgfsys@transformshift{3.799724in}{0.475959in}%
\pgfsys@useobject{currentmarker}{}%
\end{pgfscope}%
\end{pgfscope}%
\begin{pgfscope}%
\definecolor{textcolor}{rgb}{0.000000,0.000000,0.000000}%
\pgfsetstrokecolor{textcolor}%
\pgfsetfillcolor{textcolor}%
\pgftext[x=3.799724in,y=0.378737in,,top]{\color{textcolor}{\rmfamily\fontsize{10.000000}{12.000000}\selectfont\catcode`\^=\active\def^{\ifmmode\sp\else\^{}\fi}\catcode`\%=\active\def%{\%}$\pi$}}%
\end{pgfscope}%
\begin{pgfscope}%
\pgfsetbuttcap%
\pgfsetroundjoin%
\definecolor{currentfill}{rgb}{0.000000,0.000000,0.000000}%
\pgfsetfillcolor{currentfill}%
\pgfsetlinewidth{0.803000pt}%
\definecolor{currentstroke}{rgb}{0.000000,0.000000,0.000000}%
\pgfsetstrokecolor{currentstroke}%
\pgfsetdash{}{0pt}%
\pgfsys@defobject{currentmarker}{\pgfqpoint{0.000000in}{-0.048611in}}{\pgfqpoint{0.000000in}{0.000000in}}{%
\pgfpathmoveto{\pgfqpoint{0.000000in}{0.000000in}}%
\pgfpathlineto{\pgfqpoint{0.000000in}{-0.048611in}}%
\pgfusepath{stroke,fill}%
}%
\begin{pgfscope}%
\pgfsys@transformshift{4.274023in}{0.475959in}%
\pgfsys@useobject{currentmarker}{}%
\end{pgfscope}%
\end{pgfscope}%
\begin{pgfscope}%
\definecolor{textcolor}{rgb}{0.000000,0.000000,0.000000}%
\pgfsetstrokecolor{textcolor}%
\pgfsetfillcolor{textcolor}%
\pgftext[x=4.274023in,y=0.378737in,,top]{\color{textcolor}{\rmfamily\fontsize{10.000000}{12.000000}\selectfont\catcode`\^=\active\def^{\ifmmode\sp\else\^{}\fi}\catcode`\%=\active\def%{\%}$\pi \cdot \frac{3}{2}$}}%
\end{pgfscope}%
\begin{pgfscope}%
\pgfsetbuttcap%
\pgfsetroundjoin%
\definecolor{currentfill}{rgb}{0.000000,0.000000,0.000000}%
\pgfsetfillcolor{currentfill}%
\pgfsetlinewidth{0.803000pt}%
\definecolor{currentstroke}{rgb}{0.000000,0.000000,0.000000}%
\pgfsetstrokecolor{currentstroke}%
\pgfsetdash{}{0pt}%
\pgfsys@defobject{currentmarker}{\pgfqpoint{0.000000in}{-0.048611in}}{\pgfqpoint{0.000000in}{0.000000in}}{%
\pgfpathmoveto{\pgfqpoint{0.000000in}{0.000000in}}%
\pgfpathlineto{\pgfqpoint{0.000000in}{-0.048611in}}%
\pgfusepath{stroke,fill}%
}%
\begin{pgfscope}%
\pgfsys@transformshift{4.748322in}{0.475959in}%
\pgfsys@useobject{currentmarker}{}%
\end{pgfscope}%
\end{pgfscope}%
\begin{pgfscope}%
\definecolor{textcolor}{rgb}{0.000000,0.000000,0.000000}%
\pgfsetstrokecolor{textcolor}%
\pgfsetfillcolor{textcolor}%
\pgftext[x=4.748322in,y=0.378737in,,top]{\color{textcolor}{\rmfamily\fontsize{10.000000}{12.000000}\selectfont\catcode`\^=\active\def^{\ifmmode\sp\else\^{}\fi}\catcode`\%=\active\def%{\%}$2\pi$}}%
\end{pgfscope}%
\begin{pgfscope}%
\pgfsetbuttcap%
\pgfsetroundjoin%
\definecolor{currentfill}{rgb}{0.000000,0.000000,0.000000}%
\pgfsetfillcolor{currentfill}%
\pgfsetlinewidth{0.803000pt}%
\definecolor{currentstroke}{rgb}{0.000000,0.000000,0.000000}%
\pgfsetstrokecolor{currentstroke}%
\pgfsetdash{}{0pt}%
\pgfsys@defobject{currentmarker}{\pgfqpoint{-0.048611in}{0.000000in}}{\pgfqpoint{-0.000000in}{0.000000in}}{%
\pgfpathmoveto{\pgfqpoint{-0.000000in}{0.000000in}}%
\pgfpathlineto{\pgfqpoint{-0.048611in}{0.000000in}}%
\pgfusepath{stroke,fill}%
}%
\begin{pgfscope}%
\pgfsys@transformshift{0.526127in}{0.580959in}%
\pgfsys@useobject{currentmarker}{}%
\end{pgfscope}%
\end{pgfscope}%
\begin{pgfscope}%
\definecolor{textcolor}{rgb}{0.000000,0.000000,0.000000}%
\pgfsetstrokecolor{textcolor}%
\pgfsetfillcolor{textcolor}%
\pgftext[x=0.100000in, y=0.528197in, left, base]{\color{textcolor}{\rmfamily\fontsize{10.000000}{12.000000}\selectfont\catcode`\^=\active\def^{\ifmmode\sp\else\^{}\fi}\catcode`\%=\active\def%{\%}\ensuremath{-}1.0}}%
\end{pgfscope}%
\begin{pgfscope}%
\pgfsetbuttcap%
\pgfsetroundjoin%
\definecolor{currentfill}{rgb}{0.000000,0.000000,0.000000}%
\pgfsetfillcolor{currentfill}%
\pgfsetlinewidth{0.803000pt}%
\definecolor{currentstroke}{rgb}{0.000000,0.000000,0.000000}%
\pgfsetstrokecolor{currentstroke}%
\pgfsetdash{}{0pt}%
\pgfsys@defobject{currentmarker}{\pgfqpoint{-0.048611in}{0.000000in}}{\pgfqpoint{-0.000000in}{0.000000in}}{%
\pgfpathmoveto{\pgfqpoint{-0.000000in}{0.000000in}}%
\pgfpathlineto{\pgfqpoint{-0.048611in}{0.000000in}}%
\pgfusepath{stroke,fill}%
}%
\begin{pgfscope}%
\pgfsys@transformshift{0.526127in}{1.105959in}%
\pgfsys@useobject{currentmarker}{}%
\end{pgfscope}%
\end{pgfscope}%
\begin{pgfscope}%
\definecolor{textcolor}{rgb}{0.000000,0.000000,0.000000}%
\pgfsetstrokecolor{textcolor}%
\pgfsetfillcolor{textcolor}%
\pgftext[x=0.100000in, y=1.053197in, left, base]{\color{textcolor}{\rmfamily\fontsize{10.000000}{12.000000}\selectfont\catcode`\^=\active\def^{\ifmmode\sp\else\^{}\fi}\catcode`\%=\active\def%{\%}\ensuremath{-}0.5}}%
\end{pgfscope}%
\begin{pgfscope}%
\pgfsetbuttcap%
\pgfsetroundjoin%
\definecolor{currentfill}{rgb}{0.000000,0.000000,0.000000}%
\pgfsetfillcolor{currentfill}%
\pgfsetlinewidth{0.803000pt}%
\definecolor{currentstroke}{rgb}{0.000000,0.000000,0.000000}%
\pgfsetstrokecolor{currentstroke}%
\pgfsetdash{}{0pt}%
\pgfsys@defobject{currentmarker}{\pgfqpoint{-0.048611in}{0.000000in}}{\pgfqpoint{-0.000000in}{0.000000in}}{%
\pgfpathmoveto{\pgfqpoint{-0.000000in}{0.000000in}}%
\pgfpathlineto{\pgfqpoint{-0.048611in}{0.000000in}}%
\pgfusepath{stroke,fill}%
}%
\begin{pgfscope}%
\pgfsys@transformshift{0.526127in}{1.630959in}%
\pgfsys@useobject{currentmarker}{}%
\end{pgfscope}%
\end{pgfscope}%
\begin{pgfscope}%
\definecolor{textcolor}{rgb}{0.000000,0.000000,0.000000}%
\pgfsetstrokecolor{textcolor}%
\pgfsetfillcolor{textcolor}%
\pgftext[x=0.208025in, y=1.578197in, left, base]{\color{textcolor}{\rmfamily\fontsize{10.000000}{12.000000}\selectfont\catcode`\^=\active\def^{\ifmmode\sp\else\^{}\fi}\catcode`\%=\active\def%{\%}0.0}}%
\end{pgfscope}%
\begin{pgfscope}%
\pgfsetbuttcap%
\pgfsetroundjoin%
\definecolor{currentfill}{rgb}{0.000000,0.000000,0.000000}%
\pgfsetfillcolor{currentfill}%
\pgfsetlinewidth{0.803000pt}%
\definecolor{currentstroke}{rgb}{0.000000,0.000000,0.000000}%
\pgfsetstrokecolor{currentstroke}%
\pgfsetdash{}{0pt}%
\pgfsys@defobject{currentmarker}{\pgfqpoint{-0.048611in}{0.000000in}}{\pgfqpoint{-0.000000in}{0.000000in}}{%
\pgfpathmoveto{\pgfqpoint{-0.000000in}{0.000000in}}%
\pgfpathlineto{\pgfqpoint{-0.048611in}{0.000000in}}%
\pgfusepath{stroke,fill}%
}%
\begin{pgfscope}%
\pgfsys@transformshift{0.526127in}{2.155959in}%
\pgfsys@useobject{currentmarker}{}%
\end{pgfscope}%
\end{pgfscope}%
\begin{pgfscope}%
\definecolor{textcolor}{rgb}{0.000000,0.000000,0.000000}%
\pgfsetstrokecolor{textcolor}%
\pgfsetfillcolor{textcolor}%
\pgftext[x=0.208025in, y=2.103197in, left, base]{\color{textcolor}{\rmfamily\fontsize{10.000000}{12.000000}\selectfont\catcode`\^=\active\def^{\ifmmode\sp\else\^{}\fi}\catcode`\%=\active\def%{\%}0.5}}%
\end{pgfscope}%
\begin{pgfscope}%
\pgfsetbuttcap%
\pgfsetroundjoin%
\definecolor{currentfill}{rgb}{0.000000,0.000000,0.000000}%
\pgfsetfillcolor{currentfill}%
\pgfsetlinewidth{0.803000pt}%
\definecolor{currentstroke}{rgb}{0.000000,0.000000,0.000000}%
\pgfsetstrokecolor{currentstroke}%
\pgfsetdash{}{0pt}%
\pgfsys@defobject{currentmarker}{\pgfqpoint{-0.048611in}{0.000000in}}{\pgfqpoint{-0.000000in}{0.000000in}}{%
\pgfpathmoveto{\pgfqpoint{-0.000000in}{0.000000in}}%
\pgfpathlineto{\pgfqpoint{-0.048611in}{0.000000in}}%
\pgfusepath{stroke,fill}%
}%
\begin{pgfscope}%
\pgfsys@transformshift{0.526127in}{2.680959in}%
\pgfsys@useobject{currentmarker}{}%
\end{pgfscope}%
\end{pgfscope}%
\begin{pgfscope}%
\definecolor{textcolor}{rgb}{0.000000,0.000000,0.000000}%
\pgfsetstrokecolor{textcolor}%
\pgfsetfillcolor{textcolor}%
\pgftext[x=0.208025in, y=2.628197in, left, base]{\color{textcolor}{\rmfamily\fontsize{10.000000}{12.000000}\selectfont\catcode`\^=\active\def^{\ifmmode\sp\else\^{}\fi}\catcode`\%=\active\def%{\%}1.0}}%
\end{pgfscope}%
\begin{pgfscope}%
\pgfpathrectangle{\pgfqpoint{0.526127in}{0.475959in}}{\pgfqpoint{4.650000in}{2.310000in}}%
\pgfusepath{clip}%
\pgfsetrectcap%
\pgfsetroundjoin%
\pgfsetlinewidth{1.505625pt}%
\definecolor{currentstroke}{rgb}{0.000000,0.000000,0.000000}%
\pgfsetstrokecolor{currentstroke}%
\pgfsetdash{}{0pt}%
\pgfpathmoveto{\pgfqpoint{0.737490in}{0.941123in}}%
\pgfpathlineto{\pgfqpoint{0.779975in}{1.058953in}}%
\pgfpathlineto{\pgfqpoint{0.822461in}{1.188090in}}%
\pgfpathlineto{\pgfqpoint{0.864946in}{1.325979in}}%
\pgfpathlineto{\pgfqpoint{0.928673in}{1.543230in}}%
\pgfpathlineto{\pgfqpoint{1.034886in}{1.909114in}}%
\pgfpathlineto{\pgfqpoint{1.077372in}{2.048356in}}%
\pgfpathlineto{\pgfqpoint{1.119857in}{2.179348in}}%
\pgfpathlineto{\pgfqpoint{1.162342in}{2.299501in}}%
\pgfpathlineto{\pgfqpoint{1.183584in}{2.354762in}}%
\pgfpathlineto{\pgfqpoint{1.204827in}{2.406441in}}%
\pgfpathlineto{\pgfqpoint{1.226070in}{2.454284in}}%
\pgfpathlineto{\pgfqpoint{1.247312in}{2.498054in}}%
\pgfpathlineto{\pgfqpoint{1.268555in}{2.537533in}}%
\pgfpathlineto{\pgfqpoint{1.289797in}{2.572528in}}%
\pgfpathlineto{\pgfqpoint{1.311040in}{2.602865in}}%
\pgfpathlineto{\pgfqpoint{1.332282in}{2.628393in}}%
\pgfpathlineto{\pgfqpoint{1.353525in}{2.648986in}}%
\pgfpathlineto{\pgfqpoint{1.374768in}{2.664543in}}%
\pgfpathlineto{\pgfqpoint{1.396010in}{2.674987in}}%
\pgfpathlineto{\pgfqpoint{1.417253in}{2.680265in}}%
\pgfpathlineto{\pgfqpoint{1.438495in}{2.680352in}}%
\pgfpathlineto{\pgfqpoint{1.459738in}{2.675247in}}%
\pgfpathlineto{\pgfqpoint{1.480980in}{2.664976in}}%
\pgfpathlineto{\pgfqpoint{1.502223in}{2.649590in}}%
\pgfpathlineto{\pgfqpoint{1.523466in}{2.629163in}}%
\pgfpathlineto{\pgfqpoint{1.544708in}{2.603799in}}%
\pgfpathlineto{\pgfqpoint{1.565951in}{2.573621in}}%
\pgfpathlineto{\pgfqpoint{1.587193in}{2.538780in}}%
\pgfpathlineto{\pgfqpoint{1.608436in}{2.499447in}}%
\pgfpathlineto{\pgfqpoint{1.629679in}{2.455818in}}%
\pgfpathlineto{\pgfqpoint{1.650921in}{2.408108in}}%
\pgfpathlineto{\pgfqpoint{1.672164in}{2.356553in}}%
\pgfpathlineto{\pgfqpoint{1.693406in}{2.301408in}}%
\pgfpathlineto{\pgfqpoint{1.735891in}{2.181457in}}%
\pgfpathlineto{\pgfqpoint{1.778377in}{2.050626in}}%
\pgfpathlineto{\pgfqpoint{1.820862in}{1.911500in}}%
\pgfpathlineto{\pgfqpoint{1.884589in}{1.693305in}}%
\pgfpathlineto{\pgfqpoint{1.969560in}{1.399774in}}%
\pgfpathlineto{\pgfqpoint{2.012045in}{1.258421in}}%
\pgfpathlineto{\pgfqpoint{2.054530in}{1.124430in}}%
\pgfpathlineto{\pgfqpoint{2.097015in}{1.000451in}}%
\pgfpathlineto{\pgfqpoint{2.118258in}{0.942991in}}%
\pgfpathlineto{\pgfqpoint{2.139500in}{0.888934in}}%
\pgfpathlineto{\pgfqpoint{2.160743in}{0.838549in}}%
\pgfpathlineto{\pgfqpoint{2.181986in}{0.792083in}}%
\pgfpathlineto{\pgfqpoint{2.203228in}{0.749768in}}%
\pgfpathlineto{\pgfqpoint{2.224471in}{0.711813in}}%
\pgfpathlineto{\pgfqpoint{2.245713in}{0.678404in}}%
\pgfpathlineto{\pgfqpoint{2.266956in}{0.649709in}}%
\pgfpathlineto{\pgfqpoint{2.288198in}{0.625867in}}%
\pgfpathlineto{\pgfqpoint{2.309441in}{0.606999in}}%
\pgfpathlineto{\pgfqpoint{2.330684in}{0.593196in}}%
\pgfpathlineto{\pgfqpoint{2.351926in}{0.584527in}}%
\pgfpathlineto{\pgfqpoint{2.373169in}{0.581036in}}%
\pgfpathlineto{\pgfqpoint{2.394411in}{0.582739in}}%
\pgfpathlineto{\pgfqpoint{2.415654in}{0.589627in}}%
\pgfpathlineto{\pgfqpoint{2.436896in}{0.601668in}}%
\pgfpathlineto{\pgfqpoint{2.458139in}{0.618800in}}%
\pgfpathlineto{\pgfqpoint{2.479382in}{0.640941in}}%
\pgfpathlineto{\pgfqpoint{2.500624in}{0.667979in}}%
\pgfpathlineto{\pgfqpoint{2.521867in}{0.699781in}}%
\pgfpathlineto{\pgfqpoint{2.543109in}{0.736190in}}%
\pgfpathlineto{\pgfqpoint{2.564352in}{0.777026in}}%
\pgfpathlineto{\pgfqpoint{2.585594in}{0.822086in}}%
\pgfpathlineto{\pgfqpoint{2.606837in}{0.871149in}}%
\pgfpathlineto{\pgfqpoint{2.628080in}{0.923970in}}%
\pgfpathlineto{\pgfqpoint{2.649322in}{0.980289in}}%
\pgfpathlineto{\pgfqpoint{2.691807in}{1.102290in}}%
\pgfpathlineto{\pgfqpoint{2.734292in}{1.234740in}}%
\pgfpathlineto{\pgfqpoint{2.776778in}{1.375021in}}%
\pgfpathlineto{\pgfqpoint{2.840505in}{1.594032in}}%
\pgfpathlineto{\pgfqpoint{2.925476in}{1.886897in}}%
\pgfpathlineto{\pgfqpoint{2.967961in}{2.027178in}}%
\pgfpathlineto{\pgfqpoint{3.010446in}{2.159628in}}%
\pgfpathlineto{\pgfqpoint{3.052931in}{2.281628in}}%
\pgfpathlineto{\pgfqpoint{3.074174in}{2.337947in}}%
\pgfpathlineto{\pgfqpoint{3.095416in}{2.390769in}}%
\pgfpathlineto{\pgfqpoint{3.116659in}{2.439831in}}%
\pgfpathlineto{\pgfqpoint{3.137901in}{2.484892in}}%
\pgfpathlineto{\pgfqpoint{3.159144in}{2.525727in}}%
\pgfpathlineto{\pgfqpoint{3.180387in}{2.562137in}}%
\pgfpathlineto{\pgfqpoint{3.201629in}{2.593939in}}%
\pgfpathlineto{\pgfqpoint{3.222872in}{2.620977in}}%
\pgfpathlineto{\pgfqpoint{3.244114in}{2.643117in}}%
\pgfpathlineto{\pgfqpoint{3.265357in}{2.660250in}}%
\pgfpathlineto{\pgfqpoint{3.286599in}{2.672290in}}%
\pgfpathlineto{\pgfqpoint{3.307842in}{2.679179in}}%
\pgfpathlineto{\pgfqpoint{3.329085in}{2.680882in}}%
\pgfpathlineto{\pgfqpoint{3.350327in}{2.677390in}}%
\pgfpathlineto{\pgfqpoint{3.371570in}{2.668721in}}%
\pgfpathlineto{\pgfqpoint{3.392812in}{2.654919in}}%
\pgfpathlineto{\pgfqpoint{3.414055in}{2.636050in}}%
\pgfpathlineto{\pgfqpoint{3.435298in}{2.612209in}}%
\pgfpathlineto{\pgfqpoint{3.456540in}{2.583513in}}%
\pgfpathlineto{\pgfqpoint{3.477783in}{2.550105in}}%
\pgfpathlineto{\pgfqpoint{3.499025in}{2.512149in}}%
\pgfpathlineto{\pgfqpoint{3.520268in}{2.469834in}}%
\pgfpathlineto{\pgfqpoint{3.541510in}{2.423369in}}%
\pgfpathlineto{\pgfqpoint{3.562753in}{2.372983in}}%
\pgfpathlineto{\pgfqpoint{3.583996in}{2.318926in}}%
\pgfpathlineto{\pgfqpoint{3.626481in}{2.200886in}}%
\pgfpathlineto{\pgfqpoint{3.668966in}{2.071582in}}%
\pgfpathlineto{\pgfqpoint{3.711451in}{1.933569in}}%
\pgfpathlineto{\pgfqpoint{3.775179in}{1.716220in}}%
\pgfpathlineto{\pgfqpoint{3.881392in}{1.350418in}}%
\pgfpathlineto{\pgfqpoint{3.923877in}{1.211291in}}%
\pgfpathlineto{\pgfqpoint{3.966362in}{1.080460in}}%
\pgfpathlineto{\pgfqpoint{4.008847in}{0.960509in}}%
\pgfpathlineto{\pgfqpoint{4.030090in}{0.905364in}}%
\pgfpathlineto{\pgfqpoint{4.051332in}{0.853809in}}%
\pgfpathlineto{\pgfqpoint{4.072575in}{0.806099in}}%
\pgfpathlineto{\pgfqpoint{4.093817in}{0.762470in}}%
\pgfpathlineto{\pgfqpoint{4.115060in}{0.723138in}}%
\pgfpathlineto{\pgfqpoint{4.136303in}{0.688296in}}%
\pgfpathlineto{\pgfqpoint{4.157545in}{0.658119in}}%
\pgfpathlineto{\pgfqpoint{4.178788in}{0.632754in}}%
\pgfpathlineto{\pgfqpoint{4.200030in}{0.612328in}}%
\pgfpathlineto{\pgfqpoint{4.221273in}{0.596941in}}%
\pgfpathlineto{\pgfqpoint{4.242515in}{0.586670in}}%
\pgfpathlineto{\pgfqpoint{4.263758in}{0.581565in}}%
\pgfpathlineto{\pgfqpoint{4.285001in}{0.581653in}}%
\pgfpathlineto{\pgfqpoint{4.306243in}{0.586931in}}%
\pgfpathlineto{\pgfqpoint{4.327486in}{0.597374in}}%
\pgfpathlineto{\pgfqpoint{4.348728in}{0.612931in}}%
\pgfpathlineto{\pgfqpoint{4.369971in}{0.633525in}}%
\pgfpathlineto{\pgfqpoint{4.391213in}{0.659053in}}%
\pgfpathlineto{\pgfqpoint{4.412456in}{0.689389in}}%
\pgfpathlineto{\pgfqpoint{4.433699in}{0.724384in}}%
\pgfpathlineto{\pgfqpoint{4.454941in}{0.763864in}}%
\pgfpathlineto{\pgfqpoint{4.476184in}{0.807633in}}%
\pgfpathlineto{\pgfqpoint{4.497426in}{0.855476in}}%
\pgfpathlineto{\pgfqpoint{4.518669in}{0.907156in}}%
\pgfpathlineto{\pgfqpoint{4.539911in}{0.962416in}}%
\pgfpathlineto{\pgfqpoint{4.582397in}{1.082570in}}%
\pgfpathlineto{\pgfqpoint{4.624882in}{1.213562in}}%
\pgfpathlineto{\pgfqpoint{4.667367in}{1.352804in}}%
\pgfpathlineto{\pgfqpoint{4.731095in}{1.571084in}}%
\pgfpathlineto{\pgfqpoint{4.816065in}{1.864558in}}%
\pgfpathlineto{\pgfqpoint{4.858550in}{2.005810in}}%
\pgfpathlineto{\pgfqpoint{4.901035in}{2.139654in}}%
\pgfpathlineto{\pgfqpoint{4.943520in}{2.263444in}}%
\pgfpathlineto{\pgfqpoint{4.964763in}{2.320795in}}%
\pgfpathlineto{\pgfqpoint{4.964763in}{2.320795in}}%
\pgfusepath{stroke}%
\end{pgfscope}%
\begin{pgfscope}%
\pgfpathrectangle{\pgfqpoint{0.526127in}{0.475959in}}{\pgfqpoint{4.650000in}{2.310000in}}%
\pgfusepath{clip}%
\pgfsetbuttcap%
\pgfsetroundjoin%
\pgfsetlinewidth{1.505625pt}%
\definecolor{currentstroke}{rgb}{0.000000,0.000000,0.000000}%
\pgfsetstrokecolor{currentstroke}%
\pgfsetdash{{5.550000pt}{2.400000pt}}{0.000000pt}%
\pgfpathmoveto{\pgfqpoint{0.737490in}{2.422556in}}%
\pgfpathlineto{\pgfqpoint{0.758733in}{2.469089in}}%
\pgfpathlineto{\pgfqpoint{0.779975in}{2.511476in}}%
\pgfpathlineto{\pgfqpoint{0.801218in}{2.549506in}}%
\pgfpathlineto{\pgfqpoint{0.822461in}{2.582992in}}%
\pgfpathlineto{\pgfqpoint{0.843703in}{2.611768in}}%
\pgfpathlineto{\pgfqpoint{0.864946in}{2.635691in}}%
\pgfpathlineto{\pgfqpoint{0.886188in}{2.654644in}}%
\pgfpathlineto{\pgfqpoint{0.907431in}{2.668532in}}%
\pgfpathlineto{\pgfqpoint{0.928673in}{2.677287in}}%
\pgfpathlineto{\pgfqpoint{0.949916in}{2.680866in}}%
\pgfpathlineto{\pgfqpoint{0.971159in}{2.679250in}}%
\pgfpathlineto{\pgfqpoint{0.992401in}{2.672448in}}%
\pgfpathlineto{\pgfqpoint{1.013644in}{2.660494in}}%
\pgfpathlineto{\pgfqpoint{1.034886in}{2.643446in}}%
\pgfpathlineto{\pgfqpoint{1.056129in}{2.621389in}}%
\pgfpathlineto{\pgfqpoint{1.077372in}{2.594432in}}%
\pgfpathlineto{\pgfqpoint{1.098614in}{2.562708in}}%
\pgfpathlineto{\pgfqpoint{1.119857in}{2.526374in}}%
\pgfpathlineto{\pgfqpoint{1.141099in}{2.485611in}}%
\pgfpathlineto{\pgfqpoint{1.162342in}{2.440620in}}%
\pgfpathlineto{\pgfqpoint{1.183584in}{2.391622in}}%
\pgfpathlineto{\pgfqpoint{1.204827in}{2.338862in}}%
\pgfpathlineto{\pgfqpoint{1.226070in}{2.282599in}}%
\pgfpathlineto{\pgfqpoint{1.268555in}{2.160697in}}%
\pgfpathlineto{\pgfqpoint{1.311040in}{2.028324in}}%
\pgfpathlineto{\pgfqpoint{1.353525in}{1.888097in}}%
\pgfpathlineto{\pgfqpoint{1.417253in}{1.669123in}}%
\pgfpathlineto{\pgfqpoint{1.502223in}{1.376221in}}%
\pgfpathlineto{\pgfqpoint{1.544708in}{1.235886in}}%
\pgfpathlineto{\pgfqpoint{1.587193in}{1.103360in}}%
\pgfpathlineto{\pgfqpoint{1.629679in}{0.981261in}}%
\pgfpathlineto{\pgfqpoint{1.650921in}{0.924886in}}%
\pgfpathlineto{\pgfqpoint{1.672164in}{0.872004in}}%
\pgfpathlineto{\pgfqpoint{1.693406in}{0.822876in}}%
\pgfpathlineto{\pgfqpoint{1.714649in}{0.777747in}}%
\pgfpathlineto{\pgfqpoint{1.735891in}{0.736838in}}%
\pgfpathlineto{\pgfqpoint{1.757134in}{0.700354in}}%
\pgfpathlineto{\pgfqpoint{1.778377in}{0.668473in}}%
\pgfpathlineto{\pgfqpoint{1.799619in}{0.641354in}}%
\pgfpathlineto{\pgfqpoint{1.820862in}{0.619130in}}%
\pgfpathlineto{\pgfqpoint{1.842104in}{0.601913in}}%
\pgfpathlineto{\pgfqpoint{1.863347in}{0.589787in}}%
\pgfpathlineto{\pgfqpoint{1.884589in}{0.582811in}}%
\pgfpathlineto{\pgfqpoint{1.905832in}{0.581022in}}%
\pgfpathlineto{\pgfqpoint{1.927075in}{0.584426in}}%
\pgfpathlineto{\pgfqpoint{1.948317in}{0.593008in}}%
\pgfpathlineto{\pgfqpoint{1.969560in}{0.606726in}}%
\pgfpathlineto{\pgfqpoint{1.990802in}{0.625510in}}%
\pgfpathlineto{\pgfqpoint{2.012045in}{0.649269in}}%
\pgfpathlineto{\pgfqpoint{2.033287in}{0.677884in}}%
\pgfpathlineto{\pgfqpoint{2.054530in}{0.711215in}}%
\pgfpathlineto{\pgfqpoint{2.075773in}{0.749096in}}%
\pgfpathlineto{\pgfqpoint{2.097015in}{0.791339in}}%
\pgfpathlineto{\pgfqpoint{2.118258in}{0.837737in}}%
\pgfpathlineto{\pgfqpoint{2.139500in}{0.888059in}}%
\pgfpathlineto{\pgfqpoint{2.160743in}{0.942056in}}%
\pgfpathlineto{\pgfqpoint{2.203228in}{1.059992in}}%
\pgfpathlineto{\pgfqpoint{2.245713in}{1.189212in}}%
\pgfpathlineto{\pgfqpoint{2.288198in}{1.327164in}}%
\pgfpathlineto{\pgfqpoint{2.351926in}{1.544463in}}%
\pgfpathlineto{\pgfqpoint{2.458139in}{1.910307in}}%
\pgfpathlineto{\pgfqpoint{2.500624in}{2.049491in}}%
\pgfpathlineto{\pgfqpoint{2.543109in}{2.180403in}}%
\pgfpathlineto{\pgfqpoint{2.585594in}{2.300455in}}%
\pgfpathlineto{\pgfqpoint{2.606837in}{2.355658in}}%
\pgfpathlineto{\pgfqpoint{2.628080in}{2.407275in}}%
\pgfpathlineto{\pgfqpoint{2.649322in}{2.455052in}}%
\pgfpathlineto{\pgfqpoint{2.670565in}{2.498751in}}%
\pgfpathlineto{\pgfqpoint{2.691807in}{2.538157in}}%
\pgfpathlineto{\pgfqpoint{2.713050in}{2.573075in}}%
\pgfpathlineto{\pgfqpoint{2.734292in}{2.603332in}}%
\pgfpathlineto{\pgfqpoint{2.755535in}{2.628779in}}%
\pgfpathlineto{\pgfqpoint{2.776778in}{2.649289in}}%
\pgfpathlineto{\pgfqpoint{2.798020in}{2.664760in}}%
\pgfpathlineto{\pgfqpoint{2.819263in}{2.675118in}}%
\pgfpathlineto{\pgfqpoint{2.840505in}{2.680309in}}%
\pgfpathlineto{\pgfqpoint{2.861748in}{2.680309in}}%
\pgfpathlineto{\pgfqpoint{2.882991in}{2.675118in}}%
\pgfpathlineto{\pgfqpoint{2.904233in}{2.664760in}}%
\pgfpathlineto{\pgfqpoint{2.925476in}{2.649289in}}%
\pgfpathlineto{\pgfqpoint{2.946718in}{2.628779in}}%
\pgfpathlineto{\pgfqpoint{2.967961in}{2.603332in}}%
\pgfpathlineto{\pgfqpoint{2.989203in}{2.573075in}}%
\pgfpathlineto{\pgfqpoint{3.010446in}{2.538157in}}%
\pgfpathlineto{\pgfqpoint{3.031689in}{2.498751in}}%
\pgfpathlineto{\pgfqpoint{3.052931in}{2.455052in}}%
\pgfpathlineto{\pgfqpoint{3.074174in}{2.407275in}}%
\pgfpathlineto{\pgfqpoint{3.095416in}{2.355658in}}%
\pgfpathlineto{\pgfqpoint{3.116659in}{2.300455in}}%
\pgfpathlineto{\pgfqpoint{3.159144in}{2.180403in}}%
\pgfpathlineto{\pgfqpoint{3.201629in}{2.049491in}}%
\pgfpathlineto{\pgfqpoint{3.244114in}{1.910307in}}%
\pgfpathlineto{\pgfqpoint{3.307842in}{1.692069in}}%
\pgfpathlineto{\pgfqpoint{3.392812in}{1.398567in}}%
\pgfpathlineto{\pgfqpoint{3.435298in}{1.257264in}}%
\pgfpathlineto{\pgfqpoint{3.477783in}{1.123346in}}%
\pgfpathlineto{\pgfqpoint{3.520268in}{0.999462in}}%
\pgfpathlineto{\pgfqpoint{3.541510in}{0.942056in}}%
\pgfpathlineto{\pgfqpoint{3.562753in}{0.888059in}}%
\pgfpathlineto{\pgfqpoint{3.583996in}{0.837737in}}%
\pgfpathlineto{\pgfqpoint{3.605238in}{0.791339in}}%
\pgfpathlineto{\pgfqpoint{3.626481in}{0.749096in}}%
\pgfpathlineto{\pgfqpoint{3.647723in}{0.711215in}}%
\pgfpathlineto{\pgfqpoint{3.668966in}{0.677884in}}%
\pgfpathlineto{\pgfqpoint{3.690208in}{0.649269in}}%
\pgfpathlineto{\pgfqpoint{3.711451in}{0.625510in}}%
\pgfpathlineto{\pgfqpoint{3.732694in}{0.606726in}}%
\pgfpathlineto{\pgfqpoint{3.753936in}{0.593008in}}%
\pgfpathlineto{\pgfqpoint{3.775179in}{0.584426in}}%
\pgfpathlineto{\pgfqpoint{3.796421in}{0.581022in}}%
\pgfpathlineto{\pgfqpoint{3.817664in}{0.582811in}}%
\pgfpathlineto{\pgfqpoint{3.838906in}{0.589787in}}%
\pgfpathlineto{\pgfqpoint{3.860149in}{0.601913in}}%
\pgfpathlineto{\pgfqpoint{3.881392in}{0.619130in}}%
\pgfpathlineto{\pgfqpoint{3.902634in}{0.641354in}}%
\pgfpathlineto{\pgfqpoint{3.923877in}{0.668473in}}%
\pgfpathlineto{\pgfqpoint{3.945119in}{0.700354in}}%
\pgfpathlineto{\pgfqpoint{3.966362in}{0.736838in}}%
\pgfpathlineto{\pgfqpoint{3.987605in}{0.777747in}}%
\pgfpathlineto{\pgfqpoint{4.008847in}{0.822876in}}%
\pgfpathlineto{\pgfqpoint{4.030090in}{0.872004in}}%
\pgfpathlineto{\pgfqpoint{4.051332in}{0.924886in}}%
\pgfpathlineto{\pgfqpoint{4.072575in}{0.981261in}}%
\pgfpathlineto{\pgfqpoint{4.115060in}{1.103360in}}%
\pgfpathlineto{\pgfqpoint{4.157545in}{1.235886in}}%
\pgfpathlineto{\pgfqpoint{4.200030in}{1.376221in}}%
\pgfpathlineto{\pgfqpoint{4.263758in}{1.595269in}}%
\pgfpathlineto{\pgfqpoint{4.348728in}{1.888097in}}%
\pgfpathlineto{\pgfqpoint{4.391213in}{2.028324in}}%
\pgfpathlineto{\pgfqpoint{4.433699in}{2.160697in}}%
\pgfpathlineto{\pgfqpoint{4.476184in}{2.282599in}}%
\pgfpathlineto{\pgfqpoint{4.497426in}{2.338862in}}%
\pgfpathlineto{\pgfqpoint{4.518669in}{2.391622in}}%
\pgfpathlineto{\pgfqpoint{4.539911in}{2.440620in}}%
\pgfpathlineto{\pgfqpoint{4.561154in}{2.485611in}}%
\pgfpathlineto{\pgfqpoint{4.582397in}{2.526374in}}%
\pgfpathlineto{\pgfqpoint{4.603639in}{2.562708in}}%
\pgfpathlineto{\pgfqpoint{4.624882in}{2.594432in}}%
\pgfpathlineto{\pgfqpoint{4.646124in}{2.621389in}}%
\pgfpathlineto{\pgfqpoint{4.667367in}{2.643446in}}%
\pgfpathlineto{\pgfqpoint{4.688610in}{2.660494in}}%
\pgfpathlineto{\pgfqpoint{4.709852in}{2.672448in}}%
\pgfpathlineto{\pgfqpoint{4.731095in}{2.679250in}}%
\pgfpathlineto{\pgfqpoint{4.752337in}{2.680866in}}%
\pgfpathlineto{\pgfqpoint{4.773580in}{2.677287in}}%
\pgfpathlineto{\pgfqpoint{4.794822in}{2.668532in}}%
\pgfpathlineto{\pgfqpoint{4.816065in}{2.654644in}}%
\pgfpathlineto{\pgfqpoint{4.837308in}{2.635691in}}%
\pgfpathlineto{\pgfqpoint{4.858550in}{2.611768in}}%
\pgfpathlineto{\pgfqpoint{4.879793in}{2.582992in}}%
\pgfpathlineto{\pgfqpoint{4.901035in}{2.549506in}}%
\pgfpathlineto{\pgfqpoint{4.922278in}{2.511476in}}%
\pgfpathlineto{\pgfqpoint{4.943520in}{2.469089in}}%
\pgfpathlineto{\pgfqpoint{4.964763in}{2.422556in}}%
\pgfpathlineto{\pgfqpoint{4.964763in}{2.422556in}}%
\pgfusepath{stroke}%
\end{pgfscope}%
\begin{pgfscope}%
\pgfpathrectangle{\pgfqpoint{0.526127in}{0.475959in}}{\pgfqpoint{4.650000in}{2.310000in}}%
\pgfusepath{clip}%
\pgfsetrectcap%
\pgfsetroundjoin%
\pgfsetlinewidth{0.501875pt}%
\definecolor{currentstroke}{rgb}{0.000000,0.000000,0.000000}%
\pgfsetstrokecolor{currentstroke}%
\pgfsetdash{}{0pt}%
\pgfpathmoveto{\pgfqpoint{0.526127in}{1.630959in}}%
\pgfpathlineto{\pgfqpoint{5.176127in}{1.630959in}}%
\pgfusepath{stroke}%
\end{pgfscope}%
\begin{pgfscope}%
\pgfpathrectangle{\pgfqpoint{0.526127in}{0.475959in}}{\pgfqpoint{4.650000in}{2.310000in}}%
\pgfusepath{clip}%
\pgfsetrectcap%
\pgfsetroundjoin%
\pgfsetlinewidth{0.501875pt}%
\definecolor{currentstroke}{rgb}{0.000000,0.000000,0.000000}%
\pgfsetstrokecolor{currentstroke}%
\pgfsetdash{}{0pt}%
\pgfpathmoveto{\pgfqpoint{2.851127in}{0.475959in}}%
\pgfpathlineto{\pgfqpoint{2.851127in}{2.785959in}}%
\pgfusepath{stroke}%
\end{pgfscope}%
\begin{pgfscope}%
\pgfpathrectangle{\pgfqpoint{0.526127in}{0.475959in}}{\pgfqpoint{4.650000in}{2.310000in}}%
\pgfusepath{clip}%
\pgfsetbuttcap%
\pgfsetroundjoin%
\pgfsetlinewidth{0.501875pt}%
\definecolor{currentstroke}{rgb}{0.000000,0.000000,0.000000}%
\pgfsetstrokecolor{currentstroke}%
\pgfsetdash{{0.500000pt}{0.825000pt}}{0.000000pt}%
\pgfpathmoveto{\pgfqpoint{0.953931in}{0.580959in}}%
\pgfpathlineto{\pgfqpoint{0.953931in}{2.680959in}}%
\pgfusepath{stroke}%
\end{pgfscope}%
\begin{pgfscope}%
\pgfpathrectangle{\pgfqpoint{0.526127in}{0.475959in}}{\pgfqpoint{4.650000in}{2.310000in}}%
\pgfusepath{clip}%
\pgfsetbuttcap%
\pgfsetroundjoin%
\pgfsetlinewidth{0.501875pt}%
\definecolor{currentstroke}{rgb}{0.000000,0.000000,0.000000}%
\pgfsetstrokecolor{currentstroke}%
\pgfsetdash{{0.500000pt}{0.825000pt}}{0.000000pt}%
\pgfpathmoveto{\pgfqpoint{1.428230in}{0.580959in}}%
\pgfpathlineto{\pgfqpoint{1.428230in}{2.680959in}}%
\pgfusepath{stroke}%
\end{pgfscope}%
\begin{pgfscope}%
\pgfpathrectangle{\pgfqpoint{0.526127in}{0.475959in}}{\pgfqpoint{4.650000in}{2.310000in}}%
\pgfusepath{clip}%
\pgfsetbuttcap%
\pgfsetroundjoin%
\pgfsetlinewidth{0.501875pt}%
\definecolor{currentstroke}{rgb}{0.000000,0.000000,0.000000}%
\pgfsetstrokecolor{currentstroke}%
\pgfsetdash{{0.500000pt}{0.825000pt}}{0.000000pt}%
\pgfpathmoveto{\pgfqpoint{1.902529in}{0.580959in}}%
\pgfpathlineto{\pgfqpoint{1.902529in}{2.680959in}}%
\pgfusepath{stroke}%
\end{pgfscope}%
\begin{pgfscope}%
\pgfpathrectangle{\pgfqpoint{0.526127in}{0.475959in}}{\pgfqpoint{4.650000in}{2.310000in}}%
\pgfusepath{clip}%
\pgfsetbuttcap%
\pgfsetroundjoin%
\pgfsetlinewidth{0.501875pt}%
\definecolor{currentstroke}{rgb}{0.000000,0.000000,0.000000}%
\pgfsetstrokecolor{currentstroke}%
\pgfsetdash{{0.500000pt}{0.825000pt}}{0.000000pt}%
\pgfpathmoveto{\pgfqpoint{2.376828in}{0.580959in}}%
\pgfpathlineto{\pgfqpoint{2.376828in}{2.680959in}}%
\pgfusepath{stroke}%
\end{pgfscope}%
\begin{pgfscope}%
\pgfpathrectangle{\pgfqpoint{0.526127in}{0.475959in}}{\pgfqpoint{4.650000in}{2.310000in}}%
\pgfusepath{clip}%
\pgfsetbuttcap%
\pgfsetroundjoin%
\pgfsetlinewidth{0.501875pt}%
\definecolor{currentstroke}{rgb}{0.000000,0.000000,0.000000}%
\pgfsetstrokecolor{currentstroke}%
\pgfsetdash{{0.500000pt}{0.825000pt}}{0.000000pt}%
\pgfpathmoveto{\pgfqpoint{2.851127in}{0.580959in}}%
\pgfpathlineto{\pgfqpoint{2.851127in}{2.680959in}}%
\pgfusepath{stroke}%
\end{pgfscope}%
\begin{pgfscope}%
\pgfpathrectangle{\pgfqpoint{0.526127in}{0.475959in}}{\pgfqpoint{4.650000in}{2.310000in}}%
\pgfusepath{clip}%
\pgfsetbuttcap%
\pgfsetroundjoin%
\pgfsetlinewidth{0.501875pt}%
\definecolor{currentstroke}{rgb}{0.000000,0.000000,0.000000}%
\pgfsetstrokecolor{currentstroke}%
\pgfsetdash{{0.500000pt}{0.825000pt}}{0.000000pt}%
\pgfpathmoveto{\pgfqpoint{3.325426in}{0.580959in}}%
\pgfpathlineto{\pgfqpoint{3.325426in}{2.680959in}}%
\pgfusepath{stroke}%
\end{pgfscope}%
\begin{pgfscope}%
\pgfpathrectangle{\pgfqpoint{0.526127in}{0.475959in}}{\pgfqpoint{4.650000in}{2.310000in}}%
\pgfusepath{clip}%
\pgfsetbuttcap%
\pgfsetroundjoin%
\pgfsetlinewidth{0.501875pt}%
\definecolor{currentstroke}{rgb}{0.000000,0.000000,0.000000}%
\pgfsetstrokecolor{currentstroke}%
\pgfsetdash{{0.500000pt}{0.825000pt}}{0.000000pt}%
\pgfpathmoveto{\pgfqpoint{3.799724in}{0.580959in}}%
\pgfpathlineto{\pgfqpoint{3.799724in}{2.680959in}}%
\pgfusepath{stroke}%
\end{pgfscope}%
\begin{pgfscope}%
\pgfpathrectangle{\pgfqpoint{0.526127in}{0.475959in}}{\pgfqpoint{4.650000in}{2.310000in}}%
\pgfusepath{clip}%
\pgfsetbuttcap%
\pgfsetroundjoin%
\pgfsetlinewidth{0.501875pt}%
\definecolor{currentstroke}{rgb}{0.000000,0.000000,0.000000}%
\pgfsetstrokecolor{currentstroke}%
\pgfsetdash{{0.500000pt}{0.825000pt}}{0.000000pt}%
\pgfpathmoveto{\pgfqpoint{4.274023in}{0.580959in}}%
\pgfpathlineto{\pgfqpoint{4.274023in}{2.680959in}}%
\pgfusepath{stroke}%
\end{pgfscope}%
\begin{pgfscope}%
\pgfpathrectangle{\pgfqpoint{0.526127in}{0.475959in}}{\pgfqpoint{4.650000in}{2.310000in}}%
\pgfusepath{clip}%
\pgfsetbuttcap%
\pgfsetroundjoin%
\pgfsetlinewidth{0.501875pt}%
\definecolor{currentstroke}{rgb}{0.000000,0.000000,0.000000}%
\pgfsetstrokecolor{currentstroke}%
\pgfsetdash{{0.500000pt}{0.825000pt}}{0.000000pt}%
\pgfpathmoveto{\pgfqpoint{4.748322in}{0.580959in}}%
\pgfpathlineto{\pgfqpoint{4.748322in}{2.680959in}}%
\pgfusepath{stroke}%
\end{pgfscope}%
\begin{pgfscope}%
\pgfsetbuttcap%
\pgfsetmiterjoin%
\definecolor{currentfill}{rgb}{1.000000,1.000000,1.000000}%
\pgfsetfillcolor{currentfill}%
\pgfsetfillopacity{0.800000}%
\pgfsetlinewidth{1.003750pt}%
\definecolor{currentstroke}{rgb}{0.800000,0.800000,0.800000}%
\pgfsetstrokecolor{currentstroke}%
\pgfsetstrokeopacity{0.800000}%
\pgfsetdash{}{0pt}%
\pgfpathmoveto{\pgfqpoint{0.623349in}{0.545403in}}%
\pgfpathlineto{\pgfqpoint{1.451516in}{0.545403in}}%
\pgfpathquadraticcurveto{\pgfqpoint{1.479294in}{0.545403in}}{\pgfqpoint{1.479294in}{0.573181in}}%
\pgfpathlineto{\pgfqpoint{1.479294in}{0.978672in}}%
\pgfpathquadraticcurveto{\pgfqpoint{1.479294in}{1.006449in}}{\pgfqpoint{1.451516in}{1.006449in}}%
\pgfpathlineto{\pgfqpoint{0.623349in}{1.006449in}}%
\pgfpathquadraticcurveto{\pgfqpoint{0.595571in}{1.006449in}}{\pgfqpoint{0.595571in}{0.978672in}}%
\pgfpathlineto{\pgfqpoint{0.595571in}{0.573181in}}%
\pgfpathquadraticcurveto{\pgfqpoint{0.595571in}{0.545403in}}{\pgfqpoint{0.623349in}{0.545403in}}%
\pgfpathlineto{\pgfqpoint{0.623349in}{0.545403in}}%
\pgfpathclose%
\pgfusepath{stroke,fill}%
\end{pgfscope}%
\begin{pgfscope}%
\pgfsetrectcap%
\pgfsetroundjoin%
\pgfsetlinewidth{1.505625pt}%
\definecolor{currentstroke}{rgb}{0.000000,0.000000,0.000000}%
\pgfsetstrokecolor{currentstroke}%
\pgfsetdash{}{0pt}%
\pgfpathmoveto{\pgfqpoint{0.651127in}{0.893982in}}%
\pgfpathlineto{\pgfqpoint{0.790016in}{0.893982in}}%
\pgfpathlineto{\pgfqpoint{0.928904in}{0.893982in}}%
\pgfusepath{stroke}%
\end{pgfscope}%
\begin{pgfscope}%
\definecolor{textcolor}{rgb}{0.000000,0.000000,0.000000}%
\pgfsetstrokecolor{textcolor}%
\pgfsetfillcolor{textcolor}%
\pgftext[x=1.040016in,y=0.845371in,left,base]{\color{textcolor}{\rmfamily\fontsize{10.000000}{12.000000}\selectfont\catcode`\^=\active\def^{\ifmmode\sp\else\^{}\fi}\catcode`\%=\active\def%{\%}$sin(x)$}}%
\end{pgfscope}%
\begin{pgfscope}%
\pgfsetbuttcap%
\pgfsetroundjoin%
\pgfsetlinewidth{1.505625pt}%
\definecolor{currentstroke}{rgb}{0.000000,0.000000,0.000000}%
\pgfsetstrokecolor{currentstroke}%
\pgfsetdash{{5.550000pt}{2.400000pt}}{0.000000pt}%
\pgfpathmoveto{\pgfqpoint{0.651127in}{0.684292in}}%
\pgfpathlineto{\pgfqpoint{0.790016in}{0.684292in}}%
\pgfpathlineto{\pgfqpoint{0.928904in}{0.684292in}}%
\pgfusepath{stroke}%
\end{pgfscope}%
\begin{pgfscope}%
\definecolor{textcolor}{rgb}{0.000000,0.000000,0.000000}%
\pgfsetstrokecolor{textcolor}%
\pgfsetfillcolor{textcolor}%
\pgftext[x=1.040016in,y=0.635681in,left,base]{\color{textcolor}{\rmfamily\fontsize{10.000000}{12.000000}\selectfont\catcode`\^=\active\def^{\ifmmode\sp\else\^{}\fi}\catcode`\%=\active\def%{\%}$cos(x)$}}%
\end{pgfscope}%
\end{pgfpicture}%
\makeatother%
\endgroup%

	\caption{Sinus und Kosinus Funktionsverlauf.}
	\label{fig:sincos}
\end{figure}
\todo[]{x achse beschriftung fehlt.}


\begin{figure}[H]
	\centering
	%% Creator: Matplotlib, PGF backend
%%
%% To include the figure in your LaTeX document, write
%%   \input{<filename>.pgf}
%%
%% Make sure the required packages are loaded in your preamble
%%   \usepackage{pgf}
%%
%% Also ensure that all the required font packages are loaded; for instance,
%% the lmodern package is sometimes necessary when using math font.
%%   \usepackage{lmodern}
%%
%% Figures using additional raster images can only be included by \input if
%% they are in the same directory as the main LaTeX file. For loading figures
%% from other directories you can use the `import` package
%%   \usepackage{import}
%%
%% and then include the figures with
%%   \import{<path to file>}{<filename>.pgf}
%%
%% Matplotlib used the following preamble
%%   \def\mathdefault#1{#1}
%%   \everymath=\expandafter{\the\everymath\displaystyle}
%%   
%%   \usepackage{fontspec}
%%   \setmainfont{VeraSe.ttf}[Path=\detokenize{/usr/share/fonts/TTF/}]
%%   \setsansfont{DejaVuSans.ttf}[Path=\detokenize{/home/pl/miniconda3/lib/python3.12/site-packages/matplotlib/mpl-data/fonts/ttf/}]
%%   \setmonofont{DejaVuSansMono.ttf}[Path=\detokenize{/home/pl/miniconda3/lib/python3.12/site-packages/matplotlib/mpl-data/fonts/ttf/}]
%%   \makeatletter\@ifpackageloaded{underscore}{}{\usepackage[strings]{underscore}}\makeatother
%%
\begingroup%
\makeatletter%
\begin{pgfpicture}%
\pgfpathrectangle{\pgfpointorigin}{\pgfqpoint{5.364492in}{3.178933in}}%
\pgfusepath{use as bounding box, clip}%
\begin{pgfscope}%
\pgfsetbuttcap%
\pgfsetmiterjoin%
\definecolor{currentfill}{rgb}{1.000000,1.000000,1.000000}%
\pgfsetfillcolor{currentfill}%
\pgfsetlinewidth{0.000000pt}%
\definecolor{currentstroke}{rgb}{1.000000,1.000000,1.000000}%
\pgfsetstrokecolor{currentstroke}%
\pgfsetdash{}{0pt}%
\pgfpathmoveto{\pgfqpoint{0.000000in}{0.000000in}}%
\pgfpathlineto{\pgfqpoint{5.364492in}{0.000000in}}%
\pgfpathlineto{\pgfqpoint{5.364492in}{3.178933in}}%
\pgfpathlineto{\pgfqpoint{0.000000in}{3.178933in}}%
\pgfpathlineto{\pgfqpoint{0.000000in}{0.000000in}}%
\pgfpathclose%
\pgfusepath{fill}%
\end{pgfscope}%
\begin{pgfscope}%
\pgfsetbuttcap%
\pgfsetmiterjoin%
\definecolor{currentfill}{rgb}{1.000000,1.000000,1.000000}%
\pgfsetfillcolor{currentfill}%
\pgfsetlinewidth{0.000000pt}%
\definecolor{currentstroke}{rgb}{0.000000,0.000000,0.000000}%
\pgfsetstrokecolor{currentstroke}%
\pgfsetstrokeopacity{0.000000}%
\pgfsetdash{}{0pt}%
\pgfpathmoveto{\pgfqpoint{0.614492in}{0.548486in}}%
\pgfpathlineto{\pgfqpoint{5.264492in}{0.548486in}}%
\pgfpathlineto{\pgfqpoint{5.264492in}{2.858486in}}%
\pgfpathlineto{\pgfqpoint{0.614492in}{2.858486in}}%
\pgfpathlineto{\pgfqpoint{0.614492in}{0.548486in}}%
\pgfpathclose%
\pgfusepath{fill}%
\end{pgfscope}%
\begin{pgfscope}%
\pgfpathrectangle{\pgfqpoint{0.614492in}{0.548486in}}{\pgfqpoint{4.650000in}{2.310000in}}%
\pgfusepath{clip}%
\pgfsetroundcap%
\pgfsetroundjoin%
\pgfsetlinewidth{0.501875pt}%
\definecolor{currentstroke}{rgb}{0.000000,0.000000,0.000000}%
\pgfsetstrokecolor{currentstroke}%
\pgfsetdash{}{0pt}%
\pgfpathmoveto{\pgfqpoint{1.037219in}{1.737048in}}%
\pgfpathquadraticcurveto{\pgfqpoint{1.037219in}{2.165493in}}{\pgfqpoint{1.037219in}{2.593938in}}%
\pgfusepath{stroke}%
\end{pgfscope}%
\begin{pgfscope}%
\pgfpathrectangle{\pgfqpoint{0.614492in}{0.548486in}}{\pgfqpoint{4.650000in}{2.310000in}}%
\pgfusepath{clip}%
\pgfsetroundcap%
\pgfsetroundjoin%
\pgfsetlinewidth{0.501875pt}%
\definecolor{currentstroke}{rgb}{0.000000,0.000000,0.000000}%
\pgfsetstrokecolor{currentstroke}%
\pgfsetdash{}{0pt}%
\pgfpathmoveto{\pgfqpoint{1.078886in}{1.792604in}}%
\pgfpathlineto{\pgfqpoint{1.037219in}{1.737048in}}%
\pgfpathlineto{\pgfqpoint{0.995553in}{1.792604in}}%
\pgfusepath{stroke}%
\end{pgfscope}%
\begin{pgfscope}%
\pgfpathrectangle{\pgfqpoint{0.614492in}{0.548486in}}{\pgfqpoint{4.650000in}{2.310000in}}%
\pgfusepath{clip}%
\pgfsetroundcap%
\pgfsetroundjoin%
\pgfsetlinewidth{0.501875pt}%
\definecolor{currentstroke}{rgb}{0.000000,0.000000,0.000000}%
\pgfsetstrokecolor{currentstroke}%
\pgfsetdash{}{0pt}%
\pgfpathmoveto{\pgfqpoint{0.995553in}{2.538383in}}%
\pgfpathlineto{\pgfqpoint{1.037219in}{2.593938in}}%
\pgfpathlineto{\pgfqpoint{1.078886in}{2.538383in}}%
\pgfusepath{stroke}%
\end{pgfscope}%
\begin{pgfscope}%
\pgfpathrectangle{\pgfqpoint{0.614492in}{0.548486in}}{\pgfqpoint{4.650000in}{2.310000in}}%
\pgfusepath{clip}%
\pgfsetroundcap%
\pgfsetroundjoin%
\pgfsetlinewidth{0.501875pt}%
\definecolor{currentstroke}{rgb}{0.000000,0.000000,0.000000}%
\pgfsetstrokecolor{currentstroke}%
\pgfsetdash{}{0pt}%
\pgfpathmoveto{\pgfqpoint{2.412372in}{1.299236in}}%
\pgfpathquadraticcurveto{\pgfqpoint{2.659134in}{1.299236in}}{\pgfqpoint{2.905895in}{1.299236in}}%
\pgfusepath{stroke}%
\end{pgfscope}%
\begin{pgfscope}%
\pgfpathrectangle{\pgfqpoint{0.614492in}{0.548486in}}{\pgfqpoint{4.650000in}{2.310000in}}%
\pgfusepath{clip}%
\pgfsetroundcap%
\pgfsetroundjoin%
\pgfsetlinewidth{0.501875pt}%
\definecolor{currentstroke}{rgb}{0.000000,0.000000,0.000000}%
\pgfsetstrokecolor{currentstroke}%
\pgfsetdash{}{0pt}%
\pgfpathmoveto{\pgfqpoint{2.467928in}{1.257569in}}%
\pgfpathlineto{\pgfqpoint{2.412372in}{1.299236in}}%
\pgfpathlineto{\pgfqpoint{2.467928in}{1.340903in}}%
\pgfusepath{stroke}%
\end{pgfscope}%
\begin{pgfscope}%
\pgfpathrectangle{\pgfqpoint{0.614492in}{0.548486in}}{\pgfqpoint{4.650000in}{2.310000in}}%
\pgfusepath{clip}%
\pgfsetroundcap%
\pgfsetroundjoin%
\pgfsetlinewidth{0.501875pt}%
\definecolor{currentstroke}{rgb}{0.000000,0.000000,0.000000}%
\pgfsetstrokecolor{currentstroke}%
\pgfsetdash{}{0pt}%
\pgfpathmoveto{\pgfqpoint{2.850340in}{1.340903in}}%
\pgfpathlineto{\pgfqpoint{2.905895in}{1.299236in}}%
\pgfpathlineto{\pgfqpoint{2.850340in}{1.257569in}}%
\pgfusepath{stroke}%
\end{pgfscope}%
\begin{pgfscope}%
\pgfpathrectangle{\pgfqpoint{0.614492in}{0.548486in}}{\pgfqpoint{4.650000in}{2.310000in}}%
\pgfusepath{clip}%
\pgfsetroundcap%
\pgfsetroundjoin%
\pgfsetlinewidth{0.501875pt}%
\definecolor{currentstroke}{rgb}{0.000000,0.000000,0.000000}%
\pgfsetstrokecolor{currentstroke}%
\pgfsetdash{}{0pt}%
\pgfpathmoveto{\pgfqpoint{1.091360in}{2.719886in}}%
\pgfpathquadraticcurveto{\pgfqpoint{1.938472in}{2.719886in}}{\pgfqpoint{2.785585in}{2.719886in}}%
\pgfusepath{stroke}%
\end{pgfscope}%
\begin{pgfscope}%
\pgfpathrectangle{\pgfqpoint{0.614492in}{0.548486in}}{\pgfqpoint{4.650000in}{2.310000in}}%
\pgfusepath{clip}%
\pgfsetroundcap%
\pgfsetroundjoin%
\pgfsetlinewidth{0.501875pt}%
\definecolor{currentstroke}{rgb}{0.000000,0.000000,0.000000}%
\pgfsetstrokecolor{currentstroke}%
\pgfsetdash{}{0pt}%
\pgfpathmoveto{\pgfqpoint{1.146916in}{2.678219in}}%
\pgfpathlineto{\pgfqpoint{1.091360in}{2.719886in}}%
\pgfpathlineto{\pgfqpoint{1.146916in}{2.761553in}}%
\pgfusepath{stroke}%
\end{pgfscope}%
\begin{pgfscope}%
\pgfpathrectangle{\pgfqpoint{0.614492in}{0.548486in}}{\pgfqpoint{4.650000in}{2.310000in}}%
\pgfusepath{clip}%
\pgfsetroundcap%
\pgfsetroundjoin%
\pgfsetlinewidth{0.501875pt}%
\definecolor{currentstroke}{rgb}{0.000000,0.000000,0.000000}%
\pgfsetstrokecolor{currentstroke}%
\pgfsetdash{}{0pt}%
\pgfpathmoveto{\pgfqpoint{2.730029in}{2.761553in}}%
\pgfpathlineto{\pgfqpoint{2.785585in}{2.719886in}}%
\pgfpathlineto{\pgfqpoint{2.730029in}{2.678219in}}%
\pgfusepath{stroke}%
\end{pgfscope}%
\begin{pgfscope}%
\pgfsetbuttcap%
\pgfsetroundjoin%
\definecolor{currentfill}{rgb}{0.000000,0.000000,0.000000}%
\pgfsetfillcolor{currentfill}%
\pgfsetlinewidth{0.803000pt}%
\definecolor{currentstroke}{rgb}{0.000000,0.000000,0.000000}%
\pgfsetstrokecolor{currentstroke}%
\pgfsetdash{}{0pt}%
\pgfsys@defobject{currentmarker}{\pgfqpoint{0.000000in}{-0.048611in}}{\pgfqpoint{0.000000in}{0.000000in}}{%
\pgfpathmoveto{\pgfqpoint{0.000000in}{0.000000in}}%
\pgfpathlineto{\pgfqpoint{0.000000in}{-0.048611in}}%
\pgfusepath{stroke,fill}%
}%
\begin{pgfscope}%
\pgfsys@transformshift{0.825856in}{0.548486in}%
\pgfsys@useobject{currentmarker}{}%
\end{pgfscope}%
\end{pgfscope}%
\begin{pgfscope}%
\definecolor{textcolor}{rgb}{0.000000,0.000000,0.000000}%
\pgfsetstrokecolor{textcolor}%
\pgfsetfillcolor{textcolor}%
\pgftext[x=0.825856in,y=0.451264in,,top]{\color{textcolor}{\rmfamily\fontsize{10.000000}{12.000000}\selectfont\catcode`\^=\active\def^{\ifmmode\sp\else\^{}\fi}\catcode`\%=\active\def%{\%}\ensuremath{-}2.0}}%
\end{pgfscope}%
\begin{pgfscope}%
\pgfsetbuttcap%
\pgfsetroundjoin%
\definecolor{currentfill}{rgb}{0.000000,0.000000,0.000000}%
\pgfsetfillcolor{currentfill}%
\pgfsetlinewidth{0.803000pt}%
\definecolor{currentstroke}{rgb}{0.000000,0.000000,0.000000}%
\pgfsetstrokecolor{currentstroke}%
\pgfsetdash{}{0pt}%
\pgfsys@defobject{currentmarker}{\pgfqpoint{0.000000in}{-0.048611in}}{\pgfqpoint{0.000000in}{0.000000in}}{%
\pgfpathmoveto{\pgfqpoint{0.000000in}{0.000000in}}%
\pgfpathlineto{\pgfqpoint{0.000000in}{-0.048611in}}%
\pgfusepath{stroke,fill}%
}%
\begin{pgfscope}%
\pgfsys@transformshift{1.354265in}{0.548486in}%
\pgfsys@useobject{currentmarker}{}%
\end{pgfscope}%
\end{pgfscope}%
\begin{pgfscope}%
\definecolor{textcolor}{rgb}{0.000000,0.000000,0.000000}%
\pgfsetstrokecolor{textcolor}%
\pgfsetfillcolor{textcolor}%
\pgftext[x=1.354265in,y=0.451264in,,top]{\color{textcolor}{\rmfamily\fontsize{10.000000}{12.000000}\selectfont\catcode`\^=\active\def^{\ifmmode\sp\else\^{}\fi}\catcode`\%=\active\def%{\%}\ensuremath{-}1.5}}%
\end{pgfscope}%
\begin{pgfscope}%
\pgfsetbuttcap%
\pgfsetroundjoin%
\definecolor{currentfill}{rgb}{0.000000,0.000000,0.000000}%
\pgfsetfillcolor{currentfill}%
\pgfsetlinewidth{0.803000pt}%
\definecolor{currentstroke}{rgb}{0.000000,0.000000,0.000000}%
\pgfsetstrokecolor{currentstroke}%
\pgfsetdash{}{0pt}%
\pgfsys@defobject{currentmarker}{\pgfqpoint{0.000000in}{-0.048611in}}{\pgfqpoint{0.000000in}{0.000000in}}{%
\pgfpathmoveto{\pgfqpoint{0.000000in}{0.000000in}}%
\pgfpathlineto{\pgfqpoint{0.000000in}{-0.048611in}}%
\pgfusepath{stroke,fill}%
}%
\begin{pgfscope}%
\pgfsys@transformshift{1.882674in}{0.548486in}%
\pgfsys@useobject{currentmarker}{}%
\end{pgfscope}%
\end{pgfscope}%
\begin{pgfscope}%
\definecolor{textcolor}{rgb}{0.000000,0.000000,0.000000}%
\pgfsetstrokecolor{textcolor}%
\pgfsetfillcolor{textcolor}%
\pgftext[x=1.882674in,y=0.451264in,,top]{\color{textcolor}{\rmfamily\fontsize{10.000000}{12.000000}\selectfont\catcode`\^=\active\def^{\ifmmode\sp\else\^{}\fi}\catcode`\%=\active\def%{\%}\ensuremath{-}1.0}}%
\end{pgfscope}%
\begin{pgfscope}%
\pgfsetbuttcap%
\pgfsetroundjoin%
\definecolor{currentfill}{rgb}{0.000000,0.000000,0.000000}%
\pgfsetfillcolor{currentfill}%
\pgfsetlinewidth{0.803000pt}%
\definecolor{currentstroke}{rgb}{0.000000,0.000000,0.000000}%
\pgfsetstrokecolor{currentstroke}%
\pgfsetdash{}{0pt}%
\pgfsys@defobject{currentmarker}{\pgfqpoint{0.000000in}{-0.048611in}}{\pgfqpoint{0.000000in}{0.000000in}}{%
\pgfpathmoveto{\pgfqpoint{0.000000in}{0.000000in}}%
\pgfpathlineto{\pgfqpoint{0.000000in}{-0.048611in}}%
\pgfusepath{stroke,fill}%
}%
\begin{pgfscope}%
\pgfsys@transformshift{2.411083in}{0.548486in}%
\pgfsys@useobject{currentmarker}{}%
\end{pgfscope}%
\end{pgfscope}%
\begin{pgfscope}%
\definecolor{textcolor}{rgb}{0.000000,0.000000,0.000000}%
\pgfsetstrokecolor{textcolor}%
\pgfsetfillcolor{textcolor}%
\pgftext[x=2.411083in,y=0.451264in,,top]{\color{textcolor}{\rmfamily\fontsize{10.000000}{12.000000}\selectfont\catcode`\^=\active\def^{\ifmmode\sp\else\^{}\fi}\catcode`\%=\active\def%{\%}\ensuremath{-}0.5}}%
\end{pgfscope}%
\begin{pgfscope}%
\pgfsetbuttcap%
\pgfsetroundjoin%
\definecolor{currentfill}{rgb}{0.000000,0.000000,0.000000}%
\pgfsetfillcolor{currentfill}%
\pgfsetlinewidth{0.803000pt}%
\definecolor{currentstroke}{rgb}{0.000000,0.000000,0.000000}%
\pgfsetstrokecolor{currentstroke}%
\pgfsetdash{}{0pt}%
\pgfsys@defobject{currentmarker}{\pgfqpoint{0.000000in}{-0.048611in}}{\pgfqpoint{0.000000in}{0.000000in}}{%
\pgfpathmoveto{\pgfqpoint{0.000000in}{0.000000in}}%
\pgfpathlineto{\pgfqpoint{0.000000in}{-0.048611in}}%
\pgfusepath{stroke,fill}%
}%
\begin{pgfscope}%
\pgfsys@transformshift{2.939492in}{0.548486in}%
\pgfsys@useobject{currentmarker}{}%
\end{pgfscope}%
\end{pgfscope}%
\begin{pgfscope}%
\definecolor{textcolor}{rgb}{0.000000,0.000000,0.000000}%
\pgfsetstrokecolor{textcolor}%
\pgfsetfillcolor{textcolor}%
\pgftext[x=2.939492in,y=0.451264in,,top]{\color{textcolor}{\rmfamily\fontsize{10.000000}{12.000000}\selectfont\catcode`\^=\active\def^{\ifmmode\sp\else\^{}\fi}\catcode`\%=\active\def%{\%}0.0}}%
\end{pgfscope}%
\begin{pgfscope}%
\pgfsetbuttcap%
\pgfsetroundjoin%
\definecolor{currentfill}{rgb}{0.000000,0.000000,0.000000}%
\pgfsetfillcolor{currentfill}%
\pgfsetlinewidth{0.803000pt}%
\definecolor{currentstroke}{rgb}{0.000000,0.000000,0.000000}%
\pgfsetstrokecolor{currentstroke}%
\pgfsetdash{}{0pt}%
\pgfsys@defobject{currentmarker}{\pgfqpoint{0.000000in}{-0.048611in}}{\pgfqpoint{0.000000in}{0.000000in}}{%
\pgfpathmoveto{\pgfqpoint{0.000000in}{0.000000in}}%
\pgfpathlineto{\pgfqpoint{0.000000in}{-0.048611in}}%
\pgfusepath{stroke,fill}%
}%
\begin{pgfscope}%
\pgfsys@transformshift{3.467901in}{0.548486in}%
\pgfsys@useobject{currentmarker}{}%
\end{pgfscope}%
\end{pgfscope}%
\begin{pgfscope}%
\definecolor{textcolor}{rgb}{0.000000,0.000000,0.000000}%
\pgfsetstrokecolor{textcolor}%
\pgfsetfillcolor{textcolor}%
\pgftext[x=3.467901in,y=0.451264in,,top]{\color{textcolor}{\rmfamily\fontsize{10.000000}{12.000000}\selectfont\catcode`\^=\active\def^{\ifmmode\sp\else\^{}\fi}\catcode`\%=\active\def%{\%}0.5}}%
\end{pgfscope}%
\begin{pgfscope}%
\pgfsetbuttcap%
\pgfsetroundjoin%
\definecolor{currentfill}{rgb}{0.000000,0.000000,0.000000}%
\pgfsetfillcolor{currentfill}%
\pgfsetlinewidth{0.803000pt}%
\definecolor{currentstroke}{rgb}{0.000000,0.000000,0.000000}%
\pgfsetstrokecolor{currentstroke}%
\pgfsetdash{}{0pt}%
\pgfsys@defobject{currentmarker}{\pgfqpoint{0.000000in}{-0.048611in}}{\pgfqpoint{0.000000in}{0.000000in}}{%
\pgfpathmoveto{\pgfqpoint{0.000000in}{0.000000in}}%
\pgfpathlineto{\pgfqpoint{0.000000in}{-0.048611in}}%
\pgfusepath{stroke,fill}%
}%
\begin{pgfscope}%
\pgfsys@transformshift{3.996310in}{0.548486in}%
\pgfsys@useobject{currentmarker}{}%
\end{pgfscope}%
\end{pgfscope}%
\begin{pgfscope}%
\definecolor{textcolor}{rgb}{0.000000,0.000000,0.000000}%
\pgfsetstrokecolor{textcolor}%
\pgfsetfillcolor{textcolor}%
\pgftext[x=3.996310in,y=0.451264in,,top]{\color{textcolor}{\rmfamily\fontsize{10.000000}{12.000000}\selectfont\catcode`\^=\active\def^{\ifmmode\sp\else\^{}\fi}\catcode`\%=\active\def%{\%}1.0}}%
\end{pgfscope}%
\begin{pgfscope}%
\pgfsetbuttcap%
\pgfsetroundjoin%
\definecolor{currentfill}{rgb}{0.000000,0.000000,0.000000}%
\pgfsetfillcolor{currentfill}%
\pgfsetlinewidth{0.803000pt}%
\definecolor{currentstroke}{rgb}{0.000000,0.000000,0.000000}%
\pgfsetstrokecolor{currentstroke}%
\pgfsetdash{}{0pt}%
\pgfsys@defobject{currentmarker}{\pgfqpoint{0.000000in}{-0.048611in}}{\pgfqpoint{0.000000in}{0.000000in}}{%
\pgfpathmoveto{\pgfqpoint{0.000000in}{0.000000in}}%
\pgfpathlineto{\pgfqpoint{0.000000in}{-0.048611in}}%
\pgfusepath{stroke,fill}%
}%
\begin{pgfscope}%
\pgfsys@transformshift{4.524719in}{0.548486in}%
\pgfsys@useobject{currentmarker}{}%
\end{pgfscope}%
\end{pgfscope}%
\begin{pgfscope}%
\definecolor{textcolor}{rgb}{0.000000,0.000000,0.000000}%
\pgfsetstrokecolor{textcolor}%
\pgfsetfillcolor{textcolor}%
\pgftext[x=4.524719in,y=0.451264in,,top]{\color{textcolor}{\rmfamily\fontsize{10.000000}{12.000000}\selectfont\catcode`\^=\active\def^{\ifmmode\sp\else\^{}\fi}\catcode`\%=\active\def%{\%}1.5}}%
\end{pgfscope}%
\begin{pgfscope}%
\pgfsetbuttcap%
\pgfsetroundjoin%
\definecolor{currentfill}{rgb}{0.000000,0.000000,0.000000}%
\pgfsetfillcolor{currentfill}%
\pgfsetlinewidth{0.803000pt}%
\definecolor{currentstroke}{rgb}{0.000000,0.000000,0.000000}%
\pgfsetstrokecolor{currentstroke}%
\pgfsetdash{}{0pt}%
\pgfsys@defobject{currentmarker}{\pgfqpoint{0.000000in}{-0.048611in}}{\pgfqpoint{0.000000in}{0.000000in}}{%
\pgfpathmoveto{\pgfqpoint{0.000000in}{0.000000in}}%
\pgfpathlineto{\pgfqpoint{0.000000in}{-0.048611in}}%
\pgfusepath{stroke,fill}%
}%
\begin{pgfscope}%
\pgfsys@transformshift{5.053128in}{0.548486in}%
\pgfsys@useobject{currentmarker}{}%
\end{pgfscope}%
\end{pgfscope}%
\begin{pgfscope}%
\definecolor{textcolor}{rgb}{0.000000,0.000000,0.000000}%
\pgfsetstrokecolor{textcolor}%
\pgfsetfillcolor{textcolor}%
\pgftext[x=5.053128in,y=0.451264in,,top]{\color{textcolor}{\rmfamily\fontsize{10.000000}{12.000000}\selectfont\catcode`\^=\active\def^{\ifmmode\sp\else\^{}\fi}\catcode`\%=\active\def%{\%}2.0}}%
\end{pgfscope}%
\begin{pgfscope}%
\definecolor{textcolor}{rgb}{0.000000,0.000000,0.000000}%
\pgfsetstrokecolor{textcolor}%
\pgfsetfillcolor{textcolor}%
\pgftext[x=2.939492in,y=0.261295in,,top]{\color{textcolor}{\rmfamily\fontsize{12.000000}{14.400000}\selectfont\catcode`\^=\active\def^{\ifmmode\sp\else\^{}\fi}\catcode`\%=\active\def%{\%}$t$}}%
\end{pgfscope}%
\begin{pgfscope}%
\pgfsetbuttcap%
\pgfsetroundjoin%
\definecolor{currentfill}{rgb}{0.000000,0.000000,0.000000}%
\pgfsetfillcolor{currentfill}%
\pgfsetlinewidth{0.803000pt}%
\definecolor{currentstroke}{rgb}{0.000000,0.000000,0.000000}%
\pgfsetstrokecolor{currentstroke}%
\pgfsetdash{}{0pt}%
\pgfsys@defobject{currentmarker}{\pgfqpoint{-0.048611in}{0.000000in}}{\pgfqpoint{-0.000000in}{0.000000in}}{%
\pgfpathmoveto{\pgfqpoint{-0.000000in}{0.000000in}}%
\pgfpathlineto{\pgfqpoint{-0.048611in}{0.000000in}}%
\pgfusepath{stroke,fill}%
}%
\begin{pgfscope}%
\pgfsys@transformshift{0.614492in}{0.548486in}%
\pgfsys@useobject{currentmarker}{}%
\end{pgfscope}%
\end{pgfscope}%
\begin{pgfscope}%
\definecolor{textcolor}{rgb}{0.000000,0.000000,0.000000}%
\pgfsetstrokecolor{textcolor}%
\pgfsetfillcolor{textcolor}%
\pgftext[x=0.100000in, y=0.495724in, left, base]{\color{textcolor}{\rmfamily\fontsize{10.000000}{12.000000}\selectfont\catcode`\^=\active\def^{\ifmmode\sp\else\^{}\fi}\catcode`\%=\active\def%{\%}\ensuremath{-}1.00}}%
\end{pgfscope}%
\begin{pgfscope}%
\pgfsetbuttcap%
\pgfsetroundjoin%
\definecolor{currentfill}{rgb}{0.000000,0.000000,0.000000}%
\pgfsetfillcolor{currentfill}%
\pgfsetlinewidth{0.803000pt}%
\definecolor{currentstroke}{rgb}{0.000000,0.000000,0.000000}%
\pgfsetstrokecolor{currentstroke}%
\pgfsetdash{}{0pt}%
\pgfsys@defobject{currentmarker}{\pgfqpoint{-0.048611in}{0.000000in}}{\pgfqpoint{-0.000000in}{0.000000in}}{%
\pgfpathmoveto{\pgfqpoint{-0.000000in}{0.000000in}}%
\pgfpathlineto{\pgfqpoint{-0.048611in}{0.000000in}}%
\pgfusepath{stroke,fill}%
}%
\begin{pgfscope}%
\pgfsys@transformshift{0.614492in}{0.837236in}%
\pgfsys@useobject{currentmarker}{}%
\end{pgfscope}%
\end{pgfscope}%
\begin{pgfscope}%
\definecolor{textcolor}{rgb}{0.000000,0.000000,0.000000}%
\pgfsetstrokecolor{textcolor}%
\pgfsetfillcolor{textcolor}%
\pgftext[x=0.100000in, y=0.784474in, left, base]{\color{textcolor}{\rmfamily\fontsize{10.000000}{12.000000}\selectfont\catcode`\^=\active\def^{\ifmmode\sp\else\^{}\fi}\catcode`\%=\active\def%{\%}\ensuremath{-}0.75}}%
\end{pgfscope}%
\begin{pgfscope}%
\pgfsetbuttcap%
\pgfsetroundjoin%
\definecolor{currentfill}{rgb}{0.000000,0.000000,0.000000}%
\pgfsetfillcolor{currentfill}%
\pgfsetlinewidth{0.803000pt}%
\definecolor{currentstroke}{rgb}{0.000000,0.000000,0.000000}%
\pgfsetstrokecolor{currentstroke}%
\pgfsetdash{}{0pt}%
\pgfsys@defobject{currentmarker}{\pgfqpoint{-0.048611in}{0.000000in}}{\pgfqpoint{-0.000000in}{0.000000in}}{%
\pgfpathmoveto{\pgfqpoint{-0.000000in}{0.000000in}}%
\pgfpathlineto{\pgfqpoint{-0.048611in}{0.000000in}}%
\pgfusepath{stroke,fill}%
}%
\begin{pgfscope}%
\pgfsys@transformshift{0.614492in}{1.125986in}%
\pgfsys@useobject{currentmarker}{}%
\end{pgfscope}%
\end{pgfscope}%
\begin{pgfscope}%
\definecolor{textcolor}{rgb}{0.000000,0.000000,0.000000}%
\pgfsetstrokecolor{textcolor}%
\pgfsetfillcolor{textcolor}%
\pgftext[x=0.100000in, y=1.073224in, left, base]{\color{textcolor}{\rmfamily\fontsize{10.000000}{12.000000}\selectfont\catcode`\^=\active\def^{\ifmmode\sp\else\^{}\fi}\catcode`\%=\active\def%{\%}\ensuremath{-}0.50}}%
\end{pgfscope}%
\begin{pgfscope}%
\pgfsetbuttcap%
\pgfsetroundjoin%
\definecolor{currentfill}{rgb}{0.000000,0.000000,0.000000}%
\pgfsetfillcolor{currentfill}%
\pgfsetlinewidth{0.803000pt}%
\definecolor{currentstroke}{rgb}{0.000000,0.000000,0.000000}%
\pgfsetstrokecolor{currentstroke}%
\pgfsetdash{}{0pt}%
\pgfsys@defobject{currentmarker}{\pgfqpoint{-0.048611in}{0.000000in}}{\pgfqpoint{-0.000000in}{0.000000in}}{%
\pgfpathmoveto{\pgfqpoint{-0.000000in}{0.000000in}}%
\pgfpathlineto{\pgfqpoint{-0.048611in}{0.000000in}}%
\pgfusepath{stroke,fill}%
}%
\begin{pgfscope}%
\pgfsys@transformshift{0.614492in}{1.414736in}%
\pgfsys@useobject{currentmarker}{}%
\end{pgfscope}%
\end{pgfscope}%
\begin{pgfscope}%
\definecolor{textcolor}{rgb}{0.000000,0.000000,0.000000}%
\pgfsetstrokecolor{textcolor}%
\pgfsetfillcolor{textcolor}%
\pgftext[x=0.100000in, y=1.361974in, left, base]{\color{textcolor}{\rmfamily\fontsize{10.000000}{12.000000}\selectfont\catcode`\^=\active\def^{\ifmmode\sp\else\^{}\fi}\catcode`\%=\active\def%{\%}\ensuremath{-}0.25}}%
\end{pgfscope}%
\begin{pgfscope}%
\pgfsetbuttcap%
\pgfsetroundjoin%
\definecolor{currentfill}{rgb}{0.000000,0.000000,0.000000}%
\pgfsetfillcolor{currentfill}%
\pgfsetlinewidth{0.803000pt}%
\definecolor{currentstroke}{rgb}{0.000000,0.000000,0.000000}%
\pgfsetstrokecolor{currentstroke}%
\pgfsetdash{}{0pt}%
\pgfsys@defobject{currentmarker}{\pgfqpoint{-0.048611in}{0.000000in}}{\pgfqpoint{-0.000000in}{0.000000in}}{%
\pgfpathmoveto{\pgfqpoint{-0.000000in}{0.000000in}}%
\pgfpathlineto{\pgfqpoint{-0.048611in}{0.000000in}}%
\pgfusepath{stroke,fill}%
}%
\begin{pgfscope}%
\pgfsys@transformshift{0.614492in}{1.703486in}%
\pgfsys@useobject{currentmarker}{}%
\end{pgfscope}%
\end{pgfscope}%
\begin{pgfscope}%
\definecolor{textcolor}{rgb}{0.000000,0.000000,0.000000}%
\pgfsetstrokecolor{textcolor}%
\pgfsetfillcolor{textcolor}%
\pgftext[x=0.208025in, y=1.650724in, left, base]{\color{textcolor}{\rmfamily\fontsize{10.000000}{12.000000}\selectfont\catcode`\^=\active\def^{\ifmmode\sp\else\^{}\fi}\catcode`\%=\active\def%{\%}0.00}}%
\end{pgfscope}%
\begin{pgfscope}%
\pgfsetbuttcap%
\pgfsetroundjoin%
\definecolor{currentfill}{rgb}{0.000000,0.000000,0.000000}%
\pgfsetfillcolor{currentfill}%
\pgfsetlinewidth{0.803000pt}%
\definecolor{currentstroke}{rgb}{0.000000,0.000000,0.000000}%
\pgfsetstrokecolor{currentstroke}%
\pgfsetdash{}{0pt}%
\pgfsys@defobject{currentmarker}{\pgfqpoint{-0.048611in}{0.000000in}}{\pgfqpoint{-0.000000in}{0.000000in}}{%
\pgfpathmoveto{\pgfqpoint{-0.000000in}{0.000000in}}%
\pgfpathlineto{\pgfqpoint{-0.048611in}{0.000000in}}%
\pgfusepath{stroke,fill}%
}%
\begin{pgfscope}%
\pgfsys@transformshift{0.614492in}{1.992236in}%
\pgfsys@useobject{currentmarker}{}%
\end{pgfscope}%
\end{pgfscope}%
\begin{pgfscope}%
\definecolor{textcolor}{rgb}{0.000000,0.000000,0.000000}%
\pgfsetstrokecolor{textcolor}%
\pgfsetfillcolor{textcolor}%
\pgftext[x=0.208025in, y=1.939474in, left, base]{\color{textcolor}{\rmfamily\fontsize{10.000000}{12.000000}\selectfont\catcode`\^=\active\def^{\ifmmode\sp\else\^{}\fi}\catcode`\%=\active\def%{\%}0.25}}%
\end{pgfscope}%
\begin{pgfscope}%
\pgfsetbuttcap%
\pgfsetroundjoin%
\definecolor{currentfill}{rgb}{0.000000,0.000000,0.000000}%
\pgfsetfillcolor{currentfill}%
\pgfsetlinewidth{0.803000pt}%
\definecolor{currentstroke}{rgb}{0.000000,0.000000,0.000000}%
\pgfsetstrokecolor{currentstroke}%
\pgfsetdash{}{0pt}%
\pgfsys@defobject{currentmarker}{\pgfqpoint{-0.048611in}{0.000000in}}{\pgfqpoint{-0.000000in}{0.000000in}}{%
\pgfpathmoveto{\pgfqpoint{-0.000000in}{0.000000in}}%
\pgfpathlineto{\pgfqpoint{-0.048611in}{0.000000in}}%
\pgfusepath{stroke,fill}%
}%
\begin{pgfscope}%
\pgfsys@transformshift{0.614492in}{2.280986in}%
\pgfsys@useobject{currentmarker}{}%
\end{pgfscope}%
\end{pgfscope}%
\begin{pgfscope}%
\definecolor{textcolor}{rgb}{0.000000,0.000000,0.000000}%
\pgfsetstrokecolor{textcolor}%
\pgfsetfillcolor{textcolor}%
\pgftext[x=0.208025in, y=2.228224in, left, base]{\color{textcolor}{\rmfamily\fontsize{10.000000}{12.000000}\selectfont\catcode`\^=\active\def^{\ifmmode\sp\else\^{}\fi}\catcode`\%=\active\def%{\%}0.50}}%
\end{pgfscope}%
\begin{pgfscope}%
\pgfsetbuttcap%
\pgfsetroundjoin%
\definecolor{currentfill}{rgb}{0.000000,0.000000,0.000000}%
\pgfsetfillcolor{currentfill}%
\pgfsetlinewidth{0.803000pt}%
\definecolor{currentstroke}{rgb}{0.000000,0.000000,0.000000}%
\pgfsetstrokecolor{currentstroke}%
\pgfsetdash{}{0pt}%
\pgfsys@defobject{currentmarker}{\pgfqpoint{-0.048611in}{0.000000in}}{\pgfqpoint{-0.000000in}{0.000000in}}{%
\pgfpathmoveto{\pgfqpoint{-0.000000in}{0.000000in}}%
\pgfpathlineto{\pgfqpoint{-0.048611in}{0.000000in}}%
\pgfusepath{stroke,fill}%
}%
\begin{pgfscope}%
\pgfsys@transformshift{0.614492in}{2.569736in}%
\pgfsys@useobject{currentmarker}{}%
\end{pgfscope}%
\end{pgfscope}%
\begin{pgfscope}%
\definecolor{textcolor}{rgb}{0.000000,0.000000,0.000000}%
\pgfsetstrokecolor{textcolor}%
\pgfsetfillcolor{textcolor}%
\pgftext[x=0.208025in, y=2.516974in, left, base]{\color{textcolor}{\rmfamily\fontsize{10.000000}{12.000000}\selectfont\catcode`\^=\active\def^{\ifmmode\sp\else\^{}\fi}\catcode`\%=\active\def%{\%}0.75}}%
\end{pgfscope}%
\begin{pgfscope}%
\pgfsetbuttcap%
\pgfsetroundjoin%
\definecolor{currentfill}{rgb}{0.000000,0.000000,0.000000}%
\pgfsetfillcolor{currentfill}%
\pgfsetlinewidth{0.803000pt}%
\definecolor{currentstroke}{rgb}{0.000000,0.000000,0.000000}%
\pgfsetstrokecolor{currentstroke}%
\pgfsetdash{}{0pt}%
\pgfsys@defobject{currentmarker}{\pgfqpoint{-0.048611in}{0.000000in}}{\pgfqpoint{-0.000000in}{0.000000in}}{%
\pgfpathmoveto{\pgfqpoint{-0.000000in}{0.000000in}}%
\pgfpathlineto{\pgfqpoint{-0.048611in}{0.000000in}}%
\pgfusepath{stroke,fill}%
}%
\begin{pgfscope}%
\pgfsys@transformshift{0.614492in}{2.858486in}%
\pgfsys@useobject{currentmarker}{}%
\end{pgfscope}%
\end{pgfscope}%
\begin{pgfscope}%
\definecolor{textcolor}{rgb}{0.000000,0.000000,0.000000}%
\pgfsetstrokecolor{textcolor}%
\pgfsetfillcolor{textcolor}%
\pgftext[x=0.208025in, y=2.805724in, left, base]{\color{textcolor}{\rmfamily\fontsize{10.000000}{12.000000}\selectfont\catcode`\^=\active\def^{\ifmmode\sp\else\^{}\fi}\catcode`\%=\active\def%{\%}1.00}}%
\end{pgfscope}%
\begin{pgfscope}%
\pgfpathrectangle{\pgfqpoint{0.614492in}{0.548486in}}{\pgfqpoint{4.650000in}{2.310000in}}%
\pgfusepath{clip}%
\pgfsetrectcap%
\pgfsetroundjoin%
\pgfsetlinewidth{1.505625pt}%
\definecolor{currentstroke}{rgb}{0.000000,0.000000,0.000000}%
\pgfsetstrokecolor{currentstroke}%
\pgfsetdash{}{0pt}%
\pgfpathmoveto{\pgfqpoint{0.825856in}{2.328819in}}%
\pgfpathlineto{\pgfqpoint{0.851245in}{2.387781in}}%
\pgfpathlineto{\pgfqpoint{0.876634in}{2.441133in}}%
\pgfpathlineto{\pgfqpoint{0.897791in}{2.480992in}}%
\pgfpathlineto{\pgfqpoint{0.918949in}{2.516424in}}%
\pgfpathlineto{\pgfqpoint{0.940106in}{2.547227in}}%
\pgfpathlineto{\pgfqpoint{0.957032in}{2.568418in}}%
\pgfpathlineto{\pgfqpoint{0.973958in}{2.586458in}}%
\pgfpathlineto{\pgfqpoint{0.990884in}{2.601279in}}%
\pgfpathlineto{\pgfqpoint{1.007810in}{2.612828in}}%
\pgfpathlineto{\pgfqpoint{1.024736in}{2.621063in}}%
\pgfpathlineto{\pgfqpoint{1.037431in}{2.625045in}}%
\pgfpathlineto{\pgfqpoint{1.050125in}{2.627139in}}%
\pgfpathlineto{\pgfqpoint{1.062820in}{2.627338in}}%
\pgfpathlineto{\pgfqpoint{1.075514in}{2.625644in}}%
\pgfpathlineto{\pgfqpoint{1.088209in}{2.622058in}}%
\pgfpathlineto{\pgfqpoint{1.100903in}{2.616590in}}%
\pgfpathlineto{\pgfqpoint{1.117830in}{2.606389in}}%
\pgfpathlineto{\pgfqpoint{1.134756in}{2.592897in}}%
\pgfpathlineto{\pgfqpoint{1.151682in}{2.576163in}}%
\pgfpathlineto{\pgfqpoint{1.168608in}{2.556249in}}%
\pgfpathlineto{\pgfqpoint{1.185534in}{2.533227in}}%
\pgfpathlineto{\pgfqpoint{1.206691in}{2.500209in}}%
\pgfpathlineto{\pgfqpoint{1.227849in}{2.462654in}}%
\pgfpathlineto{\pgfqpoint{1.249006in}{2.420776in}}%
\pgfpathlineto{\pgfqpoint{1.274395in}{2.365155in}}%
\pgfpathlineto{\pgfqpoint{1.299784in}{2.304110in}}%
\pgfpathlineto{\pgfqpoint{1.329405in}{2.226706in}}%
\pgfpathlineto{\pgfqpoint{1.363257in}{2.131151in}}%
\pgfpathlineto{\pgfqpoint{1.401340in}{2.016280in}}%
\pgfpathlineto{\pgfqpoint{1.447887in}{1.868276in}}%
\pgfpathlineto{\pgfqpoint{1.612916in}{1.335686in}}%
\pgfpathlineto{\pgfqpoint{1.650999in}{1.224274in}}%
\pgfpathlineto{\pgfqpoint{1.684851in}{1.132595in}}%
\pgfpathlineto{\pgfqpoint{1.714472in}{1.059153in}}%
\pgfpathlineto{\pgfqpoint{1.739861in}{1.001894in}}%
\pgfpathlineto{\pgfqpoint{1.765250in}{0.950387in}}%
\pgfpathlineto{\pgfqpoint{1.786407in}{0.912163in}}%
\pgfpathlineto{\pgfqpoint{1.807565in}{0.878445in}}%
\pgfpathlineto{\pgfqpoint{1.828722in}{0.849424in}}%
\pgfpathlineto{\pgfqpoint{1.845648in}{0.829701in}}%
\pgfpathlineto{\pgfqpoint{1.862574in}{0.813163in}}%
\pgfpathlineto{\pgfqpoint{1.879500in}{0.799869in}}%
\pgfpathlineto{\pgfqpoint{1.896426in}{0.789869in}}%
\pgfpathlineto{\pgfqpoint{1.909121in}{0.784553in}}%
\pgfpathlineto{\pgfqpoint{1.921815in}{0.781120in}}%
\pgfpathlineto{\pgfqpoint{1.934510in}{0.779579in}}%
\pgfpathlineto{\pgfqpoint{1.947204in}{0.779932in}}%
\pgfpathlineto{\pgfqpoint{1.959899in}{0.782179in}}%
\pgfpathlineto{\pgfqpoint{1.972593in}{0.786314in}}%
\pgfpathlineto{\pgfqpoint{1.985288in}{0.792330in}}%
\pgfpathlineto{\pgfqpoint{2.002214in}{0.803255in}}%
\pgfpathlineto{\pgfqpoint{2.019140in}{0.817461in}}%
\pgfpathlineto{\pgfqpoint{2.036066in}{0.834896in}}%
\pgfpathlineto{\pgfqpoint{2.052992in}{0.855496in}}%
\pgfpathlineto{\pgfqpoint{2.069918in}{0.879187in}}%
\pgfpathlineto{\pgfqpoint{2.091075in}{0.913014in}}%
\pgfpathlineto{\pgfqpoint{2.112233in}{0.951341in}}%
\pgfpathlineto{\pgfqpoint{2.133391in}{0.993951in}}%
\pgfpathlineto{\pgfqpoint{2.158780in}{1.050393in}}%
\pgfpathlineto{\pgfqpoint{2.184169in}{1.112188in}}%
\pgfpathlineto{\pgfqpoint{2.213789in}{1.190370in}}%
\pgfpathlineto{\pgfqpoint{2.247641in}{1.286676in}}%
\pgfpathlineto{\pgfqpoint{2.285725in}{1.402203in}}%
\pgfpathlineto{\pgfqpoint{2.332271in}{1.550719in}}%
\pgfpathlineto{\pgfqpoint{2.488837in}{2.056839in}}%
\pgfpathlineto{\pgfqpoint{2.526920in}{2.169213in}}%
\pgfpathlineto{\pgfqpoint{2.560772in}{2.261958in}}%
\pgfpathlineto{\pgfqpoint{2.590393in}{2.336480in}}%
\pgfpathlineto{\pgfqpoint{2.615782in}{2.394766in}}%
\pgfpathlineto{\pgfqpoint{2.641171in}{2.447386in}}%
\pgfpathlineto{\pgfqpoint{2.662329in}{2.486594in}}%
\pgfpathlineto{\pgfqpoint{2.683486in}{2.521344in}}%
\pgfpathlineto{\pgfqpoint{2.704644in}{2.551437in}}%
\pgfpathlineto{\pgfqpoint{2.721570in}{2.572043in}}%
\pgfpathlineto{\pgfqpoint{2.738496in}{2.589483in}}%
\pgfpathlineto{\pgfqpoint{2.755422in}{2.603695in}}%
\pgfpathlineto{\pgfqpoint{2.772348in}{2.614625in}}%
\pgfpathlineto{\pgfqpoint{2.789274in}{2.622235in}}%
\pgfpathlineto{\pgfqpoint{2.801968in}{2.625746in}}%
\pgfpathlineto{\pgfqpoint{2.814663in}{2.627366in}}%
\pgfpathlineto{\pgfqpoint{2.827357in}{2.627092in}}%
\pgfpathlineto{\pgfqpoint{2.840052in}{2.624925in}}%
\pgfpathlineto{\pgfqpoint{2.852746in}{2.620868in}}%
\pgfpathlineto{\pgfqpoint{2.865441in}{2.614930in}}%
\pgfpathlineto{\pgfqpoint{2.882367in}{2.604108in}}%
\pgfpathlineto{\pgfqpoint{2.899293in}{2.590004in}}%
\pgfpathlineto{\pgfqpoint{2.916219in}{2.572670in}}%
\pgfpathlineto{\pgfqpoint{2.933145in}{2.552167in}}%
\pgfpathlineto{\pgfqpoint{2.950071in}{2.528572in}}%
\pgfpathlineto{\pgfqpoint{2.971228in}{2.494860in}}%
\pgfpathlineto{\pgfqpoint{2.992386in}{2.456642in}}%
\pgfpathlineto{\pgfqpoint{3.013543in}{2.414136in}}%
\pgfpathlineto{\pgfqpoint{3.038932in}{2.357811in}}%
\pgfpathlineto{\pgfqpoint{3.064321in}{2.296123in}}%
\pgfpathlineto{\pgfqpoint{3.093942in}{2.218051in}}%
\pgfpathlineto{\pgfqpoint{3.127794in}{2.121851in}}%
\pgfpathlineto{\pgfqpoint{3.165878in}{2.006417in}}%
\pgfpathlineto{\pgfqpoint{3.212424in}{1.857972in}}%
\pgfpathlineto{\pgfqpoint{3.368990in}{1.351745in}}%
\pgfpathlineto{\pgfqpoint{3.407073in}{1.239265in}}%
\pgfpathlineto{\pgfqpoint{3.440925in}{1.146404in}}%
\pgfpathlineto{\pgfqpoint{3.470546in}{1.071763in}}%
\pgfpathlineto{\pgfqpoint{3.495935in}{1.013364in}}%
\pgfpathlineto{\pgfqpoint{3.521324in}{0.960622in}}%
\pgfpathlineto{\pgfqpoint{3.542481in}{0.921305in}}%
\pgfpathlineto{\pgfqpoint{3.563639in}{0.886441in}}%
\pgfpathlineto{\pgfqpoint{3.584796in}{0.856229in}}%
\pgfpathlineto{\pgfqpoint{3.601722in}{0.835525in}}%
\pgfpathlineto{\pgfqpoint{3.618649in}{0.817985in}}%
\pgfpathlineto{\pgfqpoint{3.635575in}{0.803672in}}%
\pgfpathlineto{\pgfqpoint{3.652501in}{0.792638in}}%
\pgfpathlineto{\pgfqpoint{3.669427in}{0.784924in}}%
\pgfpathlineto{\pgfqpoint{3.682121in}{0.781334in}}%
\pgfpathlineto{\pgfqpoint{3.694816in}{0.779635in}}%
\pgfpathlineto{\pgfqpoint{3.707510in}{0.779830in}}%
\pgfpathlineto{\pgfqpoint{3.720205in}{0.781919in}}%
\pgfpathlineto{\pgfqpoint{3.732899in}{0.785898in}}%
\pgfpathlineto{\pgfqpoint{3.745594in}{0.791758in}}%
\pgfpathlineto{\pgfqpoint{3.762520in}{0.802476in}}%
\pgfpathlineto{\pgfqpoint{3.779446in}{0.816478in}}%
\pgfpathlineto{\pgfqpoint{3.796372in}{0.833712in}}%
\pgfpathlineto{\pgfqpoint{3.813298in}{0.854117in}}%
\pgfpathlineto{\pgfqpoint{3.830224in}{0.877617in}}%
\pgfpathlineto{\pgfqpoint{3.851381in}{0.911214in}}%
\pgfpathlineto{\pgfqpoint{3.872539in}{0.949321in}}%
\pgfpathlineto{\pgfqpoint{3.893696in}{0.991723in}}%
\pgfpathlineto{\pgfqpoint{3.919085in}{1.047931in}}%
\pgfpathlineto{\pgfqpoint{3.944474in}{1.109513in}}%
\pgfpathlineto{\pgfqpoint{3.974095in}{1.187474in}}%
\pgfpathlineto{\pgfqpoint{4.007947in}{1.283567in}}%
\pgfpathlineto{\pgfqpoint{4.046030in}{1.398908in}}%
\pgfpathlineto{\pgfqpoint{4.092577in}{1.547281in}}%
\pgfpathlineto{\pgfqpoint{4.249143in}{2.053615in}}%
\pgfpathlineto{\pgfqpoint{4.287226in}{2.166198in}}%
\pgfpathlineto{\pgfqpoint{4.321078in}{2.259176in}}%
\pgfpathlineto{\pgfqpoint{4.350699in}{2.333935in}}%
\pgfpathlineto{\pgfqpoint{4.376088in}{2.392447in}}%
\pgfpathlineto{\pgfqpoint{4.401477in}{2.445312in}}%
\pgfpathlineto{\pgfqpoint{4.422634in}{2.484738in}}%
\pgfpathlineto{\pgfqpoint{4.443792in}{2.519715in}}%
\pgfpathlineto{\pgfqpoint{4.464949in}{2.550046in}}%
\pgfpathlineto{\pgfqpoint{4.481875in}{2.570847in}}%
\pgfpathlineto{\pgfqpoint{4.498801in}{2.588487in}}%
\pgfpathlineto{\pgfqpoint{4.515727in}{2.602902in}}%
\pgfpathlineto{\pgfqpoint{4.532653in}{2.614039in}}%
\pgfpathlineto{\pgfqpoint{4.549579in}{2.621857in}}%
\pgfpathlineto{\pgfqpoint{4.562274in}{2.625526in}}%
\pgfpathlineto{\pgfqpoint{4.574968in}{2.627304in}}%
\pgfpathlineto{\pgfqpoint{4.587663in}{2.627187in}}%
\pgfpathlineto{\pgfqpoint{4.600357in}{2.625177in}}%
\pgfpathlineto{\pgfqpoint{4.613052in}{2.621278in}}%
\pgfpathlineto{\pgfqpoint{4.625747in}{2.615496in}}%
\pgfpathlineto{\pgfqpoint{4.642673in}{2.604881in}}%
\pgfpathlineto{\pgfqpoint{4.659599in}{2.590981in}}%
\pgfpathlineto{\pgfqpoint{4.676525in}{2.573846in}}%
\pgfpathlineto{\pgfqpoint{4.693451in}{2.553540in}}%
\pgfpathlineto{\pgfqpoint{4.710377in}{2.530135in}}%
\pgfpathlineto{\pgfqpoint{4.731534in}{2.496654in}}%
\pgfpathlineto{\pgfqpoint{4.752692in}{2.458657in}}%
\pgfpathlineto{\pgfqpoint{4.773849in}{2.416360in}}%
\pgfpathlineto{\pgfqpoint{4.799238in}{2.360268in}}%
\pgfpathlineto{\pgfqpoint{4.824627in}{2.298794in}}%
\pgfpathlineto{\pgfqpoint{4.854248in}{2.220943in}}%
\pgfpathlineto{\pgfqpoint{4.888100in}{2.124957in}}%
\pgfpathlineto{\pgfqpoint{4.926183in}{2.009709in}}%
\pgfpathlineto{\pgfqpoint{4.972730in}{1.861409in}}%
\pgfpathlineto{\pgfqpoint{5.053128in}{1.597419in}}%
\pgfpathlineto{\pgfqpoint{5.053128in}{1.597419in}}%
\pgfusepath{stroke}%
\end{pgfscope}%
\begin{pgfscope}%
\pgfpathrectangle{\pgfqpoint{0.614492in}{0.548486in}}{\pgfqpoint{4.650000in}{2.310000in}}%
\pgfusepath{clip}%
\pgfsetrectcap%
\pgfsetroundjoin%
\pgfsetlinewidth{0.501875pt}%
\definecolor{currentstroke}{rgb}{0.000000,0.000000,0.000000}%
\pgfsetstrokecolor{currentstroke}%
\pgfsetdash{}{0pt}%
\pgfpathmoveto{\pgfqpoint{0.614492in}{1.703486in}}%
\pgfpathlineto{\pgfqpoint{5.264492in}{1.703486in}}%
\pgfusepath{stroke}%
\end{pgfscope}%
\begin{pgfscope}%
\pgfpathrectangle{\pgfqpoint{0.614492in}{0.548486in}}{\pgfqpoint{4.650000in}{2.310000in}}%
\pgfusepath{clip}%
\pgfsetbuttcap%
\pgfsetroundjoin%
\pgfsetlinewidth{1.003750pt}%
\definecolor{currentstroke}{rgb}{1.000000,0.000000,0.000000}%
\pgfsetstrokecolor{currentstroke}%
\pgfsetdash{{1.000000pt}{1.650000pt}}{0.000000pt}%
\pgfpathmoveto{\pgfqpoint{2.378833in}{1.125986in}}%
\pgfpathlineto{\pgfqpoint{2.378833in}{1.703486in}}%
\pgfusepath{stroke}%
\end{pgfscope}%
\begin{pgfscope}%
\pgfpathrectangle{\pgfqpoint{0.614492in}{0.548486in}}{\pgfqpoint{4.650000in}{2.310000in}}%
\pgfusepath{clip}%
\pgfsetrectcap%
\pgfsetroundjoin%
\pgfsetlinewidth{0.501875pt}%
\definecolor{currentstroke}{rgb}{0.000000,0.000000,0.000000}%
\pgfsetstrokecolor{currentstroke}%
\pgfsetdash{}{0pt}%
\pgfpathmoveto{\pgfqpoint{2.939492in}{0.548486in}}%
\pgfpathlineto{\pgfqpoint{2.939492in}{2.858486in}}%
\pgfusepath{stroke}%
\end{pgfscope}%
\begin{pgfscope}%
\definecolor{textcolor}{rgb}{0.000000,0.000000,0.000000}%
\pgfsetstrokecolor{textcolor}%
\pgfsetfillcolor{textcolor}%
\pgftext[x=1.142901in,y=2.304375in,,base]{\color{textcolor}{\rmfamily\fontsize{13.000000}{15.600000}\selectfont\catcode`\^=\active\def^{\ifmmode\sp\else\^{}\fi}\catcode`\%=\active\def%{\%}$A$}}%
\end{pgfscope}%
\begin{pgfscope}%
\definecolor{textcolor}{rgb}{0.000000,0.000000,0.000000}%
\pgfsetstrokecolor{textcolor}%
\pgfsetfillcolor{textcolor}%
\pgftext[x=2.659162in,y=1.438125in,,base]{\color{textcolor}{\rmfamily\fontsize{13.000000}{15.600000}\selectfont\catcode`\^=\active\def^{\ifmmode\sp\else\^{}\fi}\catcode`\%=\active\def%{\%}$\frac{\phi}{\omega}$}}%
\end{pgfscope}%
\begin{pgfscope}%
\definecolor{textcolor}{rgb}{0.000000,0.000000,0.000000}%
\pgfsetstrokecolor{textcolor}%
\pgfsetfillcolor{textcolor}%
\pgftext[x=1.938492in,y=2.858775in,,base]{\color{textcolor}{\rmfamily\fontsize{13.000000}{15.600000}\selectfont\catcode`\^=\active\def^{\ifmmode\sp\else\^{}\fi}\catcode`\%=\active\def%{\%}$T = \frac{1}{f_0}$}}%
\end{pgfscope}%
\end{pgfpicture}%
\makeatother%
\endgroup%

	\caption{Eine Sinus Funktion und ihre 'Parameter'.}
	\label{fig:sinParams}
\end{figure}


\begin{equation}
-sin(x) = sin(-x)
\end{equation}
\begin{equation}
cos(x) = cos(-x)
\end{equation}

\section{$e^x$}
Eulersche Zahl $e \approx 2.718281828459045$.\footnote{Benannt nach Leonhard Euler, 15. April 1707 – 18. September 1783. Einer der wichtigsten Mathematiker der Geschichte. }

\begin{figure}[H]
	\centering
	%% Creator: Matplotlib, PGF backend
%%
%% To include the figure in your LaTeX document, write
%%   \input{<filename>.pgf}
%%
%% Make sure the required packages are loaded in your preamble
%%   \usepackage{pgf}
%%
%% Also ensure that all the required font packages are loaded; for instance,
%% the lmodern package is sometimes necessary when using math font.
%%   \usepackage{lmodern}
%%
%% Figures using additional raster images can only be included by \input if
%% they are in the same directory as the main LaTeX file. For loading figures
%% from other directories you can use the `import` package
%%   \usepackage{import}
%%
%% and then include the figures with
%%   \import{<path to file>}{<filename>.pgf}
%%
%% Matplotlib used the following preamble
%%   
%%   \usepackage{fontspec}
%%   \setmainfont{DejaVuSerif.ttf}[Path=\detokenize{/home/pl/anaconda3/lib/python3.11/site-packages/matplotlib/mpl-data/fonts/ttf/}]
%%   \setsansfont{DejaVuSans.ttf}[Path=\detokenize{/home/pl/anaconda3/lib/python3.11/site-packages/matplotlib/mpl-data/fonts/ttf/}]
%%   \setmonofont{DejaVuSansMono.ttf}[Path=\detokenize{/home/pl/anaconda3/lib/python3.11/site-packages/matplotlib/mpl-data/fonts/ttf/}]
%%   \makeatletter\@ifpackageloaded{underscore}{}{\usepackage[strings]{underscore}}\makeatother
%%
\begingroup%
\makeatletter%
\begin{pgfpicture}%
\pgfpathrectangle{\pgfpointorigin}{\pgfqpoint{5.124692in}{3.141564in}}%
\pgfusepath{use as bounding box, clip}%
\begin{pgfscope}%
\pgfsetbuttcap%
\pgfsetmiterjoin%
\definecolor{currentfill}{rgb}{1.000000,1.000000,1.000000}%
\pgfsetfillcolor{currentfill}%
\pgfsetlinewidth{0.000000pt}%
\definecolor{currentstroke}{rgb}{1.000000,1.000000,1.000000}%
\pgfsetstrokecolor{currentstroke}%
\pgfsetdash{}{0pt}%
\pgfpathmoveto{\pgfqpoint{0.000000in}{0.000000in}}%
\pgfpathlineto{\pgfqpoint{5.124692in}{0.000000in}}%
\pgfpathlineto{\pgfqpoint{5.124692in}{3.141564in}}%
\pgfpathlineto{\pgfqpoint{0.000000in}{3.141564in}}%
\pgfpathlineto{\pgfqpoint{0.000000in}{0.000000in}}%
\pgfpathclose%
\pgfusepath{fill}%
\end{pgfscope}%
\begin{pgfscope}%
\pgfsetbuttcap%
\pgfsetmiterjoin%
\definecolor{currentfill}{rgb}{1.000000,1.000000,1.000000}%
\pgfsetfillcolor{currentfill}%
\pgfsetlinewidth{0.000000pt}%
\definecolor{currentstroke}{rgb}{0.000000,0.000000,0.000000}%
\pgfsetstrokecolor{currentstroke}%
\pgfsetstrokeopacity{0.000000}%
\pgfsetdash{}{0pt}%
\pgfpathmoveto{\pgfqpoint{0.374692in}{0.521603in}}%
\pgfpathlineto{\pgfqpoint{5.024692in}{0.521603in}}%
\pgfpathlineto{\pgfqpoint{5.024692in}{2.831603in}}%
\pgfpathlineto{\pgfqpoint{0.374692in}{2.831603in}}%
\pgfpathlineto{\pgfqpoint{0.374692in}{0.521603in}}%
\pgfpathclose%
\pgfusepath{fill}%
\end{pgfscope}%
\begin{pgfscope}%
\pgfsetbuttcap%
\pgfsetroundjoin%
\definecolor{currentfill}{rgb}{0.000000,0.000000,0.000000}%
\pgfsetfillcolor{currentfill}%
\pgfsetlinewidth{0.803000pt}%
\definecolor{currentstroke}{rgb}{0.000000,0.000000,0.000000}%
\pgfsetstrokecolor{currentstroke}%
\pgfsetdash{}{0pt}%
\pgfsys@defobject{currentmarker}{\pgfqpoint{0.000000in}{-0.048611in}}{\pgfqpoint{0.000000in}{0.000000in}}{%
\pgfpathmoveto{\pgfqpoint{0.000000in}{0.000000in}}%
\pgfpathlineto{\pgfqpoint{0.000000in}{-0.048611in}}%
\pgfusepath{stroke,fill}%
}%
\begin{pgfscope}%
\pgfsys@transformshift{0.586056in}{0.521603in}%
\pgfsys@useobject{currentmarker}{}%
\end{pgfscope}%
\end{pgfscope}%
\begin{pgfscope}%
\definecolor{textcolor}{rgb}{0.000000,0.000000,0.000000}%
\pgfsetstrokecolor{textcolor}%
\pgfsetfillcolor{textcolor}%
\pgftext[x=0.586056in,y=0.424381in,,top]{\color{textcolor}\rmfamily\fontsize{10.000000}{12.000000}\selectfont \(\displaystyle {\ensuremath{-}3}\)}%
\end{pgfscope}%
\begin{pgfscope}%
\pgfsetbuttcap%
\pgfsetroundjoin%
\definecolor{currentfill}{rgb}{0.000000,0.000000,0.000000}%
\pgfsetfillcolor{currentfill}%
\pgfsetlinewidth{0.803000pt}%
\definecolor{currentstroke}{rgb}{0.000000,0.000000,0.000000}%
\pgfsetstrokecolor{currentstroke}%
\pgfsetdash{}{0pt}%
\pgfsys@defobject{currentmarker}{\pgfqpoint{0.000000in}{-0.048611in}}{\pgfqpoint{0.000000in}{0.000000in}}{%
\pgfpathmoveto{\pgfqpoint{0.000000in}{0.000000in}}%
\pgfpathlineto{\pgfqpoint{0.000000in}{-0.048611in}}%
\pgfusepath{stroke,fill}%
}%
\begin{pgfscope}%
\pgfsys@transformshift{1.290601in}{0.521603in}%
\pgfsys@useobject{currentmarker}{}%
\end{pgfscope}%
\end{pgfscope}%
\begin{pgfscope}%
\definecolor{textcolor}{rgb}{0.000000,0.000000,0.000000}%
\pgfsetstrokecolor{textcolor}%
\pgfsetfillcolor{textcolor}%
\pgftext[x=1.290601in,y=0.424381in,,top]{\color{textcolor}\rmfamily\fontsize{10.000000}{12.000000}\selectfont \(\displaystyle {\ensuremath{-}2}\)}%
\end{pgfscope}%
\begin{pgfscope}%
\pgfsetbuttcap%
\pgfsetroundjoin%
\definecolor{currentfill}{rgb}{0.000000,0.000000,0.000000}%
\pgfsetfillcolor{currentfill}%
\pgfsetlinewidth{0.803000pt}%
\definecolor{currentstroke}{rgb}{0.000000,0.000000,0.000000}%
\pgfsetstrokecolor{currentstroke}%
\pgfsetdash{}{0pt}%
\pgfsys@defobject{currentmarker}{\pgfqpoint{0.000000in}{-0.048611in}}{\pgfqpoint{0.000000in}{0.000000in}}{%
\pgfpathmoveto{\pgfqpoint{0.000000in}{0.000000in}}%
\pgfpathlineto{\pgfqpoint{0.000000in}{-0.048611in}}%
\pgfusepath{stroke,fill}%
}%
\begin{pgfscope}%
\pgfsys@transformshift{1.995146in}{0.521603in}%
\pgfsys@useobject{currentmarker}{}%
\end{pgfscope}%
\end{pgfscope}%
\begin{pgfscope}%
\definecolor{textcolor}{rgb}{0.000000,0.000000,0.000000}%
\pgfsetstrokecolor{textcolor}%
\pgfsetfillcolor{textcolor}%
\pgftext[x=1.995146in,y=0.424381in,,top]{\color{textcolor}\rmfamily\fontsize{10.000000}{12.000000}\selectfont \(\displaystyle {\ensuremath{-}1}\)}%
\end{pgfscope}%
\begin{pgfscope}%
\pgfsetbuttcap%
\pgfsetroundjoin%
\definecolor{currentfill}{rgb}{0.000000,0.000000,0.000000}%
\pgfsetfillcolor{currentfill}%
\pgfsetlinewidth{0.803000pt}%
\definecolor{currentstroke}{rgb}{0.000000,0.000000,0.000000}%
\pgfsetstrokecolor{currentstroke}%
\pgfsetdash{}{0pt}%
\pgfsys@defobject{currentmarker}{\pgfqpoint{0.000000in}{-0.048611in}}{\pgfqpoint{0.000000in}{0.000000in}}{%
\pgfpathmoveto{\pgfqpoint{0.000000in}{0.000000in}}%
\pgfpathlineto{\pgfqpoint{0.000000in}{-0.048611in}}%
\pgfusepath{stroke,fill}%
}%
\begin{pgfscope}%
\pgfsys@transformshift{2.699692in}{0.521603in}%
\pgfsys@useobject{currentmarker}{}%
\end{pgfscope}%
\end{pgfscope}%
\begin{pgfscope}%
\definecolor{textcolor}{rgb}{0.000000,0.000000,0.000000}%
\pgfsetstrokecolor{textcolor}%
\pgfsetfillcolor{textcolor}%
\pgftext[x=2.699692in,y=0.424381in,,top]{\color{textcolor}\rmfamily\fontsize{10.000000}{12.000000}\selectfont \(\displaystyle {0}\)}%
\end{pgfscope}%
\begin{pgfscope}%
\pgfsetbuttcap%
\pgfsetroundjoin%
\definecolor{currentfill}{rgb}{0.000000,0.000000,0.000000}%
\pgfsetfillcolor{currentfill}%
\pgfsetlinewidth{0.803000pt}%
\definecolor{currentstroke}{rgb}{0.000000,0.000000,0.000000}%
\pgfsetstrokecolor{currentstroke}%
\pgfsetdash{}{0pt}%
\pgfsys@defobject{currentmarker}{\pgfqpoint{0.000000in}{-0.048611in}}{\pgfqpoint{0.000000in}{0.000000in}}{%
\pgfpathmoveto{\pgfqpoint{0.000000in}{0.000000in}}%
\pgfpathlineto{\pgfqpoint{0.000000in}{-0.048611in}}%
\pgfusepath{stroke,fill}%
}%
\begin{pgfscope}%
\pgfsys@transformshift{3.404237in}{0.521603in}%
\pgfsys@useobject{currentmarker}{}%
\end{pgfscope}%
\end{pgfscope}%
\begin{pgfscope}%
\definecolor{textcolor}{rgb}{0.000000,0.000000,0.000000}%
\pgfsetstrokecolor{textcolor}%
\pgfsetfillcolor{textcolor}%
\pgftext[x=3.404237in,y=0.424381in,,top]{\color{textcolor}\rmfamily\fontsize{10.000000}{12.000000}\selectfont \(\displaystyle {1}\)}%
\end{pgfscope}%
\begin{pgfscope}%
\pgfsetbuttcap%
\pgfsetroundjoin%
\definecolor{currentfill}{rgb}{0.000000,0.000000,0.000000}%
\pgfsetfillcolor{currentfill}%
\pgfsetlinewidth{0.803000pt}%
\definecolor{currentstroke}{rgb}{0.000000,0.000000,0.000000}%
\pgfsetstrokecolor{currentstroke}%
\pgfsetdash{}{0pt}%
\pgfsys@defobject{currentmarker}{\pgfqpoint{0.000000in}{-0.048611in}}{\pgfqpoint{0.000000in}{0.000000in}}{%
\pgfpathmoveto{\pgfqpoint{0.000000in}{0.000000in}}%
\pgfpathlineto{\pgfqpoint{0.000000in}{-0.048611in}}%
\pgfusepath{stroke,fill}%
}%
\begin{pgfscope}%
\pgfsys@transformshift{4.108783in}{0.521603in}%
\pgfsys@useobject{currentmarker}{}%
\end{pgfscope}%
\end{pgfscope}%
\begin{pgfscope}%
\definecolor{textcolor}{rgb}{0.000000,0.000000,0.000000}%
\pgfsetstrokecolor{textcolor}%
\pgfsetfillcolor{textcolor}%
\pgftext[x=4.108783in,y=0.424381in,,top]{\color{textcolor}\rmfamily\fontsize{10.000000}{12.000000}\selectfont \(\displaystyle {2}\)}%
\end{pgfscope}%
\begin{pgfscope}%
\pgfsetbuttcap%
\pgfsetroundjoin%
\definecolor{currentfill}{rgb}{0.000000,0.000000,0.000000}%
\pgfsetfillcolor{currentfill}%
\pgfsetlinewidth{0.803000pt}%
\definecolor{currentstroke}{rgb}{0.000000,0.000000,0.000000}%
\pgfsetstrokecolor{currentstroke}%
\pgfsetdash{}{0pt}%
\pgfsys@defobject{currentmarker}{\pgfqpoint{0.000000in}{-0.048611in}}{\pgfqpoint{0.000000in}{0.000000in}}{%
\pgfpathmoveto{\pgfqpoint{0.000000in}{0.000000in}}%
\pgfpathlineto{\pgfqpoint{0.000000in}{-0.048611in}}%
\pgfusepath{stroke,fill}%
}%
\begin{pgfscope}%
\pgfsys@transformshift{4.813328in}{0.521603in}%
\pgfsys@useobject{currentmarker}{}%
\end{pgfscope}%
\end{pgfscope}%
\begin{pgfscope}%
\definecolor{textcolor}{rgb}{0.000000,0.000000,0.000000}%
\pgfsetstrokecolor{textcolor}%
\pgfsetfillcolor{textcolor}%
\pgftext[x=4.813328in,y=0.424381in,,top]{\color{textcolor}\rmfamily\fontsize{10.000000}{12.000000}\selectfont \(\displaystyle {3}\)}%
\end{pgfscope}%
\begin{pgfscope}%
\definecolor{textcolor}{rgb}{0.000000,0.000000,0.000000}%
\pgfsetstrokecolor{textcolor}%
\pgfsetfillcolor{textcolor}%
\pgftext[x=2.699692in,y=0.234413in,,top]{\color{textcolor}\rmfamily\fontsize{10.000000}{12.000000}\selectfont \(\displaystyle t\)}%
\end{pgfscope}%
\begin{pgfscope}%
\pgfsetbuttcap%
\pgfsetroundjoin%
\definecolor{currentfill}{rgb}{0.000000,0.000000,0.000000}%
\pgfsetfillcolor{currentfill}%
\pgfsetlinewidth{0.803000pt}%
\definecolor{currentstroke}{rgb}{0.000000,0.000000,0.000000}%
\pgfsetstrokecolor{currentstroke}%
\pgfsetdash{}{0pt}%
\pgfsys@defobject{currentmarker}{\pgfqpoint{-0.048611in}{0.000000in}}{\pgfqpoint{-0.000000in}{0.000000in}}{%
\pgfpathmoveto{\pgfqpoint{-0.000000in}{0.000000in}}%
\pgfpathlineto{\pgfqpoint{-0.048611in}{0.000000in}}%
\pgfusepath{stroke,fill}%
}%
\begin{pgfscope}%
\pgfsys@transformshift{0.374692in}{0.631603in}%
\pgfsys@useobject{currentmarker}{}%
\end{pgfscope}%
\end{pgfscope}%
\begin{pgfscope}%
\definecolor{textcolor}{rgb}{0.000000,0.000000,0.000000}%
\pgfsetstrokecolor{textcolor}%
\pgfsetfillcolor{textcolor}%
\pgftext[x=0.100000in, y=0.578842in, left, base]{\color{textcolor}\rmfamily\fontsize{10.000000}{12.000000}\selectfont \(\displaystyle {0.0}\)}%
\end{pgfscope}%
\begin{pgfscope}%
\pgfsetbuttcap%
\pgfsetroundjoin%
\definecolor{currentfill}{rgb}{0.000000,0.000000,0.000000}%
\pgfsetfillcolor{currentfill}%
\pgfsetlinewidth{0.803000pt}%
\definecolor{currentstroke}{rgb}{0.000000,0.000000,0.000000}%
\pgfsetstrokecolor{currentstroke}%
\pgfsetdash{}{0pt}%
\pgfsys@defobject{currentmarker}{\pgfqpoint{-0.048611in}{0.000000in}}{\pgfqpoint{-0.000000in}{0.000000in}}{%
\pgfpathmoveto{\pgfqpoint{-0.000000in}{0.000000in}}%
\pgfpathlineto{\pgfqpoint{-0.048611in}{0.000000in}}%
\pgfusepath{stroke,fill}%
}%
\begin{pgfscope}%
\pgfsys@transformshift{0.374692in}{1.181603in}%
\pgfsys@useobject{currentmarker}{}%
\end{pgfscope}%
\end{pgfscope}%
\begin{pgfscope}%
\definecolor{textcolor}{rgb}{0.000000,0.000000,0.000000}%
\pgfsetstrokecolor{textcolor}%
\pgfsetfillcolor{textcolor}%
\pgftext[x=0.100000in, y=1.128842in, left, base]{\color{textcolor}\rmfamily\fontsize{10.000000}{12.000000}\selectfont \(\displaystyle {0.5}\)}%
\end{pgfscope}%
\begin{pgfscope}%
\pgfsetbuttcap%
\pgfsetroundjoin%
\definecolor{currentfill}{rgb}{0.000000,0.000000,0.000000}%
\pgfsetfillcolor{currentfill}%
\pgfsetlinewidth{0.803000pt}%
\definecolor{currentstroke}{rgb}{0.000000,0.000000,0.000000}%
\pgfsetstrokecolor{currentstroke}%
\pgfsetdash{}{0pt}%
\pgfsys@defobject{currentmarker}{\pgfqpoint{-0.048611in}{0.000000in}}{\pgfqpoint{-0.000000in}{0.000000in}}{%
\pgfpathmoveto{\pgfqpoint{-0.000000in}{0.000000in}}%
\pgfpathlineto{\pgfqpoint{-0.048611in}{0.000000in}}%
\pgfusepath{stroke,fill}%
}%
\begin{pgfscope}%
\pgfsys@transformshift{0.374692in}{1.731603in}%
\pgfsys@useobject{currentmarker}{}%
\end{pgfscope}%
\end{pgfscope}%
\begin{pgfscope}%
\definecolor{textcolor}{rgb}{0.000000,0.000000,0.000000}%
\pgfsetstrokecolor{textcolor}%
\pgfsetfillcolor{textcolor}%
\pgftext[x=0.100000in, y=1.678842in, left, base]{\color{textcolor}\rmfamily\fontsize{10.000000}{12.000000}\selectfont \(\displaystyle {1.0}\)}%
\end{pgfscope}%
\begin{pgfscope}%
\pgfsetbuttcap%
\pgfsetroundjoin%
\definecolor{currentfill}{rgb}{0.000000,0.000000,0.000000}%
\pgfsetfillcolor{currentfill}%
\pgfsetlinewidth{0.803000pt}%
\definecolor{currentstroke}{rgb}{0.000000,0.000000,0.000000}%
\pgfsetstrokecolor{currentstroke}%
\pgfsetdash{}{0pt}%
\pgfsys@defobject{currentmarker}{\pgfqpoint{-0.048611in}{0.000000in}}{\pgfqpoint{-0.000000in}{0.000000in}}{%
\pgfpathmoveto{\pgfqpoint{-0.000000in}{0.000000in}}%
\pgfpathlineto{\pgfqpoint{-0.048611in}{0.000000in}}%
\pgfusepath{stroke,fill}%
}%
\begin{pgfscope}%
\pgfsys@transformshift{0.374692in}{2.281603in}%
\pgfsys@useobject{currentmarker}{}%
\end{pgfscope}%
\end{pgfscope}%
\begin{pgfscope}%
\definecolor{textcolor}{rgb}{0.000000,0.000000,0.000000}%
\pgfsetstrokecolor{textcolor}%
\pgfsetfillcolor{textcolor}%
\pgftext[x=0.100000in, y=2.228842in, left, base]{\color{textcolor}\rmfamily\fontsize{10.000000}{12.000000}\selectfont \(\displaystyle {1.5}\)}%
\end{pgfscope}%
\begin{pgfscope}%
\pgfsetbuttcap%
\pgfsetroundjoin%
\definecolor{currentfill}{rgb}{0.000000,0.000000,0.000000}%
\pgfsetfillcolor{currentfill}%
\pgfsetlinewidth{0.803000pt}%
\definecolor{currentstroke}{rgb}{0.000000,0.000000,0.000000}%
\pgfsetstrokecolor{currentstroke}%
\pgfsetdash{}{0pt}%
\pgfsys@defobject{currentmarker}{\pgfqpoint{-0.048611in}{0.000000in}}{\pgfqpoint{-0.000000in}{0.000000in}}{%
\pgfpathmoveto{\pgfqpoint{-0.000000in}{0.000000in}}%
\pgfpathlineto{\pgfqpoint{-0.048611in}{0.000000in}}%
\pgfusepath{stroke,fill}%
}%
\begin{pgfscope}%
\pgfsys@transformshift{0.374692in}{2.831603in}%
\pgfsys@useobject{currentmarker}{}%
\end{pgfscope}%
\end{pgfscope}%
\begin{pgfscope}%
\definecolor{textcolor}{rgb}{0.000000,0.000000,0.000000}%
\pgfsetstrokecolor{textcolor}%
\pgfsetfillcolor{textcolor}%
\pgftext[x=0.100000in, y=2.778842in, left, base]{\color{textcolor}\rmfamily\fontsize{10.000000}{12.000000}\selectfont \(\displaystyle {2.0}\)}%
\end{pgfscope}%
\begin{pgfscope}%
\pgfpathrectangle{\pgfqpoint{0.374692in}{0.521603in}}{\pgfqpoint{4.650000in}{2.310000in}}%
\pgfusepath{clip}%
\pgfsetrectcap%
\pgfsetroundjoin%
\pgfsetlinewidth{1.505625pt}%
\definecolor{currentstroke}{rgb}{0.121569,0.466667,0.705882}%
\pgfsetstrokecolor{currentstroke}%
\pgfsetdash{}{0pt}%
\pgfpathmoveto{\pgfqpoint{2.453959in}{2.841603in}}%
\pgfpathlineto{\pgfqpoint{2.455402in}{2.832308in}}%
\pgfpathlineto{\pgfqpoint{2.476645in}{2.703524in}}%
\pgfpathlineto{\pgfqpoint{2.497887in}{2.582276in}}%
\pgfpathlineto{\pgfqpoint{2.519130in}{2.468124in}}%
\pgfpathlineto{\pgfqpoint{2.540373in}{2.360652in}}%
\pgfpathlineto{\pgfqpoint{2.561615in}{2.259469in}}%
\pgfpathlineto{\pgfqpoint{2.582858in}{2.164208in}}%
\pgfpathlineto{\pgfqpoint{2.604100in}{2.074521in}}%
\pgfpathlineto{\pgfqpoint{2.625343in}{1.990082in}}%
\pgfpathlineto{\pgfqpoint{2.646586in}{1.910585in}}%
\pgfpathlineto{\pgfqpoint{2.667828in}{1.835740in}}%
\pgfpathlineto{\pgfqpoint{2.689071in}{1.765274in}}%
\pgfpathlineto{\pgfqpoint{2.710313in}{1.698933in}}%
\pgfpathlineto{\pgfqpoint{2.731556in}{1.636473in}}%
\pgfpathlineto{\pgfqpoint{2.752798in}{1.577669in}}%
\pgfpathlineto{\pgfqpoint{2.774041in}{1.522306in}}%
\pgfpathlineto{\pgfqpoint{2.795284in}{1.470182in}}%
\pgfpathlineto{\pgfqpoint{2.816526in}{1.421109in}}%
\pgfpathlineto{\pgfqpoint{2.837769in}{1.374908in}}%
\pgfpathlineto{\pgfqpoint{2.859011in}{1.331410in}}%
\pgfpathlineto{\pgfqpoint{2.901496in}{1.251902in}}%
\pgfpathlineto{\pgfqpoint{2.943982in}{1.181427in}}%
\pgfpathlineto{\pgfqpoint{2.986467in}{1.118959in}}%
\pgfpathlineto{\pgfqpoint{3.028952in}{1.063589in}}%
\pgfpathlineto{\pgfqpoint{3.071437in}{1.014509in}}%
\pgfpathlineto{\pgfqpoint{3.113922in}{0.971006in}}%
\pgfpathlineto{\pgfqpoint{3.156407in}{0.932445in}}%
\pgfpathlineto{\pgfqpoint{3.198892in}{0.898265in}}%
\pgfpathlineto{\pgfqpoint{3.241378in}{0.867968in}}%
\pgfpathlineto{\pgfqpoint{3.283863in}{0.841114in}}%
\pgfpathlineto{\pgfqpoint{3.326348in}{0.817310in}}%
\pgfpathlineto{\pgfqpoint{3.368833in}{0.796212in}}%
\pgfpathlineto{\pgfqpoint{3.411318in}{0.777510in}}%
\pgfpathlineto{\pgfqpoint{3.453803in}{0.760933in}}%
\pgfpathlineto{\pgfqpoint{3.496289in}{0.746239in}}%
\pgfpathlineto{\pgfqpoint{3.560016in}{0.727269in}}%
\pgfpathlineto{\pgfqpoint{3.623744in}{0.711437in}}%
\pgfpathlineto{\pgfqpoint{3.687472in}{0.698226in}}%
\pgfpathlineto{\pgfqpoint{3.751199in}{0.687201in}}%
\pgfpathlineto{\pgfqpoint{3.836170in}{0.675285in}}%
\pgfpathlineto{\pgfqpoint{3.921140in}{0.665923in}}%
\pgfpathlineto{\pgfqpoint{4.027353in}{0.656990in}}%
\pgfpathlineto{\pgfqpoint{4.133566in}{0.650382in}}%
\pgfpathlineto{\pgfqpoint{4.261021in}{0.644681in}}%
\pgfpathlineto{\pgfqpoint{4.409719in}{0.640178in}}%
\pgfpathlineto{\pgfqpoint{4.600903in}{0.636587in}}%
\pgfpathlineto{\pgfqpoint{4.813328in}{0.634330in}}%
\pgfpathlineto{\pgfqpoint{4.813328in}{0.634330in}}%
\pgfusepath{stroke}%
\end{pgfscope}%
\begin{pgfscope}%
\pgfpathrectangle{\pgfqpoint{0.374692in}{0.521603in}}{\pgfqpoint{4.650000in}{2.310000in}}%
\pgfusepath{clip}%
\pgfsetrectcap%
\pgfsetroundjoin%
\pgfsetlinewidth{1.505625pt}%
\definecolor{currentstroke}{rgb}{1.000000,0.498039,0.054902}%
\pgfsetstrokecolor{currentstroke}%
\pgfsetdash{}{0pt}%
\pgfpathmoveto{\pgfqpoint{2.208217in}{2.841603in}}%
\pgfpathlineto{\pgfqpoint{2.242977in}{2.734996in}}%
\pgfpathlineto{\pgfqpoint{2.285462in}{2.611906in}}%
\pgfpathlineto{\pgfqpoint{2.327947in}{2.496020in}}%
\pgfpathlineto{\pgfqpoint{2.370432in}{2.386916in}}%
\pgfpathlineto{\pgfqpoint{2.412917in}{2.284196in}}%
\pgfpathlineto{\pgfqpoint{2.455402in}{2.187487in}}%
\pgfpathlineto{\pgfqpoint{2.497887in}{2.096438in}}%
\pgfpathlineto{\pgfqpoint{2.540373in}{2.010717in}}%
\pgfpathlineto{\pgfqpoint{2.582858in}{1.930012in}}%
\pgfpathlineto{\pgfqpoint{2.625343in}{1.854030in}}%
\pgfpathlineto{\pgfqpoint{2.667828in}{1.782494in}}%
\pgfpathlineto{\pgfqpoint{2.710313in}{1.715145in}}%
\pgfpathlineto{\pgfqpoint{2.752798in}{1.651737in}}%
\pgfpathlineto{\pgfqpoint{2.795284in}{1.592039in}}%
\pgfpathlineto{\pgfqpoint{2.837769in}{1.535835in}}%
\pgfpathlineto{\pgfqpoint{2.880254in}{1.482920in}}%
\pgfpathlineto{\pgfqpoint{2.922739in}{1.433101in}}%
\pgfpathlineto{\pgfqpoint{2.965224in}{1.386198in}}%
\pgfpathlineto{\pgfqpoint{3.007709in}{1.342040in}}%
\pgfpathlineto{\pgfqpoint{3.050194in}{1.300466in}}%
\pgfpathlineto{\pgfqpoint{3.092680in}{1.261324in}}%
\pgfpathlineto{\pgfqpoint{3.135165in}{1.224473in}}%
\pgfpathlineto{\pgfqpoint{3.177650in}{1.189779in}}%
\pgfpathlineto{\pgfqpoint{3.220135in}{1.157115in}}%
\pgfpathlineto{\pgfqpoint{3.283863in}{1.111667in}}%
\pgfpathlineto{\pgfqpoint{3.347591in}{1.070151in}}%
\pgfpathlineto{\pgfqpoint{3.411318in}{1.032224in}}%
\pgfpathlineto{\pgfqpoint{3.475046in}{0.997578in}}%
\pgfpathlineto{\pgfqpoint{3.538774in}{0.965927in}}%
\pgfpathlineto{\pgfqpoint{3.602501in}{0.937014in}}%
\pgfpathlineto{\pgfqpoint{3.666229in}{0.910602in}}%
\pgfpathlineto{\pgfqpoint{3.729957in}{0.886473in}}%
\pgfpathlineto{\pgfqpoint{3.793685in}{0.864432in}}%
\pgfpathlineto{\pgfqpoint{3.857412in}{0.844296in}}%
\pgfpathlineto{\pgfqpoint{3.942383in}{0.820131in}}%
\pgfpathlineto{\pgfqpoint{4.027353in}{0.798712in}}%
\pgfpathlineto{\pgfqpoint{4.112323in}{0.779726in}}%
\pgfpathlineto{\pgfqpoint{4.197294in}{0.762897in}}%
\pgfpathlineto{\pgfqpoint{4.303506in}{0.744524in}}%
\pgfpathlineto{\pgfqpoint{4.409719in}{0.728722in}}%
\pgfpathlineto{\pgfqpoint{4.515932in}{0.715131in}}%
\pgfpathlineto{\pgfqpoint{4.643388in}{0.701308in}}%
\pgfpathlineto{\pgfqpoint{4.770843in}{0.689773in}}%
\pgfpathlineto{\pgfqpoint{4.813328in}{0.686369in}}%
\pgfpathlineto{\pgfqpoint{4.813328in}{0.686369in}}%
\pgfusepath{stroke}%
\end{pgfscope}%
\begin{pgfscope}%
\pgfpathrectangle{\pgfqpoint{0.374692in}{0.521603in}}{\pgfqpoint{4.650000in}{2.310000in}}%
\pgfusepath{clip}%
\pgfsetrectcap%
\pgfsetroundjoin%
\pgfsetlinewidth{1.505625pt}%
\definecolor{currentstroke}{rgb}{0.172549,0.627451,0.172549}%
\pgfsetstrokecolor{currentstroke}%
\pgfsetdash{}{0pt}%
\pgfpathmoveto{\pgfqpoint{1.716622in}{2.841603in}}%
\pgfpathlineto{\pgfqpoint{1.775640in}{2.750910in}}%
\pgfpathlineto{\pgfqpoint{1.839368in}{2.657197in}}%
\pgfpathlineto{\pgfqpoint{1.903095in}{2.567628in}}%
\pgfpathlineto{\pgfqpoint{1.966823in}{2.482020in}}%
\pgfpathlineto{\pgfqpoint{2.030551in}{2.400197in}}%
\pgfpathlineto{\pgfqpoint{2.094279in}{2.321992in}}%
\pgfpathlineto{\pgfqpoint{2.158006in}{2.247245in}}%
\pgfpathlineto{\pgfqpoint{2.221734in}{2.175804in}}%
\pgfpathlineto{\pgfqpoint{2.285462in}{2.107521in}}%
\pgfpathlineto{\pgfqpoint{2.349189in}{2.042258in}}%
\pgfpathlineto{\pgfqpoint{2.412917in}{1.979881in}}%
\pgfpathlineto{\pgfqpoint{2.476645in}{1.920262in}}%
\pgfpathlineto{\pgfqpoint{2.540373in}{1.863279in}}%
\pgfpathlineto{\pgfqpoint{2.604100in}{1.808816in}}%
\pgfpathlineto{\pgfqpoint{2.667828in}{1.756761in}}%
\pgfpathlineto{\pgfqpoint{2.731556in}{1.707008in}}%
\pgfpathlineto{\pgfqpoint{2.795284in}{1.659455in}}%
\pgfpathlineto{\pgfqpoint{2.880254in}{1.599306in}}%
\pgfpathlineto{\pgfqpoint{2.965224in}{1.542677in}}%
\pgfpathlineto{\pgfqpoint{3.050194in}{1.489361in}}%
\pgfpathlineto{\pgfqpoint{3.135165in}{1.439166in}}%
\pgfpathlineto{\pgfqpoint{3.220135in}{1.391908in}}%
\pgfpathlineto{\pgfqpoint{3.305105in}{1.347415in}}%
\pgfpathlineto{\pgfqpoint{3.390076in}{1.305526in}}%
\pgfpathlineto{\pgfqpoint{3.475046in}{1.266089in}}%
\pgfpathlineto{\pgfqpoint{3.560016in}{1.228959in}}%
\pgfpathlineto{\pgfqpoint{3.644987in}{1.194002in}}%
\pgfpathlineto{\pgfqpoint{3.729957in}{1.161091in}}%
\pgfpathlineto{\pgfqpoint{3.836170in}{1.122647in}}%
\pgfpathlineto{\pgfqpoint{3.942383in}{1.086994in}}%
\pgfpathlineto{\pgfqpoint{4.048596in}{1.053930in}}%
\pgfpathlineto{\pgfqpoint{4.154808in}{1.023266in}}%
\pgfpathlineto{\pgfqpoint{4.261021in}{0.994829in}}%
\pgfpathlineto{\pgfqpoint{4.367234in}{0.968457in}}%
\pgfpathlineto{\pgfqpoint{4.494690in}{0.939325in}}%
\pgfpathlineto{\pgfqpoint{4.622145in}{0.912713in}}%
\pgfpathlineto{\pgfqpoint{4.749601in}{0.888402in}}%
\pgfpathlineto{\pgfqpoint{4.813328in}{0.877047in}}%
\pgfpathlineto{\pgfqpoint{4.813328in}{0.877047in}}%
\pgfusepath{stroke}%
\end{pgfscope}%
\begin{pgfscope}%
\pgfpathrectangle{\pgfqpoint{0.374692in}{0.521603in}}{\pgfqpoint{4.650000in}{2.310000in}}%
\pgfusepath{clip}%
\pgfsetrectcap%
\pgfsetroundjoin%
\pgfsetlinewidth{1.505625pt}%
\definecolor{currentstroke}{rgb}{0.839216,0.152941,0.156863}%
\pgfsetstrokecolor{currentstroke}%
\pgfsetdash{}{0pt}%
\pgfpathmoveto{\pgfqpoint{0.586056in}{1.731603in}}%
\pgfpathlineto{\pgfqpoint{4.813328in}{1.731603in}}%
\pgfpathlineto{\pgfqpoint{4.813328in}{1.731603in}}%
\pgfusepath{stroke}%
\end{pgfscope}%
\begin{pgfscope}%
\pgfpathrectangle{\pgfqpoint{0.374692in}{0.521603in}}{\pgfqpoint{4.650000in}{2.310000in}}%
\pgfusepath{clip}%
\pgfsetrectcap%
\pgfsetroundjoin%
\pgfsetlinewidth{1.505625pt}%
\definecolor{currentstroke}{rgb}{0.580392,0.403922,0.741176}%
\pgfsetstrokecolor{currentstroke}%
\pgfsetdash{}{0pt}%
\pgfpathmoveto{\pgfqpoint{0.586056in}{0.877047in}}%
\pgfpathlineto{\pgfqpoint{0.713511in}{0.900282in}}%
\pgfpathlineto{\pgfqpoint{0.840966in}{0.925718in}}%
\pgfpathlineto{\pgfqpoint{0.968422in}{0.953562in}}%
\pgfpathlineto{\pgfqpoint{1.095877in}{0.984041in}}%
\pgfpathlineto{\pgfqpoint{1.202090in}{1.011634in}}%
\pgfpathlineto{\pgfqpoint{1.308303in}{1.041387in}}%
\pgfpathlineto{\pgfqpoint{1.414516in}{1.073469in}}%
\pgfpathlineto{\pgfqpoint{1.520729in}{1.108063in}}%
\pgfpathlineto{\pgfqpoint{1.626942in}{1.145365in}}%
\pgfpathlineto{\pgfqpoint{1.711912in}{1.177298in}}%
\pgfpathlineto{\pgfqpoint{1.796882in}{1.211217in}}%
\pgfpathlineto{\pgfqpoint{1.881853in}{1.247244in}}%
\pgfpathlineto{\pgfqpoint{1.966823in}{1.285510in}}%
\pgfpathlineto{\pgfqpoint{2.051793in}{1.326155in}}%
\pgfpathlineto{\pgfqpoint{2.136764in}{1.369326in}}%
\pgfpathlineto{\pgfqpoint{2.221734in}{1.415180in}}%
\pgfpathlineto{\pgfqpoint{2.306704in}{1.463885in}}%
\pgfpathlineto{\pgfqpoint{2.391675in}{1.515617in}}%
\pgfpathlineto{\pgfqpoint{2.476645in}{1.570564in}}%
\pgfpathlineto{\pgfqpoint{2.561615in}{1.628927in}}%
\pgfpathlineto{\pgfqpoint{2.625343in}{1.675068in}}%
\pgfpathlineto{\pgfqpoint{2.689071in}{1.723343in}}%
\pgfpathlineto{\pgfqpoint{2.752798in}{1.773852in}}%
\pgfpathlineto{\pgfqpoint{2.816526in}{1.826697in}}%
\pgfpathlineto{\pgfqpoint{2.880254in}{1.881988in}}%
\pgfpathlineto{\pgfqpoint{2.943982in}{1.939836in}}%
\pgfpathlineto{\pgfqpoint{3.007709in}{2.000360in}}%
\pgfpathlineto{\pgfqpoint{3.071437in}{2.063685in}}%
\pgfpathlineto{\pgfqpoint{3.135165in}{2.129940in}}%
\pgfpathlineto{\pgfqpoint{3.198892in}{2.199259in}}%
\pgfpathlineto{\pgfqpoint{3.262620in}{2.271786in}}%
\pgfpathlineto{\pgfqpoint{3.326348in}{2.347668in}}%
\pgfpathlineto{\pgfqpoint{3.390076in}{2.427061in}}%
\pgfpathlineto{\pgfqpoint{3.453803in}{2.510127in}}%
\pgfpathlineto{\pgfqpoint{3.517531in}{2.597036in}}%
\pgfpathlineto{\pgfqpoint{3.581259in}{2.687965in}}%
\pgfpathlineto{\pgfqpoint{3.644987in}{2.783102in}}%
\pgfpathlineto{\pgfqpoint{3.682762in}{2.841603in}}%
\pgfpathlineto{\pgfqpoint{3.682762in}{2.841603in}}%
\pgfusepath{stroke}%
\end{pgfscope}%
\begin{pgfscope}%
\pgfpathrectangle{\pgfqpoint{0.374692in}{0.521603in}}{\pgfqpoint{4.650000in}{2.310000in}}%
\pgfusepath{clip}%
\pgfsetrectcap%
\pgfsetroundjoin%
\pgfsetlinewidth{1.505625pt}%
\definecolor{currentstroke}{rgb}{0.549020,0.337255,0.294118}%
\pgfsetstrokecolor{currentstroke}%
\pgfsetdash{}{0pt}%
\pgfpathmoveto{\pgfqpoint{0.586056in}{0.686369in}}%
\pgfpathlineto{\pgfqpoint{0.713511in}{0.697229in}}%
\pgfpathlineto{\pgfqpoint{0.840966in}{0.710243in}}%
\pgfpathlineto{\pgfqpoint{0.947179in}{0.723038in}}%
\pgfpathlineto{\pgfqpoint{1.053392in}{0.737916in}}%
\pgfpathlineto{\pgfqpoint{1.159605in}{0.755214in}}%
\pgfpathlineto{\pgfqpoint{1.244575in}{0.771058in}}%
\pgfpathlineto{\pgfqpoint{1.329546in}{0.788933in}}%
\pgfpathlineto{\pgfqpoint{1.414516in}{0.809099in}}%
\pgfpathlineto{\pgfqpoint{1.499486in}{0.831850in}}%
\pgfpathlineto{\pgfqpoint{1.584457in}{0.857517in}}%
\pgfpathlineto{\pgfqpoint{1.648184in}{0.878904in}}%
\pgfpathlineto{\pgfqpoint{1.711912in}{0.902315in}}%
\pgfpathlineto{\pgfqpoint{1.775640in}{0.927943in}}%
\pgfpathlineto{\pgfqpoint{1.839368in}{0.955998in}}%
\pgfpathlineto{\pgfqpoint{1.903095in}{0.986708in}}%
\pgfpathlineto{\pgfqpoint{1.966823in}{1.020325in}}%
\pgfpathlineto{\pgfqpoint{2.030551in}{1.057125in}}%
\pgfpathlineto{\pgfqpoint{2.094279in}{1.097409in}}%
\pgfpathlineto{\pgfqpoint{2.158006in}{1.141507in}}%
\pgfpathlineto{\pgfqpoint{2.200491in}{1.173201in}}%
\pgfpathlineto{\pgfqpoint{2.242977in}{1.206865in}}%
\pgfpathlineto{\pgfqpoint{2.285462in}{1.242621in}}%
\pgfpathlineto{\pgfqpoint{2.327947in}{1.280600in}}%
\pgfpathlineto{\pgfqpoint{2.370432in}{1.320939in}}%
\pgfpathlineto{\pgfqpoint{2.412917in}{1.363786in}}%
\pgfpathlineto{\pgfqpoint{2.455402in}{1.409296in}}%
\pgfpathlineto{\pgfqpoint{2.497887in}{1.457635in}}%
\pgfpathlineto{\pgfqpoint{2.540373in}{1.508979in}}%
\pgfpathlineto{\pgfqpoint{2.582858in}{1.563513in}}%
\pgfpathlineto{\pgfqpoint{2.625343in}{1.621438in}}%
\pgfpathlineto{\pgfqpoint{2.667828in}{1.682963in}}%
\pgfpathlineto{\pgfqpoint{2.710313in}{1.748312in}}%
\pgfpathlineto{\pgfqpoint{2.752798in}{1.817723in}}%
\pgfpathlineto{\pgfqpoint{2.795284in}{1.891448in}}%
\pgfpathlineto{\pgfqpoint{2.837769in}{1.969756in}}%
\pgfpathlineto{\pgfqpoint{2.880254in}{2.052931in}}%
\pgfpathlineto{\pgfqpoint{2.922739in}{2.141276in}}%
\pgfpathlineto{\pgfqpoint{2.965224in}{2.235113in}}%
\pgfpathlineto{\pgfqpoint{3.007709in}{2.334782in}}%
\pgfpathlineto{\pgfqpoint{3.050194in}{2.440646in}}%
\pgfpathlineto{\pgfqpoint{3.092680in}{2.553090in}}%
\pgfpathlineto{\pgfqpoint{3.135165in}{2.672523in}}%
\pgfpathlineto{\pgfqpoint{3.177650in}{2.799380in}}%
\pgfpathlineto{\pgfqpoint{3.191167in}{2.841603in}}%
\pgfpathlineto{\pgfqpoint{3.191167in}{2.841603in}}%
\pgfusepath{stroke}%
\end{pgfscope}%
\begin{pgfscope}%
\pgfpathrectangle{\pgfqpoint{0.374692in}{0.521603in}}{\pgfqpoint{4.650000in}{2.310000in}}%
\pgfusepath{clip}%
\pgfsetrectcap%
\pgfsetroundjoin%
\pgfsetlinewidth{1.505625pt}%
\definecolor{currentstroke}{rgb}{0.890196,0.466667,0.760784}%
\pgfsetstrokecolor{currentstroke}%
\pgfsetdash{}{0pt}%
\pgfpathmoveto{\pgfqpoint{0.586056in}{0.634330in}}%
\pgfpathlineto{\pgfqpoint{0.862209in}{0.637575in}}%
\pgfpathlineto{\pgfqpoint{1.053392in}{0.641878in}}%
\pgfpathlineto{\pgfqpoint{1.202090in}{0.647274in}}%
\pgfpathlineto{\pgfqpoint{1.329546in}{0.654106in}}%
\pgfpathlineto{\pgfqpoint{1.435759in}{0.662024in}}%
\pgfpathlineto{\pgfqpoint{1.520729in}{0.670322in}}%
\pgfpathlineto{\pgfqpoint{1.605699in}{0.680884in}}%
\pgfpathlineto{\pgfqpoint{1.669427in}{0.690657in}}%
\pgfpathlineto{\pgfqpoint{1.733155in}{0.702367in}}%
\pgfpathlineto{\pgfqpoint{1.796882in}{0.716400in}}%
\pgfpathlineto{\pgfqpoint{1.860610in}{0.733215in}}%
\pgfpathlineto{\pgfqpoint{1.903095in}{0.746239in}}%
\pgfpathlineto{\pgfqpoint{1.945580in}{0.760933in}}%
\pgfpathlineto{\pgfqpoint{1.988066in}{0.777510in}}%
\pgfpathlineto{\pgfqpoint{2.030551in}{0.796212in}}%
\pgfpathlineto{\pgfqpoint{2.073036in}{0.817310in}}%
\pgfpathlineto{\pgfqpoint{2.115521in}{0.841114in}}%
\pgfpathlineto{\pgfqpoint{2.158006in}{0.867968in}}%
\pgfpathlineto{\pgfqpoint{2.200491in}{0.898265in}}%
\pgfpathlineto{\pgfqpoint{2.242977in}{0.932445in}}%
\pgfpathlineto{\pgfqpoint{2.285462in}{0.971006in}}%
\pgfpathlineto{\pgfqpoint{2.327947in}{1.014509in}}%
\pgfpathlineto{\pgfqpoint{2.370432in}{1.063589in}}%
\pgfpathlineto{\pgfqpoint{2.412917in}{1.118959in}}%
\pgfpathlineto{\pgfqpoint{2.455402in}{1.181427in}}%
\pgfpathlineto{\pgfqpoint{2.476645in}{1.215603in}}%
\pgfpathlineto{\pgfqpoint{2.497887in}{1.251902in}}%
\pgfpathlineto{\pgfqpoint{2.519130in}{1.290458in}}%
\pgfpathlineto{\pgfqpoint{2.540373in}{1.331410in}}%
\pgfpathlineto{\pgfqpoint{2.561615in}{1.374908in}}%
\pgfpathlineto{\pgfqpoint{2.582858in}{1.421109in}}%
\pgfpathlineto{\pgfqpoint{2.604100in}{1.470182in}}%
\pgfpathlineto{\pgfqpoint{2.625343in}{1.522306in}}%
\pgfpathlineto{\pgfqpoint{2.646586in}{1.577669in}}%
\pgfpathlineto{\pgfqpoint{2.667828in}{1.636473in}}%
\pgfpathlineto{\pgfqpoint{2.689071in}{1.698933in}}%
\pgfpathlineto{\pgfqpoint{2.710313in}{1.765274in}}%
\pgfpathlineto{\pgfqpoint{2.731556in}{1.835740in}}%
\pgfpathlineto{\pgfqpoint{2.752798in}{1.910585in}}%
\pgfpathlineto{\pgfqpoint{2.774041in}{1.990082in}}%
\pgfpathlineto{\pgfqpoint{2.795284in}{2.074521in}}%
\pgfpathlineto{\pgfqpoint{2.816526in}{2.164208in}}%
\pgfpathlineto{\pgfqpoint{2.837769in}{2.259469in}}%
\pgfpathlineto{\pgfqpoint{2.859011in}{2.360652in}}%
\pgfpathlineto{\pgfqpoint{2.880254in}{2.468124in}}%
\pgfpathlineto{\pgfqpoint{2.901496in}{2.582276in}}%
\pgfpathlineto{\pgfqpoint{2.922739in}{2.703524in}}%
\pgfpathlineto{\pgfqpoint{2.945425in}{2.841603in}}%
\pgfpathlineto{\pgfqpoint{2.945425in}{2.841603in}}%
\pgfusepath{stroke}%
\end{pgfscope}%
\begin{pgfscope}%
\pgfpathrectangle{\pgfqpoint{0.374692in}{0.521603in}}{\pgfqpoint{4.650000in}{2.310000in}}%
\pgfusepath{clip}%
\pgfsetrectcap%
\pgfsetroundjoin%
\pgfsetlinewidth{0.501875pt}%
\definecolor{currentstroke}{rgb}{0.000000,0.000000,0.000000}%
\pgfsetstrokecolor{currentstroke}%
\pgfsetdash{}{0pt}%
\pgfpathmoveto{\pgfqpoint{0.374692in}{0.631603in}}%
\pgfpathlineto{\pgfqpoint{5.024692in}{0.631603in}}%
\pgfusepath{stroke}%
\end{pgfscope}%
\begin{pgfscope}%
\pgfpathrectangle{\pgfqpoint{0.374692in}{0.521603in}}{\pgfqpoint{4.650000in}{2.310000in}}%
\pgfusepath{clip}%
\pgfsetrectcap%
\pgfsetroundjoin%
\pgfsetlinewidth{0.501875pt}%
\definecolor{currentstroke}{rgb}{0.000000,0.000000,0.000000}%
\pgfsetstrokecolor{currentstroke}%
\pgfsetdash{}{0pt}%
\pgfpathmoveto{\pgfqpoint{2.699692in}{0.521603in}}%
\pgfpathlineto{\pgfqpoint{2.699692in}{2.831603in}}%
\pgfusepath{stroke}%
\end{pgfscope}%
\begin{pgfscope}%
\definecolor{textcolor}{rgb}{0.000000,0.000000,0.000000}%
\pgfsetstrokecolor{textcolor}%
\pgfsetfillcolor{textcolor}%
\pgftext[x=2.699692in,y=2.914937in,,base]{\color{textcolor}\rmfamily\fontsize{12.000000}{14.400000}\selectfont \(\displaystyle e^{t \cdot \tau}\)}%
\end{pgfscope}%
\begin{pgfscope}%
\pgfsetbuttcap%
\pgfsetmiterjoin%
\definecolor{currentfill}{rgb}{1.000000,1.000000,1.000000}%
\pgfsetfillcolor{currentfill}%
\pgfsetfillopacity{0.800000}%
\pgfsetlinewidth{1.003750pt}%
\definecolor{currentstroke}{rgb}{0.800000,0.800000,0.800000}%
\pgfsetstrokecolor{currentstroke}%
\pgfsetstrokeopacity{0.800000}%
\pgfsetdash{}{0pt}%
\pgfpathmoveto{\pgfqpoint{3.935910in}{1.293492in}}%
\pgfpathlineto{\pgfqpoint{4.927470in}{1.293492in}}%
\pgfpathquadraticcurveto{\pgfqpoint{4.955247in}{1.293492in}}{\pgfqpoint{4.955247in}{1.321269in}}%
\pgfpathlineto{\pgfqpoint{4.955247in}{2.734381in}}%
\pgfpathquadraticcurveto{\pgfqpoint{4.955247in}{2.762159in}}{\pgfqpoint{4.927470in}{2.762159in}}%
\pgfpathlineto{\pgfqpoint{3.935910in}{2.762159in}}%
\pgfpathquadraticcurveto{\pgfqpoint{3.908132in}{2.762159in}}{\pgfqpoint{3.908132in}{2.734381in}}%
\pgfpathlineto{\pgfqpoint{3.908132in}{1.321269in}}%
\pgfpathquadraticcurveto{\pgfqpoint{3.908132in}{1.293492in}}{\pgfqpoint{3.935910in}{1.293492in}}%
\pgfpathlineto{\pgfqpoint{3.935910in}{1.293492in}}%
\pgfpathclose%
\pgfusepath{stroke,fill}%
\end{pgfscope}%
\begin{pgfscope}%
\pgfsetrectcap%
\pgfsetroundjoin%
\pgfsetlinewidth{1.505625pt}%
\definecolor{currentstroke}{rgb}{0.121569,0.466667,0.705882}%
\pgfsetstrokecolor{currentstroke}%
\pgfsetdash{}{0pt}%
\pgfpathmoveto{\pgfqpoint{3.963687in}{2.649691in}}%
\pgfpathlineto{\pgfqpoint{4.102576in}{2.649691in}}%
\pgfpathlineto{\pgfqpoint{4.241465in}{2.649691in}}%
\pgfusepath{stroke}%
\end{pgfscope}%
\begin{pgfscope}%
\definecolor{textcolor}{rgb}{0.000000,0.000000,0.000000}%
\pgfsetstrokecolor{textcolor}%
\pgfsetfillcolor{textcolor}%
\pgftext[x=4.352576in,y=2.601080in,left,base]{\color{textcolor}\rmfamily\fontsize{10.000000}{12.000000}\selectfont \(\displaystyle \tau = -2\)}%
\end{pgfscope}%
\begin{pgfscope}%
\pgfsetrectcap%
\pgfsetroundjoin%
\pgfsetlinewidth{1.505625pt}%
\definecolor{currentstroke}{rgb}{1.000000,0.498039,0.054902}%
\pgfsetstrokecolor{currentstroke}%
\pgfsetdash{}{0pt}%
\pgfpathmoveto{\pgfqpoint{3.963687in}{2.445834in}}%
\pgfpathlineto{\pgfqpoint{4.102576in}{2.445834in}}%
\pgfpathlineto{\pgfqpoint{4.241465in}{2.445834in}}%
\pgfusepath{stroke}%
\end{pgfscope}%
\begin{pgfscope}%
\definecolor{textcolor}{rgb}{0.000000,0.000000,0.000000}%
\pgfsetstrokecolor{textcolor}%
\pgfsetfillcolor{textcolor}%
\pgftext[x=4.352576in,y=2.397223in,left,base]{\color{textcolor}\rmfamily\fontsize{10.000000}{12.000000}\selectfont \(\displaystyle \tau = -1\)}%
\end{pgfscope}%
\begin{pgfscope}%
\pgfsetrectcap%
\pgfsetroundjoin%
\pgfsetlinewidth{1.505625pt}%
\definecolor{currentstroke}{rgb}{0.172549,0.627451,0.172549}%
\pgfsetstrokecolor{currentstroke}%
\pgfsetdash{}{0pt}%
\pgfpathmoveto{\pgfqpoint{3.963687in}{2.241977in}}%
\pgfpathlineto{\pgfqpoint{4.102576in}{2.241977in}}%
\pgfpathlineto{\pgfqpoint{4.241465in}{2.241977in}}%
\pgfusepath{stroke}%
\end{pgfscope}%
\begin{pgfscope}%
\definecolor{textcolor}{rgb}{0.000000,0.000000,0.000000}%
\pgfsetstrokecolor{textcolor}%
\pgfsetfillcolor{textcolor}%
\pgftext[x=4.352576in,y=2.193366in,left,base]{\color{textcolor}\rmfamily\fontsize{10.000000}{12.000000}\selectfont \(\displaystyle \tau = -0.5\)}%
\end{pgfscope}%
\begin{pgfscope}%
\pgfsetrectcap%
\pgfsetroundjoin%
\pgfsetlinewidth{1.505625pt}%
\definecolor{currentstroke}{rgb}{0.839216,0.152941,0.156863}%
\pgfsetstrokecolor{currentstroke}%
\pgfsetdash{}{0pt}%
\pgfpathmoveto{\pgfqpoint{3.963687in}{2.038120in}}%
\pgfpathlineto{\pgfqpoint{4.102576in}{2.038120in}}%
\pgfpathlineto{\pgfqpoint{4.241465in}{2.038120in}}%
\pgfusepath{stroke}%
\end{pgfscope}%
\begin{pgfscope}%
\definecolor{textcolor}{rgb}{0.000000,0.000000,0.000000}%
\pgfsetstrokecolor{textcolor}%
\pgfsetfillcolor{textcolor}%
\pgftext[x=4.352576in,y=1.989509in,left,base]{\color{textcolor}\rmfamily\fontsize{10.000000}{12.000000}\selectfont \(\displaystyle \tau = 0\)}%
\end{pgfscope}%
\begin{pgfscope}%
\pgfsetrectcap%
\pgfsetroundjoin%
\pgfsetlinewidth{1.505625pt}%
\definecolor{currentstroke}{rgb}{0.580392,0.403922,0.741176}%
\pgfsetstrokecolor{currentstroke}%
\pgfsetdash{}{0pt}%
\pgfpathmoveto{\pgfqpoint{3.963687in}{1.834263in}}%
\pgfpathlineto{\pgfqpoint{4.102576in}{1.834263in}}%
\pgfpathlineto{\pgfqpoint{4.241465in}{1.834263in}}%
\pgfusepath{stroke}%
\end{pgfscope}%
\begin{pgfscope}%
\definecolor{textcolor}{rgb}{0.000000,0.000000,0.000000}%
\pgfsetstrokecolor{textcolor}%
\pgfsetfillcolor{textcolor}%
\pgftext[x=4.352576in,y=1.785651in,left,base]{\color{textcolor}\rmfamily\fontsize{10.000000}{12.000000}\selectfont \(\displaystyle \tau = 0.5\)}%
\end{pgfscope}%
\begin{pgfscope}%
\pgfsetrectcap%
\pgfsetroundjoin%
\pgfsetlinewidth{1.505625pt}%
\definecolor{currentstroke}{rgb}{0.549020,0.337255,0.294118}%
\pgfsetstrokecolor{currentstroke}%
\pgfsetdash{}{0pt}%
\pgfpathmoveto{\pgfqpoint{3.963687in}{1.630405in}}%
\pgfpathlineto{\pgfqpoint{4.102576in}{1.630405in}}%
\pgfpathlineto{\pgfqpoint{4.241465in}{1.630405in}}%
\pgfusepath{stroke}%
\end{pgfscope}%
\begin{pgfscope}%
\definecolor{textcolor}{rgb}{0.000000,0.000000,0.000000}%
\pgfsetstrokecolor{textcolor}%
\pgfsetfillcolor{textcolor}%
\pgftext[x=4.352576in,y=1.581794in,left,base]{\color{textcolor}\rmfamily\fontsize{10.000000}{12.000000}\selectfont \(\displaystyle \tau = 1\)}%
\end{pgfscope}%
\begin{pgfscope}%
\pgfsetrectcap%
\pgfsetroundjoin%
\pgfsetlinewidth{1.505625pt}%
\definecolor{currentstroke}{rgb}{0.890196,0.466667,0.760784}%
\pgfsetstrokecolor{currentstroke}%
\pgfsetdash{}{0pt}%
\pgfpathmoveto{\pgfqpoint{3.963687in}{1.426548in}}%
\pgfpathlineto{\pgfqpoint{4.102576in}{1.426548in}}%
\pgfpathlineto{\pgfqpoint{4.241465in}{1.426548in}}%
\pgfusepath{stroke}%
\end{pgfscope}%
\begin{pgfscope}%
\definecolor{textcolor}{rgb}{0.000000,0.000000,0.000000}%
\pgfsetstrokecolor{textcolor}%
\pgfsetfillcolor{textcolor}%
\pgftext[x=4.352576in,y=1.377937in,left,base]{\color{textcolor}\rmfamily\fontsize{10.000000}{12.000000}\selectfont \(\displaystyle \tau = 2\)}%
\end{pgfscope}%
\end{pgfpicture}%
\makeatother%
\endgroup%

	\caption{Die exponentialfunktion mit verschiedenen multiplikatoren im Exponenten.}
	\label{fig:expVersions}
\end{figure}

\todo{TODO}


\section{Sinc}
\index{Sinc Funktion}
\index{Impulsantwort}
\index{Filter}

Die sinc Funktion ist definiert als:
\begin{equation}
 sinc (x)={\begin{cases} \frac{sin(x)}{x} ,&x\neq 0\\{1},&x=0\end{cases}}
\end{equation}

Alternativ wird manchmal die \emph{normalisierte sinc Funktion} verwendet bei der der Faktor $\pi$ in die Funktion hineingenommen wird: $sin(\pi x) / \pi x$.

Die Funktion hat für uns Relevanz u.a. da sie die Impulsantwort eines idealen Lowpass Filters beschreibt \citep{smith1997scientist}\footnote{\href{https://www.dspguide.com/ch16/1.htm}{https://www.dspguide.com/ch16/1.htm}} , \citep{enwiki:sincFilter}. Eine Möglichkeit sich diese Tatsache zu verdeutlichen ist dass die $sinc$ Funktion aus Kosinus Schwingungen unterschiedlicher Frequenzen zusammengesetzt werden kann, siehe Abb \ref{fig:sincAsSum}. 



\begin{figure}[H]
	\centering
	%% Creator: Matplotlib, PGF backend
%%
%% To include the figure in your LaTeX document, write
%%   \input{<filename>.pgf}
%%
%% Make sure the required packages are loaded in your preamble
%%   \usepackage{pgf}
%%
%% Also ensure that all the required font packages are loaded; for instance,
%% the lmodern package is sometimes necessary when using math font.
%%   \usepackage{lmodern}
%%
%% Figures using additional raster images can only be included by \input if
%% they are in the same directory as the main LaTeX file. For loading figures
%% from other directories you can use the `import` package
%%   \usepackage{import}
%%
%% and then include the figures with
%%   \import{<path to file>}{<filename>.pgf}
%%
%% Matplotlib used the following preamble
%%   
%%   \usepackage{fontspec}
%%   \setmainfont{DejaVuSerif.ttf}[Path=\detokenize{/home/pl/anaconda3/lib/python3.11/site-packages/matplotlib/mpl-data/fonts/ttf/}]
%%   \setsansfont{DejaVuSans.ttf}[Path=\detokenize{/home/pl/anaconda3/lib/python3.11/site-packages/matplotlib/mpl-data/fonts/ttf/}]
%%   \setmonofont{DejaVuSansMono.ttf}[Path=\detokenize{/home/pl/anaconda3/lib/python3.11/site-packages/matplotlib/mpl-data/fonts/ttf/}]
%%   \makeatletter\@ifpackageloaded{underscore}{}{\usepackage[strings]{underscore}}\makeatother
%%
\begingroup%
\makeatletter%
\begin{pgfpicture}%
\pgfpathrectangle{\pgfpointorigin}{\pgfqpoint{5.276127in}{2.931603in}}%
\pgfusepath{use as bounding box, clip}%
\begin{pgfscope}%
\pgfsetbuttcap%
\pgfsetmiterjoin%
\definecolor{currentfill}{rgb}{1.000000,1.000000,1.000000}%
\pgfsetfillcolor{currentfill}%
\pgfsetlinewidth{0.000000pt}%
\definecolor{currentstroke}{rgb}{1.000000,1.000000,1.000000}%
\pgfsetstrokecolor{currentstroke}%
\pgfsetdash{}{0pt}%
\pgfpathmoveto{\pgfqpoint{0.000000in}{0.000000in}}%
\pgfpathlineto{\pgfqpoint{5.276127in}{0.000000in}}%
\pgfpathlineto{\pgfqpoint{5.276127in}{2.931603in}}%
\pgfpathlineto{\pgfqpoint{0.000000in}{2.931603in}}%
\pgfpathlineto{\pgfqpoint{0.000000in}{0.000000in}}%
\pgfpathclose%
\pgfusepath{fill}%
\end{pgfscope}%
\begin{pgfscope}%
\pgfsetbuttcap%
\pgfsetmiterjoin%
\definecolor{currentfill}{rgb}{1.000000,1.000000,1.000000}%
\pgfsetfillcolor{currentfill}%
\pgfsetlinewidth{0.000000pt}%
\definecolor{currentstroke}{rgb}{0.000000,0.000000,0.000000}%
\pgfsetstrokecolor{currentstroke}%
\pgfsetstrokeopacity{0.000000}%
\pgfsetdash{}{0pt}%
\pgfpathmoveto{\pgfqpoint{0.526127in}{0.521603in}}%
\pgfpathlineto{\pgfqpoint{5.176127in}{0.521603in}}%
\pgfpathlineto{\pgfqpoint{5.176127in}{2.831603in}}%
\pgfpathlineto{\pgfqpoint{0.526127in}{2.831603in}}%
\pgfpathlineto{\pgfqpoint{0.526127in}{0.521603in}}%
\pgfpathclose%
\pgfusepath{fill}%
\end{pgfscope}%
\begin{pgfscope}%
\pgfsetbuttcap%
\pgfsetroundjoin%
\definecolor{currentfill}{rgb}{0.000000,0.000000,0.000000}%
\pgfsetfillcolor{currentfill}%
\pgfsetlinewidth{0.803000pt}%
\definecolor{currentstroke}{rgb}{0.000000,0.000000,0.000000}%
\pgfsetstrokecolor{currentstroke}%
\pgfsetdash{}{0pt}%
\pgfsys@defobject{currentmarker}{\pgfqpoint{0.000000in}{-0.048611in}}{\pgfqpoint{0.000000in}{0.000000in}}{%
\pgfpathmoveto{\pgfqpoint{0.000000in}{0.000000in}}%
\pgfpathlineto{\pgfqpoint{0.000000in}{-0.048611in}}%
\pgfusepath{stroke,fill}%
}%
\begin{pgfscope}%
\pgfsys@transformshift{0.737490in}{0.521603in}%
\pgfsys@useobject{currentmarker}{}%
\end{pgfscope}%
\end{pgfscope}%
\begin{pgfscope}%
\definecolor{textcolor}{rgb}{0.000000,0.000000,0.000000}%
\pgfsetstrokecolor{textcolor}%
\pgfsetfillcolor{textcolor}%
\pgftext[x=0.737490in,y=0.424381in,,top]{\color{textcolor}\rmfamily\fontsize{10.000000}{12.000000}\selectfont \ensuremath{-}30}%
\end{pgfscope}%
\begin{pgfscope}%
\pgfsetbuttcap%
\pgfsetroundjoin%
\definecolor{currentfill}{rgb}{0.000000,0.000000,0.000000}%
\pgfsetfillcolor{currentfill}%
\pgfsetlinewidth{0.803000pt}%
\definecolor{currentstroke}{rgb}{0.000000,0.000000,0.000000}%
\pgfsetstrokecolor{currentstroke}%
\pgfsetdash{}{0pt}%
\pgfsys@defobject{currentmarker}{\pgfqpoint{0.000000in}{-0.048611in}}{\pgfqpoint{0.000000in}{0.000000in}}{%
\pgfpathmoveto{\pgfqpoint{0.000000in}{0.000000in}}%
\pgfpathlineto{\pgfqpoint{0.000000in}{-0.048611in}}%
\pgfusepath{stroke,fill}%
}%
\begin{pgfscope}%
\pgfsys@transformshift{1.442036in}{0.521603in}%
\pgfsys@useobject{currentmarker}{}%
\end{pgfscope}%
\end{pgfscope}%
\begin{pgfscope}%
\definecolor{textcolor}{rgb}{0.000000,0.000000,0.000000}%
\pgfsetstrokecolor{textcolor}%
\pgfsetfillcolor{textcolor}%
\pgftext[x=1.442036in,y=0.424381in,,top]{\color{textcolor}\rmfamily\fontsize{10.000000}{12.000000}\selectfont \ensuremath{-}20}%
\end{pgfscope}%
\begin{pgfscope}%
\pgfsetbuttcap%
\pgfsetroundjoin%
\definecolor{currentfill}{rgb}{0.000000,0.000000,0.000000}%
\pgfsetfillcolor{currentfill}%
\pgfsetlinewidth{0.803000pt}%
\definecolor{currentstroke}{rgb}{0.000000,0.000000,0.000000}%
\pgfsetstrokecolor{currentstroke}%
\pgfsetdash{}{0pt}%
\pgfsys@defobject{currentmarker}{\pgfqpoint{0.000000in}{-0.048611in}}{\pgfqpoint{0.000000in}{0.000000in}}{%
\pgfpathmoveto{\pgfqpoint{0.000000in}{0.000000in}}%
\pgfpathlineto{\pgfqpoint{0.000000in}{-0.048611in}}%
\pgfusepath{stroke,fill}%
}%
\begin{pgfscope}%
\pgfsys@transformshift{2.146581in}{0.521603in}%
\pgfsys@useobject{currentmarker}{}%
\end{pgfscope}%
\end{pgfscope}%
\begin{pgfscope}%
\definecolor{textcolor}{rgb}{0.000000,0.000000,0.000000}%
\pgfsetstrokecolor{textcolor}%
\pgfsetfillcolor{textcolor}%
\pgftext[x=2.146581in,y=0.424381in,,top]{\color{textcolor}\rmfamily\fontsize{10.000000}{12.000000}\selectfont \ensuremath{-}10}%
\end{pgfscope}%
\begin{pgfscope}%
\pgfsetbuttcap%
\pgfsetroundjoin%
\definecolor{currentfill}{rgb}{0.000000,0.000000,0.000000}%
\pgfsetfillcolor{currentfill}%
\pgfsetlinewidth{0.803000pt}%
\definecolor{currentstroke}{rgb}{0.000000,0.000000,0.000000}%
\pgfsetstrokecolor{currentstroke}%
\pgfsetdash{}{0pt}%
\pgfsys@defobject{currentmarker}{\pgfqpoint{0.000000in}{-0.048611in}}{\pgfqpoint{0.000000in}{0.000000in}}{%
\pgfpathmoveto{\pgfqpoint{0.000000in}{0.000000in}}%
\pgfpathlineto{\pgfqpoint{0.000000in}{-0.048611in}}%
\pgfusepath{stroke,fill}%
}%
\begin{pgfscope}%
\pgfsys@transformshift{2.851127in}{0.521603in}%
\pgfsys@useobject{currentmarker}{}%
\end{pgfscope}%
\end{pgfscope}%
\begin{pgfscope}%
\definecolor{textcolor}{rgb}{0.000000,0.000000,0.000000}%
\pgfsetstrokecolor{textcolor}%
\pgfsetfillcolor{textcolor}%
\pgftext[x=2.851127in,y=0.424381in,,top]{\color{textcolor}\rmfamily\fontsize{10.000000}{12.000000}\selectfont 0}%
\end{pgfscope}%
\begin{pgfscope}%
\pgfsetbuttcap%
\pgfsetroundjoin%
\definecolor{currentfill}{rgb}{0.000000,0.000000,0.000000}%
\pgfsetfillcolor{currentfill}%
\pgfsetlinewidth{0.803000pt}%
\definecolor{currentstroke}{rgb}{0.000000,0.000000,0.000000}%
\pgfsetstrokecolor{currentstroke}%
\pgfsetdash{}{0pt}%
\pgfsys@defobject{currentmarker}{\pgfqpoint{0.000000in}{-0.048611in}}{\pgfqpoint{0.000000in}{0.000000in}}{%
\pgfpathmoveto{\pgfqpoint{0.000000in}{0.000000in}}%
\pgfpathlineto{\pgfqpoint{0.000000in}{-0.048611in}}%
\pgfusepath{stroke,fill}%
}%
\begin{pgfscope}%
\pgfsys@transformshift{3.555672in}{0.521603in}%
\pgfsys@useobject{currentmarker}{}%
\end{pgfscope}%
\end{pgfscope}%
\begin{pgfscope}%
\definecolor{textcolor}{rgb}{0.000000,0.000000,0.000000}%
\pgfsetstrokecolor{textcolor}%
\pgfsetfillcolor{textcolor}%
\pgftext[x=3.555672in,y=0.424381in,,top]{\color{textcolor}\rmfamily\fontsize{10.000000}{12.000000}\selectfont 10}%
\end{pgfscope}%
\begin{pgfscope}%
\pgfsetbuttcap%
\pgfsetroundjoin%
\definecolor{currentfill}{rgb}{0.000000,0.000000,0.000000}%
\pgfsetfillcolor{currentfill}%
\pgfsetlinewidth{0.803000pt}%
\definecolor{currentstroke}{rgb}{0.000000,0.000000,0.000000}%
\pgfsetstrokecolor{currentstroke}%
\pgfsetdash{}{0pt}%
\pgfsys@defobject{currentmarker}{\pgfqpoint{0.000000in}{-0.048611in}}{\pgfqpoint{0.000000in}{0.000000in}}{%
\pgfpathmoveto{\pgfqpoint{0.000000in}{0.000000in}}%
\pgfpathlineto{\pgfqpoint{0.000000in}{-0.048611in}}%
\pgfusepath{stroke,fill}%
}%
\begin{pgfscope}%
\pgfsys@transformshift{4.260218in}{0.521603in}%
\pgfsys@useobject{currentmarker}{}%
\end{pgfscope}%
\end{pgfscope}%
\begin{pgfscope}%
\definecolor{textcolor}{rgb}{0.000000,0.000000,0.000000}%
\pgfsetstrokecolor{textcolor}%
\pgfsetfillcolor{textcolor}%
\pgftext[x=4.260218in,y=0.424381in,,top]{\color{textcolor}\rmfamily\fontsize{10.000000}{12.000000}\selectfont 20}%
\end{pgfscope}%
\begin{pgfscope}%
\pgfsetbuttcap%
\pgfsetroundjoin%
\definecolor{currentfill}{rgb}{0.000000,0.000000,0.000000}%
\pgfsetfillcolor{currentfill}%
\pgfsetlinewidth{0.803000pt}%
\definecolor{currentstroke}{rgb}{0.000000,0.000000,0.000000}%
\pgfsetstrokecolor{currentstroke}%
\pgfsetdash{}{0pt}%
\pgfsys@defobject{currentmarker}{\pgfqpoint{0.000000in}{-0.048611in}}{\pgfqpoint{0.000000in}{0.000000in}}{%
\pgfpathmoveto{\pgfqpoint{0.000000in}{0.000000in}}%
\pgfpathlineto{\pgfqpoint{0.000000in}{-0.048611in}}%
\pgfusepath{stroke,fill}%
}%
\begin{pgfscope}%
\pgfsys@transformshift{4.964763in}{0.521603in}%
\pgfsys@useobject{currentmarker}{}%
\end{pgfscope}%
\end{pgfscope}%
\begin{pgfscope}%
\definecolor{textcolor}{rgb}{0.000000,0.000000,0.000000}%
\pgfsetstrokecolor{textcolor}%
\pgfsetfillcolor{textcolor}%
\pgftext[x=4.964763in,y=0.424381in,,top]{\color{textcolor}\rmfamily\fontsize{10.000000}{12.000000}\selectfont 30}%
\end{pgfscope}%
\begin{pgfscope}%
\definecolor{textcolor}{rgb}{0.000000,0.000000,0.000000}%
\pgfsetstrokecolor{textcolor}%
\pgfsetfillcolor{textcolor}%
\pgftext[x=2.851127in,y=0.234413in,,top]{\color{textcolor}\rmfamily\fontsize{10.000000}{12.000000}\selectfont \(\displaystyle t\)}%
\end{pgfscope}%
\begin{pgfscope}%
\pgfsetbuttcap%
\pgfsetroundjoin%
\definecolor{currentfill}{rgb}{0.000000,0.000000,0.000000}%
\pgfsetfillcolor{currentfill}%
\pgfsetlinewidth{0.803000pt}%
\definecolor{currentstroke}{rgb}{0.000000,0.000000,0.000000}%
\pgfsetstrokecolor{currentstroke}%
\pgfsetdash{}{0pt}%
\pgfsys@defobject{currentmarker}{\pgfqpoint{-0.048611in}{0.000000in}}{\pgfqpoint{-0.000000in}{0.000000in}}{%
\pgfpathmoveto{\pgfqpoint{-0.000000in}{0.000000in}}%
\pgfpathlineto{\pgfqpoint{-0.048611in}{0.000000in}}%
\pgfusepath{stroke,fill}%
}%
\begin{pgfscope}%
\pgfsys@transformshift{0.526127in}{0.659684in}%
\pgfsys@useobject{currentmarker}{}%
\end{pgfscope}%
\end{pgfscope}%
\begin{pgfscope}%
\definecolor{textcolor}{rgb}{0.000000,0.000000,0.000000}%
\pgfsetstrokecolor{textcolor}%
\pgfsetfillcolor{textcolor}%
\pgftext[x=0.100000in, y=0.606922in, left, base]{\color{textcolor}\rmfamily\fontsize{10.000000}{12.000000}\selectfont \ensuremath{-}0.2}%
\end{pgfscope}%
\begin{pgfscope}%
\pgfsetbuttcap%
\pgfsetroundjoin%
\definecolor{currentfill}{rgb}{0.000000,0.000000,0.000000}%
\pgfsetfillcolor{currentfill}%
\pgfsetlinewidth{0.803000pt}%
\definecolor{currentstroke}{rgb}{0.000000,0.000000,0.000000}%
\pgfsetstrokecolor{currentstroke}%
\pgfsetdash{}{0pt}%
\pgfsys@defobject{currentmarker}{\pgfqpoint{-0.048611in}{0.000000in}}{\pgfqpoint{-0.000000in}{0.000000in}}{%
\pgfpathmoveto{\pgfqpoint{-0.000000in}{0.000000in}}%
\pgfpathlineto{\pgfqpoint{-0.048611in}{0.000000in}}%
\pgfusepath{stroke,fill}%
}%
\begin{pgfscope}%
\pgfsys@transformshift{0.526127in}{1.005255in}%
\pgfsys@useobject{currentmarker}{}%
\end{pgfscope}%
\end{pgfscope}%
\begin{pgfscope}%
\definecolor{textcolor}{rgb}{0.000000,0.000000,0.000000}%
\pgfsetstrokecolor{textcolor}%
\pgfsetfillcolor{textcolor}%
\pgftext[x=0.208025in, y=0.952493in, left, base]{\color{textcolor}\rmfamily\fontsize{10.000000}{12.000000}\selectfont 0.0}%
\end{pgfscope}%
\begin{pgfscope}%
\pgfsetbuttcap%
\pgfsetroundjoin%
\definecolor{currentfill}{rgb}{0.000000,0.000000,0.000000}%
\pgfsetfillcolor{currentfill}%
\pgfsetlinewidth{0.803000pt}%
\definecolor{currentstroke}{rgb}{0.000000,0.000000,0.000000}%
\pgfsetstrokecolor{currentstroke}%
\pgfsetdash{}{0pt}%
\pgfsys@defobject{currentmarker}{\pgfqpoint{-0.048611in}{0.000000in}}{\pgfqpoint{-0.000000in}{0.000000in}}{%
\pgfpathmoveto{\pgfqpoint{-0.000000in}{0.000000in}}%
\pgfpathlineto{\pgfqpoint{-0.048611in}{0.000000in}}%
\pgfusepath{stroke,fill}%
}%
\begin{pgfscope}%
\pgfsys@transformshift{0.526127in}{1.350825in}%
\pgfsys@useobject{currentmarker}{}%
\end{pgfscope}%
\end{pgfscope}%
\begin{pgfscope}%
\definecolor{textcolor}{rgb}{0.000000,0.000000,0.000000}%
\pgfsetstrokecolor{textcolor}%
\pgfsetfillcolor{textcolor}%
\pgftext[x=0.208025in, y=1.298064in, left, base]{\color{textcolor}\rmfamily\fontsize{10.000000}{12.000000}\selectfont 0.2}%
\end{pgfscope}%
\begin{pgfscope}%
\pgfsetbuttcap%
\pgfsetroundjoin%
\definecolor{currentfill}{rgb}{0.000000,0.000000,0.000000}%
\pgfsetfillcolor{currentfill}%
\pgfsetlinewidth{0.803000pt}%
\definecolor{currentstroke}{rgb}{0.000000,0.000000,0.000000}%
\pgfsetstrokecolor{currentstroke}%
\pgfsetdash{}{0pt}%
\pgfsys@defobject{currentmarker}{\pgfqpoint{-0.048611in}{0.000000in}}{\pgfqpoint{-0.000000in}{0.000000in}}{%
\pgfpathmoveto{\pgfqpoint{-0.000000in}{0.000000in}}%
\pgfpathlineto{\pgfqpoint{-0.048611in}{0.000000in}}%
\pgfusepath{stroke,fill}%
}%
\begin{pgfscope}%
\pgfsys@transformshift{0.526127in}{1.696396in}%
\pgfsys@useobject{currentmarker}{}%
\end{pgfscope}%
\end{pgfscope}%
\begin{pgfscope}%
\definecolor{textcolor}{rgb}{0.000000,0.000000,0.000000}%
\pgfsetstrokecolor{textcolor}%
\pgfsetfillcolor{textcolor}%
\pgftext[x=0.208025in, y=1.643634in, left, base]{\color{textcolor}\rmfamily\fontsize{10.000000}{12.000000}\selectfont 0.4}%
\end{pgfscope}%
\begin{pgfscope}%
\pgfsetbuttcap%
\pgfsetroundjoin%
\definecolor{currentfill}{rgb}{0.000000,0.000000,0.000000}%
\pgfsetfillcolor{currentfill}%
\pgfsetlinewidth{0.803000pt}%
\definecolor{currentstroke}{rgb}{0.000000,0.000000,0.000000}%
\pgfsetstrokecolor{currentstroke}%
\pgfsetdash{}{0pt}%
\pgfsys@defobject{currentmarker}{\pgfqpoint{-0.048611in}{0.000000in}}{\pgfqpoint{-0.000000in}{0.000000in}}{%
\pgfpathmoveto{\pgfqpoint{-0.000000in}{0.000000in}}%
\pgfpathlineto{\pgfqpoint{-0.048611in}{0.000000in}}%
\pgfusepath{stroke,fill}%
}%
\begin{pgfscope}%
\pgfsys@transformshift{0.526127in}{2.041966in}%
\pgfsys@useobject{currentmarker}{}%
\end{pgfscope}%
\end{pgfscope}%
\begin{pgfscope}%
\definecolor{textcolor}{rgb}{0.000000,0.000000,0.000000}%
\pgfsetstrokecolor{textcolor}%
\pgfsetfillcolor{textcolor}%
\pgftext[x=0.208025in, y=1.989205in, left, base]{\color{textcolor}\rmfamily\fontsize{10.000000}{12.000000}\selectfont 0.6}%
\end{pgfscope}%
\begin{pgfscope}%
\pgfsetbuttcap%
\pgfsetroundjoin%
\definecolor{currentfill}{rgb}{0.000000,0.000000,0.000000}%
\pgfsetfillcolor{currentfill}%
\pgfsetlinewidth{0.803000pt}%
\definecolor{currentstroke}{rgb}{0.000000,0.000000,0.000000}%
\pgfsetstrokecolor{currentstroke}%
\pgfsetdash{}{0pt}%
\pgfsys@defobject{currentmarker}{\pgfqpoint{-0.048611in}{0.000000in}}{\pgfqpoint{-0.000000in}{0.000000in}}{%
\pgfpathmoveto{\pgfqpoint{-0.000000in}{0.000000in}}%
\pgfpathlineto{\pgfqpoint{-0.048611in}{0.000000in}}%
\pgfusepath{stroke,fill}%
}%
\begin{pgfscope}%
\pgfsys@transformshift{0.526127in}{2.387537in}%
\pgfsys@useobject{currentmarker}{}%
\end{pgfscope}%
\end{pgfscope}%
\begin{pgfscope}%
\definecolor{textcolor}{rgb}{0.000000,0.000000,0.000000}%
\pgfsetstrokecolor{textcolor}%
\pgfsetfillcolor{textcolor}%
\pgftext[x=0.208025in, y=2.334775in, left, base]{\color{textcolor}\rmfamily\fontsize{10.000000}{12.000000}\selectfont 0.8}%
\end{pgfscope}%
\begin{pgfscope}%
\pgfsetbuttcap%
\pgfsetroundjoin%
\definecolor{currentfill}{rgb}{0.000000,0.000000,0.000000}%
\pgfsetfillcolor{currentfill}%
\pgfsetlinewidth{0.803000pt}%
\definecolor{currentstroke}{rgb}{0.000000,0.000000,0.000000}%
\pgfsetstrokecolor{currentstroke}%
\pgfsetdash{}{0pt}%
\pgfsys@defobject{currentmarker}{\pgfqpoint{-0.048611in}{0.000000in}}{\pgfqpoint{-0.000000in}{0.000000in}}{%
\pgfpathmoveto{\pgfqpoint{-0.000000in}{0.000000in}}%
\pgfpathlineto{\pgfqpoint{-0.048611in}{0.000000in}}%
\pgfusepath{stroke,fill}%
}%
\begin{pgfscope}%
\pgfsys@transformshift{0.526127in}{2.733107in}%
\pgfsys@useobject{currentmarker}{}%
\end{pgfscope}%
\end{pgfscope}%
\begin{pgfscope}%
\definecolor{textcolor}{rgb}{0.000000,0.000000,0.000000}%
\pgfsetstrokecolor{textcolor}%
\pgfsetfillcolor{textcolor}%
\pgftext[x=0.208025in, y=2.680346in, left, base]{\color{textcolor}\rmfamily\fontsize{10.000000}{12.000000}\selectfont 1.0}%
\end{pgfscope}%
\begin{pgfscope}%
\pgfpathrectangle{\pgfqpoint{0.526127in}{0.521603in}}{\pgfqpoint{4.650000in}{2.310000in}}%
\pgfusepath{clip}%
\pgfsetrectcap%
\pgfsetroundjoin%
\pgfsetlinewidth{0.501875pt}%
\definecolor{currentstroke}{rgb}{0.000000,0.000000,0.000000}%
\pgfsetstrokecolor{currentstroke}%
\pgfsetdash{}{0pt}%
\pgfpathmoveto{\pgfqpoint{0.737490in}{0.871626in}}%
\pgfpathlineto{\pgfqpoint{0.864946in}{0.899483in}}%
\pgfpathlineto{\pgfqpoint{1.013644in}{0.934177in}}%
\pgfpathlineto{\pgfqpoint{1.226070in}{0.986334in}}%
\pgfpathlineto{\pgfqpoint{1.608436in}{1.080772in}}%
\pgfpathlineto{\pgfqpoint{1.757134in}{1.115208in}}%
\pgfpathlineto{\pgfqpoint{1.884589in}{1.142768in}}%
\pgfpathlineto{\pgfqpoint{2.012045in}{1.168036in}}%
\pgfpathlineto{\pgfqpoint{2.118258in}{1.187035in}}%
\pgfpathlineto{\pgfqpoint{2.224471in}{1.203929in}}%
\pgfpathlineto{\pgfqpoint{2.330684in}{1.218521in}}%
\pgfpathlineto{\pgfqpoint{2.436896in}{1.230642in}}%
\pgfpathlineto{\pgfqpoint{2.543109in}{1.240153in}}%
\pgfpathlineto{\pgfqpoint{2.649322in}{1.246943in}}%
\pgfpathlineto{\pgfqpoint{2.734292in}{1.250361in}}%
\pgfpathlineto{\pgfqpoint{2.819263in}{1.251962in}}%
\pgfpathlineto{\pgfqpoint{2.904233in}{1.251733in}}%
\pgfpathlineto{\pgfqpoint{2.989203in}{1.249676in}}%
\pgfpathlineto{\pgfqpoint{3.074174in}{1.245807in}}%
\pgfpathlineto{\pgfqpoint{3.180387in}{1.238465in}}%
\pgfpathlineto{\pgfqpoint{3.286599in}{1.228423in}}%
\pgfpathlineto{\pgfqpoint{3.392812in}{1.215795in}}%
\pgfpathlineto{\pgfqpoint{3.499025in}{1.200728in}}%
\pgfpathlineto{\pgfqpoint{3.605238in}{1.183397in}}%
\pgfpathlineto{\pgfqpoint{3.711451in}{1.164002in}}%
\pgfpathlineto{\pgfqpoint{3.838906in}{1.138322in}}%
\pgfpathlineto{\pgfqpoint{3.966362in}{1.110424in}}%
\pgfpathlineto{\pgfqpoint{4.115060in}{1.075693in}}%
\pgfpathlineto{\pgfqpoint{4.327486in}{1.023510in}}%
\pgfpathlineto{\pgfqpoint{4.709852in}{0.929103in}}%
\pgfpathlineto{\pgfqpoint{4.858550in}{0.894705in}}%
\pgfpathlineto{\pgfqpoint{4.964763in}{0.871626in}}%
\pgfpathlineto{\pgfqpoint{4.964763in}{0.871626in}}%
\pgfusepath{stroke}%
\end{pgfscope}%
\begin{pgfscope}%
\pgfpathrectangle{\pgfqpoint{0.526127in}{0.521603in}}{\pgfqpoint{4.650000in}{2.310000in}}%
\pgfusepath{clip}%
\pgfsetrectcap%
\pgfsetroundjoin%
\pgfsetlinewidth{0.501875pt}%
\definecolor{currentstroke}{rgb}{0.000000,0.000000,0.000000}%
\pgfsetstrokecolor{currentstroke}%
\pgfsetdash{}{0pt}%
\pgfpathmoveto{\pgfqpoint{0.737490in}{1.249487in}}%
\pgfpathlineto{\pgfqpoint{0.779975in}{1.252058in}}%
\pgfpathlineto{\pgfqpoint{0.822461in}{1.250515in}}%
\pgfpathlineto{\pgfqpoint{0.864946in}{1.244882in}}%
\pgfpathlineto{\pgfqpoint{0.907431in}{1.235254in}}%
\pgfpathlineto{\pgfqpoint{0.949916in}{1.221791in}}%
\pgfpathlineto{\pgfqpoint{0.992401in}{1.204717in}}%
\pgfpathlineto{\pgfqpoint{1.034886in}{1.184317in}}%
\pgfpathlineto{\pgfqpoint{1.077372in}{1.160932in}}%
\pgfpathlineto{\pgfqpoint{1.119857in}{1.134951in}}%
\pgfpathlineto{\pgfqpoint{1.162342in}{1.106807in}}%
\pgfpathlineto{\pgfqpoint{1.226070in}{1.061571in}}%
\pgfpathlineto{\pgfqpoint{1.438495in}{0.905528in}}%
\pgfpathlineto{\pgfqpoint{1.480980in}{0.877264in}}%
\pgfpathlineto{\pgfqpoint{1.523466in}{0.851134in}}%
\pgfpathlineto{\pgfqpoint{1.565951in}{0.827574in}}%
\pgfpathlineto{\pgfqpoint{1.608436in}{0.806977in}}%
\pgfpathlineto{\pgfqpoint{1.650921in}{0.789686in}}%
\pgfpathlineto{\pgfqpoint{1.693406in}{0.775989in}}%
\pgfpathlineto{\pgfqpoint{1.735891in}{0.766114in}}%
\pgfpathlineto{\pgfqpoint{1.778377in}{0.760228in}}%
\pgfpathlineto{\pgfqpoint{1.820862in}{0.758426in}}%
\pgfpathlineto{\pgfqpoint{1.863347in}{0.760741in}}%
\pgfpathlineto{\pgfqpoint{1.905832in}{0.767132in}}%
\pgfpathlineto{\pgfqpoint{1.948317in}{0.777494in}}%
\pgfpathlineto{\pgfqpoint{1.990802in}{0.791654in}}%
\pgfpathlineto{\pgfqpoint{2.033287in}{0.809375in}}%
\pgfpathlineto{\pgfqpoint{2.075773in}{0.830362in}}%
\pgfpathlineto{\pgfqpoint{2.118258in}{0.854265in}}%
\pgfpathlineto{\pgfqpoint{2.160743in}{0.880686in}}%
\pgfpathlineto{\pgfqpoint{2.224471in}{0.924065in}}%
\pgfpathlineto{\pgfqpoint{2.288198in}{0.970484in}}%
\pgfpathlineto{\pgfqpoint{2.436896in}{1.080772in}}%
\pgfpathlineto{\pgfqpoint{2.500624in}{1.124622in}}%
\pgfpathlineto{\pgfqpoint{2.543109in}{1.151467in}}%
\pgfpathlineto{\pgfqpoint{2.585594in}{1.175875in}}%
\pgfpathlineto{\pgfqpoint{2.628080in}{1.197437in}}%
\pgfpathlineto{\pgfqpoint{2.670565in}{1.215795in}}%
\pgfpathlineto{\pgfqpoint{2.713050in}{1.230642in}}%
\pgfpathlineto{\pgfqpoint{2.755535in}{1.241731in}}%
\pgfpathlineto{\pgfqpoint{2.798020in}{1.248878in}}%
\pgfpathlineto{\pgfqpoint{2.840505in}{1.251962in}}%
\pgfpathlineto{\pgfqpoint{2.882991in}{1.250932in}}%
\pgfpathlineto{\pgfqpoint{2.925476in}{1.245807in}}%
\pgfpathlineto{\pgfqpoint{2.967961in}{1.236670in}}%
\pgfpathlineto{\pgfqpoint{3.010446in}{1.223674in}}%
\pgfpathlineto{\pgfqpoint{3.052931in}{1.207037in}}%
\pgfpathlineto{\pgfqpoint{3.095416in}{1.187035in}}%
\pgfpathlineto{\pgfqpoint{3.137901in}{1.164002in}}%
\pgfpathlineto{\pgfqpoint{3.180387in}{1.138322in}}%
\pgfpathlineto{\pgfqpoint{3.222872in}{1.110424in}}%
\pgfpathlineto{\pgfqpoint{3.286599in}{1.065441in}}%
\pgfpathlineto{\pgfqpoint{3.392812in}{0.986334in}}%
\pgfpathlineto{\pgfqpoint{3.477783in}{0.924065in}}%
\pgfpathlineto{\pgfqpoint{3.541510in}{0.880686in}}%
\pgfpathlineto{\pgfqpoint{3.583996in}{0.854265in}}%
\pgfpathlineto{\pgfqpoint{3.626481in}{0.830362in}}%
\pgfpathlineto{\pgfqpoint{3.668966in}{0.809375in}}%
\pgfpathlineto{\pgfqpoint{3.711451in}{0.791654in}}%
\pgfpathlineto{\pgfqpoint{3.753936in}{0.777494in}}%
\pgfpathlineto{\pgfqpoint{3.796421in}{0.767132in}}%
\pgfpathlineto{\pgfqpoint{3.838906in}{0.760741in}}%
\pgfpathlineto{\pgfqpoint{3.881392in}{0.758426in}}%
\pgfpathlineto{\pgfqpoint{3.923877in}{0.760228in}}%
\pgfpathlineto{\pgfqpoint{3.966362in}{0.766114in}}%
\pgfpathlineto{\pgfqpoint{4.008847in}{0.775989in}}%
\pgfpathlineto{\pgfqpoint{4.051332in}{0.789686in}}%
\pgfpathlineto{\pgfqpoint{4.093817in}{0.806977in}}%
\pgfpathlineto{\pgfqpoint{4.136303in}{0.827574in}}%
\pgfpathlineto{\pgfqpoint{4.178788in}{0.851134in}}%
\pgfpathlineto{\pgfqpoint{4.221273in}{0.877264in}}%
\pgfpathlineto{\pgfqpoint{4.263758in}{0.905528in}}%
\pgfpathlineto{\pgfqpoint{4.327486in}{0.950887in}}%
\pgfpathlineto{\pgfqpoint{4.539911in}{1.106807in}}%
\pgfpathlineto{\pgfqpoint{4.582397in}{1.134951in}}%
\pgfpathlineto{\pgfqpoint{4.624882in}{1.160932in}}%
\pgfpathlineto{\pgfqpoint{4.667367in}{1.184317in}}%
\pgfpathlineto{\pgfqpoint{4.709852in}{1.204717in}}%
\pgfpathlineto{\pgfqpoint{4.752337in}{1.221791in}}%
\pgfpathlineto{\pgfqpoint{4.794822in}{1.235254in}}%
\pgfpathlineto{\pgfqpoint{4.837308in}{1.244882in}}%
\pgfpathlineto{\pgfqpoint{4.879793in}{1.250515in}}%
\pgfpathlineto{\pgfqpoint{4.922278in}{1.252058in}}%
\pgfpathlineto{\pgfqpoint{4.964763in}{1.249487in}}%
\pgfpathlineto{\pgfqpoint{4.964763in}{1.249487in}}%
\pgfusepath{stroke}%
\end{pgfscope}%
\begin{pgfscope}%
\pgfpathrectangle{\pgfqpoint{0.526127in}{0.521603in}}{\pgfqpoint{4.650000in}{2.310000in}}%
\pgfusepath{clip}%
\pgfsetrectcap%
\pgfsetroundjoin%
\pgfsetlinewidth{0.501875pt}%
\definecolor{currentstroke}{rgb}{0.000000,0.000000,0.000000}%
\pgfsetstrokecolor{currentstroke}%
\pgfsetdash{}{0pt}%
\pgfpathmoveto{\pgfqpoint{0.737490in}{0.936734in}}%
\pgfpathlineto{\pgfqpoint{0.779975in}{0.887641in}}%
\pgfpathlineto{\pgfqpoint{0.801218in}{0.864999in}}%
\pgfpathlineto{\pgfqpoint{0.822461in}{0.843982in}}%
\pgfpathlineto{\pgfqpoint{0.843703in}{0.824833in}}%
\pgfpathlineto{\pgfqpoint{0.864946in}{0.807774in}}%
\pgfpathlineto{\pgfqpoint{0.886188in}{0.793002in}}%
\pgfpathlineto{\pgfqpoint{0.907431in}{0.780690in}}%
\pgfpathlineto{\pgfqpoint{0.928673in}{0.770978in}}%
\pgfpathlineto{\pgfqpoint{0.949916in}{0.763981in}}%
\pgfpathlineto{\pgfqpoint{0.971159in}{0.759778in}}%
\pgfpathlineto{\pgfqpoint{0.992401in}{0.758419in}}%
\pgfpathlineto{\pgfqpoint{1.013644in}{0.759920in}}%
\pgfpathlineto{\pgfqpoint{1.034886in}{0.764262in}}%
\pgfpathlineto{\pgfqpoint{1.056129in}{0.771396in}}%
\pgfpathlineto{\pgfqpoint{1.077372in}{0.781239in}}%
\pgfpathlineto{\pgfqpoint{1.098614in}{0.793677in}}%
\pgfpathlineto{\pgfqpoint{1.119857in}{0.808566in}}%
\pgfpathlineto{\pgfqpoint{1.141099in}{0.825734in}}%
\pgfpathlineto{\pgfqpoint{1.162342in}{0.844981in}}%
\pgfpathlineto{\pgfqpoint{1.183584in}{0.866085in}}%
\pgfpathlineto{\pgfqpoint{1.226070in}{0.912865in}}%
\pgfpathlineto{\pgfqpoint{1.268555in}{0.963915in}}%
\pgfpathlineto{\pgfqpoint{1.374768in}{1.094545in}}%
\pgfpathlineto{\pgfqpoint{1.417253in}{1.141659in}}%
\pgfpathlineto{\pgfqpoint{1.438495in}{1.162979in}}%
\pgfpathlineto{\pgfqpoint{1.459738in}{1.182472in}}%
\pgfpathlineto{\pgfqpoint{1.480980in}{1.199911in}}%
\pgfpathlineto{\pgfqpoint{1.502223in}{1.215096in}}%
\pgfpathlineto{\pgfqpoint{1.523466in}{1.227850in}}%
\pgfpathlineto{\pgfqpoint{1.544708in}{1.238025in}}%
\pgfpathlineto{\pgfqpoint{1.565951in}{1.245504in}}%
\pgfpathlineto{\pgfqpoint{1.587193in}{1.250200in}}%
\pgfpathlineto{\pgfqpoint{1.608436in}{1.252058in}}%
\pgfpathlineto{\pgfqpoint{1.629679in}{1.251058in}}%
\pgfpathlineto{\pgfqpoint{1.650921in}{1.247210in}}%
\pgfpathlineto{\pgfqpoint{1.672164in}{1.240559in}}%
\pgfpathlineto{\pgfqpoint{1.693406in}{1.231182in}}%
\pgfpathlineto{\pgfqpoint{1.714649in}{1.219189in}}%
\pgfpathlineto{\pgfqpoint{1.735891in}{1.204717in}}%
\pgfpathlineto{\pgfqpoint{1.757134in}{1.187934in}}%
\pgfpathlineto{\pgfqpoint{1.778377in}{1.169035in}}%
\pgfpathlineto{\pgfqpoint{1.799619in}{1.148239in}}%
\pgfpathlineto{\pgfqpoint{1.842104in}{1.101939in}}%
\pgfpathlineto{\pgfqpoint{1.884589in}{1.051171in}}%
\pgfpathlineto{\pgfqpoint{1.990802in}{0.920314in}}%
\pgfpathlineto{\pgfqpoint{2.033287in}{0.872749in}}%
\pgfpathlineto{\pgfqpoint{2.054530in}{0.851134in}}%
\pgfpathlineto{\pgfqpoint{2.075773in}{0.831305in}}%
\pgfpathlineto{\pgfqpoint{2.097015in}{0.813491in}}%
\pgfpathlineto{\pgfqpoint{2.118258in}{0.797899in}}%
\pgfpathlineto{\pgfqpoint{2.139500in}{0.784708in}}%
\pgfpathlineto{\pgfqpoint{2.160743in}{0.774072in}}%
\pgfpathlineto{\pgfqpoint{2.181986in}{0.766114in}}%
\pgfpathlineto{\pgfqpoint{2.203228in}{0.760927in}}%
\pgfpathlineto{\pgfqpoint{2.224471in}{0.758570in}}%
\pgfpathlineto{\pgfqpoint{2.245713in}{0.759070in}}%
\pgfpathlineto{\pgfqpoint{2.266956in}{0.762422in}}%
\pgfpathlineto{\pgfqpoint{2.288198in}{0.768587in}}%
\pgfpathlineto{\pgfqpoint{2.309441in}{0.777494in}}%
\pgfpathlineto{\pgfqpoint{2.330684in}{0.789039in}}%
\pgfpathlineto{\pgfqpoint{2.351926in}{0.803089in}}%
\pgfpathlineto{\pgfqpoint{2.373169in}{0.819481in}}%
\pgfpathlineto{\pgfqpoint{2.394411in}{0.838024in}}%
\pgfpathlineto{\pgfqpoint{2.415654in}{0.858505in}}%
\pgfpathlineto{\pgfqpoint{2.458139in}{0.904310in}}%
\pgfpathlineto{\pgfqpoint{2.500624in}{0.954779in}}%
\pgfpathlineto{\pgfqpoint{2.606837in}{1.085815in}}%
\pgfpathlineto{\pgfqpoint{2.649322in}{1.133814in}}%
\pgfpathlineto{\pgfqpoint{2.670565in}{1.155716in}}%
\pgfpathlineto{\pgfqpoint{2.691807in}{1.175875in}}%
\pgfpathlineto{\pgfqpoint{2.713050in}{1.194057in}}%
\pgfpathlineto{\pgfqpoint{2.734292in}{1.210052in}}%
\pgfpathlineto{\pgfqpoint{2.755535in}{1.223674in}}%
\pgfpathlineto{\pgfqpoint{2.776778in}{1.234767in}}%
\pgfpathlineto{\pgfqpoint{2.798020in}{1.243200in}}%
\pgfpathlineto{\pgfqpoint{2.819263in}{1.248878in}}%
\pgfpathlineto{\pgfqpoint{2.840505in}{1.251733in}}%
\pgfpathlineto{\pgfqpoint{2.861748in}{1.251733in}}%
\pgfpathlineto{\pgfqpoint{2.882991in}{1.248878in}}%
\pgfpathlineto{\pgfqpoint{2.904233in}{1.243200in}}%
\pgfpathlineto{\pgfqpoint{2.925476in}{1.234767in}}%
\pgfpathlineto{\pgfqpoint{2.946718in}{1.223674in}}%
\pgfpathlineto{\pgfqpoint{2.967961in}{1.210052in}}%
\pgfpathlineto{\pgfqpoint{2.989203in}{1.194057in}}%
\pgfpathlineto{\pgfqpoint{3.010446in}{1.175875in}}%
\pgfpathlineto{\pgfqpoint{3.031689in}{1.155716in}}%
\pgfpathlineto{\pgfqpoint{3.074174in}{1.110424in}}%
\pgfpathlineto{\pgfqpoint{3.116659in}{1.060272in}}%
\pgfpathlineto{\pgfqpoint{3.244114in}{0.904310in}}%
\pgfpathlineto{\pgfqpoint{3.286599in}{0.858505in}}%
\pgfpathlineto{\pgfqpoint{3.307842in}{0.838024in}}%
\pgfpathlineto{\pgfqpoint{3.329085in}{0.819481in}}%
\pgfpathlineto{\pgfqpoint{3.350327in}{0.803089in}}%
\pgfpathlineto{\pgfqpoint{3.371570in}{0.789039in}}%
\pgfpathlineto{\pgfqpoint{3.392812in}{0.777494in}}%
\pgfpathlineto{\pgfqpoint{3.414055in}{0.768587in}}%
\pgfpathlineto{\pgfqpoint{3.435298in}{0.762422in}}%
\pgfpathlineto{\pgfqpoint{3.456540in}{0.759070in}}%
\pgfpathlineto{\pgfqpoint{3.477783in}{0.758570in}}%
\pgfpathlineto{\pgfqpoint{3.499025in}{0.760927in}}%
\pgfpathlineto{\pgfqpoint{3.520268in}{0.766114in}}%
\pgfpathlineto{\pgfqpoint{3.541510in}{0.774072in}}%
\pgfpathlineto{\pgfqpoint{3.562753in}{0.784708in}}%
\pgfpathlineto{\pgfqpoint{3.583996in}{0.797899in}}%
\pgfpathlineto{\pgfqpoint{3.605238in}{0.813491in}}%
\pgfpathlineto{\pgfqpoint{3.626481in}{0.831305in}}%
\pgfpathlineto{\pgfqpoint{3.647723in}{0.851134in}}%
\pgfpathlineto{\pgfqpoint{3.690208in}{0.895898in}}%
\pgfpathlineto{\pgfqpoint{3.732694in}{0.945714in}}%
\pgfpathlineto{\pgfqpoint{3.860149in}{1.101939in}}%
\pgfpathlineto{\pgfqpoint{3.902634in}{1.148239in}}%
\pgfpathlineto{\pgfqpoint{3.923877in}{1.169035in}}%
\pgfpathlineto{\pgfqpoint{3.945119in}{1.187934in}}%
\pgfpathlineto{\pgfqpoint{3.966362in}{1.204717in}}%
\pgfpathlineto{\pgfqpoint{3.987605in}{1.219189in}}%
\pgfpathlineto{\pgfqpoint{4.008847in}{1.231182in}}%
\pgfpathlineto{\pgfqpoint{4.030090in}{1.240559in}}%
\pgfpathlineto{\pgfqpoint{4.051332in}{1.247210in}}%
\pgfpathlineto{\pgfqpoint{4.072575in}{1.251058in}}%
\pgfpathlineto{\pgfqpoint{4.093817in}{1.252058in}}%
\pgfpathlineto{\pgfqpoint{4.115060in}{1.250200in}}%
\pgfpathlineto{\pgfqpoint{4.136303in}{1.245504in}}%
\pgfpathlineto{\pgfqpoint{4.157545in}{1.238025in}}%
\pgfpathlineto{\pgfqpoint{4.178788in}{1.227850in}}%
\pgfpathlineto{\pgfqpoint{4.200030in}{1.215096in}}%
\pgfpathlineto{\pgfqpoint{4.221273in}{1.199911in}}%
\pgfpathlineto{\pgfqpoint{4.242515in}{1.182472in}}%
\pgfpathlineto{\pgfqpoint{4.263758in}{1.162979in}}%
\pgfpathlineto{\pgfqpoint{4.285001in}{1.141659in}}%
\pgfpathlineto{\pgfqpoint{4.327486in}{1.094545in}}%
\pgfpathlineto{\pgfqpoint{4.369971in}{1.043305in}}%
\pgfpathlineto{\pgfqpoint{4.476184in}{0.912865in}}%
\pgfpathlineto{\pgfqpoint{4.518669in}{0.866085in}}%
\pgfpathlineto{\pgfqpoint{4.539911in}{0.844981in}}%
\pgfpathlineto{\pgfqpoint{4.561154in}{0.825734in}}%
\pgfpathlineto{\pgfqpoint{4.582397in}{0.808566in}}%
\pgfpathlineto{\pgfqpoint{4.603639in}{0.793677in}}%
\pgfpathlineto{\pgfqpoint{4.624882in}{0.781239in}}%
\pgfpathlineto{\pgfqpoint{4.646124in}{0.771396in}}%
\pgfpathlineto{\pgfqpoint{4.667367in}{0.764262in}}%
\pgfpathlineto{\pgfqpoint{4.688610in}{0.759920in}}%
\pgfpathlineto{\pgfqpoint{4.709852in}{0.758419in}}%
\pgfpathlineto{\pgfqpoint{4.731095in}{0.759778in}}%
\pgfpathlineto{\pgfqpoint{4.752337in}{0.763981in}}%
\pgfpathlineto{\pgfqpoint{4.773580in}{0.770978in}}%
\pgfpathlineto{\pgfqpoint{4.794822in}{0.780690in}}%
\pgfpathlineto{\pgfqpoint{4.816065in}{0.793002in}}%
\pgfpathlineto{\pgfqpoint{4.837308in}{0.807774in}}%
\pgfpathlineto{\pgfqpoint{4.858550in}{0.824833in}}%
\pgfpathlineto{\pgfqpoint{4.879793in}{0.843982in}}%
\pgfpathlineto{\pgfqpoint{4.901035in}{0.864999in}}%
\pgfpathlineto{\pgfqpoint{4.943520in}{0.911646in}}%
\pgfpathlineto{\pgfqpoint{4.964763in}{0.936734in}}%
\pgfpathlineto{\pgfqpoint{4.964763in}{0.936734in}}%
\pgfusepath{stroke}%
\end{pgfscope}%
\begin{pgfscope}%
\pgfpathrectangle{\pgfqpoint{0.526127in}{0.521603in}}{\pgfqpoint{4.650000in}{2.310000in}}%
\pgfusepath{clip}%
\pgfsetrectcap%
\pgfsetroundjoin%
\pgfsetlinewidth{0.501875pt}%
\definecolor{currentstroke}{rgb}{0.000000,0.000000,0.000000}%
\pgfsetstrokecolor{currentstroke}%
\pgfsetdash{}{0pt}%
\pgfpathmoveto{\pgfqpoint{0.737490in}{0.817736in}}%
\pgfpathlineto{\pgfqpoint{0.758733in}{0.843970in}}%
\pgfpathlineto{\pgfqpoint{0.779975in}{0.873862in}}%
\pgfpathlineto{\pgfqpoint{0.801218in}{0.906734in}}%
\pgfpathlineto{\pgfqpoint{0.843703in}{0.978387in}}%
\pgfpathlineto{\pgfqpoint{0.907431in}{1.088315in}}%
\pgfpathlineto{\pgfqpoint{0.928673in}{1.122282in}}%
\pgfpathlineto{\pgfqpoint{0.949916in}{1.153594in}}%
\pgfpathlineto{\pgfqpoint{0.971159in}{1.181541in}}%
\pgfpathlineto{\pgfqpoint{0.992401in}{1.205490in}}%
\pgfpathlineto{\pgfqpoint{1.013644in}{1.224896in}}%
\pgfpathlineto{\pgfqpoint{1.034886in}{1.239320in}}%
\pgfpathlineto{\pgfqpoint{1.056129in}{1.248435in}}%
\pgfpathlineto{\pgfqpoint{1.077372in}{1.252033in}}%
\pgfpathlineto{\pgfqpoint{1.098614in}{1.250034in}}%
\pgfpathlineto{\pgfqpoint{1.119857in}{1.242482in}}%
\pgfpathlineto{\pgfqpoint{1.141099in}{1.229549in}}%
\pgfpathlineto{\pgfqpoint{1.162342in}{1.211528in}}%
\pgfpathlineto{\pgfqpoint{1.183584in}{1.188828in}}%
\pgfpathlineto{\pgfqpoint{1.204827in}{1.161964in}}%
\pgfpathlineto{\pgfqpoint{1.226070in}{1.131545in}}%
\pgfpathlineto{\pgfqpoint{1.247312in}{1.098261in}}%
\pgfpathlineto{\pgfqpoint{1.289797in}{1.026168in}}%
\pgfpathlineto{\pgfqpoint{1.353525in}{0.916586in}}%
\pgfpathlineto{\pgfqpoint{1.374768in}{0.882995in}}%
\pgfpathlineto{\pgfqpoint{1.396010in}{0.852178in}}%
\pgfpathlineto{\pgfqpoint{1.417253in}{0.824833in}}%
\pgfpathlineto{\pgfqpoint{1.438495in}{0.801580in}}%
\pgfpathlineto{\pgfqpoint{1.459738in}{0.782948in}}%
\pgfpathlineto{\pgfqpoint{1.480980in}{0.769359in}}%
\pgfpathlineto{\pgfqpoint{1.502223in}{0.761120in}}%
\pgfpathlineto{\pgfqpoint{1.523466in}{0.758419in}}%
\pgfpathlineto{\pgfqpoint{1.544708in}{0.761318in}}%
\pgfpathlineto{\pgfqpoint{1.565951in}{0.769750in}}%
\pgfpathlineto{\pgfqpoint{1.587193in}{0.783524in}}%
\pgfpathlineto{\pgfqpoint{1.608436in}{0.802327in}}%
\pgfpathlineto{\pgfqpoint{1.629679in}{0.825734in}}%
\pgfpathlineto{\pgfqpoint{1.650921in}{0.853213in}}%
\pgfpathlineto{\pgfqpoint{1.672164in}{0.884141in}}%
\pgfpathlineto{\pgfqpoint{1.714649in}{0.953474in}}%
\pgfpathlineto{\pgfqpoint{1.799619in}{1.099480in}}%
\pgfpathlineto{\pgfqpoint{1.820862in}{1.132675in}}%
\pgfpathlineto{\pgfqpoint{1.842104in}{1.162979in}}%
\pgfpathlineto{\pgfqpoint{1.863347in}{1.189706in}}%
\pgfpathlineto{\pgfqpoint{1.884589in}{1.212248in}}%
\pgfpathlineto{\pgfqpoint{1.905832in}{1.230095in}}%
\pgfpathlineto{\pgfqpoint{1.927075in}{1.242842in}}%
\pgfpathlineto{\pgfqpoint{1.948317in}{1.250200in}}%
\pgfpathlineto{\pgfqpoint{1.969560in}{1.252001in}}%
\pgfpathlineto{\pgfqpoint{1.990802in}{1.248206in}}%
\pgfpathlineto{\pgfqpoint{2.012045in}{1.238899in}}%
\pgfpathlineto{\pgfqpoint{2.033287in}{1.224292in}}%
\pgfpathlineto{\pgfqpoint{2.054530in}{1.204717in}}%
\pgfpathlineto{\pgfqpoint{2.075773in}{1.180617in}}%
\pgfpathlineto{\pgfqpoint{2.097015in}{1.152539in}}%
\pgfpathlineto{\pgfqpoint{2.118258in}{1.121120in}}%
\pgfpathlineto{\pgfqpoint{2.160743in}{1.051171in}}%
\pgfpathlineto{\pgfqpoint{2.245713in}{0.905528in}}%
\pgfpathlineto{\pgfqpoint{2.266956in}{0.872749in}}%
\pgfpathlineto{\pgfqpoint{2.288198in}{0.842975in}}%
\pgfpathlineto{\pgfqpoint{2.309441in}{0.816882in}}%
\pgfpathlineto{\pgfqpoint{2.330684in}{0.795063in}}%
\pgfpathlineto{\pgfqpoint{2.351926in}{0.778011in}}%
\pgfpathlineto{\pgfqpoint{2.373169in}{0.766114in}}%
\pgfpathlineto{\pgfqpoint{2.394411in}{0.759642in}}%
\pgfpathlineto{\pgfqpoint{2.415654in}{0.758741in}}%
\pgfpathlineto{\pgfqpoint{2.436896in}{0.763432in}}%
\pgfpathlineto{\pgfqpoint{2.458139in}{0.773608in}}%
\pgfpathlineto{\pgfqpoint{2.479382in}{0.789039in}}%
\pgfpathlineto{\pgfqpoint{2.500624in}{0.809375in}}%
\pgfpathlineto{\pgfqpoint{2.521867in}{0.834154in}}%
\pgfpathlineto{\pgfqpoint{2.543109in}{0.862813in}}%
\pgfpathlineto{\pgfqpoint{2.564352in}{0.894705in}}%
\pgfpathlineto{\pgfqpoint{2.606837in}{0.965229in}}%
\pgfpathlineto{\pgfqpoint{2.670565in}{1.075693in}}%
\pgfpathlineto{\pgfqpoint{2.691807in}{1.110424in}}%
\pgfpathlineto{\pgfqpoint{2.713050in}{1.142768in}}%
\pgfpathlineto{\pgfqpoint{2.734292in}{1.171994in}}%
\pgfpathlineto{\pgfqpoint{2.755535in}{1.197437in}}%
\pgfpathlineto{\pgfqpoint{2.776778in}{1.218521in}}%
\pgfpathlineto{\pgfqpoint{2.798020in}{1.234767in}}%
\pgfpathlineto{\pgfqpoint{2.819263in}{1.245807in}}%
\pgfpathlineto{\pgfqpoint{2.840505in}{1.251390in}}%
\pgfpathlineto{\pgfqpoint{2.861748in}{1.251390in}}%
\pgfpathlineto{\pgfqpoint{2.882991in}{1.245807in}}%
\pgfpathlineto{\pgfqpoint{2.904233in}{1.234767in}}%
\pgfpathlineto{\pgfqpoint{2.925476in}{1.218521in}}%
\pgfpathlineto{\pgfqpoint{2.946718in}{1.197437in}}%
\pgfpathlineto{\pgfqpoint{2.967961in}{1.171994in}}%
\pgfpathlineto{\pgfqpoint{2.989203in}{1.142768in}}%
\pgfpathlineto{\pgfqpoint{3.010446in}{1.110424in}}%
\pgfpathlineto{\pgfqpoint{3.052931in}{1.039365in}}%
\pgfpathlineto{\pgfqpoint{3.116659in}{0.929103in}}%
\pgfpathlineto{\pgfqpoint{3.137901in}{0.894705in}}%
\pgfpathlineto{\pgfqpoint{3.159144in}{0.862813in}}%
\pgfpathlineto{\pgfqpoint{3.180387in}{0.834154in}}%
\pgfpathlineto{\pgfqpoint{3.201629in}{0.809375in}}%
\pgfpathlineto{\pgfqpoint{3.222872in}{0.789039in}}%
\pgfpathlineto{\pgfqpoint{3.244114in}{0.773608in}}%
\pgfpathlineto{\pgfqpoint{3.265357in}{0.763432in}}%
\pgfpathlineto{\pgfqpoint{3.286599in}{0.758741in}}%
\pgfpathlineto{\pgfqpoint{3.307842in}{0.759642in}}%
\pgfpathlineto{\pgfqpoint{3.329085in}{0.766114in}}%
\pgfpathlineto{\pgfqpoint{3.350327in}{0.778011in}}%
\pgfpathlineto{\pgfqpoint{3.371570in}{0.795063in}}%
\pgfpathlineto{\pgfqpoint{3.392812in}{0.816882in}}%
\pgfpathlineto{\pgfqpoint{3.414055in}{0.842975in}}%
\pgfpathlineto{\pgfqpoint{3.435298in}{0.872749in}}%
\pgfpathlineto{\pgfqpoint{3.456540in}{0.905528in}}%
\pgfpathlineto{\pgfqpoint{3.499025in}{0.977078in}}%
\pgfpathlineto{\pgfqpoint{3.562753in}{1.087074in}}%
\pgfpathlineto{\pgfqpoint{3.583996in}{1.121120in}}%
\pgfpathlineto{\pgfqpoint{3.605238in}{1.152539in}}%
\pgfpathlineto{\pgfqpoint{3.626481in}{1.180617in}}%
\pgfpathlineto{\pgfqpoint{3.647723in}{1.204717in}}%
\pgfpathlineto{\pgfqpoint{3.668966in}{1.224292in}}%
\pgfpathlineto{\pgfqpoint{3.690208in}{1.238899in}}%
\pgfpathlineto{\pgfqpoint{3.711451in}{1.248206in}}%
\pgfpathlineto{\pgfqpoint{3.732694in}{1.252001in}}%
\pgfpathlineto{\pgfqpoint{3.753936in}{1.250200in}}%
\pgfpathlineto{\pgfqpoint{3.775179in}{1.242842in}}%
\pgfpathlineto{\pgfqpoint{3.796421in}{1.230095in}}%
\pgfpathlineto{\pgfqpoint{3.817664in}{1.212248in}}%
\pgfpathlineto{\pgfqpoint{3.838906in}{1.189706in}}%
\pgfpathlineto{\pgfqpoint{3.860149in}{1.162979in}}%
\pgfpathlineto{\pgfqpoint{3.881392in}{1.132675in}}%
\pgfpathlineto{\pgfqpoint{3.902634in}{1.099480in}}%
\pgfpathlineto{\pgfqpoint{3.945119in}{1.027480in}}%
\pgfpathlineto{\pgfqpoint{4.008847in}{0.917816in}}%
\pgfpathlineto{\pgfqpoint{4.030090in}{0.884141in}}%
\pgfpathlineto{\pgfqpoint{4.051332in}{0.853213in}}%
\pgfpathlineto{\pgfqpoint{4.072575in}{0.825734in}}%
\pgfpathlineto{\pgfqpoint{4.093817in}{0.802327in}}%
\pgfpathlineto{\pgfqpoint{4.115060in}{0.783524in}}%
\pgfpathlineto{\pgfqpoint{4.136303in}{0.769750in}}%
\pgfpathlineto{\pgfqpoint{4.157545in}{0.761318in}}%
\pgfpathlineto{\pgfqpoint{4.178788in}{0.758419in}}%
\pgfpathlineto{\pgfqpoint{4.200030in}{0.761120in}}%
\pgfpathlineto{\pgfqpoint{4.221273in}{0.769359in}}%
\pgfpathlineto{\pgfqpoint{4.242515in}{0.782948in}}%
\pgfpathlineto{\pgfqpoint{4.263758in}{0.801580in}}%
\pgfpathlineto{\pgfqpoint{4.285001in}{0.824833in}}%
\pgfpathlineto{\pgfqpoint{4.306243in}{0.852178in}}%
\pgfpathlineto{\pgfqpoint{4.327486in}{0.882995in}}%
\pgfpathlineto{\pgfqpoint{4.369971in}{0.952188in}}%
\pgfpathlineto{\pgfqpoint{4.454941in}{1.098261in}}%
\pgfpathlineto{\pgfqpoint{4.476184in}{1.131545in}}%
\pgfpathlineto{\pgfqpoint{4.497426in}{1.161964in}}%
\pgfpathlineto{\pgfqpoint{4.518669in}{1.188828in}}%
\pgfpathlineto{\pgfqpoint{4.539911in}{1.211528in}}%
\pgfpathlineto{\pgfqpoint{4.561154in}{1.229549in}}%
\pgfpathlineto{\pgfqpoint{4.582397in}{1.242482in}}%
\pgfpathlineto{\pgfqpoint{4.603639in}{1.250034in}}%
\pgfpathlineto{\pgfqpoint{4.624882in}{1.252033in}}%
\pgfpathlineto{\pgfqpoint{4.646124in}{1.248435in}}%
\pgfpathlineto{\pgfqpoint{4.667367in}{1.239320in}}%
\pgfpathlineto{\pgfqpoint{4.688610in}{1.224896in}}%
\pgfpathlineto{\pgfqpoint{4.709852in}{1.205490in}}%
\pgfpathlineto{\pgfqpoint{4.731095in}{1.181541in}}%
\pgfpathlineto{\pgfqpoint{4.752337in}{1.153594in}}%
\pgfpathlineto{\pgfqpoint{4.773580in}{1.122282in}}%
\pgfpathlineto{\pgfqpoint{4.816065in}{1.052464in}}%
\pgfpathlineto{\pgfqpoint{4.901035in}{0.906734in}}%
\pgfpathlineto{\pgfqpoint{4.922278in}{0.873862in}}%
\pgfpathlineto{\pgfqpoint{4.943520in}{0.843970in}}%
\pgfpathlineto{\pgfqpoint{4.964763in}{0.817736in}}%
\pgfpathlineto{\pgfqpoint{4.964763in}{0.817736in}}%
\pgfusepath{stroke}%
\end{pgfscope}%
\begin{pgfscope}%
\pgfpathrectangle{\pgfqpoint{0.526127in}{0.521603in}}{\pgfqpoint{4.650000in}{2.310000in}}%
\pgfusepath{clip}%
\pgfsetrectcap%
\pgfsetroundjoin%
\pgfsetlinewidth{0.501875pt}%
\definecolor{currentstroke}{rgb}{0.000000,0.000000,0.000000}%
\pgfsetstrokecolor{currentstroke}%
\pgfsetdash{}{0pt}%
\pgfpathmoveto{\pgfqpoint{0.737490in}{1.228982in}}%
\pgfpathlineto{\pgfqpoint{0.758733in}{1.244878in}}%
\pgfpathlineto{\pgfqpoint{0.779975in}{1.251800in}}%
\pgfpathlineto{\pgfqpoint{0.801218in}{1.249489in}}%
\pgfpathlineto{\pgfqpoint{0.822461in}{1.238031in}}%
\pgfpathlineto{\pgfqpoint{0.843703in}{1.217855in}}%
\pgfpathlineto{\pgfqpoint{0.864946in}{1.189716in}}%
\pgfpathlineto{\pgfqpoint{0.886188in}{1.154670in}}%
\pgfpathlineto{\pgfqpoint{0.907431in}{1.114028in}}%
\pgfpathlineto{\pgfqpoint{0.949916in}{1.022197in}}%
\pgfpathlineto{\pgfqpoint{0.992401in}{0.927852in}}%
\pgfpathlineto{\pgfqpoint{1.013644in}{0.884155in}}%
\pgfpathlineto{\pgfqpoint{1.034886in}{0.844993in}}%
\pgfpathlineto{\pgfqpoint{1.056129in}{0.811834in}}%
\pgfpathlineto{\pgfqpoint{1.077372in}{0.785918in}}%
\pgfpathlineto{\pgfqpoint{1.098614in}{0.768217in}}%
\pgfpathlineto{\pgfqpoint{1.119857in}{0.759393in}}%
\pgfpathlineto{\pgfqpoint{1.141099in}{0.759777in}}%
\pgfpathlineto{\pgfqpoint{1.162342in}{0.769354in}}%
\pgfpathlineto{\pgfqpoint{1.183584in}{0.787766in}}%
\pgfpathlineto{\pgfqpoint{1.204827in}{0.814323in}}%
\pgfpathlineto{\pgfqpoint{1.226070in}{0.848031in}}%
\pgfpathlineto{\pgfqpoint{1.247312in}{0.887627in}}%
\pgfpathlineto{\pgfqpoint{1.268555in}{0.931628in}}%
\pgfpathlineto{\pgfqpoint{1.353525in}{1.117574in}}%
\pgfpathlineto{\pgfqpoint{1.374768in}{1.157808in}}%
\pgfpathlineto{\pgfqpoint{1.396010in}{1.192328in}}%
\pgfpathlineto{\pgfqpoint{1.417253in}{1.219842in}}%
\pgfpathlineto{\pgfqpoint{1.438495in}{1.239320in}}%
\pgfpathlineto{\pgfqpoint{1.459738in}{1.250032in}}%
\pgfpathlineto{\pgfqpoint{1.480980in}{1.251576in}}%
\pgfpathlineto{\pgfqpoint{1.502223in}{1.243896in}}%
\pgfpathlineto{\pgfqpoint{1.523466in}{1.227278in}}%
\pgfpathlineto{\pgfqpoint{1.544708in}{1.202345in}}%
\pgfpathlineto{\pgfqpoint{1.565951in}{1.170030in}}%
\pgfpathlineto{\pgfqpoint{1.587193in}{1.131545in}}%
\pgfpathlineto{\pgfqpoint{1.608436in}{1.088330in}}%
\pgfpathlineto{\pgfqpoint{1.693406in}{0.901898in}}%
\pgfpathlineto{\pgfqpoint{1.714649in}{0.860658in}}%
\pgfpathlineto{\pgfqpoint{1.735891in}{0.824833in}}%
\pgfpathlineto{\pgfqpoint{1.757134in}{0.795765in}}%
\pgfpathlineto{\pgfqpoint{1.778377in}{0.774543in}}%
\pgfpathlineto{\pgfqpoint{1.799619in}{0.761961in}}%
\pgfpathlineto{\pgfqpoint{1.820862in}{0.758491in}}%
\pgfpathlineto{\pgfqpoint{1.842104in}{0.764262in}}%
\pgfpathlineto{\pgfqpoint{1.863347in}{0.779059in}}%
\pgfpathlineto{\pgfqpoint{1.884589in}{0.802327in}}%
\pgfpathlineto{\pgfqpoint{1.905832in}{0.833195in}}%
\pgfpathlineto{\pgfqpoint{1.927075in}{0.870507in}}%
\pgfpathlineto{\pgfqpoint{1.948317in}{0.912865in}}%
\pgfpathlineto{\pgfqpoint{1.990802in}{1.006246in}}%
\pgfpathlineto{\pgfqpoint{2.033287in}{1.099480in}}%
\pgfpathlineto{\pgfqpoint{2.054530in}{1.141659in}}%
\pgfpathlineto{\pgfqpoint{2.075773in}{1.178730in}}%
\pgfpathlineto{\pgfqpoint{2.097015in}{1.209305in}}%
\pgfpathlineto{\pgfqpoint{2.118258in}{1.232237in}}%
\pgfpathlineto{\pgfqpoint{2.139500in}{1.246668in}}%
\pgfpathlineto{\pgfqpoint{2.160743in}{1.252058in}}%
\pgfpathlineto{\pgfqpoint{2.181986in}{1.248206in}}%
\pgfpathlineto{\pgfqpoint{2.203228in}{1.235254in}}%
\pgfpathlineto{\pgfqpoint{2.224471in}{1.213689in}}%
\pgfpathlineto{\pgfqpoint{2.245713in}{1.184317in}}%
\pgfpathlineto{\pgfqpoint{2.266956in}{1.148239in}}%
\pgfpathlineto{\pgfqpoint{2.288198in}{1.106807in}}%
\pgfpathlineto{\pgfqpoint{2.330684in}{1.014226in}}%
\pgfpathlineto{\pgfqpoint{2.373169in}{0.920314in}}%
\pgfpathlineto{\pgfqpoint{2.394411in}{0.877264in}}%
\pgfpathlineto{\pgfqpoint{2.415654in}{0.839007in}}%
\pgfpathlineto{\pgfqpoint{2.436896in}{0.806977in}}%
\pgfpathlineto{\pgfqpoint{2.458139in}{0.782372in}}%
\pgfpathlineto{\pgfqpoint{2.479382in}{0.766114in}}%
\pgfpathlineto{\pgfqpoint{2.500624in}{0.758813in}}%
\pgfpathlineto{\pgfqpoint{2.521867in}{0.760741in}}%
\pgfpathlineto{\pgfqpoint{2.543109in}{0.771826in}}%
\pgfpathlineto{\pgfqpoint{2.564352in}{0.791654in}}%
\pgfpathlineto{\pgfqpoint{2.585594in}{0.819481in}}%
\pgfpathlineto{\pgfqpoint{2.606837in}{0.854265in}}%
\pgfpathlineto{\pgfqpoint{2.628080in}{0.894705in}}%
\pgfpathlineto{\pgfqpoint{2.670565in}{0.986334in}}%
\pgfpathlineto{\pgfqpoint{2.713050in}{1.080772in}}%
\pgfpathlineto{\pgfqpoint{2.734292in}{1.124622in}}%
\pgfpathlineto{\pgfqpoint{2.755535in}{1.164002in}}%
\pgfpathlineto{\pgfqpoint{2.776778in}{1.197437in}}%
\pgfpathlineto{\pgfqpoint{2.798020in}{1.223674in}}%
\pgfpathlineto{\pgfqpoint{2.819263in}{1.241731in}}%
\pgfpathlineto{\pgfqpoint{2.840505in}{1.250932in}}%
\pgfpathlineto{\pgfqpoint{2.861748in}{1.250932in}}%
\pgfpathlineto{\pgfqpoint{2.882991in}{1.241731in}}%
\pgfpathlineto{\pgfqpoint{2.904233in}{1.223674in}}%
\pgfpathlineto{\pgfqpoint{2.925476in}{1.197437in}}%
\pgfpathlineto{\pgfqpoint{2.946718in}{1.164002in}}%
\pgfpathlineto{\pgfqpoint{2.967961in}{1.124622in}}%
\pgfpathlineto{\pgfqpoint{2.989203in}{1.080772in}}%
\pgfpathlineto{\pgfqpoint{3.074174in}{0.894705in}}%
\pgfpathlineto{\pgfqpoint{3.095416in}{0.854265in}}%
\pgfpathlineto{\pgfqpoint{3.116659in}{0.819481in}}%
\pgfpathlineto{\pgfqpoint{3.137901in}{0.791654in}}%
\pgfpathlineto{\pgfqpoint{3.159144in}{0.771826in}}%
\pgfpathlineto{\pgfqpoint{3.180387in}{0.760741in}}%
\pgfpathlineto{\pgfqpoint{3.201629in}{0.758813in}}%
\pgfpathlineto{\pgfqpoint{3.222872in}{0.766114in}}%
\pgfpathlineto{\pgfqpoint{3.244114in}{0.782372in}}%
\pgfpathlineto{\pgfqpoint{3.265357in}{0.806977in}}%
\pgfpathlineto{\pgfqpoint{3.286599in}{0.839007in}}%
\pgfpathlineto{\pgfqpoint{3.307842in}{0.877264in}}%
\pgfpathlineto{\pgfqpoint{3.329085in}{0.920314in}}%
\pgfpathlineto{\pgfqpoint{3.414055in}{1.106807in}}%
\pgfpathlineto{\pgfqpoint{3.435298in}{1.148239in}}%
\pgfpathlineto{\pgfqpoint{3.456540in}{1.184317in}}%
\pgfpathlineto{\pgfqpoint{3.477783in}{1.213689in}}%
\pgfpathlineto{\pgfqpoint{3.499025in}{1.235254in}}%
\pgfpathlineto{\pgfqpoint{3.520268in}{1.248206in}}%
\pgfpathlineto{\pgfqpoint{3.541510in}{1.252058in}}%
\pgfpathlineto{\pgfqpoint{3.562753in}{1.246668in}}%
\pgfpathlineto{\pgfqpoint{3.583996in}{1.232237in}}%
\pgfpathlineto{\pgfqpoint{3.605238in}{1.209305in}}%
\pgfpathlineto{\pgfqpoint{3.626481in}{1.178730in}}%
\pgfpathlineto{\pgfqpoint{3.647723in}{1.141659in}}%
\pgfpathlineto{\pgfqpoint{3.668966in}{1.099480in}}%
\pgfpathlineto{\pgfqpoint{3.711451in}{1.006246in}}%
\pgfpathlineto{\pgfqpoint{3.753936in}{0.912865in}}%
\pgfpathlineto{\pgfqpoint{3.775179in}{0.870507in}}%
\pgfpathlineto{\pgfqpoint{3.796421in}{0.833195in}}%
\pgfpathlineto{\pgfqpoint{3.817664in}{0.802327in}}%
\pgfpathlineto{\pgfqpoint{3.838906in}{0.779059in}}%
\pgfpathlineto{\pgfqpoint{3.860149in}{0.764262in}}%
\pgfpathlineto{\pgfqpoint{3.881392in}{0.758491in}}%
\pgfpathlineto{\pgfqpoint{3.902634in}{0.761961in}}%
\pgfpathlineto{\pgfqpoint{3.923877in}{0.774543in}}%
\pgfpathlineto{\pgfqpoint{3.945119in}{0.795765in}}%
\pgfpathlineto{\pgfqpoint{3.966362in}{0.824833in}}%
\pgfpathlineto{\pgfqpoint{3.987605in}{0.860658in}}%
\pgfpathlineto{\pgfqpoint{4.008847in}{0.901898in}}%
\pgfpathlineto{\pgfqpoint{4.051332in}{0.994301in}}%
\pgfpathlineto{\pgfqpoint{4.093817in}{1.088330in}}%
\pgfpathlineto{\pgfqpoint{4.115060in}{1.131545in}}%
\pgfpathlineto{\pgfqpoint{4.136303in}{1.170030in}}%
\pgfpathlineto{\pgfqpoint{4.157545in}{1.202345in}}%
\pgfpathlineto{\pgfqpoint{4.178788in}{1.227278in}}%
\pgfpathlineto{\pgfqpoint{4.200030in}{1.243896in}}%
\pgfpathlineto{\pgfqpoint{4.221273in}{1.251576in}}%
\pgfpathlineto{\pgfqpoint{4.242515in}{1.250032in}}%
\pgfpathlineto{\pgfqpoint{4.263758in}{1.239320in}}%
\pgfpathlineto{\pgfqpoint{4.285001in}{1.219842in}}%
\pgfpathlineto{\pgfqpoint{4.306243in}{1.192328in}}%
\pgfpathlineto{\pgfqpoint{4.327486in}{1.157808in}}%
\pgfpathlineto{\pgfqpoint{4.348728in}{1.117574in}}%
\pgfpathlineto{\pgfqpoint{4.391213in}{1.026152in}}%
\pgfpathlineto{\pgfqpoint{4.433699in}{0.931628in}}%
\pgfpathlineto{\pgfqpoint{4.454941in}{0.887627in}}%
\pgfpathlineto{\pgfqpoint{4.476184in}{0.848031in}}%
\pgfpathlineto{\pgfqpoint{4.497426in}{0.814323in}}%
\pgfpathlineto{\pgfqpoint{4.518669in}{0.787766in}}%
\pgfpathlineto{\pgfqpoint{4.539911in}{0.769354in}}%
\pgfpathlineto{\pgfqpoint{4.561154in}{0.759777in}}%
\pgfpathlineto{\pgfqpoint{4.582397in}{0.759393in}}%
\pgfpathlineto{\pgfqpoint{4.603639in}{0.768217in}}%
\pgfpathlineto{\pgfqpoint{4.624882in}{0.785918in}}%
\pgfpathlineto{\pgfqpoint{4.646124in}{0.811834in}}%
\pgfpathlineto{\pgfqpoint{4.667367in}{0.844993in}}%
\pgfpathlineto{\pgfqpoint{4.688610in}{0.884155in}}%
\pgfpathlineto{\pgfqpoint{4.709852in}{0.927852in}}%
\pgfpathlineto{\pgfqpoint{4.794822in}{1.114028in}}%
\pgfpathlineto{\pgfqpoint{4.816065in}{1.154670in}}%
\pgfpathlineto{\pgfqpoint{4.837308in}{1.189716in}}%
\pgfpathlineto{\pgfqpoint{4.858550in}{1.217855in}}%
\pgfpathlineto{\pgfqpoint{4.879793in}{1.238031in}}%
\pgfpathlineto{\pgfqpoint{4.901035in}{1.249489in}}%
\pgfpathlineto{\pgfqpoint{4.922278in}{1.251800in}}%
\pgfpathlineto{\pgfqpoint{4.943520in}{1.244878in}}%
\pgfpathlineto{\pgfqpoint{4.964763in}{1.228982in}}%
\pgfpathlineto{\pgfqpoint{4.964763in}{1.228982in}}%
\pgfusepath{stroke}%
\end{pgfscope}%
\begin{pgfscope}%
\pgfpathrectangle{\pgfqpoint{0.526127in}{0.521603in}}{\pgfqpoint{4.650000in}{2.310000in}}%
\pgfusepath{clip}%
\pgfsetrectcap%
\pgfsetroundjoin%
\pgfsetlinewidth{0.501875pt}%
\definecolor{currentstroke}{rgb}{0.000000,0.000000,0.000000}%
\pgfsetstrokecolor{currentstroke}%
\pgfsetdash{}{0pt}%
\pgfpathmoveto{\pgfqpoint{0.737490in}{1.007595in}}%
\pgfpathlineto{\pgfqpoint{0.779975in}{0.894719in}}%
\pgfpathlineto{\pgfqpoint{0.801218in}{0.846010in}}%
\pgfpathlineto{\pgfqpoint{0.822461in}{0.806195in}}%
\pgfpathlineto{\pgfqpoint{0.843703in}{0.777500in}}%
\pgfpathlineto{\pgfqpoint{0.864946in}{0.761528in}}%
\pgfpathlineto{\pgfqpoint{0.886188in}{0.759169in}}%
\pgfpathlineto{\pgfqpoint{0.907431in}{0.770557in}}%
\pgfpathlineto{\pgfqpoint{0.928673in}{0.795054in}}%
\pgfpathlineto{\pgfqpoint{0.949916in}{0.831294in}}%
\pgfpathlineto{\pgfqpoint{0.971159in}{0.877250in}}%
\pgfpathlineto{\pgfqpoint{0.992401in}{0.930357in}}%
\pgfpathlineto{\pgfqpoint{1.056129in}{1.101924in}}%
\pgfpathlineto{\pgfqpoint{1.077372in}{1.152526in}}%
\pgfpathlineto{\pgfqpoint{1.098614in}{1.194902in}}%
\pgfpathlineto{\pgfqpoint{1.119857in}{1.226685in}}%
\pgfpathlineto{\pgfqpoint{1.141099in}{1.246098in}}%
\pgfpathlineto{\pgfqpoint{1.162342in}{1.252059in}}%
\pgfpathlineto{\pgfqpoint{1.183584in}{1.244233in}}%
\pgfpathlineto{\pgfqpoint{1.204827in}{1.223058in}}%
\pgfpathlineto{\pgfqpoint{1.226070in}{1.189716in}}%
\pgfpathlineto{\pgfqpoint{1.247312in}{1.146071in}}%
\pgfpathlineto{\pgfqpoint{1.268555in}{1.094560in}}%
\pgfpathlineto{\pgfqpoint{1.353525in}{0.870521in}}%
\pgfpathlineto{\pgfqpoint{1.374768in}{0.825745in}}%
\pgfpathlineto{\pgfqpoint{1.396010in}{0.790997in}}%
\pgfpathlineto{\pgfqpoint{1.417253in}{0.768217in}}%
\pgfpathlineto{\pgfqpoint{1.438495in}{0.758677in}}%
\pgfpathlineto{\pgfqpoint{1.459738in}{0.762912in}}%
\pgfpathlineto{\pgfqpoint{1.480980in}{0.780683in}}%
\pgfpathlineto{\pgfqpoint{1.502223in}{0.810998in}}%
\pgfpathlineto{\pgfqpoint{1.523466in}{0.852165in}}%
\pgfpathlineto{\pgfqpoint{1.544708in}{0.901883in}}%
\pgfpathlineto{\pgfqpoint{1.587193in}{1.015542in}}%
\pgfpathlineto{\pgfqpoint{1.608436in}{1.073134in}}%
\pgfpathlineto{\pgfqpoint{1.629679in}{1.126935in}}%
\pgfpathlineto{\pgfqpoint{1.650921in}{1.173938in}}%
\pgfpathlineto{\pgfqpoint{1.672164in}{1.211519in}}%
\pgfpathlineto{\pgfqpoint{1.693406in}{1.237578in}}%
\pgfpathlineto{\pgfqpoint{1.714649in}{1.250660in}}%
\pgfpathlineto{\pgfqpoint{1.735891in}{1.250034in}}%
\pgfpathlineto{\pgfqpoint{1.757134in}{1.235734in}}%
\pgfpathlineto{\pgfqpoint{1.778377in}{1.208561in}}%
\pgfpathlineto{\pgfqpoint{1.799619in}{1.170030in}}%
\pgfpathlineto{\pgfqpoint{1.820862in}{1.122296in}}%
\pgfpathlineto{\pgfqpoint{1.842104in}{1.068024in}}%
\pgfpathlineto{\pgfqpoint{1.905832in}{0.897095in}}%
\pgfpathlineto{\pgfqpoint{1.927075in}{0.848043in}}%
\pgfpathlineto{\pgfqpoint{1.948317in}{0.807774in}}%
\pgfpathlineto{\pgfqpoint{1.969560in}{0.778535in}}%
\pgfpathlineto{\pgfqpoint{1.990802in}{0.761961in}}%
\pgfpathlineto{\pgfqpoint{2.012045in}{0.758977in}}%
\pgfpathlineto{\pgfqpoint{2.033287in}{0.769750in}}%
\pgfpathlineto{\pgfqpoint{2.054530in}{0.793677in}}%
\pgfpathlineto{\pgfqpoint{2.075773in}{0.829424in}}%
\pgfpathlineto{\pgfqpoint{2.097015in}{0.874992in}}%
\pgfpathlineto{\pgfqpoint{2.118258in}{0.927836in}}%
\pgfpathlineto{\pgfqpoint{2.181986in}{1.099480in}}%
\pgfpathlineto{\pgfqpoint{2.203228in}{1.150391in}}%
\pgfpathlineto{\pgfqpoint{2.224471in}{1.193195in}}%
\pgfpathlineto{\pgfqpoint{2.245713in}{1.225501in}}%
\pgfpathlineto{\pgfqpoint{2.266956in}{1.245504in}}%
\pgfpathlineto{\pgfqpoint{2.288198in}{1.252087in}}%
\pgfpathlineto{\pgfqpoint{2.309441in}{1.244882in}}%
\pgfpathlineto{\pgfqpoint{2.330684in}{1.224292in}}%
\pgfpathlineto{\pgfqpoint{2.351926in}{1.191467in}}%
\pgfpathlineto{\pgfqpoint{2.373169in}{1.148239in}}%
\pgfpathlineto{\pgfqpoint{2.394411in}{1.097025in}}%
\pgfpathlineto{\pgfqpoint{2.479382in}{0.872749in}}%
\pgfpathlineto{\pgfqpoint{2.500624in}{0.827574in}}%
\pgfpathlineto{\pgfqpoint{2.521867in}{0.792325in}}%
\pgfpathlineto{\pgfqpoint{2.543109in}{0.768970in}}%
\pgfpathlineto{\pgfqpoint{2.564352in}{0.758813in}}%
\pgfpathlineto{\pgfqpoint{2.585594in}{0.762422in}}%
\pgfpathlineto{\pgfqpoint{2.606837in}{0.779596in}}%
\pgfpathlineto{\pgfqpoint{2.628080in}{0.809375in}}%
\pgfpathlineto{\pgfqpoint{2.649322in}{0.850095in}}%
\pgfpathlineto{\pgfqpoint{2.670565in}{0.899483in}}%
\pgfpathlineto{\pgfqpoint{2.713050in}{1.012894in}}%
\pgfpathlineto{\pgfqpoint{2.734292in}{1.070583in}}%
\pgfpathlineto{\pgfqpoint{2.755535in}{1.124622in}}%
\pgfpathlineto{\pgfqpoint{2.776778in}{1.171994in}}%
\pgfpathlineto{\pgfqpoint{2.798020in}{1.210052in}}%
\pgfpathlineto{\pgfqpoint{2.819263in}{1.236670in}}%
\pgfpathlineto{\pgfqpoint{2.840505in}{1.250361in}}%
\pgfpathlineto{\pgfqpoint{2.861748in}{1.250361in}}%
\pgfpathlineto{\pgfqpoint{2.882991in}{1.236670in}}%
\pgfpathlineto{\pgfqpoint{2.904233in}{1.210052in}}%
\pgfpathlineto{\pgfqpoint{2.925476in}{1.171994in}}%
\pgfpathlineto{\pgfqpoint{2.946718in}{1.124622in}}%
\pgfpathlineto{\pgfqpoint{2.967961in}{1.070583in}}%
\pgfpathlineto{\pgfqpoint{3.031689in}{0.899483in}}%
\pgfpathlineto{\pgfqpoint{3.052931in}{0.850095in}}%
\pgfpathlineto{\pgfqpoint{3.074174in}{0.809375in}}%
\pgfpathlineto{\pgfqpoint{3.095416in}{0.779596in}}%
\pgfpathlineto{\pgfqpoint{3.116659in}{0.762422in}}%
\pgfpathlineto{\pgfqpoint{3.137901in}{0.758813in}}%
\pgfpathlineto{\pgfqpoint{3.159144in}{0.768970in}}%
\pgfpathlineto{\pgfqpoint{3.180387in}{0.792325in}}%
\pgfpathlineto{\pgfqpoint{3.201629in}{0.827574in}}%
\pgfpathlineto{\pgfqpoint{3.222872in}{0.872749in}}%
\pgfpathlineto{\pgfqpoint{3.244114in}{0.925325in}}%
\pgfpathlineto{\pgfqpoint{3.307842in}{1.097025in}}%
\pgfpathlineto{\pgfqpoint{3.329085in}{1.148239in}}%
\pgfpathlineto{\pgfqpoint{3.350327in}{1.191467in}}%
\pgfpathlineto{\pgfqpoint{3.371570in}{1.224292in}}%
\pgfpathlineto{\pgfqpoint{3.392812in}{1.244882in}}%
\pgfpathlineto{\pgfqpoint{3.414055in}{1.252087in}}%
\pgfpathlineto{\pgfqpoint{3.435298in}{1.245504in}}%
\pgfpathlineto{\pgfqpoint{3.456540in}{1.225501in}}%
\pgfpathlineto{\pgfqpoint{3.477783in}{1.193195in}}%
\pgfpathlineto{\pgfqpoint{3.499025in}{1.150391in}}%
\pgfpathlineto{\pgfqpoint{3.520268in}{1.099480in}}%
\pgfpathlineto{\pgfqpoint{3.605238in}{0.874992in}}%
\pgfpathlineto{\pgfqpoint{3.626481in}{0.829424in}}%
\pgfpathlineto{\pgfqpoint{3.647723in}{0.793677in}}%
\pgfpathlineto{\pgfqpoint{3.668966in}{0.769750in}}%
\pgfpathlineto{\pgfqpoint{3.690208in}{0.758977in}}%
\pgfpathlineto{\pgfqpoint{3.711451in}{0.761961in}}%
\pgfpathlineto{\pgfqpoint{3.732694in}{0.778535in}}%
\pgfpathlineto{\pgfqpoint{3.753936in}{0.807774in}}%
\pgfpathlineto{\pgfqpoint{3.775179in}{0.848043in}}%
\pgfpathlineto{\pgfqpoint{3.796421in}{0.897095in}}%
\pgfpathlineto{\pgfqpoint{3.838906in}{1.010245in}}%
\pgfpathlineto{\pgfqpoint{3.860149in}{1.068024in}}%
\pgfpathlineto{\pgfqpoint{3.881392in}{1.122296in}}%
\pgfpathlineto{\pgfqpoint{3.902634in}{1.170030in}}%
\pgfpathlineto{\pgfqpoint{3.923877in}{1.208561in}}%
\pgfpathlineto{\pgfqpoint{3.945119in}{1.235734in}}%
\pgfpathlineto{\pgfqpoint{3.966362in}{1.250034in}}%
\pgfpathlineto{\pgfqpoint{3.987605in}{1.250660in}}%
\pgfpathlineto{\pgfqpoint{4.008847in}{1.237578in}}%
\pgfpathlineto{\pgfqpoint{4.030090in}{1.211519in}}%
\pgfpathlineto{\pgfqpoint{4.051332in}{1.173938in}}%
\pgfpathlineto{\pgfqpoint{4.072575in}{1.126935in}}%
\pgfpathlineto{\pgfqpoint{4.093817in}{1.073134in}}%
\pgfpathlineto{\pgfqpoint{4.157545in}{0.901883in}}%
\pgfpathlineto{\pgfqpoint{4.178788in}{0.852165in}}%
\pgfpathlineto{\pgfqpoint{4.200030in}{0.810998in}}%
\pgfpathlineto{\pgfqpoint{4.221273in}{0.780683in}}%
\pgfpathlineto{\pgfqpoint{4.242515in}{0.762912in}}%
\pgfpathlineto{\pgfqpoint{4.263758in}{0.758677in}}%
\pgfpathlineto{\pgfqpoint{4.285001in}{0.768217in}}%
\pgfpathlineto{\pgfqpoint{4.306243in}{0.790997in}}%
\pgfpathlineto{\pgfqpoint{4.327486in}{0.825745in}}%
\pgfpathlineto{\pgfqpoint{4.348728in}{0.870521in}}%
\pgfpathlineto{\pgfqpoint{4.369971in}{0.922822in}}%
\pgfpathlineto{\pgfqpoint{4.433699in}{1.094560in}}%
\pgfpathlineto{\pgfqpoint{4.454941in}{1.146071in}}%
\pgfpathlineto{\pgfqpoint{4.476184in}{1.189716in}}%
\pgfpathlineto{\pgfqpoint{4.497426in}{1.223058in}}%
\pgfpathlineto{\pgfqpoint{4.518669in}{1.244233in}}%
\pgfpathlineto{\pgfqpoint{4.539911in}{1.252059in}}%
\pgfpathlineto{\pgfqpoint{4.561154in}{1.246098in}}%
\pgfpathlineto{\pgfqpoint{4.582397in}{1.226685in}}%
\pgfpathlineto{\pgfqpoint{4.603639in}{1.194902in}}%
\pgfpathlineto{\pgfqpoint{4.624882in}{1.152526in}}%
\pgfpathlineto{\pgfqpoint{4.646124in}{1.101924in}}%
\pgfpathlineto{\pgfqpoint{4.688610in}{0.987647in}}%
\pgfpathlineto{\pgfqpoint{4.709852in}{0.930357in}}%
\pgfpathlineto{\pgfqpoint{4.731095in}{0.877250in}}%
\pgfpathlineto{\pgfqpoint{4.752337in}{0.831294in}}%
\pgfpathlineto{\pgfqpoint{4.773580in}{0.795054in}}%
\pgfpathlineto{\pgfqpoint{4.794822in}{0.770557in}}%
\pgfpathlineto{\pgfqpoint{4.816065in}{0.759169in}}%
\pgfpathlineto{\pgfqpoint{4.837308in}{0.761528in}}%
\pgfpathlineto{\pgfqpoint{4.858550in}{0.777500in}}%
\pgfpathlineto{\pgfqpoint{4.879793in}{0.806195in}}%
\pgfpathlineto{\pgfqpoint{4.901035in}{0.846010in}}%
\pgfpathlineto{\pgfqpoint{4.922278in}{0.894719in}}%
\pgfpathlineto{\pgfqpoint{4.943520in}{0.949603in}}%
\pgfpathlineto{\pgfqpoint{4.964763in}{1.007595in}}%
\pgfpathlineto{\pgfqpoint{4.964763in}{1.007595in}}%
\pgfusepath{stroke}%
\end{pgfscope}%
\begin{pgfscope}%
\pgfpathrectangle{\pgfqpoint{0.526127in}{0.521603in}}{\pgfqpoint{4.650000in}{2.310000in}}%
\pgfusepath{clip}%
\pgfsetrectcap%
\pgfsetroundjoin%
\pgfsetlinewidth{0.501875pt}%
\definecolor{currentstroke}{rgb}{0.000000,0.000000,0.000000}%
\pgfsetstrokecolor{currentstroke}%
\pgfsetdash{}{0pt}%
\pgfpathmoveto{\pgfqpoint{0.737490in}{0.779589in}}%
\pgfpathlineto{\pgfqpoint{0.758733in}{0.816013in}}%
\pgfpathlineto{\pgfqpoint{0.779975in}{0.867174in}}%
\pgfpathlineto{\pgfqpoint{0.801218in}{0.929088in}}%
\pgfpathlineto{\pgfqpoint{0.864946in}{1.129233in}}%
\pgfpathlineto{\pgfqpoint{0.886188in}{1.183386in}}%
\pgfpathlineto{\pgfqpoint{0.907431in}{1.223667in}}%
\pgfpathlineto{\pgfqpoint{0.928673in}{1.246939in}}%
\pgfpathlineto{\pgfqpoint{0.949916in}{1.251391in}}%
\pgfpathlineto{\pgfqpoint{0.971159in}{1.236675in}}%
\pgfpathlineto{\pgfqpoint{0.992401in}{1.203938in}}%
\pgfpathlineto{\pgfqpoint{1.013644in}{1.155729in}}%
\pgfpathlineto{\pgfqpoint{1.034886in}{1.095802in}}%
\pgfpathlineto{\pgfqpoint{1.098614in}{0.894719in}}%
\pgfpathlineto{\pgfqpoint{1.119857in}{0.838036in}}%
\pgfpathlineto{\pgfqpoint{1.141099in}{0.794376in}}%
\pgfpathlineto{\pgfqpoint{1.162342in}{0.767137in}}%
\pgfpathlineto{\pgfqpoint{1.183584in}{0.758441in}}%
\pgfpathlineto{\pgfqpoint{1.204827in}{0.768965in}}%
\pgfpathlineto{\pgfqpoint{1.226070in}{0.797890in}}%
\pgfpathlineto{\pgfqpoint{1.247312in}{0.842963in}}%
\pgfpathlineto{\pgfqpoint{1.268555in}{0.900674in}}%
\pgfpathlineto{\pgfqpoint{1.311040in}{1.035400in}}%
\pgfpathlineto{\pgfqpoint{1.332282in}{1.101924in}}%
\pgfpathlineto{\pgfqpoint{1.353525in}{1.160919in}}%
\pgfpathlineto{\pgfqpoint{1.374768in}{1.207792in}}%
\pgfpathlineto{\pgfqpoint{1.396010in}{1.238894in}}%
\pgfpathlineto{\pgfqpoint{1.417253in}{1.251800in}}%
\pgfpathlineto{\pgfqpoint{1.438495in}{1.245508in}}%
\pgfpathlineto{\pgfqpoint{1.459738in}{1.220506in}}%
\pgfpathlineto{\pgfqpoint{1.480980in}{1.178742in}}%
\pgfpathlineto{\pgfqpoint{1.502223in}{1.123468in}}%
\pgfpathlineto{\pgfqpoint{1.523466in}{1.058988in}}%
\pgfpathlineto{\pgfqpoint{1.565951in}{0.922822in}}%
\pgfpathlineto{\pgfqpoint{1.587193in}{0.861740in}}%
\pgfpathlineto{\pgfqpoint{1.608436in}{0.811834in}}%
\pgfpathlineto{\pgfqpoint{1.629679in}{0.776990in}}%
\pgfpathlineto{\pgfqpoint{1.650921in}{0.759922in}}%
\pgfpathlineto{\pgfqpoint{1.672164in}{0.761958in}}%
\pgfpathlineto{\pgfqpoint{1.693406in}{0.782941in}}%
\pgfpathlineto{\pgfqpoint{1.714649in}{0.821236in}}%
\pgfpathlineto{\pgfqpoint{1.735891in}{0.873862in}}%
\pgfpathlineto{\pgfqpoint{1.757134in}{0.936719in}}%
\pgfpathlineto{\pgfqpoint{1.799619in}{1.073134in}}%
\pgfpathlineto{\pgfqpoint{1.820862in}{1.136069in}}%
\pgfpathlineto{\pgfqpoint{1.842104in}{1.188817in}}%
\pgfpathlineto{\pgfqpoint{1.863347in}{1.227271in}}%
\pgfpathlineto{\pgfqpoint{1.884589in}{1.248435in}}%
\pgfpathlineto{\pgfqpoint{1.905832in}{1.250662in}}%
\pgfpathlineto{\pgfqpoint{1.927075in}{1.233778in}}%
\pgfpathlineto{\pgfqpoint{1.948317in}{1.199099in}}%
\pgfpathlineto{\pgfqpoint{1.969560in}{1.149324in}}%
\pgfpathlineto{\pgfqpoint{1.990802in}{1.088330in}}%
\pgfpathlineto{\pgfqpoint{2.054530in}{0.887641in}}%
\pgfpathlineto{\pgfqpoint{2.075773in}{0.832253in}}%
\pgfpathlineto{\pgfqpoint{2.097015in}{0.790338in}}%
\pgfpathlineto{\pgfqpoint{2.118258in}{0.765159in}}%
\pgfpathlineto{\pgfqpoint{2.139500in}{0.758677in}}%
\pgfpathlineto{\pgfqpoint{2.160743in}{0.771396in}}%
\pgfpathlineto{\pgfqpoint{2.181986in}{0.802327in}}%
\pgfpathlineto{\pgfqpoint{2.203228in}{0.849061in}}%
\pgfpathlineto{\pgfqpoint{2.224471in}{0.907958in}}%
\pgfpathlineto{\pgfqpoint{2.288198in}{1.109216in}}%
\pgfpathlineto{\pgfqpoint{2.309441in}{1.167031in}}%
\pgfpathlineto{\pgfqpoint{2.330684in}{1.212248in}}%
\pgfpathlineto{\pgfqpoint{2.351926in}{1.241346in}}%
\pgfpathlineto{\pgfqpoint{2.373169in}{1.252058in}}%
\pgfpathlineto{\pgfqpoint{2.394411in}{1.243551in}}%
\pgfpathlineto{\pgfqpoint{2.415654in}{1.216488in}}%
\pgfpathlineto{\pgfqpoint{2.436896in}{1.172974in}}%
\pgfpathlineto{\pgfqpoint{2.458139in}{1.116400in}}%
\pgfpathlineto{\pgfqpoint{2.500624in}{0.982366in}}%
\pgfpathlineto{\pgfqpoint{2.521867in}{0.915343in}}%
\pgfpathlineto{\pgfqpoint{2.543109in}{0.855322in}}%
\pgfpathlineto{\pgfqpoint{2.564352in}{0.806977in}}%
\pgfpathlineto{\pgfqpoint{2.585594in}{0.774072in}}%
\pgfpathlineto{\pgfqpoint{2.606837in}{0.759170in}}%
\pgfpathlineto{\pgfqpoint{2.628080in}{0.763432in}}%
\pgfpathlineto{\pgfqpoint{2.649322in}{0.786525in}}%
\pgfpathlineto{\pgfqpoint{2.670565in}{0.826651in}}%
\pgfpathlineto{\pgfqpoint{2.691807in}{0.880686in}}%
\pgfpathlineto{\pgfqpoint{2.713050in}{0.944421in}}%
\pgfpathlineto{\pgfqpoint{2.755535in}{1.080772in}}%
\pgfpathlineto{\pgfqpoint{2.776778in}{1.142768in}}%
\pgfpathlineto{\pgfqpoint{2.798020in}{1.194057in}}%
\pgfpathlineto{\pgfqpoint{2.819263in}{1.230642in}}%
\pgfpathlineto{\pgfqpoint{2.840505in}{1.249676in}}%
\pgfpathlineto{\pgfqpoint{2.861748in}{1.249676in}}%
\pgfpathlineto{\pgfqpoint{2.882991in}{1.230642in}}%
\pgfpathlineto{\pgfqpoint{2.904233in}{1.194057in}}%
\pgfpathlineto{\pgfqpoint{2.925476in}{1.142768in}}%
\pgfpathlineto{\pgfqpoint{2.946718in}{1.080772in}}%
\pgfpathlineto{\pgfqpoint{3.010446in}{0.880686in}}%
\pgfpathlineto{\pgfqpoint{3.031689in}{0.826651in}}%
\pgfpathlineto{\pgfqpoint{3.052931in}{0.786525in}}%
\pgfpathlineto{\pgfqpoint{3.074174in}{0.763432in}}%
\pgfpathlineto{\pgfqpoint{3.095416in}{0.759170in}}%
\pgfpathlineto{\pgfqpoint{3.116659in}{0.774072in}}%
\pgfpathlineto{\pgfqpoint{3.137901in}{0.806977in}}%
\pgfpathlineto{\pgfqpoint{3.159144in}{0.855322in}}%
\pgfpathlineto{\pgfqpoint{3.180387in}{0.915343in}}%
\pgfpathlineto{\pgfqpoint{3.244114in}{1.116400in}}%
\pgfpathlineto{\pgfqpoint{3.265357in}{1.172974in}}%
\pgfpathlineto{\pgfqpoint{3.286599in}{1.216488in}}%
\pgfpathlineto{\pgfqpoint{3.307842in}{1.243551in}}%
\pgfpathlineto{\pgfqpoint{3.329085in}{1.252058in}}%
\pgfpathlineto{\pgfqpoint{3.350327in}{1.241346in}}%
\pgfpathlineto{\pgfqpoint{3.371570in}{1.212248in}}%
\pgfpathlineto{\pgfqpoint{3.392812in}{1.167031in}}%
\pgfpathlineto{\pgfqpoint{3.414055in}{1.109216in}}%
\pgfpathlineto{\pgfqpoint{3.499025in}{0.849061in}}%
\pgfpathlineto{\pgfqpoint{3.520268in}{0.802327in}}%
\pgfpathlineto{\pgfqpoint{3.541510in}{0.771396in}}%
\pgfpathlineto{\pgfqpoint{3.562753in}{0.758677in}}%
\pgfpathlineto{\pgfqpoint{3.583996in}{0.765159in}}%
\pgfpathlineto{\pgfqpoint{3.605238in}{0.790338in}}%
\pgfpathlineto{\pgfqpoint{3.626481in}{0.832253in}}%
\pgfpathlineto{\pgfqpoint{3.647723in}{0.887641in}}%
\pgfpathlineto{\pgfqpoint{3.668966in}{0.952188in}}%
\pgfpathlineto{\pgfqpoint{3.711451in}{1.088330in}}%
\pgfpathlineto{\pgfqpoint{3.732694in}{1.149324in}}%
\pgfpathlineto{\pgfqpoint{3.753936in}{1.199099in}}%
\pgfpathlineto{\pgfqpoint{3.775179in}{1.233778in}}%
\pgfpathlineto{\pgfqpoint{3.796421in}{1.250662in}}%
\pgfpathlineto{\pgfqpoint{3.817664in}{1.248435in}}%
\pgfpathlineto{\pgfqpoint{3.838906in}{1.227271in}}%
\pgfpathlineto{\pgfqpoint{3.860149in}{1.188817in}}%
\pgfpathlineto{\pgfqpoint{3.881392in}{1.136069in}}%
\pgfpathlineto{\pgfqpoint{3.902634in}{1.073134in}}%
\pgfpathlineto{\pgfqpoint{3.945119in}{0.936719in}}%
\pgfpathlineto{\pgfqpoint{3.966362in}{0.873862in}}%
\pgfpathlineto{\pgfqpoint{3.987605in}{0.821236in}}%
\pgfpathlineto{\pgfqpoint{4.008847in}{0.782941in}}%
\pgfpathlineto{\pgfqpoint{4.030090in}{0.761958in}}%
\pgfpathlineto{\pgfqpoint{4.051332in}{0.759922in}}%
\pgfpathlineto{\pgfqpoint{4.072575in}{0.776990in}}%
\pgfpathlineto{\pgfqpoint{4.093817in}{0.811834in}}%
\pgfpathlineto{\pgfqpoint{4.115060in}{0.861740in}}%
\pgfpathlineto{\pgfqpoint{4.136303in}{0.922822in}}%
\pgfpathlineto{\pgfqpoint{4.200030in}{1.123468in}}%
\pgfpathlineto{\pgfqpoint{4.221273in}{1.178742in}}%
\pgfpathlineto{\pgfqpoint{4.242515in}{1.220506in}}%
\pgfpathlineto{\pgfqpoint{4.263758in}{1.245508in}}%
\pgfpathlineto{\pgfqpoint{4.285001in}{1.251800in}}%
\pgfpathlineto{\pgfqpoint{4.306243in}{1.238894in}}%
\pgfpathlineto{\pgfqpoint{4.327486in}{1.207792in}}%
\pgfpathlineto{\pgfqpoint{4.348728in}{1.160919in}}%
\pgfpathlineto{\pgfqpoint{4.369971in}{1.101924in}}%
\pgfpathlineto{\pgfqpoint{4.433699in}{0.900674in}}%
\pgfpathlineto{\pgfqpoint{4.454941in}{0.842963in}}%
\pgfpathlineto{\pgfqpoint{4.476184in}{0.797890in}}%
\pgfpathlineto{\pgfqpoint{4.497426in}{0.768965in}}%
\pgfpathlineto{\pgfqpoint{4.518669in}{0.758441in}}%
\pgfpathlineto{\pgfqpoint{4.539911in}{0.767137in}}%
\pgfpathlineto{\pgfqpoint{4.561154in}{0.794376in}}%
\pgfpathlineto{\pgfqpoint{4.582397in}{0.838036in}}%
\pgfpathlineto{\pgfqpoint{4.603639in}{0.894719in}}%
\pgfpathlineto{\pgfqpoint{4.646124in}{1.028823in}}%
\pgfpathlineto{\pgfqpoint{4.667367in}{1.095802in}}%
\pgfpathlineto{\pgfqpoint{4.688610in}{1.155729in}}%
\pgfpathlineto{\pgfqpoint{4.709852in}{1.203938in}}%
\pgfpathlineto{\pgfqpoint{4.731095in}{1.236675in}}%
\pgfpathlineto{\pgfqpoint{4.752337in}{1.251391in}}%
\pgfpathlineto{\pgfqpoint{4.773580in}{1.246939in}}%
\pgfpathlineto{\pgfqpoint{4.794822in}{1.223667in}}%
\pgfpathlineto{\pgfqpoint{4.816065in}{1.183386in}}%
\pgfpathlineto{\pgfqpoint{4.837308in}{1.129233in}}%
\pgfpathlineto{\pgfqpoint{4.858550in}{1.065426in}}%
\pgfpathlineto{\pgfqpoint{4.901035in}{0.929088in}}%
\pgfpathlineto{\pgfqpoint{4.922278in}{0.867174in}}%
\pgfpathlineto{\pgfqpoint{4.943520in}{0.816013in}}%
\pgfpathlineto{\pgfqpoint{4.964763in}{0.779589in}}%
\pgfpathlineto{\pgfqpoint{4.964763in}{0.779589in}}%
\pgfusepath{stroke}%
\end{pgfscope}%
\begin{pgfscope}%
\pgfpathrectangle{\pgfqpoint{0.526127in}{0.521603in}}{\pgfqpoint{4.650000in}{2.310000in}}%
\pgfusepath{clip}%
\pgfsetrectcap%
\pgfsetroundjoin%
\pgfsetlinewidth{1.505625pt}%
\definecolor{currentstroke}{rgb}{0.000000,0.000000,0.000000}%
\pgfsetstrokecolor{currentstroke}%
\pgfsetdash{}{0pt}%
\pgfpathmoveto{\pgfqpoint{0.737490in}{0.860223in}}%
\pgfpathlineto{\pgfqpoint{0.758733in}{0.861994in}}%
\pgfpathlineto{\pgfqpoint{0.779975in}{0.876413in}}%
\pgfpathlineto{\pgfqpoint{0.801218in}{0.901897in}}%
\pgfpathlineto{\pgfqpoint{0.822461in}{0.935947in}}%
\pgfpathlineto{\pgfqpoint{0.886188in}{1.056033in}}%
\pgfpathlineto{\pgfqpoint{0.907431in}{1.090166in}}%
\pgfpathlineto{\pgfqpoint{0.928673in}{1.116130in}}%
\pgfpathlineto{\pgfqpoint{0.949916in}{1.131783in}}%
\pgfpathlineto{\pgfqpoint{0.971159in}{1.135918in}}%
\pgfpathlineto{\pgfqpoint{0.992401in}{1.128348in}}%
\pgfpathlineto{\pgfqpoint{1.013644in}{1.109909in}}%
\pgfpathlineto{\pgfqpoint{1.034886in}{1.082372in}}%
\pgfpathlineto{\pgfqpoint{1.056129in}{1.048280in}}%
\pgfpathlineto{\pgfqpoint{1.098614in}{0.973040in}}%
\pgfpathlineto{\pgfqpoint{1.119857in}{0.938579in}}%
\pgfpathlineto{\pgfqpoint{1.141099in}{0.910356in}}%
\pgfpathlineto{\pgfqpoint{1.162342in}{0.890821in}}%
\pgfpathlineto{\pgfqpoint{1.183584in}{0.881648in}}%
\pgfpathlineto{\pgfqpoint{1.204827in}{0.883591in}}%
\pgfpathlineto{\pgfqpoint{1.226070in}{0.896425in}}%
\pgfpathlineto{\pgfqpoint{1.247312in}{0.918971in}}%
\pgfpathlineto{\pgfqpoint{1.268555in}{0.949201in}}%
\pgfpathlineto{\pgfqpoint{1.311040in}{1.021481in}}%
\pgfpathlineto{\pgfqpoint{1.332282in}{1.057105in}}%
\pgfpathlineto{\pgfqpoint{1.353525in}{1.088119in}}%
\pgfpathlineto{\pgfqpoint{1.374768in}{1.111755in}}%
\pgfpathlineto{\pgfqpoint{1.396010in}{1.125890in}}%
\pgfpathlineto{\pgfqpoint{1.417253in}{1.129231in}}%
\pgfpathlineto{\pgfqpoint{1.438495in}{1.121431in}}%
\pgfpathlineto{\pgfqpoint{1.459738in}{1.103122in}}%
\pgfpathlineto{\pgfqpoint{1.480980in}{1.075868in}}%
\pgfpathlineto{\pgfqpoint{1.502223in}{1.042033in}}%
\pgfpathlineto{\pgfqpoint{1.544708in}{0.966816in}}%
\pgfpathlineto{\pgfqpoint{1.565951in}{0.932111in}}%
\pgfpathlineto{\pgfqpoint{1.587193in}{0.903599in}}%
\pgfpathlineto{\pgfqpoint{1.608436in}{0.883905in}}%
\pgfpathlineto{\pgfqpoint{1.629679in}{0.874906in}}%
\pgfpathlineto{\pgfqpoint{1.650921in}{0.877562in}}%
\pgfpathlineto{\pgfqpoint{1.672164in}{0.891817in}}%
\pgfpathlineto{\pgfqpoint{1.693406in}{0.916587in}}%
\pgfpathlineto{\pgfqpoint{1.714649in}{0.949842in}}%
\pgfpathlineto{\pgfqpoint{1.757134in}{1.029978in}}%
\pgfpathlineto{\pgfqpoint{1.778377in}{1.069842in}}%
\pgfpathlineto{\pgfqpoint{1.799619in}{1.104752in}}%
\pgfpathlineto{\pgfqpoint{1.820862in}{1.131464in}}%
\pgfpathlineto{\pgfqpoint{1.842104in}{1.147378in}}%
\pgfpathlineto{\pgfqpoint{1.863347in}{1.150786in}}%
\pgfpathlineto{\pgfqpoint{1.884589in}{1.141042in}}%
\pgfpathlineto{\pgfqpoint{1.905832in}{1.118642in}}%
\pgfpathlineto{\pgfqpoint{1.927075in}{1.085219in}}%
\pgfpathlineto{\pgfqpoint{1.948317in}{1.043425in}}%
\pgfpathlineto{\pgfqpoint{2.012045in}{0.905006in}}%
\pgfpathlineto{\pgfqpoint{2.033287in}{0.868299in}}%
\pgfpathlineto{\pgfqpoint{2.054530in}{0.842657in}}%
\pgfpathlineto{\pgfqpoint{2.075773in}{0.830841in}}%
\pgfpathlineto{\pgfqpoint{2.097015in}{0.834507in}}%
\pgfpathlineto{\pgfqpoint{2.118258in}{0.854024in}}%
\pgfpathlineto{\pgfqpoint{2.139500in}{0.888381in}}%
\pgfpathlineto{\pgfqpoint{2.160743in}{0.935218in}}%
\pgfpathlineto{\pgfqpoint{2.181986in}{0.990967in}}%
\pgfpathlineto{\pgfqpoint{2.224471in}{1.110446in}}%
\pgfpathlineto{\pgfqpoint{2.245713in}{1.163641in}}%
\pgfpathlineto{\pgfqpoint{2.266956in}{1.205523in}}%
\pgfpathlineto{\pgfqpoint{2.288198in}{1.231601in}}%
\pgfpathlineto{\pgfqpoint{2.309441in}{1.238463in}}%
\pgfpathlineto{\pgfqpoint{2.330684in}{1.224125in}}%
\pgfpathlineto{\pgfqpoint{2.351926in}{1.188284in}}%
\pgfpathlineto{\pgfqpoint{2.373169in}{1.132447in}}%
\pgfpathlineto{\pgfqpoint{2.394411in}{1.059940in}}%
\pgfpathlineto{\pgfqpoint{2.415654in}{0.975763in}}%
\pgfpathlineto{\pgfqpoint{2.458139in}{0.799032in}}%
\pgfpathlineto{\pgfqpoint{2.479382in}{0.721839in}}%
\pgfpathlineto{\pgfqpoint{2.500624in}{0.662670in}}%
\pgfpathlineto{\pgfqpoint{2.521867in}{0.628861in}}%
\pgfpathlineto{\pgfqpoint{2.543109in}{0.626603in}}%
\pgfpathlineto{\pgfqpoint{2.564352in}{0.660447in}}%
\pgfpathlineto{\pgfqpoint{2.585594in}{0.732898in}}%
\pgfpathlineto{\pgfqpoint{2.606837in}{0.844142in}}%
\pgfpathlineto{\pgfqpoint{2.628080in}{0.991915in}}%
\pgfpathlineto{\pgfqpoint{2.649322in}{1.171536in}}%
\pgfpathlineto{\pgfqpoint{2.670565in}{1.376112in}}%
\pgfpathlineto{\pgfqpoint{2.755535in}{2.251644in}}%
\pgfpathlineto{\pgfqpoint{2.776778in}{2.431155in}}%
\pgfpathlineto{\pgfqpoint{2.798020in}{2.574833in}}%
\pgfpathlineto{\pgfqpoint{2.819263in}{2.675095in}}%
\pgfpathlineto{\pgfqpoint{2.840505in}{2.726603in}}%
\pgfpathlineto{\pgfqpoint{2.861748in}{2.726603in}}%
\pgfpathlineto{\pgfqpoint{2.882991in}{2.675095in}}%
\pgfpathlineto{\pgfqpoint{2.904233in}{2.574833in}}%
\pgfpathlineto{\pgfqpoint{2.925476in}{2.431155in}}%
\pgfpathlineto{\pgfqpoint{2.946718in}{2.251644in}}%
\pgfpathlineto{\pgfqpoint{2.967961in}{2.045648in}}%
\pgfpathlineto{\pgfqpoint{3.031689in}{1.376112in}}%
\pgfpathlineto{\pgfqpoint{3.052931in}{1.171536in}}%
\pgfpathlineto{\pgfqpoint{3.074174in}{0.991915in}}%
\pgfpathlineto{\pgfqpoint{3.095416in}{0.844142in}}%
\pgfpathlineto{\pgfqpoint{3.116659in}{0.732898in}}%
\pgfpathlineto{\pgfqpoint{3.137901in}{0.660447in}}%
\pgfpathlineto{\pgfqpoint{3.159144in}{0.626603in}}%
\pgfpathlineto{\pgfqpoint{3.180387in}{0.628861in}}%
\pgfpathlineto{\pgfqpoint{3.201629in}{0.662670in}}%
\pgfpathlineto{\pgfqpoint{3.222872in}{0.721839in}}%
\pgfpathlineto{\pgfqpoint{3.244114in}{0.799032in}}%
\pgfpathlineto{\pgfqpoint{3.307842in}{1.059940in}}%
\pgfpathlineto{\pgfqpoint{3.329085in}{1.132447in}}%
\pgfpathlineto{\pgfqpoint{3.350327in}{1.188284in}}%
\pgfpathlineto{\pgfqpoint{3.371570in}{1.224125in}}%
\pgfpathlineto{\pgfqpoint{3.392812in}{1.238463in}}%
\pgfpathlineto{\pgfqpoint{3.414055in}{1.231601in}}%
\pgfpathlineto{\pgfqpoint{3.435298in}{1.205523in}}%
\pgfpathlineto{\pgfqpoint{3.456540in}{1.163641in}}%
\pgfpathlineto{\pgfqpoint{3.477783in}{1.110446in}}%
\pgfpathlineto{\pgfqpoint{3.541510in}{0.935218in}}%
\pgfpathlineto{\pgfqpoint{3.562753in}{0.888381in}}%
\pgfpathlineto{\pgfqpoint{3.583996in}{0.854024in}}%
\pgfpathlineto{\pgfqpoint{3.605238in}{0.834507in}}%
\pgfpathlineto{\pgfqpoint{3.626481in}{0.830841in}}%
\pgfpathlineto{\pgfqpoint{3.647723in}{0.842657in}}%
\pgfpathlineto{\pgfqpoint{3.668966in}{0.868299in}}%
\pgfpathlineto{\pgfqpoint{3.690208in}{0.905006in}}%
\pgfpathlineto{\pgfqpoint{3.711451in}{0.949185in}}%
\pgfpathlineto{\pgfqpoint{3.753936in}{1.043425in}}%
\pgfpathlineto{\pgfqpoint{3.775179in}{1.085219in}}%
\pgfpathlineto{\pgfqpoint{3.796421in}{1.118642in}}%
\pgfpathlineto{\pgfqpoint{3.817664in}{1.141042in}}%
\pgfpathlineto{\pgfqpoint{3.838906in}{1.150786in}}%
\pgfpathlineto{\pgfqpoint{3.860149in}{1.147378in}}%
\pgfpathlineto{\pgfqpoint{3.881392in}{1.131464in}}%
\pgfpathlineto{\pgfqpoint{3.902634in}{1.104752in}}%
\pgfpathlineto{\pgfqpoint{3.923877in}{1.069842in}}%
\pgfpathlineto{\pgfqpoint{3.987605in}{0.949842in}}%
\pgfpathlineto{\pgfqpoint{4.008847in}{0.916587in}}%
\pgfpathlineto{\pgfqpoint{4.030090in}{0.891817in}}%
\pgfpathlineto{\pgfqpoint{4.051332in}{0.877562in}}%
\pgfpathlineto{\pgfqpoint{4.072575in}{0.874906in}}%
\pgfpathlineto{\pgfqpoint{4.093817in}{0.883905in}}%
\pgfpathlineto{\pgfqpoint{4.115060in}{0.903599in}}%
\pgfpathlineto{\pgfqpoint{4.136303in}{0.932111in}}%
\pgfpathlineto{\pgfqpoint{4.157545in}{0.966816in}}%
\pgfpathlineto{\pgfqpoint{4.200030in}{1.042033in}}%
\pgfpathlineto{\pgfqpoint{4.221273in}{1.075868in}}%
\pgfpathlineto{\pgfqpoint{4.242515in}{1.103122in}}%
\pgfpathlineto{\pgfqpoint{4.263758in}{1.121431in}}%
\pgfpathlineto{\pgfqpoint{4.285001in}{1.129231in}}%
\pgfpathlineto{\pgfqpoint{4.306243in}{1.125890in}}%
\pgfpathlineto{\pgfqpoint{4.327486in}{1.111755in}}%
\pgfpathlineto{\pgfqpoint{4.348728in}{1.088119in}}%
\pgfpathlineto{\pgfqpoint{4.369971in}{1.057105in}}%
\pgfpathlineto{\pgfqpoint{4.433699in}{0.949201in}}%
\pgfpathlineto{\pgfqpoint{4.454941in}{0.918971in}}%
\pgfpathlineto{\pgfqpoint{4.476184in}{0.896425in}}%
\pgfpathlineto{\pgfqpoint{4.497426in}{0.883591in}}%
\pgfpathlineto{\pgfqpoint{4.518669in}{0.881648in}}%
\pgfpathlineto{\pgfqpoint{4.539911in}{0.890821in}}%
\pgfpathlineto{\pgfqpoint{4.561154in}{0.910356in}}%
\pgfpathlineto{\pgfqpoint{4.582397in}{0.938579in}}%
\pgfpathlineto{\pgfqpoint{4.603639in}{0.973040in}}%
\pgfpathlineto{\pgfqpoint{4.646124in}{1.048280in}}%
\pgfpathlineto{\pgfqpoint{4.667367in}{1.082372in}}%
\pgfpathlineto{\pgfqpoint{4.688610in}{1.109909in}}%
\pgfpathlineto{\pgfqpoint{4.709852in}{1.128348in}}%
\pgfpathlineto{\pgfqpoint{4.731095in}{1.135918in}}%
\pgfpathlineto{\pgfqpoint{4.752337in}{1.131783in}}%
\pgfpathlineto{\pgfqpoint{4.773580in}{1.116130in}}%
\pgfpathlineto{\pgfqpoint{4.794822in}{1.090166in}}%
\pgfpathlineto{\pgfqpoint{4.816065in}{1.056033in}}%
\pgfpathlineto{\pgfqpoint{4.879793in}{0.935947in}}%
\pgfpathlineto{\pgfqpoint{4.901035in}{0.901897in}}%
\pgfpathlineto{\pgfqpoint{4.922278in}{0.876413in}}%
\pgfpathlineto{\pgfqpoint{4.943520in}{0.861994in}}%
\pgfpathlineto{\pgfqpoint{4.964763in}{0.860223in}}%
\pgfpathlineto{\pgfqpoint{4.964763in}{0.860223in}}%
\pgfusepath{stroke}%
\end{pgfscope}%
\begin{pgfscope}%
\pgfpathrectangle{\pgfqpoint{0.526127in}{0.521603in}}{\pgfqpoint{4.650000in}{2.310000in}}%
\pgfusepath{clip}%
\pgfsetbuttcap%
\pgfsetroundjoin%
\pgfsetlinewidth{1.505625pt}%
\definecolor{currentstroke}{rgb}{1.000000,0.000000,0.000000}%
\pgfsetstrokecolor{currentstroke}%
\pgfsetdash{{5.550000pt}{2.400000pt}}{0.000000pt}%
\pgfpathmoveto{\pgfqpoint{0.737490in}{0.948349in}}%
\pgfpathlineto{\pgfqpoint{0.758733in}{0.947699in}}%
\pgfpathlineto{\pgfqpoint{0.779975in}{0.952282in}}%
\pgfpathlineto{\pgfqpoint{0.801218in}{0.961789in}}%
\pgfpathlineto{\pgfqpoint{0.822461in}{0.975457in}}%
\pgfpathlineto{\pgfqpoint{0.843703in}{0.992131in}}%
\pgfpathlineto{\pgfqpoint{0.886188in}{1.028515in}}%
\pgfpathlineto{\pgfqpoint{0.907431in}{1.044947in}}%
\pgfpathlineto{\pgfqpoint{0.928673in}{1.058121in}}%
\pgfpathlineto{\pgfqpoint{0.949916in}{1.066767in}}%
\pgfpathlineto{\pgfqpoint{0.971159in}{1.069995in}}%
\pgfpathlineto{\pgfqpoint{0.992401in}{1.067390in}}%
\pgfpathlineto{\pgfqpoint{1.013644in}{1.059054in}}%
\pgfpathlineto{\pgfqpoint{1.034886in}{1.045612in}}%
\pgfpathlineto{\pgfqpoint{1.056129in}{1.028168in}}%
\pgfpathlineto{\pgfqpoint{1.119857in}{0.967911in}}%
\pgfpathlineto{\pgfqpoint{1.141099in}{0.951240in}}%
\pgfpathlineto{\pgfqpoint{1.162342in}{0.939084in}}%
\pgfpathlineto{\pgfqpoint{1.183584in}{0.932665in}}%
\pgfpathlineto{\pgfqpoint{1.204827in}{0.932713in}}%
\pgfpathlineto{\pgfqpoint{1.226070in}{0.939393in}}%
\pgfpathlineto{\pgfqpoint{1.247312in}{0.952270in}}%
\pgfpathlineto{\pgfqpoint{1.268555in}{0.970338in}}%
\pgfpathlineto{\pgfqpoint{1.289797in}{0.992090in}}%
\pgfpathlineto{\pgfqpoint{1.332282in}{1.038911in}}%
\pgfpathlineto{\pgfqpoint{1.353525in}{1.059756in}}%
\pgfpathlineto{\pgfqpoint{1.374768in}{1.076213in}}%
\pgfpathlineto{\pgfqpoint{1.396010in}{1.086656in}}%
\pgfpathlineto{\pgfqpoint{1.417253in}{1.089954in}}%
\pgfpathlineto{\pgfqpoint{1.438495in}{1.085596in}}%
\pgfpathlineto{\pgfqpoint{1.459738in}{1.073745in}}%
\pgfpathlineto{\pgfqpoint{1.480980in}{1.055252in}}%
\pgfpathlineto{\pgfqpoint{1.502223in}{1.031594in}}%
\pgfpathlineto{\pgfqpoint{1.565951in}{0.951119in}}%
\pgfpathlineto{\pgfqpoint{1.587193in}{0.929222in}}%
\pgfpathlineto{\pgfqpoint{1.608436in}{0.913554in}}%
\pgfpathlineto{\pgfqpoint{1.629679in}{0.905757in}}%
\pgfpathlineto{\pgfqpoint{1.650921in}{0.906820in}}%
\pgfpathlineto{\pgfqpoint{1.672164in}{0.916962in}}%
\pgfpathlineto{\pgfqpoint{1.693406in}{0.935589in}}%
\pgfpathlineto{\pgfqpoint{1.714649in}{0.961314in}}%
\pgfpathlineto{\pgfqpoint{1.735891in}{0.992059in}}%
\pgfpathlineto{\pgfqpoint{1.778377in}{1.057841in}}%
\pgfpathlineto{\pgfqpoint{1.799619in}{1.086949in}}%
\pgfpathlineto{\pgfqpoint{1.820862in}{1.109735in}}%
\pgfpathlineto{\pgfqpoint{1.842104in}{1.123855in}}%
\pgfpathlineto{\pgfqpoint{1.863347in}{1.127652in}}%
\pgfpathlineto{\pgfqpoint{1.884589in}{1.120331in}}%
\pgfpathlineto{\pgfqpoint{1.905832in}{1.102065in}}%
\pgfpathlineto{\pgfqpoint{1.927075in}{1.074025in}}%
\pgfpathlineto{\pgfqpoint{1.948317in}{1.038314in}}%
\pgfpathlineto{\pgfqpoint{2.012045in}{0.916667in}}%
\pgfpathlineto{\pgfqpoint{2.033287in}{0.883461in}}%
\pgfpathlineto{\pgfqpoint{2.054530in}{0.859765in}}%
\pgfpathlineto{\pgfqpoint{2.075773in}{0.848256in}}%
\pgfpathlineto{\pgfqpoint{2.097015in}{0.850662in}}%
\pgfpathlineto{\pgfqpoint{2.118258in}{0.867559in}}%
\pgfpathlineto{\pgfqpoint{2.139500in}{0.898259in}}%
\pgfpathlineto{\pgfqpoint{2.160743in}{0.940798in}}%
\pgfpathlineto{\pgfqpoint{2.181986in}{0.992038in}}%
\pgfpathlineto{\pgfqpoint{2.224471in}{1.103511in}}%
\pgfpathlineto{\pgfqpoint{2.245713in}{1.153882in}}%
\pgfpathlineto{\pgfqpoint{2.266956in}{1.194019in}}%
\pgfpathlineto{\pgfqpoint{2.288198in}{1.219512in}}%
\pgfpathlineto{\pgfqpoint{2.309441in}{1.226916in}}%
\pgfpathlineto{\pgfqpoint{2.330684in}{1.214110in}}%
\pgfpathlineto{\pgfqpoint{2.351926in}{1.180570in}}%
\pgfpathlineto{\pgfqpoint{2.373169in}{1.127528in}}%
\pgfpathlineto{\pgfqpoint{2.394411in}{1.058007in}}%
\pgfpathlineto{\pgfqpoint{2.415654in}{0.976712in}}%
\pgfpathlineto{\pgfqpoint{2.458139in}{0.804444in}}%
\pgfpathlineto{\pgfqpoint{2.479382in}{0.728501in}}%
\pgfpathlineto{\pgfqpoint{2.500624in}{0.669835in}}%
\pgfpathlineto{\pgfqpoint{2.521867in}{0.635813in}}%
\pgfpathlineto{\pgfqpoint{2.543109in}{0.632727in}}%
\pgfpathlineto{\pgfqpoint{2.564352in}{0.665284in}}%
\pgfpathlineto{\pgfqpoint{2.585594in}{0.736176in}}%
\pgfpathlineto{\pgfqpoint{2.606837in}{0.845784in}}%
\pgfpathlineto{\pgfqpoint{2.628080in}{0.992028in}}%
\pgfpathlineto{\pgfqpoint{2.649322in}{1.170379in}}%
\pgfpathlineto{\pgfqpoint{2.670565in}{1.374044in}}%
\pgfpathlineto{\pgfqpoint{2.755535in}{2.249693in}}%
\pgfpathlineto{\pgfqpoint{2.776778in}{2.429806in}}%
\pgfpathlineto{\pgfqpoint{2.798020in}{2.574075in}}%
\pgfpathlineto{\pgfqpoint{2.819263in}{2.674804in}}%
\pgfpathlineto{\pgfqpoint{2.840505in}{2.726570in}}%
\pgfpathlineto{\pgfqpoint{2.861748in}{2.726570in}}%
\pgfpathlineto{\pgfqpoint{2.882991in}{2.674804in}}%
\pgfpathlineto{\pgfqpoint{2.904233in}{2.574075in}}%
\pgfpathlineto{\pgfqpoint{2.925476in}{2.429806in}}%
\pgfpathlineto{\pgfqpoint{2.946718in}{2.249693in}}%
\pgfpathlineto{\pgfqpoint{2.967961in}{2.043216in}}%
\pgfpathlineto{\pgfqpoint{3.031689in}{1.374044in}}%
\pgfpathlineto{\pgfqpoint{3.052931in}{1.170379in}}%
\pgfpathlineto{\pgfqpoint{3.074174in}{0.992028in}}%
\pgfpathlineto{\pgfqpoint{3.095416in}{0.845784in}}%
\pgfpathlineto{\pgfqpoint{3.116659in}{0.736176in}}%
\pgfpathlineto{\pgfqpoint{3.137901in}{0.665284in}}%
\pgfpathlineto{\pgfqpoint{3.159144in}{0.632727in}}%
\pgfpathlineto{\pgfqpoint{3.180387in}{0.635813in}}%
\pgfpathlineto{\pgfqpoint{3.201629in}{0.669835in}}%
\pgfpathlineto{\pgfqpoint{3.222872in}{0.728501in}}%
\pgfpathlineto{\pgfqpoint{3.244114in}{0.804444in}}%
\pgfpathlineto{\pgfqpoint{3.307842in}{1.058007in}}%
\pgfpathlineto{\pgfqpoint{3.329085in}{1.127528in}}%
\pgfpathlineto{\pgfqpoint{3.350327in}{1.180570in}}%
\pgfpathlineto{\pgfqpoint{3.371570in}{1.214110in}}%
\pgfpathlineto{\pgfqpoint{3.392812in}{1.226916in}}%
\pgfpathlineto{\pgfqpoint{3.414055in}{1.219512in}}%
\pgfpathlineto{\pgfqpoint{3.435298in}{1.194019in}}%
\pgfpathlineto{\pgfqpoint{3.456540in}{1.153882in}}%
\pgfpathlineto{\pgfqpoint{3.477783in}{1.103511in}}%
\pgfpathlineto{\pgfqpoint{3.520268in}{0.992038in}}%
\pgfpathlineto{\pgfqpoint{3.541510in}{0.940798in}}%
\pgfpathlineto{\pgfqpoint{3.562753in}{0.898259in}}%
\pgfpathlineto{\pgfqpoint{3.583996in}{0.867559in}}%
\pgfpathlineto{\pgfqpoint{3.605238in}{0.850662in}}%
\pgfpathlineto{\pgfqpoint{3.626481in}{0.848256in}}%
\pgfpathlineto{\pgfqpoint{3.647723in}{0.859765in}}%
\pgfpathlineto{\pgfqpoint{3.668966in}{0.883461in}}%
\pgfpathlineto{\pgfqpoint{3.690208in}{0.916667in}}%
\pgfpathlineto{\pgfqpoint{3.711451in}{0.956029in}}%
\pgfpathlineto{\pgfqpoint{3.753936in}{1.038314in}}%
\pgfpathlineto{\pgfqpoint{3.775179in}{1.074025in}}%
\pgfpathlineto{\pgfqpoint{3.796421in}{1.102065in}}%
\pgfpathlineto{\pgfqpoint{3.817664in}{1.120331in}}%
\pgfpathlineto{\pgfqpoint{3.838906in}{1.127652in}}%
\pgfpathlineto{\pgfqpoint{3.860149in}{1.123855in}}%
\pgfpathlineto{\pgfqpoint{3.881392in}{1.109735in}}%
\pgfpathlineto{\pgfqpoint{3.902634in}{1.086949in}}%
\pgfpathlineto{\pgfqpoint{3.923877in}{1.057841in}}%
\pgfpathlineto{\pgfqpoint{3.987605in}{0.961314in}}%
\pgfpathlineto{\pgfqpoint{4.008847in}{0.935589in}}%
\pgfpathlineto{\pgfqpoint{4.030090in}{0.916962in}}%
\pgfpathlineto{\pgfqpoint{4.051332in}{0.906820in}}%
\pgfpathlineto{\pgfqpoint{4.072575in}{0.905757in}}%
\pgfpathlineto{\pgfqpoint{4.093817in}{0.913554in}}%
\pgfpathlineto{\pgfqpoint{4.115060in}{0.929222in}}%
\pgfpathlineto{\pgfqpoint{4.136303in}{0.951119in}}%
\pgfpathlineto{\pgfqpoint{4.178788in}{1.004765in}}%
\pgfpathlineto{\pgfqpoint{4.200030in}{1.031594in}}%
\pgfpathlineto{\pgfqpoint{4.221273in}{1.055252in}}%
\pgfpathlineto{\pgfqpoint{4.242515in}{1.073745in}}%
\pgfpathlineto{\pgfqpoint{4.263758in}{1.085596in}}%
\pgfpathlineto{\pgfqpoint{4.285001in}{1.089954in}}%
\pgfpathlineto{\pgfqpoint{4.306243in}{1.086656in}}%
\pgfpathlineto{\pgfqpoint{4.327486in}{1.076213in}}%
\pgfpathlineto{\pgfqpoint{4.348728in}{1.059756in}}%
\pgfpathlineto{\pgfqpoint{4.369971in}{1.038911in}}%
\pgfpathlineto{\pgfqpoint{4.433699in}{0.970338in}}%
\pgfpathlineto{\pgfqpoint{4.454941in}{0.952270in}}%
\pgfpathlineto{\pgfqpoint{4.476184in}{0.939393in}}%
\pgfpathlineto{\pgfqpoint{4.497426in}{0.932713in}}%
\pgfpathlineto{\pgfqpoint{4.518669in}{0.932665in}}%
\pgfpathlineto{\pgfqpoint{4.539911in}{0.939084in}}%
\pgfpathlineto{\pgfqpoint{4.561154in}{0.951240in}}%
\pgfpathlineto{\pgfqpoint{4.582397in}{0.967911in}}%
\pgfpathlineto{\pgfqpoint{4.667367in}{1.045612in}}%
\pgfpathlineto{\pgfqpoint{4.688610in}{1.059054in}}%
\pgfpathlineto{\pgfqpoint{4.709852in}{1.067390in}}%
\pgfpathlineto{\pgfqpoint{4.731095in}{1.069995in}}%
\pgfpathlineto{\pgfqpoint{4.752337in}{1.066767in}}%
\pgfpathlineto{\pgfqpoint{4.773580in}{1.058121in}}%
\pgfpathlineto{\pgfqpoint{4.794822in}{1.044947in}}%
\pgfpathlineto{\pgfqpoint{4.816065in}{1.028515in}}%
\pgfpathlineto{\pgfqpoint{4.879793in}{0.975457in}}%
\pgfpathlineto{\pgfqpoint{4.901035in}{0.961789in}}%
\pgfpathlineto{\pgfqpoint{4.922278in}{0.952282in}}%
\pgfpathlineto{\pgfqpoint{4.943520in}{0.947699in}}%
\pgfpathlineto{\pgfqpoint{4.964763in}{0.948349in}}%
\pgfpathlineto{\pgfqpoint{4.964763in}{0.948349in}}%
\pgfusepath{stroke}%
\end{pgfscope}%
\begin{pgfscope}%
\pgfpathrectangle{\pgfqpoint{0.526127in}{0.521603in}}{\pgfqpoint{4.650000in}{2.310000in}}%
\pgfusepath{clip}%
\pgfsetrectcap%
\pgfsetroundjoin%
\pgfsetlinewidth{0.501875pt}%
\definecolor{currentstroke}{rgb}{0.000000,0.000000,0.000000}%
\pgfsetstrokecolor{currentstroke}%
\pgfsetdash{}{0pt}%
\pgfpathmoveto{\pgfqpoint{0.526127in}{1.005255in}}%
\pgfpathlineto{\pgfqpoint{5.176127in}{1.005255in}}%
\pgfusepath{stroke}%
\end{pgfscope}%
\begin{pgfscope}%
\pgfpathrectangle{\pgfqpoint{0.526127in}{0.521603in}}{\pgfqpoint{4.650000in}{2.310000in}}%
\pgfusepath{clip}%
\pgfsetrectcap%
\pgfsetroundjoin%
\pgfsetlinewidth{0.501875pt}%
\definecolor{currentstroke}{rgb}{0.000000,0.000000,0.000000}%
\pgfsetstrokecolor{currentstroke}%
\pgfsetdash{}{0pt}%
\pgfpathmoveto{\pgfqpoint{2.851127in}{0.521603in}}%
\pgfpathlineto{\pgfqpoint{2.851127in}{2.831603in}}%
\pgfusepath{stroke}%
\end{pgfscope}%
\begin{pgfscope}%
\pgfsetbuttcap%
\pgfsetmiterjoin%
\definecolor{currentfill}{rgb}{1.000000,1.000000,1.000000}%
\pgfsetfillcolor{currentfill}%
\pgfsetfillopacity{0.800000}%
\pgfsetlinewidth{1.003750pt}%
\definecolor{currentstroke}{rgb}{0.800000,0.800000,0.800000}%
\pgfsetstrokecolor{currentstroke}%
\pgfsetstrokeopacity{0.800000}%
\pgfsetdash{}{0pt}%
\pgfpathmoveto{\pgfqpoint{3.574822in}{2.306945in}}%
\pgfpathlineto{\pgfqpoint{5.078904in}{2.306945in}}%
\pgfpathquadraticcurveto{\pgfqpoint{5.106682in}{2.306945in}}{\pgfqpoint{5.106682in}{2.334723in}}%
\pgfpathlineto{\pgfqpoint{5.106682in}{2.734381in}}%
\pgfpathquadraticcurveto{\pgfqpoint{5.106682in}{2.762159in}}{\pgfqpoint{5.078904in}{2.762159in}}%
\pgfpathlineto{\pgfqpoint{3.574822in}{2.762159in}}%
\pgfpathquadraticcurveto{\pgfqpoint{3.547044in}{2.762159in}}{\pgfqpoint{3.547044in}{2.734381in}}%
\pgfpathlineto{\pgfqpoint{3.547044in}{2.334723in}}%
\pgfpathquadraticcurveto{\pgfqpoint{3.547044in}{2.306945in}}{\pgfqpoint{3.574822in}{2.306945in}}%
\pgfpathlineto{\pgfqpoint{3.574822in}{2.306945in}}%
\pgfpathclose%
\pgfusepath{stroke,fill}%
\end{pgfscope}%
\begin{pgfscope}%
\pgfsetrectcap%
\pgfsetroundjoin%
\pgfsetlinewidth{1.505625pt}%
\definecolor{currentstroke}{rgb}{0.000000,0.000000,0.000000}%
\pgfsetstrokecolor{currentstroke}%
\pgfsetdash{}{0pt}%
\pgfpathmoveto{\pgfqpoint{3.602600in}{2.649691in}}%
\pgfpathlineto{\pgfqpoint{3.741488in}{2.649691in}}%
\pgfpathlineto{\pgfqpoint{3.880377in}{2.649691in}}%
\pgfusepath{stroke}%
\end{pgfscope}%
\begin{pgfscope}%
\definecolor{textcolor}{rgb}{0.000000,0.000000,0.000000}%
\pgfsetstrokecolor{textcolor}%
\pgfsetfillcolor{textcolor}%
\pgftext[x=3.991488in,y=2.601080in,left,base]{\color{textcolor}\rmfamily\fontsize{10.000000}{12.000000}\selectfont Sum of cosines}%
\end{pgfscope}%
\begin{pgfscope}%
\pgfsetbuttcap%
\pgfsetroundjoin%
\pgfsetlinewidth{1.505625pt}%
\definecolor{currentstroke}{rgb}{1.000000,0.000000,0.000000}%
\pgfsetstrokecolor{currentstroke}%
\pgfsetdash{{5.550000pt}{2.400000pt}}{0.000000pt}%
\pgfpathmoveto{\pgfqpoint{3.602600in}{2.445834in}}%
\pgfpathlineto{\pgfqpoint{3.741488in}{2.445834in}}%
\pgfpathlineto{\pgfqpoint{3.880377in}{2.445834in}}%
\pgfusepath{stroke}%
\end{pgfscope}%
\begin{pgfscope}%
\definecolor{textcolor}{rgb}{0.000000,0.000000,0.000000}%
\pgfsetstrokecolor{textcolor}%
\pgfsetfillcolor{textcolor}%
\pgftext[x=3.991488in,y=2.397223in,left,base]{\color{textcolor}\rmfamily\fontsize{10.000000}{12.000000}\selectfont \(\displaystyle sinc(t)\)}%
\end{pgfscope}%
\end{pgfpicture}%
\makeatother%
\endgroup%

	\caption{Die sinc Funktion angenähert durch eine Summe von Kosiunus Funktionen.}
	\label{fig:sincAsSum}
\end{figure}

\todo{TODO}
\section{Rauschen}
\todo{TODO}
\section{Polynome}

\important{
	Ein Ausdruck der Form
	\begin{equation}
		P_n(x) = a_n x^n + a_{n-1}x^{n-1}+...+a_1x + a_0 \label{eq:defPolynomial}
	\end{equation}
mit reellen Koeffizienten $a_k, k = 0,1...n$ und $a_n \neq 0$ wird Polynom n-ten Grades genannt.
}
\index{Nullstellen}
\index{Polstellen}

Für die Tontechnik und Signalverarbeitung gibt es hier viele Anwendungen, seien es Filter, Verzerrungen etc. Oft müssen wir \emph{Nullstellen} (Engl. 'roots') oder \emph{Polstellen} einer Funktion finden. 

\subsection{Nullstellen von quadratischen Gleichungen}

Eine Gleichung der Form $x^2 + p \cdot x + q = 0$ kann mit der \emph{p-q-Formel}, Gl. \ref{eq:pq-formel}, gelöst werden.

% \label{eq:pq-formel}

\begin{equation}
x_{1,2} = -\frac{p}{2} \pm \sqrt{\left( \frac{p}{2} \right)^2 - q } \label{eq:pq-formel}
\end{equation}

\subsection{Der allgemeine Fall: Nullstellen von Polynomen beliebigen Grades}
Es gibt keine allgemeingültigen Lösungsformeln für Polynome beliebigen Grades, aber es gibt einiges über diese Situation zu sagen.

\important{ \textbf{Fundamentalsatz der Algebra}: \\
Werden die Nullstellen mit der richtigen Vielfachheit (wiederholte Nullstellen) gezählt so gilt: \emph{\textbf{Ein Polynom vom Grad $n$ hat $n$ komplexe Nullstellen.}} }
\todo{Proof?}


Zunächst sei ganz pragmatisch darauf verwiesen, dass diverse Computersysteme sehr gut in dieser Situation funktionieren, siehe zB. Abschnitt \ref{sec:python}. \\
Bei hochgradingen Polynomen ist es sicherlich ein guter Schritt sich den Funktionsverlauf zu plotten und den Verlauf anzusehen. Durchaus lassen sich hier Nullstellen 'raten', die dann als Linearfaktor abgespalten werden können,siehe Abschnitt \ref{sec:linearfaktoren}. \emph{Numerische Verfahren} wie \emph{Newton-Raphson} sind hier auch äußerst wichtig, speziell in der Audio Technik. Ein Beispiel für die numerische Suche nach Nullstellen im Kontext der Klangprozessierung wäre die Simulation von nicht-linearen Schaltkreisen ('Diode clipper'), siehe zB. \cite{holmes2015improving}. \\

Da man in der Gestaltung von Filtern, Hall Algorithmen etc oft die Notwendigkeit hat Funktionen zu Gestalten und deren Nullstellen an bestimmte Orte zu verschieben ist es nicht ausreichend, nur ein Polynom einem Computer zuzuführen um die Nullstellen 'ausgespuckt' zu bekommen. Es ist oft notwendig Systeme umzuformulieren um 'Zugriff' auf die Orte der Nullstellen zu haben, zb. durch Zerlegung in Linearfaktoren. 
\todo[inline]{Rational root theorem? https://en.wikipedia.org/wiki/Rational\_root\_theorem}


\subsection{Gebrochen Rationale Ausdrücke}\label{subsec:rationalExp}
\index{Filter}
\index{Biquad}
Wenn wir ein wenig über die Hintergründe von digitalen Filter recherchieren stoßen wir schnell auf 
\cite{FILTERSWEB07}\footnote{\href{http://ccrma.stanford.edu/\~jos/filters/BiQuad\_Section.html}{http://ccrma.stanford.edu/\~jos/filters/BiQuad\_Section.html}}. Eines der Standardwerke wenn es um digitale Signalverarbeitung im musikalischen Kontext geht. Hier finden wir einen Filter der sich 'Biquad'\footnote{Der Name 'Biquad' deutet darauf hin dass hier 2 ('Bi') quadratische Gleichungen vorliegen.} nennt und durch Gleichung \ref{eq:biquad} beschrieben wird.

% 
	% H(z) \eqsp g\frac{1 + b_1 z^{-1}+ b_2 z^{-2}}{1 + a_1 z^{-1}+ a_2 z^{-2}}. \label{eq:biquad}

\begin{equation}
H(z) = g\frac{1 + b_1 z^{-1}+ b_2 z^{-2}}{1 + a_1 z^{-1}+ a_2 z^{-2}}. \label{eq:biquad}
\end{equation}

Wenngleich die Bedeutung, Herkunft, Anwendung etc dieser Funktion für den Moment rätselhaft bleiben soll ist 
\begin{itemize}
\item erstens festzuhalten dass dieser Filter (und damit diesee Gleichung) einer der verbreitetsten Filter überhaupt ist und 
\item zweitens es sich hier um eine Gleichung handelt die zwei Polynome in Beziehung setzt, sie dividiert. \\
\end{itemize}

\faust{Einfacher Tiefpass Filter, 'one pole'. Im Beispiel kann der Ort der Pol-Stelle und damit die Grenzfrequenz des Filters Manipuliert werden.}{https://faustide.grame.fr/?autorun=1&voices=0&name=filterEx&inline=aW1wb3J0KCJzdGRmYXVzdC5saWIiKTsKcCA9IGhzbGlkZXIoInBvbGUiLCAwLjksIDAuNywgMC45OTksIDAuMDAwMSk6c2kuc21vbzsKcHJvY2VzcyA9IG5vLm5vaXNlOmZpLnBvbGUocCk6XyowLjAyPDpfLF87}


\important{
Allgemein wird der Quotient zweier Polynome als \textbf{gebrochen Rationaler Ausdruck} bezeichnet, siehe Gleichung \ref{eq:rationaleFunktion}.


\begin{equation}
R(x) = \frac{Q_m(x)}{P_n(x)} = \frac{b_mx^m + b_{m-1}x^{m-1} + ... +b_1x + b_0}{a_nx^n + a_{n-1}x^{n-1} + ... +a_1x + a_0} \label{eq:rationaleFunktion}
\end{equation}


Es gilt hier für alles weitere eine wichtige Unterscheidung zu machen: Ein solcher Ausdruck heißt \textbf{echt gebrochen} \emph{nur wenn} der Grad des Nenner-Polynoms größer ist als die des Zähler-Polynoms: $n>m$. Ansonsten gilt dieser als \textbf{unecht gebrochen}.
}


Betrachten wir nun die vielleicht einfachste Gleichung solcher Art\footnote{Ja, genau genommen ist $f(x) = x$ oder sogar $f(x) = 0$ eine Gleichung 'solcher Art' im Sinne der Definition in Gleichung \ref{eq:rationaleFunktion}. So gesehen, betrachten wir nun also den vielleicht einfachsten 'echt gebrochenrationalen' Ausdruck.}:
\begin{equation}
	f(x) = x^{-1} = \frac{1}{x} \label{eq:simpleNegExp}
\end{equation}

Der Funktionsverlauf des Ausdrucks in Gleichung \ref{eq:simpleNegExp} kann in Abbildung \ref{fig:ratio1} betrachtet werden. Wie beim Betrachten der Gleichung und der Abbildung vermutet werden kann gibt es ein 'Problem' an der Stelle $x=0$. Man sagt es liegt eine sogenannte \textbf{Polstelle} oder ein \textbf{Pol} bei $x=0$. Hier ist die Funktion undefiniert. 

Zurückgreifend auf die Definition aus Gleichung \ref{eq:rationaleFunktion} lässt sich allgemeiner sagen dass der Definitionsbereich $D$ von $R$ durch die Menge an Zahlen beschnitten wird an denen das Nenner-Polynom $P_n$ den Wert 0 annimmt: $D(f) = \mathbb{R} \setminus \{x | P_n(x)=0\}$. 

\praxis{Speziell in der Analyse von Filtern interessieren wir uns für Nullstellen und Polstellen. Nullstellen sind hier schlicht die Nullstellen des Zähler-Polynoms. Polstellen sind die Nullstellen des Nenner-Polynoms.}

\begin{figure}[H]
	\centering
	%% Creator: Matplotlib, PGF backend
%%
%% To include the figure in your LaTeX document, write
%%   \input{<filename>.pgf}
%%
%% Make sure the required packages are loaded in your preamble
%%   \usepackage{pgf}
%%
%% Also ensure that all the required font packages are loaded; for instance,
%% the lmodern package is sometimes necessary when using math font.
%%   \usepackage{lmodern}
%%
%% Figures using additional raster images can only be included by \input if
%% they are in the same directory as the main LaTeX file. For loading figures
%% from other directories you can use the `import` package
%%   \usepackage{import}
%%
%% and then include the figures with
%%   \import{<path to file>}{<filename>.pgf}
%%
%% Matplotlib used the following preamble
%%   \def\mathdefault#1{#1}
%%   \everymath=\expandafter{\the\everymath\displaystyle}
%%   
%%   \usepackage{fontspec}
%%   \setmainfont{VeraSe.ttf}[Path=\detokenize{/usr/share/fonts/TTF/}]
%%   \setsansfont{DejaVuSans.ttf}[Path=\detokenize{/home/pl/miniconda3/lib/python3.12/site-packages/matplotlib/mpl-data/fonts/ttf/}]
%%   \setmonofont{DejaVuSansMono.ttf}[Path=\detokenize{/home/pl/miniconda3/lib/python3.12/site-packages/matplotlib/mpl-data/fonts/ttf/}]
%%   \makeatletter\@ifpackageloaded{underscore}{}{\usepackage[strings]{underscore}}\makeatother
%%
\begingroup%
\makeatletter%
\begin{pgfpicture}%
\pgfpathrectangle{\pgfpointorigin}{\pgfqpoint{5.365835in}{3.011248in}}%
\pgfusepath{use as bounding box, clip}%
\begin{pgfscope}%
\pgfsetbuttcap%
\pgfsetmiterjoin%
\definecolor{currentfill}{rgb}{1.000000,1.000000,1.000000}%
\pgfsetfillcolor{currentfill}%
\pgfsetlinewidth{0.000000pt}%
\definecolor{currentstroke}{rgb}{1.000000,1.000000,1.000000}%
\pgfsetstrokecolor{currentstroke}%
\pgfsetdash{}{0pt}%
\pgfpathmoveto{\pgfqpoint{0.000000in}{0.000000in}}%
\pgfpathlineto{\pgfqpoint{5.365835in}{0.000000in}}%
\pgfpathlineto{\pgfqpoint{5.365835in}{3.011248in}}%
\pgfpathlineto{\pgfqpoint{0.000000in}{3.011248in}}%
\pgfpathlineto{\pgfqpoint{0.000000in}{0.000000in}}%
\pgfpathclose%
\pgfusepath{fill}%
\end{pgfscope}%
\begin{pgfscope}%
\pgfsetbuttcap%
\pgfsetmiterjoin%
\definecolor{currentfill}{rgb}{1.000000,1.000000,1.000000}%
\pgfsetfillcolor{currentfill}%
\pgfsetlinewidth{0.000000pt}%
\definecolor{currentstroke}{rgb}{0.000000,0.000000,0.000000}%
\pgfsetstrokecolor{currentstroke}%
\pgfsetstrokeopacity{0.000000}%
\pgfsetdash{}{0pt}%
\pgfpathmoveto{\pgfqpoint{0.615835in}{0.548486in}}%
\pgfpathlineto{\pgfqpoint{2.729471in}{0.548486in}}%
\pgfpathlineto{\pgfqpoint{2.729471in}{2.858486in}}%
\pgfpathlineto{\pgfqpoint{0.615835in}{2.858486in}}%
\pgfpathlineto{\pgfqpoint{0.615835in}{0.548486in}}%
\pgfpathclose%
\pgfusepath{fill}%
\end{pgfscope}%
\begin{pgfscope}%
\pgfsetbuttcap%
\pgfsetroundjoin%
\definecolor{currentfill}{rgb}{0.000000,0.000000,0.000000}%
\pgfsetfillcolor{currentfill}%
\pgfsetlinewidth{0.803000pt}%
\definecolor{currentstroke}{rgb}{0.000000,0.000000,0.000000}%
\pgfsetstrokecolor{currentstroke}%
\pgfsetdash{}{0pt}%
\pgfsys@defobject{currentmarker}{\pgfqpoint{0.000000in}{-0.048611in}}{\pgfqpoint{0.000000in}{0.000000in}}{%
\pgfpathmoveto{\pgfqpoint{0.000000in}{0.000000in}}%
\pgfpathlineto{\pgfqpoint{0.000000in}{-0.048611in}}%
\pgfusepath{stroke,fill}%
}%
\begin{pgfscope}%
\pgfsys@transformshift{0.711909in}{0.548486in}%
\pgfsys@useobject{currentmarker}{}%
\end{pgfscope}%
\end{pgfscope}%
\begin{pgfscope}%
\definecolor{textcolor}{rgb}{0.000000,0.000000,0.000000}%
\pgfsetstrokecolor{textcolor}%
\pgfsetfillcolor{textcolor}%
\pgftext[x=0.711909in,y=0.451264in,,top]{\color{textcolor}{\rmfamily\fontsize{10.000000}{12.000000}\selectfont\catcode`\^=\active\def^{\ifmmode\sp\else\^{}\fi}\catcode`\%=\active\def%{\%}\ensuremath{-}10}}%
\end{pgfscope}%
\begin{pgfscope}%
\pgfsetbuttcap%
\pgfsetroundjoin%
\definecolor{currentfill}{rgb}{0.000000,0.000000,0.000000}%
\pgfsetfillcolor{currentfill}%
\pgfsetlinewidth{0.803000pt}%
\definecolor{currentstroke}{rgb}{0.000000,0.000000,0.000000}%
\pgfsetstrokecolor{currentstroke}%
\pgfsetdash{}{0pt}%
\pgfsys@defobject{currentmarker}{\pgfqpoint{0.000000in}{-0.048611in}}{\pgfqpoint{0.000000in}{0.000000in}}{%
\pgfpathmoveto{\pgfqpoint{0.000000in}{0.000000in}}%
\pgfpathlineto{\pgfqpoint{0.000000in}{-0.048611in}}%
\pgfusepath{stroke,fill}%
}%
\begin{pgfscope}%
\pgfsys@transformshift{1.192281in}{0.548486in}%
\pgfsys@useobject{currentmarker}{}%
\end{pgfscope}%
\end{pgfscope}%
\begin{pgfscope}%
\definecolor{textcolor}{rgb}{0.000000,0.000000,0.000000}%
\pgfsetstrokecolor{textcolor}%
\pgfsetfillcolor{textcolor}%
\pgftext[x=1.192281in,y=0.451264in,,top]{\color{textcolor}{\rmfamily\fontsize{10.000000}{12.000000}\selectfont\catcode`\^=\active\def^{\ifmmode\sp\else\^{}\fi}\catcode`\%=\active\def%{\%}\ensuremath{-}5}}%
\end{pgfscope}%
\begin{pgfscope}%
\pgfsetbuttcap%
\pgfsetroundjoin%
\definecolor{currentfill}{rgb}{0.000000,0.000000,0.000000}%
\pgfsetfillcolor{currentfill}%
\pgfsetlinewidth{0.803000pt}%
\definecolor{currentstroke}{rgb}{0.000000,0.000000,0.000000}%
\pgfsetstrokecolor{currentstroke}%
\pgfsetdash{}{0pt}%
\pgfsys@defobject{currentmarker}{\pgfqpoint{0.000000in}{-0.048611in}}{\pgfqpoint{0.000000in}{0.000000in}}{%
\pgfpathmoveto{\pgfqpoint{0.000000in}{0.000000in}}%
\pgfpathlineto{\pgfqpoint{0.000000in}{-0.048611in}}%
\pgfusepath{stroke,fill}%
}%
\begin{pgfscope}%
\pgfsys@transformshift{1.672653in}{0.548486in}%
\pgfsys@useobject{currentmarker}{}%
\end{pgfscope}%
\end{pgfscope}%
\begin{pgfscope}%
\definecolor{textcolor}{rgb}{0.000000,0.000000,0.000000}%
\pgfsetstrokecolor{textcolor}%
\pgfsetfillcolor{textcolor}%
\pgftext[x=1.672653in,y=0.451264in,,top]{\color{textcolor}{\rmfamily\fontsize{10.000000}{12.000000}\selectfont\catcode`\^=\active\def^{\ifmmode\sp\else\^{}\fi}\catcode`\%=\active\def%{\%}0}}%
\end{pgfscope}%
\begin{pgfscope}%
\pgfsetbuttcap%
\pgfsetroundjoin%
\definecolor{currentfill}{rgb}{0.000000,0.000000,0.000000}%
\pgfsetfillcolor{currentfill}%
\pgfsetlinewidth{0.803000pt}%
\definecolor{currentstroke}{rgb}{0.000000,0.000000,0.000000}%
\pgfsetstrokecolor{currentstroke}%
\pgfsetdash{}{0pt}%
\pgfsys@defobject{currentmarker}{\pgfqpoint{0.000000in}{-0.048611in}}{\pgfqpoint{0.000000in}{0.000000in}}{%
\pgfpathmoveto{\pgfqpoint{0.000000in}{0.000000in}}%
\pgfpathlineto{\pgfqpoint{0.000000in}{-0.048611in}}%
\pgfusepath{stroke,fill}%
}%
\begin{pgfscope}%
\pgfsys@transformshift{2.153025in}{0.548486in}%
\pgfsys@useobject{currentmarker}{}%
\end{pgfscope}%
\end{pgfscope}%
\begin{pgfscope}%
\definecolor{textcolor}{rgb}{0.000000,0.000000,0.000000}%
\pgfsetstrokecolor{textcolor}%
\pgfsetfillcolor{textcolor}%
\pgftext[x=2.153025in,y=0.451264in,,top]{\color{textcolor}{\rmfamily\fontsize{10.000000}{12.000000}\selectfont\catcode`\^=\active\def^{\ifmmode\sp\else\^{}\fi}\catcode`\%=\active\def%{\%}5}}%
\end{pgfscope}%
\begin{pgfscope}%
\pgfsetbuttcap%
\pgfsetroundjoin%
\definecolor{currentfill}{rgb}{0.000000,0.000000,0.000000}%
\pgfsetfillcolor{currentfill}%
\pgfsetlinewidth{0.803000pt}%
\definecolor{currentstroke}{rgb}{0.000000,0.000000,0.000000}%
\pgfsetstrokecolor{currentstroke}%
\pgfsetdash{}{0pt}%
\pgfsys@defobject{currentmarker}{\pgfqpoint{0.000000in}{-0.048611in}}{\pgfqpoint{0.000000in}{0.000000in}}{%
\pgfpathmoveto{\pgfqpoint{0.000000in}{0.000000in}}%
\pgfpathlineto{\pgfqpoint{0.000000in}{-0.048611in}}%
\pgfusepath{stroke,fill}%
}%
\begin{pgfscope}%
\pgfsys@transformshift{2.633397in}{0.548486in}%
\pgfsys@useobject{currentmarker}{}%
\end{pgfscope}%
\end{pgfscope}%
\begin{pgfscope}%
\definecolor{textcolor}{rgb}{0.000000,0.000000,0.000000}%
\pgfsetstrokecolor{textcolor}%
\pgfsetfillcolor{textcolor}%
\pgftext[x=2.633397in,y=0.451264in,,top]{\color{textcolor}{\rmfamily\fontsize{10.000000}{12.000000}\selectfont\catcode`\^=\active\def^{\ifmmode\sp\else\^{}\fi}\catcode`\%=\active\def%{\%}10}}%
\end{pgfscope}%
\begin{pgfscope}%
\definecolor{textcolor}{rgb}{0.000000,0.000000,0.000000}%
\pgfsetstrokecolor{textcolor}%
\pgfsetfillcolor{textcolor}%
\pgftext[x=1.672653in,y=0.261295in,,top]{\color{textcolor}{\rmfamily\fontsize{12.000000}{14.400000}\selectfont\catcode`\^=\active\def^{\ifmmode\sp\else\^{}\fi}\catcode`\%=\active\def%{\%}$x$}}%
\end{pgfscope}%
\begin{pgfscope}%
\pgfsetbuttcap%
\pgfsetroundjoin%
\definecolor{currentfill}{rgb}{0.000000,0.000000,0.000000}%
\pgfsetfillcolor{currentfill}%
\pgfsetlinewidth{0.803000pt}%
\definecolor{currentstroke}{rgb}{0.000000,0.000000,0.000000}%
\pgfsetstrokecolor{currentstroke}%
\pgfsetdash{}{0pt}%
\pgfsys@defobject{currentmarker}{\pgfqpoint{-0.048611in}{0.000000in}}{\pgfqpoint{-0.000000in}{0.000000in}}{%
\pgfpathmoveto{\pgfqpoint{-0.000000in}{0.000000in}}%
\pgfpathlineto{\pgfqpoint{-0.048611in}{0.000000in}}%
\pgfusepath{stroke,fill}%
}%
\begin{pgfscope}%
\pgfsys@transformshift{0.615835in}{0.548486in}%
\pgfsys@useobject{currentmarker}{}%
\end{pgfscope}%
\end{pgfscope}%
\begin{pgfscope}%
\definecolor{textcolor}{rgb}{0.000000,0.000000,0.000000}%
\pgfsetstrokecolor{textcolor}%
\pgfsetfillcolor{textcolor}%
\pgftext[x=0.322222in, y=0.495724in, left, base]{\color{textcolor}{\rmfamily\fontsize{10.000000}{12.000000}\selectfont\catcode`\^=\active\def^{\ifmmode\sp\else\^{}\fi}\catcode`\%=\active\def%{\%}\ensuremath{-}3}}%
\end{pgfscope}%
\begin{pgfscope}%
\pgfsetbuttcap%
\pgfsetroundjoin%
\definecolor{currentfill}{rgb}{0.000000,0.000000,0.000000}%
\pgfsetfillcolor{currentfill}%
\pgfsetlinewidth{0.803000pt}%
\definecolor{currentstroke}{rgb}{0.000000,0.000000,0.000000}%
\pgfsetstrokecolor{currentstroke}%
\pgfsetdash{}{0pt}%
\pgfsys@defobject{currentmarker}{\pgfqpoint{-0.048611in}{0.000000in}}{\pgfqpoint{-0.000000in}{0.000000in}}{%
\pgfpathmoveto{\pgfqpoint{-0.000000in}{0.000000in}}%
\pgfpathlineto{\pgfqpoint{-0.048611in}{0.000000in}}%
\pgfusepath{stroke,fill}%
}%
\begin{pgfscope}%
\pgfsys@transformshift{0.615835in}{0.933486in}%
\pgfsys@useobject{currentmarker}{}%
\end{pgfscope}%
\end{pgfscope}%
\begin{pgfscope}%
\definecolor{textcolor}{rgb}{0.000000,0.000000,0.000000}%
\pgfsetstrokecolor{textcolor}%
\pgfsetfillcolor{textcolor}%
\pgftext[x=0.322222in, y=0.880724in, left, base]{\color{textcolor}{\rmfamily\fontsize{10.000000}{12.000000}\selectfont\catcode`\^=\active\def^{\ifmmode\sp\else\^{}\fi}\catcode`\%=\active\def%{\%}\ensuremath{-}2}}%
\end{pgfscope}%
\begin{pgfscope}%
\pgfsetbuttcap%
\pgfsetroundjoin%
\definecolor{currentfill}{rgb}{0.000000,0.000000,0.000000}%
\pgfsetfillcolor{currentfill}%
\pgfsetlinewidth{0.803000pt}%
\definecolor{currentstroke}{rgb}{0.000000,0.000000,0.000000}%
\pgfsetstrokecolor{currentstroke}%
\pgfsetdash{}{0pt}%
\pgfsys@defobject{currentmarker}{\pgfqpoint{-0.048611in}{0.000000in}}{\pgfqpoint{-0.000000in}{0.000000in}}{%
\pgfpathmoveto{\pgfqpoint{-0.000000in}{0.000000in}}%
\pgfpathlineto{\pgfqpoint{-0.048611in}{0.000000in}}%
\pgfusepath{stroke,fill}%
}%
\begin{pgfscope}%
\pgfsys@transformshift{0.615835in}{1.318486in}%
\pgfsys@useobject{currentmarker}{}%
\end{pgfscope}%
\end{pgfscope}%
\begin{pgfscope}%
\definecolor{textcolor}{rgb}{0.000000,0.000000,0.000000}%
\pgfsetstrokecolor{textcolor}%
\pgfsetfillcolor{textcolor}%
\pgftext[x=0.322222in, y=1.265724in, left, base]{\color{textcolor}{\rmfamily\fontsize{10.000000}{12.000000}\selectfont\catcode`\^=\active\def^{\ifmmode\sp\else\^{}\fi}\catcode`\%=\active\def%{\%}\ensuremath{-}1}}%
\end{pgfscope}%
\begin{pgfscope}%
\pgfsetbuttcap%
\pgfsetroundjoin%
\definecolor{currentfill}{rgb}{0.000000,0.000000,0.000000}%
\pgfsetfillcolor{currentfill}%
\pgfsetlinewidth{0.803000pt}%
\definecolor{currentstroke}{rgb}{0.000000,0.000000,0.000000}%
\pgfsetstrokecolor{currentstroke}%
\pgfsetdash{}{0pt}%
\pgfsys@defobject{currentmarker}{\pgfqpoint{-0.048611in}{0.000000in}}{\pgfqpoint{-0.000000in}{0.000000in}}{%
\pgfpathmoveto{\pgfqpoint{-0.000000in}{0.000000in}}%
\pgfpathlineto{\pgfqpoint{-0.048611in}{0.000000in}}%
\pgfusepath{stroke,fill}%
}%
\begin{pgfscope}%
\pgfsys@transformshift{0.615835in}{1.703486in}%
\pgfsys@useobject{currentmarker}{}%
\end{pgfscope}%
\end{pgfscope}%
\begin{pgfscope}%
\definecolor{textcolor}{rgb}{0.000000,0.000000,0.000000}%
\pgfsetstrokecolor{textcolor}%
\pgfsetfillcolor{textcolor}%
\pgftext[x=0.430247in, y=1.650724in, left, base]{\color{textcolor}{\rmfamily\fontsize{10.000000}{12.000000}\selectfont\catcode`\^=\active\def^{\ifmmode\sp\else\^{}\fi}\catcode`\%=\active\def%{\%}0}}%
\end{pgfscope}%
\begin{pgfscope}%
\pgfsetbuttcap%
\pgfsetroundjoin%
\definecolor{currentfill}{rgb}{0.000000,0.000000,0.000000}%
\pgfsetfillcolor{currentfill}%
\pgfsetlinewidth{0.803000pt}%
\definecolor{currentstroke}{rgb}{0.000000,0.000000,0.000000}%
\pgfsetstrokecolor{currentstroke}%
\pgfsetdash{}{0pt}%
\pgfsys@defobject{currentmarker}{\pgfqpoint{-0.048611in}{0.000000in}}{\pgfqpoint{-0.000000in}{0.000000in}}{%
\pgfpathmoveto{\pgfqpoint{-0.000000in}{0.000000in}}%
\pgfpathlineto{\pgfqpoint{-0.048611in}{0.000000in}}%
\pgfusepath{stroke,fill}%
}%
\begin{pgfscope}%
\pgfsys@transformshift{0.615835in}{2.088486in}%
\pgfsys@useobject{currentmarker}{}%
\end{pgfscope}%
\end{pgfscope}%
\begin{pgfscope}%
\definecolor{textcolor}{rgb}{0.000000,0.000000,0.000000}%
\pgfsetstrokecolor{textcolor}%
\pgfsetfillcolor{textcolor}%
\pgftext[x=0.430247in, y=2.035724in, left, base]{\color{textcolor}{\rmfamily\fontsize{10.000000}{12.000000}\selectfont\catcode`\^=\active\def^{\ifmmode\sp\else\^{}\fi}\catcode`\%=\active\def%{\%}1}}%
\end{pgfscope}%
\begin{pgfscope}%
\pgfsetbuttcap%
\pgfsetroundjoin%
\definecolor{currentfill}{rgb}{0.000000,0.000000,0.000000}%
\pgfsetfillcolor{currentfill}%
\pgfsetlinewidth{0.803000pt}%
\definecolor{currentstroke}{rgb}{0.000000,0.000000,0.000000}%
\pgfsetstrokecolor{currentstroke}%
\pgfsetdash{}{0pt}%
\pgfsys@defobject{currentmarker}{\pgfqpoint{-0.048611in}{0.000000in}}{\pgfqpoint{-0.000000in}{0.000000in}}{%
\pgfpathmoveto{\pgfqpoint{-0.000000in}{0.000000in}}%
\pgfpathlineto{\pgfqpoint{-0.048611in}{0.000000in}}%
\pgfusepath{stroke,fill}%
}%
\begin{pgfscope}%
\pgfsys@transformshift{0.615835in}{2.473486in}%
\pgfsys@useobject{currentmarker}{}%
\end{pgfscope}%
\end{pgfscope}%
\begin{pgfscope}%
\definecolor{textcolor}{rgb}{0.000000,0.000000,0.000000}%
\pgfsetstrokecolor{textcolor}%
\pgfsetfillcolor{textcolor}%
\pgftext[x=0.430247in, y=2.420724in, left, base]{\color{textcolor}{\rmfamily\fontsize{10.000000}{12.000000}\selectfont\catcode`\^=\active\def^{\ifmmode\sp\else\^{}\fi}\catcode`\%=\active\def%{\%}2}}%
\end{pgfscope}%
\begin{pgfscope}%
\pgfsetbuttcap%
\pgfsetroundjoin%
\definecolor{currentfill}{rgb}{0.000000,0.000000,0.000000}%
\pgfsetfillcolor{currentfill}%
\pgfsetlinewidth{0.803000pt}%
\definecolor{currentstroke}{rgb}{0.000000,0.000000,0.000000}%
\pgfsetstrokecolor{currentstroke}%
\pgfsetdash{}{0pt}%
\pgfsys@defobject{currentmarker}{\pgfqpoint{-0.048611in}{0.000000in}}{\pgfqpoint{-0.000000in}{0.000000in}}{%
\pgfpathmoveto{\pgfqpoint{-0.000000in}{0.000000in}}%
\pgfpathlineto{\pgfqpoint{-0.048611in}{0.000000in}}%
\pgfusepath{stroke,fill}%
}%
\begin{pgfscope}%
\pgfsys@transformshift{0.615835in}{2.858486in}%
\pgfsys@useobject{currentmarker}{}%
\end{pgfscope}%
\end{pgfscope}%
\begin{pgfscope}%
\definecolor{textcolor}{rgb}{0.000000,0.000000,0.000000}%
\pgfsetstrokecolor{textcolor}%
\pgfsetfillcolor{textcolor}%
\pgftext[x=0.430247in, y=2.805724in, left, base]{\color{textcolor}{\rmfamily\fontsize{10.000000}{12.000000}\selectfont\catcode`\^=\active\def^{\ifmmode\sp\else\^{}\fi}\catcode`\%=\active\def%{\%}3}}%
\end{pgfscope}%
\begin{pgfscope}%
\definecolor{textcolor}{rgb}{0.000000,0.000000,0.000000}%
\pgfsetstrokecolor{textcolor}%
\pgfsetfillcolor{textcolor}%
\pgftext[x=0.266667in,y=1.703486in,,bottom,rotate=90.000000]{\color{textcolor}{\rmfamily\fontsize{12.000000}{14.400000}\selectfont\catcode`\^=\active\def^{\ifmmode\sp\else\^{}\fi}\catcode`\%=\active\def%{\%}$f(x)$}}%
\end{pgfscope}%
\begin{pgfscope}%
\pgfpathrectangle{\pgfqpoint{0.615835in}{0.548486in}}{\pgfqpoint{2.113636in}{2.310000in}}%
\pgfusepath{clip}%
\pgfsetrectcap%
\pgfsetroundjoin%
\pgfsetlinewidth{1.505625pt}%
\definecolor{currentstroke}{rgb}{0.000000,0.000000,0.000000}%
\pgfsetstrokecolor{currentstroke}%
\pgfsetdash{}{0pt}%
\pgfpathmoveto{\pgfqpoint{0.711909in}{1.664986in}}%
\pgfpathlineto{\pgfqpoint{0.861920in}{1.657862in}}%
\pgfpathlineto{\pgfqpoint{0.979236in}{1.650143in}}%
\pgfpathlineto{\pgfqpoint{1.072513in}{1.641853in}}%
\pgfpathlineto{\pgfqpoint{1.147518in}{1.633050in}}%
\pgfpathlineto{\pgfqpoint{1.210023in}{1.623533in}}%
\pgfpathlineto{\pgfqpoint{1.261950in}{1.613424in}}%
\pgfpathlineto{\pgfqpoint{1.305222in}{1.602818in}}%
\pgfpathlineto{\pgfqpoint{1.342725in}{1.591375in}}%
\pgfpathlineto{\pgfqpoint{1.374458in}{1.579444in}}%
\pgfpathlineto{\pgfqpoint{1.402345in}{1.566647in}}%
\pgfpathlineto{\pgfqpoint{1.427346in}{1.552701in}}%
\pgfpathlineto{\pgfqpoint{1.449463in}{1.537758in}}%
\pgfpathlineto{\pgfqpoint{1.468696in}{1.522131in}}%
\pgfpathlineto{\pgfqpoint{1.486005in}{1.505313in}}%
\pgfpathlineto{\pgfqpoint{1.502352in}{1.486290in}}%
\pgfpathlineto{\pgfqpoint{1.516776in}{1.466192in}}%
\pgfpathlineto{\pgfqpoint{1.530239in}{1.443760in}}%
\pgfpathlineto{\pgfqpoint{1.542740in}{1.418768in}}%
\pgfpathlineto{\pgfqpoint{1.554279in}{1.391013in}}%
\pgfpathlineto{\pgfqpoint{1.565818in}{1.357263in}}%
\pgfpathlineto{\pgfqpoint{1.576396in}{1.319216in}}%
\pgfpathlineto{\pgfqpoint{1.586012in}{1.276567in}}%
\pgfpathlineto{\pgfqpoint{1.595628in}{1.223269in}}%
\pgfpathlineto{\pgfqpoint{1.604283in}{1.162482in}}%
\pgfpathlineto{\pgfqpoint{1.612937in}{1.084075in}}%
\pgfpathlineto{\pgfqpoint{1.620630in}{0.992480in}}%
\pgfpathlineto{\pgfqpoint{1.627361in}{0.886810in}}%
\pgfpathlineto{\pgfqpoint{1.634092in}{0.744248in}}%
\pgfpathlineto{\pgfqpoint{1.639862in}{0.575467in}}%
\pgfpathlineto{\pgfqpoint{1.640901in}{0.538486in}}%
\pgfpathlineto{\pgfqpoint{1.640901in}{0.538486in}}%
\pgfusepath{stroke}%
\end{pgfscope}%
\begin{pgfscope}%
\pgfpathrectangle{\pgfqpoint{0.615835in}{0.548486in}}{\pgfqpoint{2.113636in}{2.310000in}}%
\pgfusepath{clip}%
\pgfsetrectcap%
\pgfsetroundjoin%
\pgfsetlinewidth{1.505625pt}%
\definecolor{currentstroke}{rgb}{0.000000,0.000000,0.000000}%
\pgfsetstrokecolor{currentstroke}%
\pgfsetdash{}{0pt}%
\pgfpathmoveto{\pgfqpoint{2.633397in}{1.741986in}}%
\pgfpathlineto{\pgfqpoint{2.483386in}{1.749110in}}%
\pgfpathlineto{\pgfqpoint{2.366069in}{1.756829in}}%
\pgfpathlineto{\pgfqpoint{2.272793in}{1.765119in}}%
\pgfpathlineto{\pgfqpoint{2.197788in}{1.773922in}}%
\pgfpathlineto{\pgfqpoint{2.135283in}{1.783439in}}%
\pgfpathlineto{\pgfqpoint{2.083356in}{1.793548in}}%
\pgfpathlineto{\pgfqpoint{2.040084in}{1.804154in}}%
\pgfpathlineto{\pgfqpoint{2.002581in}{1.815597in}}%
\pgfpathlineto{\pgfqpoint{1.970848in}{1.827528in}}%
\pgfpathlineto{\pgfqpoint{1.942961in}{1.840325in}}%
\pgfpathlineto{\pgfqpoint{1.917959in}{1.854271in}}%
\pgfpathlineto{\pgfqpoint{1.895842in}{1.869214in}}%
\pgfpathlineto{\pgfqpoint{1.876610in}{1.884841in}}%
\pgfpathlineto{\pgfqpoint{1.859301in}{1.901659in}}%
\pgfpathlineto{\pgfqpoint{1.842954in}{1.920682in}}%
\pgfpathlineto{\pgfqpoint{1.828530in}{1.940780in}}%
\pgfpathlineto{\pgfqpoint{1.815067in}{1.963212in}}%
\pgfpathlineto{\pgfqpoint{1.802566in}{1.988204in}}%
\pgfpathlineto{\pgfqpoint{1.791027in}{2.015959in}}%
\pgfpathlineto{\pgfqpoint{1.779488in}{2.049709in}}%
\pgfpathlineto{\pgfqpoint{1.768910in}{2.087756in}}%
\pgfpathlineto{\pgfqpoint{1.759294in}{2.130405in}}%
\pgfpathlineto{\pgfqpoint{1.749678in}{2.183703in}}%
\pgfpathlineto{\pgfqpoint{1.741023in}{2.244490in}}%
\pgfpathlineto{\pgfqpoint{1.732369in}{2.322897in}}%
\pgfpathlineto{\pgfqpoint{1.724676in}{2.414492in}}%
\pgfpathlineto{\pgfqpoint{1.717945in}{2.520162in}}%
\pgfpathlineto{\pgfqpoint{1.711213in}{2.662724in}}%
\pgfpathlineto{\pgfqpoint{1.705444in}{2.831505in}}%
\pgfpathlineto{\pgfqpoint{1.704405in}{2.868486in}}%
\pgfpathlineto{\pgfqpoint{1.704405in}{2.868486in}}%
\pgfusepath{stroke}%
\end{pgfscope}%
\begin{pgfscope}%
\pgfpathrectangle{\pgfqpoint{0.615835in}{0.548486in}}{\pgfqpoint{2.113636in}{2.310000in}}%
\pgfusepath{clip}%
\pgfsetrectcap%
\pgfsetroundjoin%
\pgfsetlinewidth{0.501875pt}%
\definecolor{currentstroke}{rgb}{0.000000,0.000000,0.000000}%
\pgfsetstrokecolor{currentstroke}%
\pgfsetdash{}{0pt}%
\pgfpathmoveto{\pgfqpoint{0.615835in}{1.703486in}}%
\pgfpathlineto{\pgfqpoint{2.729471in}{1.703486in}}%
\pgfusepath{stroke}%
\end{pgfscope}%
\begin{pgfscope}%
\pgfpathrectangle{\pgfqpoint{0.615835in}{0.548486in}}{\pgfqpoint{2.113636in}{2.310000in}}%
\pgfusepath{clip}%
\pgfsetrectcap%
\pgfsetroundjoin%
\pgfsetlinewidth{0.501875pt}%
\definecolor{currentstroke}{rgb}{0.000000,0.000000,0.000000}%
\pgfsetstrokecolor{currentstroke}%
\pgfsetdash{}{0pt}%
\pgfpathmoveto{\pgfqpoint{1.672653in}{0.548486in}}%
\pgfpathlineto{\pgfqpoint{1.672653in}{2.858486in}}%
\pgfusepath{stroke}%
\end{pgfscope}%
\begin{pgfscope}%
\pgfsetbuttcap%
\pgfsetmiterjoin%
\definecolor{currentfill}{rgb}{1.000000,1.000000,1.000000}%
\pgfsetfillcolor{currentfill}%
\pgfsetfillopacity{0.800000}%
\pgfsetlinewidth{1.003750pt}%
\definecolor{currentstroke}{rgb}{0.800000,0.800000,0.800000}%
\pgfsetstrokecolor{currentstroke}%
\pgfsetstrokeopacity{0.800000}%
\pgfsetdash{}{0pt}%
\pgfpathmoveto{\pgfqpoint{0.713057in}{0.617930in}}%
\pgfpathlineto{\pgfqpoint{1.841529in}{0.617930in}}%
\pgfpathquadraticcurveto{\pgfqpoint{1.869307in}{0.617930in}}{\pgfqpoint{1.869307in}{0.645708in}}%
\pgfpathlineto{\pgfqpoint{1.869307in}{0.855986in}}%
\pgfpathquadraticcurveto{\pgfqpoint{1.869307in}{0.883764in}}{\pgfqpoint{1.841529in}{0.883764in}}%
\pgfpathlineto{\pgfqpoint{0.713057in}{0.883764in}}%
\pgfpathquadraticcurveto{\pgfqpoint{0.685279in}{0.883764in}}{\pgfqpoint{0.685279in}{0.855986in}}%
\pgfpathlineto{\pgfqpoint{0.685279in}{0.645708in}}%
\pgfpathquadraticcurveto{\pgfqpoint{0.685279in}{0.617930in}}{\pgfqpoint{0.713057in}{0.617930in}}%
\pgfpathlineto{\pgfqpoint{0.713057in}{0.617930in}}%
\pgfpathclose%
\pgfusepath{stroke,fill}%
\end{pgfscope}%
\begin{pgfscope}%
\pgfsetrectcap%
\pgfsetroundjoin%
\pgfsetlinewidth{1.505625pt}%
\definecolor{currentstroke}{rgb}{0.000000,0.000000,0.000000}%
\pgfsetstrokecolor{currentstroke}%
\pgfsetdash{}{0pt}%
\pgfpathmoveto{\pgfqpoint{0.740835in}{0.756819in}}%
\pgfpathlineto{\pgfqpoint{0.879724in}{0.756819in}}%
\pgfpathlineto{\pgfqpoint{1.018613in}{0.756819in}}%
\pgfusepath{stroke}%
\end{pgfscope}%
\begin{pgfscope}%
\definecolor{textcolor}{rgb}{0.000000,0.000000,0.000000}%
\pgfsetstrokecolor{textcolor}%
\pgfsetfillcolor{textcolor}%
\pgftext[x=1.129724in,y=0.708208in,left,base]{\color{textcolor}{\rmfamily\fontsize{10.000000}{12.000000}\selectfont\catcode`\^=\active\def^{\ifmmode\sp\else\^{}\fi}\catcode`\%=\active\def%{\%}$f(x) = x^{-1}$}}%
\end{pgfscope}%
\begin{pgfscope}%
\pgfsetbuttcap%
\pgfsetmiterjoin%
\definecolor{currentfill}{rgb}{1.000000,1.000000,1.000000}%
\pgfsetfillcolor{currentfill}%
\pgfsetlinewidth{0.000000pt}%
\definecolor{currentstroke}{rgb}{0.000000,0.000000,0.000000}%
\pgfsetstrokecolor{currentstroke}%
\pgfsetstrokeopacity{0.000000}%
\pgfsetdash{}{0pt}%
\pgfpathmoveto{\pgfqpoint{3.152198in}{0.548486in}}%
\pgfpathlineto{\pgfqpoint{5.265835in}{0.548486in}}%
\pgfpathlineto{\pgfqpoint{5.265835in}{2.858486in}}%
\pgfpathlineto{\pgfqpoint{3.152198in}{2.858486in}}%
\pgfpathlineto{\pgfqpoint{3.152198in}{0.548486in}}%
\pgfpathclose%
\pgfusepath{fill}%
\end{pgfscope}%
\begin{pgfscope}%
\pgfsetbuttcap%
\pgfsetroundjoin%
\definecolor{currentfill}{rgb}{0.000000,0.000000,0.000000}%
\pgfsetfillcolor{currentfill}%
\pgfsetlinewidth{0.803000pt}%
\definecolor{currentstroke}{rgb}{0.000000,0.000000,0.000000}%
\pgfsetstrokecolor{currentstroke}%
\pgfsetdash{}{0pt}%
\pgfsys@defobject{currentmarker}{\pgfqpoint{0.000000in}{-0.048611in}}{\pgfqpoint{0.000000in}{0.000000in}}{%
\pgfpathmoveto{\pgfqpoint{0.000000in}{0.000000in}}%
\pgfpathlineto{\pgfqpoint{0.000000in}{-0.048611in}}%
\pgfusepath{stroke,fill}%
}%
\begin{pgfscope}%
\pgfsys@transformshift{3.248273in}{0.548486in}%
\pgfsys@useobject{currentmarker}{}%
\end{pgfscope}%
\end{pgfscope}%
\begin{pgfscope}%
\definecolor{textcolor}{rgb}{0.000000,0.000000,0.000000}%
\pgfsetstrokecolor{textcolor}%
\pgfsetfillcolor{textcolor}%
\pgftext[x=3.248273in,y=0.451264in,,top]{\color{textcolor}{\rmfamily\fontsize{10.000000}{12.000000}\selectfont\catcode`\^=\active\def^{\ifmmode\sp\else\^{}\fi}\catcode`\%=\active\def%{\%}\ensuremath{-}10}}%
\end{pgfscope}%
\begin{pgfscope}%
\pgfsetbuttcap%
\pgfsetroundjoin%
\definecolor{currentfill}{rgb}{0.000000,0.000000,0.000000}%
\pgfsetfillcolor{currentfill}%
\pgfsetlinewidth{0.803000pt}%
\definecolor{currentstroke}{rgb}{0.000000,0.000000,0.000000}%
\pgfsetstrokecolor{currentstroke}%
\pgfsetdash{}{0pt}%
\pgfsys@defobject{currentmarker}{\pgfqpoint{0.000000in}{-0.048611in}}{\pgfqpoint{0.000000in}{0.000000in}}{%
\pgfpathmoveto{\pgfqpoint{0.000000in}{0.000000in}}%
\pgfpathlineto{\pgfqpoint{0.000000in}{-0.048611in}}%
\pgfusepath{stroke,fill}%
}%
\begin{pgfscope}%
\pgfsys@transformshift{3.728645in}{0.548486in}%
\pgfsys@useobject{currentmarker}{}%
\end{pgfscope}%
\end{pgfscope}%
\begin{pgfscope}%
\definecolor{textcolor}{rgb}{0.000000,0.000000,0.000000}%
\pgfsetstrokecolor{textcolor}%
\pgfsetfillcolor{textcolor}%
\pgftext[x=3.728645in,y=0.451264in,,top]{\color{textcolor}{\rmfamily\fontsize{10.000000}{12.000000}\selectfont\catcode`\^=\active\def^{\ifmmode\sp\else\^{}\fi}\catcode`\%=\active\def%{\%}\ensuremath{-}5}}%
\end{pgfscope}%
\begin{pgfscope}%
\pgfsetbuttcap%
\pgfsetroundjoin%
\definecolor{currentfill}{rgb}{0.000000,0.000000,0.000000}%
\pgfsetfillcolor{currentfill}%
\pgfsetlinewidth{0.803000pt}%
\definecolor{currentstroke}{rgb}{0.000000,0.000000,0.000000}%
\pgfsetstrokecolor{currentstroke}%
\pgfsetdash{}{0pt}%
\pgfsys@defobject{currentmarker}{\pgfqpoint{0.000000in}{-0.048611in}}{\pgfqpoint{0.000000in}{0.000000in}}{%
\pgfpathmoveto{\pgfqpoint{0.000000in}{0.000000in}}%
\pgfpathlineto{\pgfqpoint{0.000000in}{-0.048611in}}%
\pgfusepath{stroke,fill}%
}%
\begin{pgfscope}%
\pgfsys@transformshift{4.209017in}{0.548486in}%
\pgfsys@useobject{currentmarker}{}%
\end{pgfscope}%
\end{pgfscope}%
\begin{pgfscope}%
\definecolor{textcolor}{rgb}{0.000000,0.000000,0.000000}%
\pgfsetstrokecolor{textcolor}%
\pgfsetfillcolor{textcolor}%
\pgftext[x=4.209017in,y=0.451264in,,top]{\color{textcolor}{\rmfamily\fontsize{10.000000}{12.000000}\selectfont\catcode`\^=\active\def^{\ifmmode\sp\else\^{}\fi}\catcode`\%=\active\def%{\%}0}}%
\end{pgfscope}%
\begin{pgfscope}%
\pgfsetbuttcap%
\pgfsetroundjoin%
\definecolor{currentfill}{rgb}{0.000000,0.000000,0.000000}%
\pgfsetfillcolor{currentfill}%
\pgfsetlinewidth{0.803000pt}%
\definecolor{currentstroke}{rgb}{0.000000,0.000000,0.000000}%
\pgfsetstrokecolor{currentstroke}%
\pgfsetdash{}{0pt}%
\pgfsys@defobject{currentmarker}{\pgfqpoint{0.000000in}{-0.048611in}}{\pgfqpoint{0.000000in}{0.000000in}}{%
\pgfpathmoveto{\pgfqpoint{0.000000in}{0.000000in}}%
\pgfpathlineto{\pgfqpoint{0.000000in}{-0.048611in}}%
\pgfusepath{stroke,fill}%
}%
\begin{pgfscope}%
\pgfsys@transformshift{4.689388in}{0.548486in}%
\pgfsys@useobject{currentmarker}{}%
\end{pgfscope}%
\end{pgfscope}%
\begin{pgfscope}%
\definecolor{textcolor}{rgb}{0.000000,0.000000,0.000000}%
\pgfsetstrokecolor{textcolor}%
\pgfsetfillcolor{textcolor}%
\pgftext[x=4.689388in,y=0.451264in,,top]{\color{textcolor}{\rmfamily\fontsize{10.000000}{12.000000}\selectfont\catcode`\^=\active\def^{\ifmmode\sp\else\^{}\fi}\catcode`\%=\active\def%{\%}5}}%
\end{pgfscope}%
\begin{pgfscope}%
\pgfsetbuttcap%
\pgfsetroundjoin%
\definecolor{currentfill}{rgb}{0.000000,0.000000,0.000000}%
\pgfsetfillcolor{currentfill}%
\pgfsetlinewidth{0.803000pt}%
\definecolor{currentstroke}{rgb}{0.000000,0.000000,0.000000}%
\pgfsetstrokecolor{currentstroke}%
\pgfsetdash{}{0pt}%
\pgfsys@defobject{currentmarker}{\pgfqpoint{0.000000in}{-0.048611in}}{\pgfqpoint{0.000000in}{0.000000in}}{%
\pgfpathmoveto{\pgfqpoint{0.000000in}{0.000000in}}%
\pgfpathlineto{\pgfqpoint{0.000000in}{-0.048611in}}%
\pgfusepath{stroke,fill}%
}%
\begin{pgfscope}%
\pgfsys@transformshift{5.169760in}{0.548486in}%
\pgfsys@useobject{currentmarker}{}%
\end{pgfscope}%
\end{pgfscope}%
\begin{pgfscope}%
\definecolor{textcolor}{rgb}{0.000000,0.000000,0.000000}%
\pgfsetstrokecolor{textcolor}%
\pgfsetfillcolor{textcolor}%
\pgftext[x=5.169760in,y=0.451264in,,top]{\color{textcolor}{\rmfamily\fontsize{10.000000}{12.000000}\selectfont\catcode`\^=\active\def^{\ifmmode\sp\else\^{}\fi}\catcode`\%=\active\def%{\%}10}}%
\end{pgfscope}%
\begin{pgfscope}%
\definecolor{textcolor}{rgb}{0.000000,0.000000,0.000000}%
\pgfsetstrokecolor{textcolor}%
\pgfsetfillcolor{textcolor}%
\pgftext[x=4.209017in,y=0.261295in,,top]{\color{textcolor}{\rmfamily\fontsize{12.000000}{14.400000}\selectfont\catcode`\^=\active\def^{\ifmmode\sp\else\^{}\fi}\catcode`\%=\active\def%{\%}$x$}}%
\end{pgfscope}%
\begin{pgfscope}%
\pgfsetbuttcap%
\pgfsetroundjoin%
\definecolor{currentfill}{rgb}{0.000000,0.000000,0.000000}%
\pgfsetfillcolor{currentfill}%
\pgfsetlinewidth{0.803000pt}%
\definecolor{currentstroke}{rgb}{0.000000,0.000000,0.000000}%
\pgfsetstrokecolor{currentstroke}%
\pgfsetdash{}{0pt}%
\pgfsys@defobject{currentmarker}{\pgfqpoint{-0.048611in}{0.000000in}}{\pgfqpoint{-0.000000in}{0.000000in}}{%
\pgfpathmoveto{\pgfqpoint{-0.000000in}{0.000000in}}%
\pgfpathlineto{\pgfqpoint{-0.048611in}{0.000000in}}%
\pgfusepath{stroke,fill}%
}%
\begin{pgfscope}%
\pgfsys@transformshift{3.152198in}{0.548486in}%
\pgfsys@useobject{currentmarker}{}%
\end{pgfscope}%
\end{pgfscope}%
\begin{pgfscope}%
\definecolor{textcolor}{rgb}{0.000000,0.000000,0.000000}%
\pgfsetstrokecolor{textcolor}%
\pgfsetfillcolor{textcolor}%
\pgftext[x=2.858586in, y=0.495724in, left, base]{\color{textcolor}{\rmfamily\fontsize{10.000000}{12.000000}\selectfont\catcode`\^=\active\def^{\ifmmode\sp\else\^{}\fi}\catcode`\%=\active\def%{\%}\ensuremath{-}3}}%
\end{pgfscope}%
\begin{pgfscope}%
\pgfsetbuttcap%
\pgfsetroundjoin%
\definecolor{currentfill}{rgb}{0.000000,0.000000,0.000000}%
\pgfsetfillcolor{currentfill}%
\pgfsetlinewidth{0.803000pt}%
\definecolor{currentstroke}{rgb}{0.000000,0.000000,0.000000}%
\pgfsetstrokecolor{currentstroke}%
\pgfsetdash{}{0pt}%
\pgfsys@defobject{currentmarker}{\pgfqpoint{-0.048611in}{0.000000in}}{\pgfqpoint{-0.000000in}{0.000000in}}{%
\pgfpathmoveto{\pgfqpoint{-0.000000in}{0.000000in}}%
\pgfpathlineto{\pgfqpoint{-0.048611in}{0.000000in}}%
\pgfusepath{stroke,fill}%
}%
\begin{pgfscope}%
\pgfsys@transformshift{3.152198in}{0.933486in}%
\pgfsys@useobject{currentmarker}{}%
\end{pgfscope}%
\end{pgfscope}%
\begin{pgfscope}%
\definecolor{textcolor}{rgb}{0.000000,0.000000,0.000000}%
\pgfsetstrokecolor{textcolor}%
\pgfsetfillcolor{textcolor}%
\pgftext[x=2.858586in, y=0.880724in, left, base]{\color{textcolor}{\rmfamily\fontsize{10.000000}{12.000000}\selectfont\catcode`\^=\active\def^{\ifmmode\sp\else\^{}\fi}\catcode`\%=\active\def%{\%}\ensuremath{-}2}}%
\end{pgfscope}%
\begin{pgfscope}%
\pgfsetbuttcap%
\pgfsetroundjoin%
\definecolor{currentfill}{rgb}{0.000000,0.000000,0.000000}%
\pgfsetfillcolor{currentfill}%
\pgfsetlinewidth{0.803000pt}%
\definecolor{currentstroke}{rgb}{0.000000,0.000000,0.000000}%
\pgfsetstrokecolor{currentstroke}%
\pgfsetdash{}{0pt}%
\pgfsys@defobject{currentmarker}{\pgfqpoint{-0.048611in}{0.000000in}}{\pgfqpoint{-0.000000in}{0.000000in}}{%
\pgfpathmoveto{\pgfqpoint{-0.000000in}{0.000000in}}%
\pgfpathlineto{\pgfqpoint{-0.048611in}{0.000000in}}%
\pgfusepath{stroke,fill}%
}%
\begin{pgfscope}%
\pgfsys@transformshift{3.152198in}{1.318486in}%
\pgfsys@useobject{currentmarker}{}%
\end{pgfscope}%
\end{pgfscope}%
\begin{pgfscope}%
\definecolor{textcolor}{rgb}{0.000000,0.000000,0.000000}%
\pgfsetstrokecolor{textcolor}%
\pgfsetfillcolor{textcolor}%
\pgftext[x=2.858586in, y=1.265724in, left, base]{\color{textcolor}{\rmfamily\fontsize{10.000000}{12.000000}\selectfont\catcode`\^=\active\def^{\ifmmode\sp\else\^{}\fi}\catcode`\%=\active\def%{\%}\ensuremath{-}1}}%
\end{pgfscope}%
\begin{pgfscope}%
\pgfsetbuttcap%
\pgfsetroundjoin%
\definecolor{currentfill}{rgb}{0.000000,0.000000,0.000000}%
\pgfsetfillcolor{currentfill}%
\pgfsetlinewidth{0.803000pt}%
\definecolor{currentstroke}{rgb}{0.000000,0.000000,0.000000}%
\pgfsetstrokecolor{currentstroke}%
\pgfsetdash{}{0pt}%
\pgfsys@defobject{currentmarker}{\pgfqpoint{-0.048611in}{0.000000in}}{\pgfqpoint{-0.000000in}{0.000000in}}{%
\pgfpathmoveto{\pgfqpoint{-0.000000in}{0.000000in}}%
\pgfpathlineto{\pgfqpoint{-0.048611in}{0.000000in}}%
\pgfusepath{stroke,fill}%
}%
\begin{pgfscope}%
\pgfsys@transformshift{3.152198in}{1.703486in}%
\pgfsys@useobject{currentmarker}{}%
\end{pgfscope}%
\end{pgfscope}%
\begin{pgfscope}%
\definecolor{textcolor}{rgb}{0.000000,0.000000,0.000000}%
\pgfsetstrokecolor{textcolor}%
\pgfsetfillcolor{textcolor}%
\pgftext[x=2.966611in, y=1.650724in, left, base]{\color{textcolor}{\rmfamily\fontsize{10.000000}{12.000000}\selectfont\catcode`\^=\active\def^{\ifmmode\sp\else\^{}\fi}\catcode`\%=\active\def%{\%}0}}%
\end{pgfscope}%
\begin{pgfscope}%
\pgfsetbuttcap%
\pgfsetroundjoin%
\definecolor{currentfill}{rgb}{0.000000,0.000000,0.000000}%
\pgfsetfillcolor{currentfill}%
\pgfsetlinewidth{0.803000pt}%
\definecolor{currentstroke}{rgb}{0.000000,0.000000,0.000000}%
\pgfsetstrokecolor{currentstroke}%
\pgfsetdash{}{0pt}%
\pgfsys@defobject{currentmarker}{\pgfqpoint{-0.048611in}{0.000000in}}{\pgfqpoint{-0.000000in}{0.000000in}}{%
\pgfpathmoveto{\pgfqpoint{-0.000000in}{0.000000in}}%
\pgfpathlineto{\pgfqpoint{-0.048611in}{0.000000in}}%
\pgfusepath{stroke,fill}%
}%
\begin{pgfscope}%
\pgfsys@transformshift{3.152198in}{2.088486in}%
\pgfsys@useobject{currentmarker}{}%
\end{pgfscope}%
\end{pgfscope}%
\begin{pgfscope}%
\definecolor{textcolor}{rgb}{0.000000,0.000000,0.000000}%
\pgfsetstrokecolor{textcolor}%
\pgfsetfillcolor{textcolor}%
\pgftext[x=2.966611in, y=2.035724in, left, base]{\color{textcolor}{\rmfamily\fontsize{10.000000}{12.000000}\selectfont\catcode`\^=\active\def^{\ifmmode\sp\else\^{}\fi}\catcode`\%=\active\def%{\%}1}}%
\end{pgfscope}%
\begin{pgfscope}%
\pgfsetbuttcap%
\pgfsetroundjoin%
\definecolor{currentfill}{rgb}{0.000000,0.000000,0.000000}%
\pgfsetfillcolor{currentfill}%
\pgfsetlinewidth{0.803000pt}%
\definecolor{currentstroke}{rgb}{0.000000,0.000000,0.000000}%
\pgfsetstrokecolor{currentstroke}%
\pgfsetdash{}{0pt}%
\pgfsys@defobject{currentmarker}{\pgfqpoint{-0.048611in}{0.000000in}}{\pgfqpoint{-0.000000in}{0.000000in}}{%
\pgfpathmoveto{\pgfqpoint{-0.000000in}{0.000000in}}%
\pgfpathlineto{\pgfqpoint{-0.048611in}{0.000000in}}%
\pgfusepath{stroke,fill}%
}%
\begin{pgfscope}%
\pgfsys@transformshift{3.152198in}{2.473486in}%
\pgfsys@useobject{currentmarker}{}%
\end{pgfscope}%
\end{pgfscope}%
\begin{pgfscope}%
\definecolor{textcolor}{rgb}{0.000000,0.000000,0.000000}%
\pgfsetstrokecolor{textcolor}%
\pgfsetfillcolor{textcolor}%
\pgftext[x=2.966611in, y=2.420724in, left, base]{\color{textcolor}{\rmfamily\fontsize{10.000000}{12.000000}\selectfont\catcode`\^=\active\def^{\ifmmode\sp\else\^{}\fi}\catcode`\%=\active\def%{\%}2}}%
\end{pgfscope}%
\begin{pgfscope}%
\pgfsetbuttcap%
\pgfsetroundjoin%
\definecolor{currentfill}{rgb}{0.000000,0.000000,0.000000}%
\pgfsetfillcolor{currentfill}%
\pgfsetlinewidth{0.803000pt}%
\definecolor{currentstroke}{rgb}{0.000000,0.000000,0.000000}%
\pgfsetstrokecolor{currentstroke}%
\pgfsetdash{}{0pt}%
\pgfsys@defobject{currentmarker}{\pgfqpoint{-0.048611in}{0.000000in}}{\pgfqpoint{-0.000000in}{0.000000in}}{%
\pgfpathmoveto{\pgfqpoint{-0.000000in}{0.000000in}}%
\pgfpathlineto{\pgfqpoint{-0.048611in}{0.000000in}}%
\pgfusepath{stroke,fill}%
}%
\begin{pgfscope}%
\pgfsys@transformshift{3.152198in}{2.858486in}%
\pgfsys@useobject{currentmarker}{}%
\end{pgfscope}%
\end{pgfscope}%
\begin{pgfscope}%
\definecolor{textcolor}{rgb}{0.000000,0.000000,0.000000}%
\pgfsetstrokecolor{textcolor}%
\pgfsetfillcolor{textcolor}%
\pgftext[x=2.966611in, y=2.805724in, left, base]{\color{textcolor}{\rmfamily\fontsize{10.000000}{12.000000}\selectfont\catcode`\^=\active\def^{\ifmmode\sp\else\^{}\fi}\catcode`\%=\active\def%{\%}3}}%
\end{pgfscope}%
\begin{pgfscope}%
\pgfpathrectangle{\pgfqpoint{3.152198in}{0.548486in}}{\pgfqpoint{2.113636in}{2.310000in}}%
\pgfusepath{clip}%
\pgfsetrectcap%
\pgfsetroundjoin%
\pgfsetlinewidth{1.505625pt}%
\definecolor{currentstroke}{rgb}{0.000000,0.000000,0.000000}%
\pgfsetstrokecolor{currentstroke}%
\pgfsetdash{}{0pt}%
\pgfpathmoveto{\pgfqpoint{3.248273in}{1.707336in}}%
\pgfpathlineto{\pgfqpoint{3.492546in}{1.710409in}}%
\pgfpathlineto{\pgfqpoint{3.634878in}{1.714267in}}%
\pgfpathlineto{\pgfqpoint{3.729125in}{1.718917in}}%
\pgfpathlineto{\pgfqpoint{3.796445in}{1.724363in}}%
\pgfpathlineto{\pgfqpoint{3.846454in}{1.730520in}}%
\pgfpathlineto{\pgfqpoint{3.884922in}{1.737318in}}%
\pgfpathlineto{\pgfqpoint{3.917620in}{1.745337in}}%
\pgfpathlineto{\pgfqpoint{3.944548in}{1.754293in}}%
\pgfpathlineto{\pgfqpoint{3.965705in}{1.763514in}}%
\pgfpathlineto{\pgfqpoint{3.984939in}{1.774261in}}%
\pgfpathlineto{\pgfqpoint{4.002250in}{1.786608in}}%
\pgfpathlineto{\pgfqpoint{4.017637in}{1.800511in}}%
\pgfpathlineto{\pgfqpoint{4.031101in}{1.815752in}}%
\pgfpathlineto{\pgfqpoint{4.042641in}{1.831867in}}%
\pgfpathlineto{\pgfqpoint{4.054182in}{1.851717in}}%
\pgfpathlineto{\pgfqpoint{4.065722in}{1.876555in}}%
\pgfpathlineto{\pgfqpoint{4.075339in}{1.902353in}}%
\pgfpathlineto{\pgfqpoint{4.084957in}{1.934380in}}%
\pgfpathlineto{\pgfqpoint{4.094574in}{1.974816in}}%
\pgfpathlineto{\pgfqpoint{4.104191in}{2.026885in}}%
\pgfpathlineto{\pgfqpoint{4.111884in}{2.080146in}}%
\pgfpathlineto{\pgfqpoint{4.119578in}{2.147734in}}%
\pgfpathlineto{\pgfqpoint{4.127272in}{2.235293in}}%
\pgfpathlineto{\pgfqpoint{4.134965in}{2.351539in}}%
\pgfpathlineto{\pgfqpoint{4.142659in}{2.510523in}}%
\pgfpathlineto{\pgfqpoint{4.148429in}{2.671565in}}%
\pgfpathlineto{\pgfqpoint{4.153770in}{2.868486in}}%
\pgfpathmoveto{\pgfqpoint{4.264264in}{2.868486in}}%
\pgfpathlineto{\pgfqpoint{4.271527in}{2.612907in}}%
\pgfpathlineto{\pgfqpoint{4.279221in}{2.424504in}}%
\pgfpathlineto{\pgfqpoint{4.286915in}{2.289114in}}%
\pgfpathlineto{\pgfqpoint{4.294608in}{2.188564in}}%
\pgfpathlineto{\pgfqpoint{4.302302in}{2.111851in}}%
\pgfpathlineto{\pgfqpoint{4.311919in}{2.039088in}}%
\pgfpathlineto{\pgfqpoint{4.321536in}{1.984171in}}%
\pgfpathlineto{\pgfqpoint{4.331153in}{1.941709in}}%
\pgfpathlineto{\pgfqpoint{4.340770in}{1.908201in}}%
\pgfpathlineto{\pgfqpoint{4.352311in}{1.876555in}}%
\pgfpathlineto{\pgfqpoint{4.363851in}{1.851717in}}%
\pgfpathlineto{\pgfqpoint{4.375392in}{1.831867in}}%
\pgfpathlineto{\pgfqpoint{4.388855in}{1.813363in}}%
\pgfpathlineto{\pgfqpoint{4.402319in}{1.798590in}}%
\pgfpathlineto{\pgfqpoint{4.417707in}{1.785083in}}%
\pgfpathlineto{\pgfqpoint{4.435017in}{1.773061in}}%
\pgfpathlineto{\pgfqpoint{4.454251in}{1.762576in}}%
\pgfpathlineto{\pgfqpoint{4.475409in}{1.753562in}}%
\pgfpathlineto{\pgfqpoint{4.500413in}{1.745337in}}%
\pgfpathlineto{\pgfqpoint{4.531188in}{1.737723in}}%
\pgfpathlineto{\pgfqpoint{4.567733in}{1.731103in}}%
\pgfpathlineto{\pgfqpoint{4.613895in}{1.725164in}}%
\pgfpathlineto{\pgfqpoint{4.671597in}{1.720093in}}%
\pgfpathlineto{\pgfqpoint{4.748533in}{1.715695in}}%
\pgfpathlineto{\pgfqpoint{4.854321in}{1.712020in}}%
\pgfpathlineto{\pgfqpoint{5.010117in}{1.709023in}}%
\pgfpathlineto{\pgfqpoint{5.169760in}{1.707336in}}%
\pgfpathlineto{\pgfqpoint{5.169760in}{1.707336in}}%
\pgfusepath{stroke}%
\end{pgfscope}%
\begin{pgfscope}%
\pgfpathrectangle{\pgfqpoint{3.152198in}{0.548486in}}{\pgfqpoint{2.113636in}{2.310000in}}%
\pgfusepath{clip}%
\pgfsetrectcap%
\pgfsetroundjoin%
\pgfsetlinewidth{0.501875pt}%
\definecolor{currentstroke}{rgb}{0.000000,0.000000,0.000000}%
\pgfsetstrokecolor{currentstroke}%
\pgfsetdash{}{0pt}%
\pgfpathmoveto{\pgfqpoint{3.152198in}{1.703486in}}%
\pgfpathlineto{\pgfqpoint{5.265835in}{1.703486in}}%
\pgfusepath{stroke}%
\end{pgfscope}%
\begin{pgfscope}%
\pgfpathrectangle{\pgfqpoint{3.152198in}{0.548486in}}{\pgfqpoint{2.113636in}{2.310000in}}%
\pgfusepath{clip}%
\pgfsetrectcap%
\pgfsetroundjoin%
\pgfsetlinewidth{0.501875pt}%
\definecolor{currentstroke}{rgb}{0.000000,0.000000,0.000000}%
\pgfsetstrokecolor{currentstroke}%
\pgfsetdash{}{0pt}%
\pgfpathmoveto{\pgfqpoint{4.209017in}{0.548486in}}%
\pgfpathlineto{\pgfqpoint{4.209017in}{2.858486in}}%
\pgfusepath{stroke}%
\end{pgfscope}%
\begin{pgfscope}%
\pgfsetbuttcap%
\pgfsetmiterjoin%
\definecolor{currentfill}{rgb}{1.000000,1.000000,1.000000}%
\pgfsetfillcolor{currentfill}%
\pgfsetfillopacity{0.800000}%
\pgfsetlinewidth{1.003750pt}%
\definecolor{currentstroke}{rgb}{0.800000,0.800000,0.800000}%
\pgfsetstrokecolor{currentstroke}%
\pgfsetstrokeopacity{0.800000}%
\pgfsetdash{}{0pt}%
\pgfpathmoveto{\pgfqpoint{3.249421in}{0.617930in}}%
\pgfpathlineto{\pgfqpoint{4.377892in}{0.617930in}}%
\pgfpathquadraticcurveto{\pgfqpoint{4.405670in}{0.617930in}}{\pgfqpoint{4.405670in}{0.645708in}}%
\pgfpathlineto{\pgfqpoint{4.405670in}{0.855986in}}%
\pgfpathquadraticcurveto{\pgfqpoint{4.405670in}{0.883764in}}{\pgfqpoint{4.377892in}{0.883764in}}%
\pgfpathlineto{\pgfqpoint{3.249421in}{0.883764in}}%
\pgfpathquadraticcurveto{\pgfqpoint{3.221643in}{0.883764in}}{\pgfqpoint{3.221643in}{0.855986in}}%
\pgfpathlineto{\pgfqpoint{3.221643in}{0.645708in}}%
\pgfpathquadraticcurveto{\pgfqpoint{3.221643in}{0.617930in}}{\pgfqpoint{3.249421in}{0.617930in}}%
\pgfpathlineto{\pgfqpoint{3.249421in}{0.617930in}}%
\pgfpathclose%
\pgfusepath{stroke,fill}%
\end{pgfscope}%
\begin{pgfscope}%
\pgfsetrectcap%
\pgfsetroundjoin%
\pgfsetlinewidth{1.505625pt}%
\definecolor{currentstroke}{rgb}{0.000000,0.000000,0.000000}%
\pgfsetstrokecolor{currentstroke}%
\pgfsetdash{}{0pt}%
\pgfpathmoveto{\pgfqpoint{3.277198in}{0.756819in}}%
\pgfpathlineto{\pgfqpoint{3.416087in}{0.756819in}}%
\pgfpathlineto{\pgfqpoint{3.554976in}{0.756819in}}%
\pgfusepath{stroke}%
\end{pgfscope}%
\begin{pgfscope}%
\definecolor{textcolor}{rgb}{0.000000,0.000000,0.000000}%
\pgfsetstrokecolor{textcolor}%
\pgfsetfillcolor{textcolor}%
\pgftext[x=3.666087in,y=0.708208in,left,base]{\color{textcolor}{\rmfamily\fontsize{10.000000}{12.000000}\selectfont\catcode`\^=\active\def^{\ifmmode\sp\else\^{}\fi}\catcode`\%=\active\def%{\%}$f(x) = x^{-2}$}}%
\end{pgfscope}%
\end{pgfpicture}%
\makeatother%
\endgroup%

	\caption{Potenzfunktionen mit Negativen Exponenten.}
	\label{fig:ratio1}
\end{figure}



% \begin{equation}
% \displaystyle x

% \end{equation}

\subsection{Polynomdivision}
Polynomdivision (Engl. 'long division of polynomials' und oft kurz 'long division') ist die Division von 2 Polynomen. Siehe das folgende Beispiel:

\polylongdiv[style=C]{x^{3} + 5 x^{2} - 2 x - 24}{x-2}


Hier wurde manuell das Polynom $P(x) = x^{3} + 5 x^{2} - 2 x - 24$ durch den Faktor $(x-2)$ geteilt. In diesem Fall gab es keinen Rest. Wir haben erreicht dass das ursprüngliche Polynom $P$ vom Grad $3$ auf ein Produkt zweier Polynome reduziert wurde wobei eines vom Grad $1$ und eines vom Grad $2$ ist. 

$$P(x) = x^{3} + 5 x^{2} - 2 x - 24 = (x-2) \cdot (x^2 + 7x + 12)$$


Was hier interessant ist, ist dass $(x-2)$ ein \emph{linearer Faktor} ist, also keine Potenzen größer 1 enthält. Es ist offensichtlich dass der Faktor $(x-2)$ $0$ wird an der Stelle $2$. Da $P(x)$ ein Produkt ua. dieses Faktors ist haben wir eine Nullstelle von $P$.


\subsection{Zerlegung in Linearfaktoren}\label{sec:linearfaktoren}

\todo[inline]{TODO}
% \polylongdiv[style=C]{x^{3} + 5 x^{2} - 2 x - 24}{x-2}



\subsection{Partialbruchzerlegung}
\todo{Vereinfachen.}
Zum Beispiel in \cite{liski2019converting} und \cite{freeman2011} kann klar gesehen welche zentrale Rolle Partialbruchzerlegung und Polynomdivision im Bereich der Signalverarbeitung spielen.

Die idee der Partialbruchzerlegung ist es einen gebrochen rationalen Ausdruck in eine \emph{summe} von anderen gebrochen rationalen Ausdrücken zu verwandeln. 

\begin{equation}
{\displaystyle {f(x) = \frac {P(x)}{Q(x)}}=p(x)+\sum _{j}{\frac {P_{j}(x)}{Q_{j}(x)}}}
\end{equation}

Wie zB. in \citep{freeman2011} demonstriert, kann dies verwendet werden um eine komplizierte Filterschaltung in eine Summe von einfacheren Filtern 'umzubauen'.

Das eigentliche verfahren dies händisch zu vollziehen ist nicht unkompliziert und wir werden daher recht praxisnah schlicht \texttt{Python} bemühen.

\section{Rekursive Funktionen}
\todo{TODO. not urgent.}
\begin{equation}
	 \displaystyle \mathrm{fib}(x) = \left\{ \begin{array}{ll} 0, & \mathrm{if} \ x = 0 \\ 1, & \mathrm{if} \ x = 1 \\ \mathrm{fib} \mathopen{}\left( x - 1 \mathclose{}\right) + \mathrm{fib} \mathopen{}\left( x - 2 \mathclose{}\right), & \mathrm{otherwise} \end{array} \right. \label{eq:fib}
\end{equation}

\section{Python}\label{sec:python}

In diesem Unterricht wird python verwendet. Alternativ könnte man vieles andere verwenden, MATLAB, Octave,  Mathematica, R, Julia etc. Alle diese Sprachen haben Vor- und Nachteile. Hier einige der Vorteile von Python:
\begin{itemize}
	\item Gratis
	\item Open Source
	\item General Purpose Langage (Industriestandard für Datascience, Machine Learnung, ...) 
\end{itemize}
Allgemein versucht dieser Unterricht die details des Programmierens eher in den Hintergrund zu Stellen und es wird versucht ein Fokus auf die dahinterliegenden Mathematischen Probleme zu legen, weshalb zu Teil die Informatik ein wenig kurz und bündig ausfallen wird. 

\subsection{Laden von Bibliotheken}
Oft wollen wir \texttt{python} einfach als 'Taschenrechner' verwenden. Da für \texttt{python} zahllose Zusatzpakete und Programmbibliotheken existieren können wir sehr komplexe Abläufe auslagern durch die Einbindung diese Bibliothheken. Sehr oft werden wir unsere 'Session' mit folgenden importen beginnen:

\begin{python}{Üblicher beginn einer Python session für unsere Zwecke.}
%pylab
import sympy as sp
\end{python}
Hier werden in Zeile \texttt{1} durch \texttt{\%pylab} 2 Pakete importiert: \texttt{numpy} und \texttt{matplotlib}\footnote{\texttt{numpy} ist das gängigste Paket für numerische Operationen im allgemeinen und speziell auf arrays. \texttt{matplotlib} ist zuständig für das Anzeigen von plots.}. In Zeile \texttt{2} wird \texttt{sympy} importiert und zwar unter dem kürzel \texttt{sp}\footnote{Wir könnten statt \texttt{sp} auch \texttt{hans} verwenden. \texttt{sp} ist sehr verbreitet und wirkt übersichtlich.}. Das kürzel erlaubt uns funktionen in \texttt{sympy} durch dieses zu verwenden, z.B..: \texttt{sp.roots()}. \\


\subsection{Numerische Berechnug vs Symbolische Bearbeitung}

Durch den obigen import haben wir nun die Möglichkeit mathematische Probleme sowohl numerisch als auf analytisch zu lösen. Es gilt hier vorsichtig zu sein, und sich immer bewusst zu sein ob wir symbolisch oder numerisch arbeiten. 

\example{\textbf{Beispiel} \\
Das polynom $p(x) = x^3 - 4 \cdot x^2 - 7 \cdot x + 10$ soll auf Nullstellen untersucht werden.

Eine variante wäre es einfach mit verschiedenen Zahlen zu evaluieren:
}

\begin{python}{Einfache Evaluierung eines Polynoms.}
In [1]: x = 3
In [2]: x**3 - 4*x**2 - 7*x + 10
Out[2]: -20
\end{python}
\todo{Make lisings work inside 'examples'.}
\example{
	Anstatt mehrfach verschiedene Werte zu probieren kann ein sog. \texttt{array} verwendet werden um viele Zahlen gleichzeitig zu evalieren und das resultat kann in einem Graphen ('plot') angezeigt werden. Ein array mit aufsteigenden Zahlen kann erzeugt werden via \texttt{linspace()}. Zum Beispiel werden im Folgenden 11 Zahlen zwischen 0 und 5 erzeugt:
}
\begin{python}{Erzeugung eines Arrays mit aufsteigenden Zahlen.}
In [1]: linspace(0, 5, 11)
Out[1]: array([0. , 0.5, 1. , 1.5, 2. , 2.5, 3. , 3.5, 4. , 4.5, 5. ])
\end{python}

\example{
	Nun können wir eine Variable \texttt{x} durch einen array definieren und das Polynom für alle Werte in \texttt{x} evaluieren. Das Ergebnis weisen wir der Variable \texttt{y} zu und wir nutzten den \texttt{plot} Befehl um uns den Grafen anzuzeigen.
}

\begin{python}{Numerische Auswertung eines Polynoms.}
In [1]: x = linspace(-6, 6, 300)
In [2]: y = x**3 - 4*x**2 - 7*x + 10
In [3]: plot(x,y)
\end{python}



\begin{figure}[h]
	\centering
	%% Creator: Matplotlib, PGF backend
%%
%% To include the figure in your LaTeX document, write
%%   \input{<filename>.pgf}
%%
%% Make sure the required packages are loaded in your preamble
%%   \usepackage{pgf}
%%
%% Also ensure that all the required font packages are loaded; for instance,
%% the lmodern package is sometimes necessary when using math font.
%%   \usepackage{lmodern}
%%
%% Figures using additional raster images can only be included by \input if
%% they are in the same directory as the main LaTeX file. For loading figures
%% from other directories you can use the `import` package
%%   \usepackage{import}
%%
%% and then include the figures with
%%   \import{<path to file>}{<filename>.pgf}
%%
%% Matplotlib used the following preamble
%%   \def\mathdefault#1{#1}
%%   \everymath=\expandafter{\the\everymath\displaystyle}
%%   
%%   \usepackage{fontspec}
%%   \setmainfont{VeraSe.ttf}[Path=\detokenize{/usr/share/fonts/TTF/}]
%%   \setsansfont{DejaVuSans.ttf}[Path=\detokenize{/home/pl/miniconda3/lib/python3.12/site-packages/matplotlib/mpl-data/fonts/ttf/}]
%%   \setmonofont{DejaVuSansMono.ttf}[Path=\detokenize{/home/pl/miniconda3/lib/python3.12/site-packages/matplotlib/mpl-data/fonts/ttf/}]
%%   \makeatletter\@ifpackageloaded{underscore}{}{\usepackage[strings]{underscore}}\makeatother
%%
\begingroup%
\makeatletter%
\begin{pgfpicture}%
\pgfpathrectangle{\pgfpointorigin}{\pgfqpoint{5.852566in}{4.344486in}}%
\pgfusepath{use as bounding box, clip}%
\begin{pgfscope}%
\pgfsetbuttcap%
\pgfsetmiterjoin%
\definecolor{currentfill}{rgb}{1.000000,1.000000,1.000000}%
\pgfsetfillcolor{currentfill}%
\pgfsetlinewidth{0.000000pt}%
\definecolor{currentstroke}{rgb}{1.000000,1.000000,1.000000}%
\pgfsetstrokecolor{currentstroke}%
\pgfsetdash{}{0pt}%
\pgfpathmoveto{\pgfqpoint{0.000000in}{0.000000in}}%
\pgfpathlineto{\pgfqpoint{5.852566in}{0.000000in}}%
\pgfpathlineto{\pgfqpoint{5.852566in}{4.344486in}}%
\pgfpathlineto{\pgfqpoint{0.000000in}{4.344486in}}%
\pgfpathlineto{\pgfqpoint{0.000000in}{0.000000in}}%
\pgfpathclose%
\pgfusepath{fill}%
\end{pgfscope}%
\begin{pgfscope}%
\pgfsetbuttcap%
\pgfsetmiterjoin%
\definecolor{currentfill}{rgb}{1.000000,1.000000,1.000000}%
\pgfsetfillcolor{currentfill}%
\pgfsetlinewidth{0.000000pt}%
\definecolor{currentstroke}{rgb}{0.000000,0.000000,0.000000}%
\pgfsetstrokecolor{currentstroke}%
\pgfsetstrokeopacity{0.000000}%
\pgfsetdash{}{0pt}%
\pgfpathmoveto{\pgfqpoint{0.792566in}{0.548486in}}%
\pgfpathlineto{\pgfqpoint{5.752566in}{0.548486in}}%
\pgfpathlineto{\pgfqpoint{5.752566in}{4.244486in}}%
\pgfpathlineto{\pgfqpoint{0.792566in}{4.244486in}}%
\pgfpathlineto{\pgfqpoint{0.792566in}{0.548486in}}%
\pgfpathclose%
\pgfusepath{fill}%
\end{pgfscope}%
\begin{pgfscope}%
\pgfsetbuttcap%
\pgfsetroundjoin%
\definecolor{currentfill}{rgb}{0.000000,0.000000,0.000000}%
\pgfsetfillcolor{currentfill}%
\pgfsetlinewidth{0.803000pt}%
\definecolor{currentstroke}{rgb}{0.000000,0.000000,0.000000}%
\pgfsetstrokecolor{currentstroke}%
\pgfsetdash{}{0pt}%
\pgfsys@defobject{currentmarker}{\pgfqpoint{0.000000in}{-0.048611in}}{\pgfqpoint{0.000000in}{0.000000in}}{%
\pgfpathmoveto{\pgfqpoint{0.000000in}{0.000000in}}%
\pgfpathlineto{\pgfqpoint{0.000000in}{-0.048611in}}%
\pgfusepath{stroke,fill}%
}%
\begin{pgfscope}%
\pgfsys@transformshift{1.018020in}{0.548486in}%
\pgfsys@useobject{currentmarker}{}%
\end{pgfscope}%
\end{pgfscope}%
\begin{pgfscope}%
\definecolor{textcolor}{rgb}{0.000000,0.000000,0.000000}%
\pgfsetstrokecolor{textcolor}%
\pgfsetfillcolor{textcolor}%
\pgftext[x=1.018020in,y=0.451264in,,top]{\color{textcolor}{\rmfamily\fontsize{10.000000}{12.000000}\selectfont\catcode`\^=\active\def^{\ifmmode\sp\else\^{}\fi}\catcode`\%=\active\def%{\%}\ensuremath{-}6}}%
\end{pgfscope}%
\begin{pgfscope}%
\pgfsetbuttcap%
\pgfsetroundjoin%
\definecolor{currentfill}{rgb}{0.000000,0.000000,0.000000}%
\pgfsetfillcolor{currentfill}%
\pgfsetlinewidth{0.803000pt}%
\definecolor{currentstroke}{rgb}{0.000000,0.000000,0.000000}%
\pgfsetstrokecolor{currentstroke}%
\pgfsetdash{}{0pt}%
\pgfsys@defobject{currentmarker}{\pgfqpoint{0.000000in}{-0.048611in}}{\pgfqpoint{0.000000in}{0.000000in}}{%
\pgfpathmoveto{\pgfqpoint{0.000000in}{0.000000in}}%
\pgfpathlineto{\pgfqpoint{0.000000in}{-0.048611in}}%
\pgfusepath{stroke,fill}%
}%
\begin{pgfscope}%
\pgfsys@transformshift{1.769535in}{0.548486in}%
\pgfsys@useobject{currentmarker}{}%
\end{pgfscope}%
\end{pgfscope}%
\begin{pgfscope}%
\definecolor{textcolor}{rgb}{0.000000,0.000000,0.000000}%
\pgfsetstrokecolor{textcolor}%
\pgfsetfillcolor{textcolor}%
\pgftext[x=1.769535in,y=0.451264in,,top]{\color{textcolor}{\rmfamily\fontsize{10.000000}{12.000000}\selectfont\catcode`\^=\active\def^{\ifmmode\sp\else\^{}\fi}\catcode`\%=\active\def%{\%}\ensuremath{-}4}}%
\end{pgfscope}%
\begin{pgfscope}%
\pgfsetbuttcap%
\pgfsetroundjoin%
\definecolor{currentfill}{rgb}{0.000000,0.000000,0.000000}%
\pgfsetfillcolor{currentfill}%
\pgfsetlinewidth{0.803000pt}%
\definecolor{currentstroke}{rgb}{0.000000,0.000000,0.000000}%
\pgfsetstrokecolor{currentstroke}%
\pgfsetdash{}{0pt}%
\pgfsys@defobject{currentmarker}{\pgfqpoint{0.000000in}{-0.048611in}}{\pgfqpoint{0.000000in}{0.000000in}}{%
\pgfpathmoveto{\pgfqpoint{0.000000in}{0.000000in}}%
\pgfpathlineto{\pgfqpoint{0.000000in}{-0.048611in}}%
\pgfusepath{stroke,fill}%
}%
\begin{pgfscope}%
\pgfsys@transformshift{2.521050in}{0.548486in}%
\pgfsys@useobject{currentmarker}{}%
\end{pgfscope}%
\end{pgfscope}%
\begin{pgfscope}%
\definecolor{textcolor}{rgb}{0.000000,0.000000,0.000000}%
\pgfsetstrokecolor{textcolor}%
\pgfsetfillcolor{textcolor}%
\pgftext[x=2.521050in,y=0.451264in,,top]{\color{textcolor}{\rmfamily\fontsize{10.000000}{12.000000}\selectfont\catcode`\^=\active\def^{\ifmmode\sp\else\^{}\fi}\catcode`\%=\active\def%{\%}\ensuremath{-}2}}%
\end{pgfscope}%
\begin{pgfscope}%
\pgfsetbuttcap%
\pgfsetroundjoin%
\definecolor{currentfill}{rgb}{0.000000,0.000000,0.000000}%
\pgfsetfillcolor{currentfill}%
\pgfsetlinewidth{0.803000pt}%
\definecolor{currentstroke}{rgb}{0.000000,0.000000,0.000000}%
\pgfsetstrokecolor{currentstroke}%
\pgfsetdash{}{0pt}%
\pgfsys@defobject{currentmarker}{\pgfqpoint{0.000000in}{-0.048611in}}{\pgfqpoint{0.000000in}{0.000000in}}{%
\pgfpathmoveto{\pgfqpoint{0.000000in}{0.000000in}}%
\pgfpathlineto{\pgfqpoint{0.000000in}{-0.048611in}}%
\pgfusepath{stroke,fill}%
}%
\begin{pgfscope}%
\pgfsys@transformshift{3.272566in}{0.548486in}%
\pgfsys@useobject{currentmarker}{}%
\end{pgfscope}%
\end{pgfscope}%
\begin{pgfscope}%
\definecolor{textcolor}{rgb}{0.000000,0.000000,0.000000}%
\pgfsetstrokecolor{textcolor}%
\pgfsetfillcolor{textcolor}%
\pgftext[x=3.272566in,y=0.451264in,,top]{\color{textcolor}{\rmfamily\fontsize{10.000000}{12.000000}\selectfont\catcode`\^=\active\def^{\ifmmode\sp\else\^{}\fi}\catcode`\%=\active\def%{\%}0}}%
\end{pgfscope}%
\begin{pgfscope}%
\pgfsetbuttcap%
\pgfsetroundjoin%
\definecolor{currentfill}{rgb}{0.000000,0.000000,0.000000}%
\pgfsetfillcolor{currentfill}%
\pgfsetlinewidth{0.803000pt}%
\definecolor{currentstroke}{rgb}{0.000000,0.000000,0.000000}%
\pgfsetstrokecolor{currentstroke}%
\pgfsetdash{}{0pt}%
\pgfsys@defobject{currentmarker}{\pgfqpoint{0.000000in}{-0.048611in}}{\pgfqpoint{0.000000in}{0.000000in}}{%
\pgfpathmoveto{\pgfqpoint{0.000000in}{0.000000in}}%
\pgfpathlineto{\pgfqpoint{0.000000in}{-0.048611in}}%
\pgfusepath{stroke,fill}%
}%
\begin{pgfscope}%
\pgfsys@transformshift{4.024081in}{0.548486in}%
\pgfsys@useobject{currentmarker}{}%
\end{pgfscope}%
\end{pgfscope}%
\begin{pgfscope}%
\definecolor{textcolor}{rgb}{0.000000,0.000000,0.000000}%
\pgfsetstrokecolor{textcolor}%
\pgfsetfillcolor{textcolor}%
\pgftext[x=4.024081in,y=0.451264in,,top]{\color{textcolor}{\rmfamily\fontsize{10.000000}{12.000000}\selectfont\catcode`\^=\active\def^{\ifmmode\sp\else\^{}\fi}\catcode`\%=\active\def%{\%}2}}%
\end{pgfscope}%
\begin{pgfscope}%
\pgfsetbuttcap%
\pgfsetroundjoin%
\definecolor{currentfill}{rgb}{0.000000,0.000000,0.000000}%
\pgfsetfillcolor{currentfill}%
\pgfsetlinewidth{0.803000pt}%
\definecolor{currentstroke}{rgb}{0.000000,0.000000,0.000000}%
\pgfsetstrokecolor{currentstroke}%
\pgfsetdash{}{0pt}%
\pgfsys@defobject{currentmarker}{\pgfqpoint{0.000000in}{-0.048611in}}{\pgfqpoint{0.000000in}{0.000000in}}{%
\pgfpathmoveto{\pgfqpoint{0.000000in}{0.000000in}}%
\pgfpathlineto{\pgfqpoint{0.000000in}{-0.048611in}}%
\pgfusepath{stroke,fill}%
}%
\begin{pgfscope}%
\pgfsys@transformshift{4.775596in}{0.548486in}%
\pgfsys@useobject{currentmarker}{}%
\end{pgfscope}%
\end{pgfscope}%
\begin{pgfscope}%
\definecolor{textcolor}{rgb}{0.000000,0.000000,0.000000}%
\pgfsetstrokecolor{textcolor}%
\pgfsetfillcolor{textcolor}%
\pgftext[x=4.775596in,y=0.451264in,,top]{\color{textcolor}{\rmfamily\fontsize{10.000000}{12.000000}\selectfont\catcode`\^=\active\def^{\ifmmode\sp\else\^{}\fi}\catcode`\%=\active\def%{\%}4}}%
\end{pgfscope}%
\begin{pgfscope}%
\pgfsetbuttcap%
\pgfsetroundjoin%
\definecolor{currentfill}{rgb}{0.000000,0.000000,0.000000}%
\pgfsetfillcolor{currentfill}%
\pgfsetlinewidth{0.803000pt}%
\definecolor{currentstroke}{rgb}{0.000000,0.000000,0.000000}%
\pgfsetstrokecolor{currentstroke}%
\pgfsetdash{}{0pt}%
\pgfsys@defobject{currentmarker}{\pgfqpoint{0.000000in}{-0.048611in}}{\pgfqpoint{0.000000in}{0.000000in}}{%
\pgfpathmoveto{\pgfqpoint{0.000000in}{0.000000in}}%
\pgfpathlineto{\pgfqpoint{0.000000in}{-0.048611in}}%
\pgfusepath{stroke,fill}%
}%
\begin{pgfscope}%
\pgfsys@transformshift{5.527111in}{0.548486in}%
\pgfsys@useobject{currentmarker}{}%
\end{pgfscope}%
\end{pgfscope}%
\begin{pgfscope}%
\definecolor{textcolor}{rgb}{0.000000,0.000000,0.000000}%
\pgfsetstrokecolor{textcolor}%
\pgfsetfillcolor{textcolor}%
\pgftext[x=5.527111in,y=0.451264in,,top]{\color{textcolor}{\rmfamily\fontsize{10.000000}{12.000000}\selectfont\catcode`\^=\active\def^{\ifmmode\sp\else\^{}\fi}\catcode`\%=\active\def%{\%}6}}%
\end{pgfscope}%
\begin{pgfscope}%
\definecolor{textcolor}{rgb}{0.000000,0.000000,0.000000}%
\pgfsetstrokecolor{textcolor}%
\pgfsetfillcolor{textcolor}%
\pgftext[x=3.272566in,y=0.261295in,,top]{\color{textcolor}{\rmfamily\fontsize{12.000000}{14.400000}\selectfont\catcode`\^=\active\def^{\ifmmode\sp\else\^{}\fi}\catcode`\%=\active\def%{\%}$x$}}%
\end{pgfscope}%
\begin{pgfscope}%
\pgfsetbuttcap%
\pgfsetroundjoin%
\definecolor{currentfill}{rgb}{0.000000,0.000000,0.000000}%
\pgfsetfillcolor{currentfill}%
\pgfsetlinewidth{0.803000pt}%
\definecolor{currentstroke}{rgb}{0.000000,0.000000,0.000000}%
\pgfsetstrokecolor{currentstroke}%
\pgfsetdash{}{0pt}%
\pgfsys@defobject{currentmarker}{\pgfqpoint{-0.048611in}{0.000000in}}{\pgfqpoint{-0.000000in}{0.000000in}}{%
\pgfpathmoveto{\pgfqpoint{-0.000000in}{0.000000in}}%
\pgfpathlineto{\pgfqpoint{-0.048611in}{0.000000in}}%
\pgfusepath{stroke,fill}%
}%
\begin{pgfscope}%
\pgfsys@transformshift{0.792566in}{0.793727in}%
\pgfsys@useobject{currentmarker}{}%
\end{pgfscope}%
\end{pgfscope}%
\begin{pgfscope}%
\definecolor{textcolor}{rgb}{0.000000,0.000000,0.000000}%
\pgfsetstrokecolor{textcolor}%
\pgfsetfillcolor{textcolor}%
\pgftext[x=0.322222in, y=0.740966in, left, base]{\color{textcolor}{\rmfamily\fontsize{10.000000}{12.000000}\selectfont\catcode`\^=\active\def^{\ifmmode\sp\else\^{}\fi}\catcode`\%=\active\def%{\%}\ensuremath{-}300}}%
\end{pgfscope}%
\begin{pgfscope}%
\pgfsetbuttcap%
\pgfsetroundjoin%
\definecolor{currentfill}{rgb}{0.000000,0.000000,0.000000}%
\pgfsetfillcolor{currentfill}%
\pgfsetlinewidth{0.803000pt}%
\definecolor{currentstroke}{rgb}{0.000000,0.000000,0.000000}%
\pgfsetstrokecolor{currentstroke}%
\pgfsetdash{}{0pt}%
\pgfsys@defobject{currentmarker}{\pgfqpoint{-0.048611in}{0.000000in}}{\pgfqpoint{-0.000000in}{0.000000in}}{%
\pgfpathmoveto{\pgfqpoint{-0.000000in}{0.000000in}}%
\pgfpathlineto{\pgfqpoint{-0.048611in}{0.000000in}}%
\pgfusepath{stroke,fill}%
}%
\begin{pgfscope}%
\pgfsys@transformshift{0.792566in}{1.276486in}%
\pgfsys@useobject{currentmarker}{}%
\end{pgfscope}%
\end{pgfscope}%
\begin{pgfscope}%
\definecolor{textcolor}{rgb}{0.000000,0.000000,0.000000}%
\pgfsetstrokecolor{textcolor}%
\pgfsetfillcolor{textcolor}%
\pgftext[x=0.322222in, y=1.223724in, left, base]{\color{textcolor}{\rmfamily\fontsize{10.000000}{12.000000}\selectfont\catcode`\^=\active\def^{\ifmmode\sp\else\^{}\fi}\catcode`\%=\active\def%{\%}\ensuremath{-}250}}%
\end{pgfscope}%
\begin{pgfscope}%
\pgfsetbuttcap%
\pgfsetroundjoin%
\definecolor{currentfill}{rgb}{0.000000,0.000000,0.000000}%
\pgfsetfillcolor{currentfill}%
\pgfsetlinewidth{0.803000pt}%
\definecolor{currentstroke}{rgb}{0.000000,0.000000,0.000000}%
\pgfsetstrokecolor{currentstroke}%
\pgfsetdash{}{0pt}%
\pgfsys@defobject{currentmarker}{\pgfqpoint{-0.048611in}{0.000000in}}{\pgfqpoint{-0.000000in}{0.000000in}}{%
\pgfpathmoveto{\pgfqpoint{-0.000000in}{0.000000in}}%
\pgfpathlineto{\pgfqpoint{-0.048611in}{0.000000in}}%
\pgfusepath{stroke,fill}%
}%
\begin{pgfscope}%
\pgfsys@transformshift{0.792566in}{1.759245in}%
\pgfsys@useobject{currentmarker}{}%
\end{pgfscope}%
\end{pgfscope}%
\begin{pgfscope}%
\definecolor{textcolor}{rgb}{0.000000,0.000000,0.000000}%
\pgfsetstrokecolor{textcolor}%
\pgfsetfillcolor{textcolor}%
\pgftext[x=0.322222in, y=1.706483in, left, base]{\color{textcolor}{\rmfamily\fontsize{10.000000}{12.000000}\selectfont\catcode`\^=\active\def^{\ifmmode\sp\else\^{}\fi}\catcode`\%=\active\def%{\%}\ensuremath{-}200}}%
\end{pgfscope}%
\begin{pgfscope}%
\pgfsetbuttcap%
\pgfsetroundjoin%
\definecolor{currentfill}{rgb}{0.000000,0.000000,0.000000}%
\pgfsetfillcolor{currentfill}%
\pgfsetlinewidth{0.803000pt}%
\definecolor{currentstroke}{rgb}{0.000000,0.000000,0.000000}%
\pgfsetstrokecolor{currentstroke}%
\pgfsetdash{}{0pt}%
\pgfsys@defobject{currentmarker}{\pgfqpoint{-0.048611in}{0.000000in}}{\pgfqpoint{-0.000000in}{0.000000in}}{%
\pgfpathmoveto{\pgfqpoint{-0.000000in}{0.000000in}}%
\pgfpathlineto{\pgfqpoint{-0.048611in}{0.000000in}}%
\pgfusepath{stroke,fill}%
}%
\begin{pgfscope}%
\pgfsys@transformshift{0.792566in}{2.242003in}%
\pgfsys@useobject{currentmarker}{}%
\end{pgfscope}%
\end{pgfscope}%
\begin{pgfscope}%
\definecolor{textcolor}{rgb}{0.000000,0.000000,0.000000}%
\pgfsetstrokecolor{textcolor}%
\pgfsetfillcolor{textcolor}%
\pgftext[x=0.322222in, y=2.189242in, left, base]{\color{textcolor}{\rmfamily\fontsize{10.000000}{12.000000}\selectfont\catcode`\^=\active\def^{\ifmmode\sp\else\^{}\fi}\catcode`\%=\active\def%{\%}\ensuremath{-}150}}%
\end{pgfscope}%
\begin{pgfscope}%
\pgfsetbuttcap%
\pgfsetroundjoin%
\definecolor{currentfill}{rgb}{0.000000,0.000000,0.000000}%
\pgfsetfillcolor{currentfill}%
\pgfsetlinewidth{0.803000pt}%
\definecolor{currentstroke}{rgb}{0.000000,0.000000,0.000000}%
\pgfsetstrokecolor{currentstroke}%
\pgfsetdash{}{0pt}%
\pgfsys@defobject{currentmarker}{\pgfqpoint{-0.048611in}{0.000000in}}{\pgfqpoint{-0.000000in}{0.000000in}}{%
\pgfpathmoveto{\pgfqpoint{-0.000000in}{0.000000in}}%
\pgfpathlineto{\pgfqpoint{-0.048611in}{0.000000in}}%
\pgfusepath{stroke,fill}%
}%
\begin{pgfscope}%
\pgfsys@transformshift{0.792566in}{2.724762in}%
\pgfsys@useobject{currentmarker}{}%
\end{pgfscope}%
\end{pgfscope}%
\begin{pgfscope}%
\definecolor{textcolor}{rgb}{0.000000,0.000000,0.000000}%
\pgfsetstrokecolor{textcolor}%
\pgfsetfillcolor{textcolor}%
\pgftext[x=0.322222in, y=2.672000in, left, base]{\color{textcolor}{\rmfamily\fontsize{10.000000}{12.000000}\selectfont\catcode`\^=\active\def^{\ifmmode\sp\else\^{}\fi}\catcode`\%=\active\def%{\%}\ensuremath{-}100}}%
\end{pgfscope}%
\begin{pgfscope}%
\pgfsetbuttcap%
\pgfsetroundjoin%
\definecolor{currentfill}{rgb}{0.000000,0.000000,0.000000}%
\pgfsetfillcolor{currentfill}%
\pgfsetlinewidth{0.803000pt}%
\definecolor{currentstroke}{rgb}{0.000000,0.000000,0.000000}%
\pgfsetstrokecolor{currentstroke}%
\pgfsetdash{}{0pt}%
\pgfsys@defobject{currentmarker}{\pgfqpoint{-0.048611in}{0.000000in}}{\pgfqpoint{-0.000000in}{0.000000in}}{%
\pgfpathmoveto{\pgfqpoint{-0.000000in}{0.000000in}}%
\pgfpathlineto{\pgfqpoint{-0.048611in}{0.000000in}}%
\pgfusepath{stroke,fill}%
}%
\begin{pgfscope}%
\pgfsys@transformshift{0.792566in}{3.207520in}%
\pgfsys@useobject{currentmarker}{}%
\end{pgfscope}%
\end{pgfscope}%
\begin{pgfscope}%
\definecolor{textcolor}{rgb}{0.000000,0.000000,0.000000}%
\pgfsetstrokecolor{textcolor}%
\pgfsetfillcolor{textcolor}%
\pgftext[x=0.410588in, y=3.154759in, left, base]{\color{textcolor}{\rmfamily\fontsize{10.000000}{12.000000}\selectfont\catcode`\^=\active\def^{\ifmmode\sp\else\^{}\fi}\catcode`\%=\active\def%{\%}\ensuremath{-}50}}%
\end{pgfscope}%
\begin{pgfscope}%
\pgfsetbuttcap%
\pgfsetroundjoin%
\definecolor{currentfill}{rgb}{0.000000,0.000000,0.000000}%
\pgfsetfillcolor{currentfill}%
\pgfsetlinewidth{0.803000pt}%
\definecolor{currentstroke}{rgb}{0.000000,0.000000,0.000000}%
\pgfsetstrokecolor{currentstroke}%
\pgfsetdash{}{0pt}%
\pgfsys@defobject{currentmarker}{\pgfqpoint{-0.048611in}{0.000000in}}{\pgfqpoint{-0.000000in}{0.000000in}}{%
\pgfpathmoveto{\pgfqpoint{-0.000000in}{0.000000in}}%
\pgfpathlineto{\pgfqpoint{-0.048611in}{0.000000in}}%
\pgfusepath{stroke,fill}%
}%
\begin{pgfscope}%
\pgfsys@transformshift{0.792566in}{3.690279in}%
\pgfsys@useobject{currentmarker}{}%
\end{pgfscope}%
\end{pgfscope}%
\begin{pgfscope}%
\definecolor{textcolor}{rgb}{0.000000,0.000000,0.000000}%
\pgfsetstrokecolor{textcolor}%
\pgfsetfillcolor{textcolor}%
\pgftext[x=0.606978in, y=3.637518in, left, base]{\color{textcolor}{\rmfamily\fontsize{10.000000}{12.000000}\selectfont\catcode`\^=\active\def^{\ifmmode\sp\else\^{}\fi}\catcode`\%=\active\def%{\%}0}}%
\end{pgfscope}%
\begin{pgfscope}%
\pgfsetbuttcap%
\pgfsetroundjoin%
\definecolor{currentfill}{rgb}{0.000000,0.000000,0.000000}%
\pgfsetfillcolor{currentfill}%
\pgfsetlinewidth{0.803000pt}%
\definecolor{currentstroke}{rgb}{0.000000,0.000000,0.000000}%
\pgfsetstrokecolor{currentstroke}%
\pgfsetdash{}{0pt}%
\pgfsys@defobject{currentmarker}{\pgfqpoint{-0.048611in}{0.000000in}}{\pgfqpoint{-0.000000in}{0.000000in}}{%
\pgfpathmoveto{\pgfqpoint{-0.000000in}{0.000000in}}%
\pgfpathlineto{\pgfqpoint{-0.048611in}{0.000000in}}%
\pgfusepath{stroke,fill}%
}%
\begin{pgfscope}%
\pgfsys@transformshift{0.792566in}{4.173038in}%
\pgfsys@useobject{currentmarker}{}%
\end{pgfscope}%
\end{pgfscope}%
\begin{pgfscope}%
\definecolor{textcolor}{rgb}{0.000000,0.000000,0.000000}%
\pgfsetstrokecolor{textcolor}%
\pgfsetfillcolor{textcolor}%
\pgftext[x=0.518613in, y=4.120276in, left, base]{\color{textcolor}{\rmfamily\fontsize{10.000000}{12.000000}\selectfont\catcode`\^=\active\def^{\ifmmode\sp\else\^{}\fi}\catcode`\%=\active\def%{\%}50}}%
\end{pgfscope}%
\begin{pgfscope}%
\definecolor{textcolor}{rgb}{0.000000,0.000000,0.000000}%
\pgfsetstrokecolor{textcolor}%
\pgfsetfillcolor{textcolor}%
\pgftext[x=0.266667in,y=2.396486in,,bottom,rotate=90.000000]{\color{textcolor}{\rmfamily\fontsize{12.000000}{14.400000}\selectfont\catcode`\^=\active\def^{\ifmmode\sp\else\^{}\fi}\catcode`\%=\active\def%{\%}$p(x)$}}%
\end{pgfscope}%
\begin{pgfscope}%
\pgfpathrectangle{\pgfqpoint{0.792566in}{0.548486in}}{\pgfqpoint{4.960000in}{3.696000in}}%
\pgfusepath{clip}%
\pgfsetrectcap%
\pgfsetroundjoin%
\pgfsetlinewidth{1.505625pt}%
\definecolor{currentstroke}{rgb}{0.000000,0.000000,0.000000}%
\pgfsetstrokecolor{currentstroke}%
\pgfsetdash{}{0pt}%
\pgfpathmoveto{\pgfqpoint{1.018020in}{0.716486in}}%
\pgfpathlineto{\pgfqpoint{1.072183in}{0.919470in}}%
\pgfpathlineto{\pgfqpoint{1.126347in}{1.113801in}}%
\pgfpathlineto{\pgfqpoint{1.180510in}{1.299652in}}%
\pgfpathlineto{\pgfqpoint{1.234673in}{1.477197in}}%
\pgfpathlineto{\pgfqpoint{1.288836in}{1.646608in}}%
\pgfpathlineto{\pgfqpoint{1.338486in}{1.794906in}}%
\pgfpathlineto{\pgfqpoint{1.388136in}{1.936650in}}%
\pgfpathlineto{\pgfqpoint{1.437785in}{2.071972in}}%
\pgfpathlineto{\pgfqpoint{1.487435in}{2.201008in}}%
\pgfpathlineto{\pgfqpoint{1.537085in}{2.323889in}}%
\pgfpathlineto{\pgfqpoint{1.586734in}{2.440752in}}%
\pgfpathlineto{\pgfqpoint{1.636384in}{2.551727in}}%
\pgfpathlineto{\pgfqpoint{1.686034in}{2.656950in}}%
\pgfpathlineto{\pgfqpoint{1.735683in}{2.756555in}}%
\pgfpathlineto{\pgfqpoint{1.780819in}{2.842341in}}%
\pgfpathlineto{\pgfqpoint{1.825955in}{2.923693in}}%
\pgfpathlineto{\pgfqpoint{1.871091in}{3.000714in}}%
\pgfpathlineto{\pgfqpoint{1.916227in}{3.073502in}}%
\pgfpathlineto{\pgfqpoint{1.961363in}{3.142159in}}%
\pgfpathlineto{\pgfqpoint{2.006499in}{3.206784in}}%
\pgfpathlineto{\pgfqpoint{2.051636in}{3.267478in}}%
\pgfpathlineto{\pgfqpoint{2.096772in}{3.324342in}}%
\pgfpathlineto{\pgfqpoint{2.141908in}{3.377476in}}%
\pgfpathlineto{\pgfqpoint{2.187044in}{3.426980in}}%
\pgfpathlineto{\pgfqpoint{2.232180in}{3.472955in}}%
\pgfpathlineto{\pgfqpoint{2.277316in}{3.515501in}}%
\pgfpathlineto{\pgfqpoint{2.322452in}{3.554719in}}%
\pgfpathlineto{\pgfqpoint{2.367588in}{3.590708in}}%
\pgfpathlineto{\pgfqpoint{2.412724in}{3.623570in}}%
\pgfpathlineto{\pgfqpoint{2.457860in}{3.653405in}}%
\pgfpathlineto{\pgfqpoint{2.502996in}{3.680313in}}%
\pgfpathlineto{\pgfqpoint{2.548132in}{3.704394in}}%
\pgfpathlineto{\pgfqpoint{2.593268in}{3.725750in}}%
\pgfpathlineto{\pgfqpoint{2.638404in}{3.744480in}}%
\pgfpathlineto{\pgfqpoint{2.683540in}{3.760684in}}%
\pgfpathlineto{\pgfqpoint{2.728676in}{3.774464in}}%
\pgfpathlineto{\pgfqpoint{2.773812in}{3.785920in}}%
\pgfpathlineto{\pgfqpoint{2.818948in}{3.795151in}}%
\pgfpathlineto{\pgfqpoint{2.864084in}{3.802259in}}%
\pgfpathlineto{\pgfqpoint{2.913734in}{3.807745in}}%
\pgfpathlineto{\pgfqpoint{2.963384in}{3.810916in}}%
\pgfpathlineto{\pgfqpoint{3.013033in}{3.811906in}}%
\pgfpathlineto{\pgfqpoint{3.067197in}{3.810657in}}%
\pgfpathlineto{\pgfqpoint{3.121360in}{3.807145in}}%
\pgfpathlineto{\pgfqpoint{3.180037in}{3.800988in}}%
\pgfpathlineto{\pgfqpoint{3.238714in}{3.792599in}}%
\pgfpathlineto{\pgfqpoint{3.301904in}{3.781323in}}%
\pgfpathlineto{\pgfqpoint{3.369608in}{3.766967in}}%
\pgfpathlineto{\pgfqpoint{3.441826in}{3.749433in}}%
\pgfpathlineto{\pgfqpoint{3.523071in}{3.727469in}}%
\pgfpathlineto{\pgfqpoint{3.622370in}{3.698232in}}%
\pgfpathlineto{\pgfqpoint{3.766805in}{3.653089in}}%
\pgfpathlineto{\pgfqpoint{3.956377in}{3.594124in}}%
\pgfpathlineto{\pgfqpoint{4.055676in}{3.565629in}}%
\pgfpathlineto{\pgfqpoint{4.136921in}{3.544526in}}%
\pgfpathlineto{\pgfqpoint{4.209138in}{3.527948in}}%
\pgfpathlineto{\pgfqpoint{4.276843in}{3.514651in}}%
\pgfpathlineto{\pgfqpoint{4.340033in}{3.504506in}}%
\pgfpathlineto{\pgfqpoint{4.398710in}{3.497292in}}%
\pgfpathlineto{\pgfqpoint{4.452873in}{3.492714in}}%
\pgfpathlineto{\pgfqpoint{4.507036in}{3.490311in}}%
\pgfpathlineto{\pgfqpoint{4.561200in}{3.490258in}}%
\pgfpathlineto{\pgfqpoint{4.610849in}{3.492421in}}%
\pgfpathlineto{\pgfqpoint{4.660499in}{3.496839in}}%
\pgfpathlineto{\pgfqpoint{4.710149in}{3.503643in}}%
\pgfpathlineto{\pgfqpoint{4.755285in}{3.512013in}}%
\pgfpathlineto{\pgfqpoint{4.800421in}{3.522567in}}%
\pgfpathlineto{\pgfqpoint{4.845557in}{3.535405in}}%
\pgfpathlineto{\pgfqpoint{4.890693in}{3.550627in}}%
\pgfpathlineto{\pgfqpoint{4.935829in}{3.568335in}}%
\pgfpathlineto{\pgfqpoint{4.980965in}{3.588628in}}%
\pgfpathlineto{\pgfqpoint{5.026101in}{3.611607in}}%
\pgfpathlineto{\pgfqpoint{5.066723in}{3.634668in}}%
\pgfpathlineto{\pgfqpoint{5.107346in}{3.660058in}}%
\pgfpathlineto{\pgfqpoint{5.147968in}{3.687852in}}%
\pgfpathlineto{\pgfqpoint{5.188591in}{3.718122in}}%
\pgfpathlineto{\pgfqpoint{5.229213in}{3.750942in}}%
\pgfpathlineto{\pgfqpoint{5.269836in}{3.786385in}}%
\pgfpathlineto{\pgfqpoint{5.310458in}{3.824524in}}%
\pgfpathlineto{\pgfqpoint{5.351080in}{3.865432in}}%
\pgfpathlineto{\pgfqpoint{5.396216in}{3.914222in}}%
\pgfpathlineto{\pgfqpoint{5.441353in}{3.966622in}}%
\pgfpathlineto{\pgfqpoint{5.486489in}{4.022732in}}%
\pgfpathlineto{\pgfqpoint{5.527111in}{4.076486in}}%
\pgfpathlineto{\pgfqpoint{5.527111in}{4.076486in}}%
\pgfusepath{stroke}%
\end{pgfscope}%
\begin{pgfscope}%
\pgfpathrectangle{\pgfqpoint{0.792566in}{0.548486in}}{\pgfqpoint{4.960000in}{3.696000in}}%
\pgfusepath{clip}%
\pgfsetrectcap%
\pgfsetroundjoin%
\pgfsetlinewidth{0.501875pt}%
\definecolor{currentstroke}{rgb}{0.000000,0.000000,0.000000}%
\pgfsetstrokecolor{currentstroke}%
\pgfsetdash{}{0pt}%
\pgfpathmoveto{\pgfqpoint{0.792566in}{3.690279in}}%
\pgfpathlineto{\pgfqpoint{5.752566in}{3.690279in}}%
\pgfusepath{stroke}%
\end{pgfscope}%
\begin{pgfscope}%
\pgfpathrectangle{\pgfqpoint{0.792566in}{0.548486in}}{\pgfqpoint{4.960000in}{3.696000in}}%
\pgfusepath{clip}%
\pgfsetrectcap%
\pgfsetroundjoin%
\pgfsetlinewidth{0.501875pt}%
\definecolor{currentstroke}{rgb}{0.000000,0.000000,0.000000}%
\pgfsetstrokecolor{currentstroke}%
\pgfsetdash{}{0pt}%
\pgfpathmoveto{\pgfqpoint{3.272566in}{0.548486in}}%
\pgfpathlineto{\pgfqpoint{3.272566in}{4.244486in}}%
\pgfusepath{stroke}%
\end{pgfscope}%
\end{pgfpicture}%
\makeatother%
\endgroup%

	\caption{Der Funktiongraph des Polynoms  $p(x) = x^3 - 4x^2 - 7x + 10$.}
	\label{fig:poly1}
\end{figure}

% \begin{python}{Rekursive Python Implementierung von Gleichung \ref{eq:fib}}
% def fib(x):
% 	if x == 0:
% 		return 0
% 	elif x == 1:
% 		return 1
% 	else:
% 		return fib(x-1) + fib(x-2)
% \end{python}


% \begin{python}{import von Paketen und Definition eines symbolischen Polynoms.}
% import sympy as sp
% sp.init_session()
% %matplotlib inline # Nur sinnvoll in Notebook environments

% p= x**5 +2*x**4 -x**3 -2*x**2
% \end{python}



\begin{python}{Symbolische extraktion von Nullstellen in Python}
import sympy as sp
from sympy.abc import x
p = x**3 - 4*x**2 - 7*x + 10
sp.roots(p,x)

Out[1]: {5: 1, 1: 1, -2: 1}
\end{python}

Hier ist hervorzuheben dass \texttt{\{5: 1, 1: 1, -2: 1\}} ein sog. \texttt{Dictionary} darstellt das sowohl die Position der Nullstellen als auch die \emph{Anzahl} der Nullstellen an der jeweiligen Stelle enthält. 
Beispielsweise das Polynom $f(x) = x^2 - 6^x + 9$ liefert das \texttt{Dictionary} \texttt{\{3: 2\}}, also eine \emph{Doppelte Nullstelle} an der Position $x=3$. 




% \begin{python}
% 	In [1]: 64**(2/3)
% Out[1]: 15.999999999999998

% In [2]: 32**(1/5)
% Out[2]: 2.0

% In [3]:


\subsection{Symbolische Linearfaktorzerlegung}
\begin{python}{Symbolische Linearfaktorzerlegung in Python}
p = x**2 - 6*x +9
sp.factor(p)
Out[1]: (x - 3)**2
\end{python}


\subsection{Symbolische Partialbruchzerlegung}

\begin{python}{Symbolische Partialbruchzerlegung in Python}
p = (x + 2)/(x + 1)
sp.apart(p)
Out[1]: 1 + 1/(x + 1)
\end{python}




\section{Aufgaben}
\begin{enumerate}
\item Versuchen Sie Python Code zu entwerfen der die Funktion in Abbildung \ref{fig:functManipul} erzeugt.
\item In Abschnitt \ref{sec:manipFuncts} wurde die Auswirkung von verschiedenen 'Manipulationen' besprochen unter der Annahme dass das argument der Funktion die Zeitachse darstellt ($f(t)$ wobei $t$ die zeit in Sekunden). Welche Interpretationen drängen sich auf unter der Annahme dass das Argument die Frequenzachse ist, also $f(\omega)$ wobei $\omega$ die Frequenz? An dieser Stelle ist es nicht wichtig ob es sich um die Kreisfrequenz, normierte Frequenz oder Frequenz in Hz handelt.

\item $4^x = 262144$, löse für $x$.

\item Sie finden folgende funktion vor $g(t) = u(t-3) \cdot sin(t) \cdot \frac{1}{t-3}$. Plotten / Skizzieren sie diese. Könnten Sie sie zeitlich 'umdrehen' (spiegelung um die ordinate)?

\item Gegeben $$ \frac{x^{2} + 3 x + 2}{\left(x - 2\right) \left(x + 1\right) \left(x^{2} + x + 1\right)}$$ ermitteln Sie die Partialbruchzerlegung in python!

\item Führen Sie eine Polynomdivision von $\frac{g}{f}$ durch:
$$g(x) = 2 x^{5} + 4 x^{4} - 13 x^{3} - 28 x^{2} - 7 x $$
$$f(x) = x^2-7 $$

\end{enumerate}
